
\documentclass{amsbook}

\usepackage{parskip} % No indent, but space between paragraphs.
\usepackage{multicol}

\usepackage[shortlabels]{enumitem} % For formatting HW subproblems
\newcommand{\ssp}{\begin{enumerate}[(a), leftmargin=*]}
\newcommand{\esp}{\end{enumerate}}

% Exercise/Answer Setup:
\usepackage{answers}
\Newassociation{sol}{Solution}{ans}
\newtheorem{exc}{}
\newenvironment{ex}{\begin{exc}\normalfont}{\end{exc}}

\usepackage[hidelinks]{hyperref} % Clickable TOC, index, etc.
\usepackage{tikz}
\usetikzlibrary{quotes, angles, patterns}
\usepackage{pgfplots}
\usepgfplotslibrary{external}
\tikzexternalize % faster compiling by creating separate figure files.

\usepackage{xcolor} % For color environments
\newcommand{\red}{\color{red}}
\newcommand{\blue}{\color{blue}}
\definecolor{darkgreen}{HTML}{006400}
\newcommand{\green}{\color{darkgreen}}
\newcommand{\purple}{\color{purple}}

%%%%%%%%%%%%%%%%%%%%%%%%%%%%%%%%%%%%%
\usepackage{comment}
%\includecomment{unfinished}
\excludecomment{unfinished}
%%%%%%%%%%%%%%%%%%%%%%%%%%%%%%%%%%%%%

%load last to mark pdfs?
\usepackage{hyperxmp,}\usepackage[type={CC},modifier={by-nc-sa},version={4.0},]{doclicense}

\numberwithin{section}{chapter}
\numberwithin{equation}{chapter}

\makeatletter  % added for exercises numbering reset each section
\@addtoreset{exc}{section}
\makeatother

%Joe's Macros
\newcommand{\drawgridxxyy}[4]{
	\draw[thin, color=lightgray] (#1,#3) grid (#2,#4);
	\draw[thick,->] (#1,0) -- (#2,0) node[right] {$x$}; 
	\draw[thick, ->] (0,#3) -- (0,#4) node[above] {$y$};
}
\newcommand{\drawgridxxyyb}[4]{
	%\draw[thin, color=lightgray] (#1,#3) grid (#2,#4);
	\draw[thick,->] (#1,0) -- (#2,0) node[right] {$x$}; 
	\draw[thick, ->] (0,#3) -- (0,#4) node[above] {$y$};
}
\newcommand{\drawgridxxyyll}[6]{
	\draw[thin, color=lightgray] (#1,#3) grid (#2,#4);
	\draw[thick,->] (#1,0) -- (#2,0) node[right] {$#5$}; 
	\draw[thick, ->] (0,#3) -- (0,#4) node[above] {$#6$};
}
\newcommand{\drawgridxxyyllb}[6]{
	%	\draw[thin, color=lightgray] (#1,#3) grid (#2,#4);
	\draw[thick,->] (#1,0) -- (#2,0) node[right] {$#5$}; 
	\draw[thick, ->] (0,#3) -- (0,#4) node[above] {$#6$};
}
\newcommand{\qi}[1]{\begin{itemize}\item #1 \end{itemize}}
\newcommand{\ds}{\displaystyle}

% Extra Symbol shortcuts:
\newcommand{\Q}{\mathbb{Q}}
\newcommand{\Z}{\mathbb{Z}}
\newcommand{\R}{\mathbb{R}}
\newcommand{\N}{\mathbb{N}}
\newcommand{\dg}{{^\circ}}


\makeindex

\begin{document}

\frontmatter

\title{Horizontal Precalculus 0.2}

\author{J. Herning}
\address{Northern Virginia Community College, Loudoun Campus}
\email{jherning@nvcc.edu}
\date{\today}

\begin{abstract} 
This course is designed to prepare NOVA students for Calculus and cover all the required topics for MTH 167.

A traditional Precalculus class proceeds ``vertically'' though the big topics -- the theory of functions, polynomials and rational functions, logarithms, and trigonometry -- completing each one before progressing to the next. This course cut across, presenting the easier parts of all the big topics first, then progressively ratcheting up the difficulty.

The author hopes that this approach will prepare NOVA students for Calculus by smoothing the progression of difficulty and visiting the main topics repeatedly through the course.

\bigskip
\noindent Some exercises in this book are taken from \textit{Precalculus}, 3rd edition, by Carl Stitz and Jeffrey Zeager, and \textit{Precalculus} by Collingwood, Prince, and Conroy. Exercises from the former were sometimes retyped and sometimes copied from the source-code. Exercises from the latter are used with permission and were retyped.


\bigskip
\doclicenseThis

% (https://creativecommons.org/licenses/by-nc-sa/4.0/)
\end{abstract}

\maketitle

\tableofcontents

\mainmatter

\chapter{First Cycle}

\section{Numbers, statements, equations, and functions}


\medskip
\begin{tikzpicture}[scale=0.9]
\draw (14,10) rectangle (0,0) node [below right] {All Real Numbers\index{real numbers}, $\R$};
\draw (9,5) circle [radius=4.5];
\draw (9,5) circle [radius=3];
\draw (9,5) circle [radius=1.5];
\draw (9,1) node [above] {Rational\index{rational numbers}, $\Q$};
\draw (6,2.7) node {$\frac{1}{2}$};
\draw (6,7) node {$\frac{20}{8}$};
\draw (8,8.5) node {$-\frac{2}{3}$};
\draw (12.7,5) node {\small $1.33\bar{3}$};
\draw (11,2) node {$4.756$};
\draw (9,2.5) node [above] {Integral\index{integers (integral numbers)}, $\Z$};
\draw (7, 3.6) node {-2};
\draw (7.5, 6.5) node {-1};
\draw (11.3, 5.5) node {\small-324};
\draw (11, 4) node {0};
\draw (9,5) node [above] {Natural\index{natural numbers}, $\N$};
\draw (9,5) node [below] {$1,\ 2,\ 3,\ \cdots$};
\draw (2.5,2.5) node [above] {Irrational\index{irrational numbers}};
\draw (3,8) node {\large $\pi$};
\draw (1.5,6) node {\large $e$};
\draw (3.5,4) node {$\sqrt{2}$};
\draw (1,0.5) node [right] {\small $0.12345678910111213141516\cdots$};
\draw (2.5,2.5) node [below] {\small (Real but not rational)};
\end{tikzpicture}

\textit{Expressions}\index{expressions} -- AKA numerical expressions -- something that takes a numerical value (if its variables are fixed).

\textit{Statements}\index{statements}: true or false, possibly depending on the values of variables.

\textit{Equivalent statements}:\index{equivalent statements} statements whose truth values are always the same (``always'' meaning over different values of the variables, if any).

The two ``Algebra I'' rules of creating equivalent statements:
\begin{enumerate}
	\item Add the same number to the expression on both sides.
	\item Multiply both sides by the same \underline{non-zero} number.
	\item Be careful doing anything else. ``Doing the same thing to both sides'' does \underline{not} always create an equivalent statement.
\end{enumerate}

\textit{AND vs. OR}\index{and}\index{or}: when ``and'' connects two statements the result is true only if both are true and is false otherwise. When ``or'' is used, the result is true if one, the other, or both are true; it is only false if both are false.
\qi{``$x^2=4$'' is equivalent to ``$x=2$ \underline{or} $x=-2$'' not ``$x=2$ \underline{and} $x=-2$.''}

If an equation or inequality has one variable, the \textit{solution set}\index{solution set} is the set of values that \textit{satisfy}\index{satisfy} the equation, i.e. when substituted for the variable, they make it true.


\begin{comment}
\begin{itemize}
\item\ \ $2\cdot3+1 = 7$\ \ \ [true]
\item\ \ $2 < 7$\ \ \ [true]
\item\ \ $2\cdot3+1 = 6+8$\ \ \ [false]
\item\ \ $2\cdot3+1 \neq 6+8$\ \ \ [true]
\item\ \ $2x-5 = 13$\ \ [true if $x$ is 9 and false otherwise.]
\end{itemize}
\end{comment}



\textit{Sets \index{sets} of numbers}: defined by which numbers are included, e.g.:

\begin{itemize}
	\item \textit{intervals}: $(1,3)$, $[-3.5, 2)$, $(4,\infty)$, etc. Represent a contiguous range of numbers. The interval $(3.2, 4.9]$ includes $4.9$ but not $3.2$.
	\item explicit sets: $\{0, 3, 4.1 \}$, $\{ 2, 7, -10, \pi \}$, $\{\}$ (the empty set; also called $\emptyset$).
	\item The set $\{ 1,2,3, 4 \}$ is the same set as $\{ 3,4,2,1 \}$.
\end{itemize}


\textit{Solving...}\index{solving}
\begin{enumerate}
	\item ...an equation means finding the corresponding solution set (even if empty), as in, ``solve this equation.''
	\item ...can mean ``isolating'' a variable or other expression, as in, ``solve for $t$.''
	\item ...can mean understanding and articulating what is needed to answer some question or perform some exercise, as in, ``solve this problem.''
	\item These meanings are distinct but related.
\end{enumerate}

Example of $y$ (explicity) as a \textit{function of}\index{function of (informal)} $x$: $y = x^2-x+7$


Example of $s$ as a function of $t$: $s = 2^{t}-7$


Example of $A$ as a function of $r$: $A=\pi r^2$\ \  \ [Recall $\pi$ is just a number].


Example of $V$ as a function of $\ell$, $w$, and $h$: $V = \frac{1}{3}\ell w h$




\Opensolutionfile{ans}[ans11]
\subsection*{Exercises \nopunct} \hfill

Determine whether each number is integral, rational, irrational, or real. Write all that apply.
\begin{multicols}{2}

\begin{ex}
	17
	\begin{sol}
		integral, rational, real
	\end{sol}
\end{ex}


\begin{ex}
	$\frac{5}{3}$
	\begin{sol}
		rational, real.
	\end{sol}
\end{ex}


\begin{ex}
	$-\frac{144}{24}$
	\begin{sol}
		integral, rational, real.
	\end{sol}
\end{ex}

\begin{ex}
	-7.9
	\begin{sol}
		rational, real.
	\end{sol}
\end{ex}


\begin{ex}
	$0.002345454545\cdots$
	\begin{sol}
		rational, real.
	\end{sol}
\end{ex}


\begin{ex}
	The ratio of a circle's circumference to its diameter.
	\begin{sol}
		irrational, real.
	\end{sol}
\end{ex}
\end{multicols}

Determine the solution set of each. Use interval notation if possible.

\begin{multicols}{2}
	\begin{ex}
		$x=9$
		\begin{sol}
			$\{9\}$
		\end{sol}
	\end{ex}
	\begin{ex}
	$3x+6=0$
	\begin{sol}
		$\{-2\}$
	\end{sol}
\end{ex}
\begin{ex}
	$14>7x$
	\begin{sol}
		$(-\infty, 2)$
	\end{sol}
\end{ex}

	\begin{ex}
	$7x \leq 14$
	\begin{sol}
		$(-\infty, 2]$
	\end{sol}
\end{ex}

\begin{ex}
	$-7x \leq 14$
	\begin{sol}
		$[-2,\infty)$
	\end{sol}
\end{ex}


\begin{ex}
	$x^2=25$
	\begin{sol}
		$\{5, -5\}$
	\end{sol}
\end{ex}


\begin{ex}
	$x^2=x$
	\begin{sol}
		$\{0, 1\}$
	\end{sol}
\end{ex}

\begin{ex}
	$\sqrt{x} < 9$
	\begin{sol}
		$[0,81)$
	\end{sol}
\end{ex}

\begin{ex}
	If $R$ is a constant, the formula \\$\ds{P = \frac{nRT}{V}}$\\ represents \underline{\ \ \ \ } as a function of \underline{\ \ \ \ \ \ \ \ \ }.
	\begin{sol}
		$P$, $n$ and $T$. [$PV=nRT$ is called the ``ideal gas law'']
	\end{sol}
\end{ex}

\begin{ex}
	If $x_0$, $v_0$, and $a$ are constants, the formula \\$x = x_0+v_0t+ \frac{1}{2}a t^2$\\ represents \underline{\ \ \ \ } as a function of \underline{\ \ \ \ \ \ \ \ \ }.
	\begin{sol}
		$x$, $t$. [This formula computes the position of an object in one dimension under a constant acceleration.]
	\end{sol}
\end{ex}

\end{multicols}

Are the following statements true or false?
\begin{multicols}{2}
	\begin{ex}
		$2+4 = 2\cdot 3$
\begin{sol}
	true
\end{sol}
\end{ex}

	\begin{ex}
	$17 > 4\cdot 4+1$
	\begin{sol}
		false
	\end{sol}
\end{ex}

\begin{ex}
	$1+1=56$ or $2+4 = 2\cdot 3$
	\begin{sol}
		true
	\end{sol}
\end{ex}

\begin{ex}
	$1+1=2$ and $1+2\geq 4$
	\begin{sol}
		false
	\end{sol}
\end{ex}

\begin{ex}
	$3-5=-2$ or $7\cdot 2 = 14$
	\begin{sol}
		true
	\end{sol}
\end{ex}

\begin{ex}
	Every rational number is real.
	\begin{sol}
		true
	\end{sol}
\end{ex}


\begin{ex}
	Every irrational number is real.
	\begin{sol}
		true
	\end{sol}
\end{ex}



\begin{ex}
	Every real number is rational.
	\begin{sol}
		false
	\end{sol}
\end{ex}

\end{multicols}

\Closesolutionfile{ans}
\subsection*{Answers \nopunct} \hfill
\begin{multicols}{2}
\input{ans11}
\end{multicols}


\newpage
\section{Proportions, conversions, areas, and volumes} \label{conversionsSection}

Basic Conversion Factors:

Length\\
$ 1 \text{ in} = 2.54 \text{ cm}$\\
%$ 1 \text{ ft} = 0.3048 \text{ m}$\\
$ 1 \text{ mi} = 5,280 \text{ ft}$\\
%$ 1 \text{ km} \approx 0.63127 \text{ mi}$\\

\smallskip
Area\\
$ 1 \text{ acre} \approx 43,560 \text{ ft}^2$\\

\smallskip
Volume\\
$ 1 \text{ gal} = 3.785411784  \text{ liter}$\\

\smallskip
Mass\\
$ 1 \text{ lb} = 0.45359237 \text{ kg}$\\

\Opensolutionfile{ans}[ans12]
\subsection*{Exercises \nopunct} \hfill

Perform the following conversions by dimensional analysis using the conversion factors above. Report answers rounded to three significant figures.

\begin{multicols}{2}
	\begin{ex}
		72 in to cm
		\begin{sol}
			183 cm
		\end{sol}
	\end{ex}
	
	\begin{ex}
		4,520 ft to mi
		\begin{sol}
			0.856 mi
		\end{sol}
	\end{ex}
	
	\begin{ex}
		155 lb to kg
		\begin{sol}
			70.3 kg
		\end{sol}
	\end{ex}
	
\begin{ex}
	1.74 in to meters
	\begin{sol}
		0.0442 m
	\end{sol}
\end{ex}

\end{multicols}
\begin{ex} \label{ftm}
 How many feet are in one meter? (Report 5 significant figures.)
	\begin{sol}
		3.2808
	\end{sol}
\end{ex}

\begin{ex}
	How many square centimeters are in one square meter?
	\begin{sol}
		10,000
	\end{sol}
\end{ex}


\begin{ex} \label{boxml}
There are 1,000 milliliters (ml) in one liter. A milliliter is also a cubic centimeter. What is the volume, to the nearest milliliter, of a shipping box that measures 4 in by 8 in by 10 in?
	\begin{sol}
		5,244 ml. This is 5.244 liters.
	\end{sol}
\end{ex}

\begin{ex}
	How many square meters are in 2.58 acres? (Report 3 significant figures -- use \#\ref{ftm}.)
	\begin{sol}
		10,400
	\end{sol}
\end{ex}

\begin{ex}
	How many square miles are in 253 square kilometers? (Report 3 significant figures.)
	\begin{sol}
		97.7
	\end{sol}
\end{ex}

\begin{ex}
	Assume the amount of gas used by Jane's care is proportional to how far she has driven. If she uses 11 gallons driving 385 miles, how much gas will she need on a 1,200 mile road trip?
	\begin{sol}
		$\frac{240}{7} \approx 34.3$ gal.
	\end{sol}
\end{ex}

\begin{ex}
	Jane (from the previous problem) has \$120. If gas costs \$3.75 per gallon, how far can she afford to go?
	\begin{sol}
		1120 mi
	\end{sol}
\end{ex}

\begin{ex}
	Jane likes to drive 60 mi/h on the highway. How fast is this in feet per second (ft/s)?
	\begin{sol}
		88 ft/s
	\end{sol}
\end{ex}

\begin{ex}
	Jane crosses into Canada and drives 600 kilometers in 6 hours. What was her average speed in km/h (kilometers per hour)? In mi/h?
	\begin{sol}
		100 km/h, about 62.1 mi/h
	\end{sol}
\end{ex}

\begin{ex} %UW 1.7
	Which is a better deal: a 10 inch  diameter pizza for \$8 or a 15 inch diameter pizza for \$16?
	\begin{sol}
		Go for the 15 inch pie.
	\end{sol}
\end{ex}

\begin{ex} %UW 1.9
	During a typical evening in Seattle, \textit{Pagliacci} receives phone orders for pizza delivery at a constant rate: 18 orders in a typical 4 minute period. How many pies are sold in 4 hours? Assume \textit{Pagliacci} starts taking orders at 5:00 pm and the profit is a constant rate of \$11 on 10 orders. When will phone order profit exceed \$1,000?
	\begin{sol}
		1,080 pizzas sold in 4 hours. Profit reaches \$1,000 at about 8:22 pm.
	\end{sol}
\end{ex}

\begin{ex} %UW 1.3
	The density of lead is 11.34 g/cm$^3$ and the density of aluminum is 2.69 g/cm$^3$. Find the radius of lead and aluminum spheres each having mass 50 kg. Round results to two decimal places.
	\begin{sol}
		$r=10.17$ cm for lead and $r=16.43$ cm for aluminum.
	\end{sol}
\end{ex}

\begin{ex} %UW 1.4
	The Eiffel Tower has a mass of 7.3 million kilograms and a height of 324 meters. Its base is a square with side length of 125 meters. The steel used to make the Tower occupies a volume of 930 cubic meters. Air has a density of 1.225 kg per cubic meter. Suppose the tower was contained in a cylinder. Find the mass of the air in the cylinder. Is this more or less than the mass of the Tower?
	\begin{sol}
		The mass of the air is about 9.74 million kg, more than the mass of the Tower.
	\end{sol}
\end{ex}

\begin{ex} %UW 1.13
	During the 1950s, Seattle was dumping an average of 20 million gallons of sewage into Lake Washington each day. How many gallons of sewage were dumped in a week? In a year? In order to illustrate the amounts involved, imagine a rectangular prism whose base is the size of a football field (100 yards by 50 yards). What is the approximate height of such a prism containing the sewage dumped into Lake Washington in a single day? There are 7.5 gallons in one cubic foot.
	\begin{sol}
		140 million gallons per week; about 7300 million gallons per year. The prism should be about 20 yards in height.
	\end{sol}
\end{ex}

\begin{ex} %UW 1.12
	A water pipe mounted to the ceiling has a leak and is dripping onto the floor below, creating a circular puddle of water. The area of the circular puddle is increasing at a constant rate of 11 cm$^2$/hour. Find the area and radius of the puddle after 1 minute, 92 minutes, 5 hours, and 1 day. Is the radius of the puddle increasing at a constant rate?
	\begin{sol}
		The radii are about 0.2416, 2.317, 4.184, and 9.167 cm. No, the radius is not increasing at a constant rate.
		\end{sol}
\end{ex}

\begin{ex}
	Referring to Exercise \ref{boxml}, write the volume, $V$, in ml as a function of the height, $h$, width $w$, and length $\ell$ of the box where these lengths are measured in inches.
	\begin{sol}
		$V = 2.54^3\ell w h$
		\end{sol}
\end{ex}


\Closesolutionfile{ans}
\subsection*{Answers \nopunct} \hfill
\begin{multicols}{2}
	\input{ans12}
\end{multicols}


\newpage
\section{Linear equations} \label{lineq}

slope = $\ds{\frac{\text{rise}}{\text{run}} = \frac{\Delta y}{\Delta x}}$

slope-intercept form of a line: $y=mx+b$

point-slope form of a line: $y-y_0 = m(x-x_0)$

Intercepts:
\begin{itemize}
	\item $y$-intercept(s) of a graph: points on the graph that are also on the $y$-axis. From an equation, find them by letting $x=0$.
	\item $x$-intercept(s) of a graph: points on the graph that are also on the $x$-axis. From an equation, find them by letting $y=0$.
\end{itemize}

\Opensolutionfile{ans}[ans13]
\subsection*{Exercises \nopunct} \hfill

Solve the given linear equations. Find exact answers.

\begin{multicols}{2}
	\begin{ex}
		$\ds{3x-4=2-4(x-3)}$
		\begin{sol}
			$x=\frac{18}{7}$ [Solution set $\left\{\frac{18}{7}\right\}$ is also acceptable.]
		\end{sol}
	\end{ex}
	\begin{ex}
	$\ds{\frac{3-2t}{4} = 7t+1}$
	\begin{sol}
		$t=-\frac{1}{30}$
	\end{sol}
\end{ex}
	\begin{ex}
	$\ds{\frac{2(w-3)}{5} = \frac{4}{15}-\frac{3w+1}{9}}$
	\begin{sol}
		$w=\frac{61}{33}$
	\end{sol}
\end{ex}
	\begin{ex}
	$\ds{-0.02y+1000=0}$
	\begin{sol}
		$y=50000$
	\end{sol}
\end{ex}
	\begin{ex}
	$\ds{ \frac{49w-14}{7} =3w-(2-4w)   }$
	\begin{sol}
		All real numbers, i.e. $(-\infty,\infty)$ or $\R$.
	\end{sol}
\end{ex}
	\begin{ex}
	$\ds{ 7-(4-x) = \frac{2x-3}{2}   }$
	\begin{sol}
		No solution (The solution set is $\emptyset$.)
	\end{sol}
\end{ex}
	\begin{ex}
	$\ds{ 3t\sqrt{7}+5=0 }$
	\begin{sol}
		$t=-\frac{5}{3\sqrt{7}} = -\frac{5\sqrt{7}}{21}$. Changing to the second form will generally \textit{not be required}.
	\end{sol}
\end{ex}
\end{multicols}

\begin{ex}
	Solve for $y$: $3x+2y=4$
	\begin{sol}
		$y=\frac{4-3x}{2}$ or $y=-\frac{3}{2}x+2$
	\end{sol}
\end{ex}

\begin{ex}
	Solve for $y$: $x(y-3) = 2y+1$
	\begin{sol}
		$y=\frac{3x+1}{x-2}$ provided $x\neq 2$.
	\end{sol}
\end{ex}

\begin{ex}
	Solve for $m$: $E = \frac{1}{2}mv^2$
	\begin{sol}
		$m=\frac{2E}{v^2}$ provided $v \neq 0$.
	\end{sol}
\end{ex}

Find both the point-slope and slope-intercept form of the line with the given slope which passes through the given point
\begin{multicols}{2}
	\begin{ex}
		$m=3$, $P(3,-1)$
		\begin{sol}
			$y+1=3(x-3)$, $y=3x-10$
		\end{sol}
	\end{ex}
	\begin{ex}
	$m=-2$, $P(-5,8)$
	\begin{sol}
		$y-8=-2(x+5)$, $y=-2x-2$
	\end{sol}
\end{ex}
	\begin{ex}
	$m=-\frac{1}{5}$, $P(10,4)$
	\begin{sol}
		$y-4=-\frac{1}{5}(x-10)$, $y=-\frac{1}{5}x+6$
	\end{sol}
\end{ex}
\end{multicols}

Graph the function. Find the slope, $y$-intercept, and $x$-intercept, if any exist.
\begin{multicols}{2}
	\begin{ex}
		$y=2x-1$
		\begin{sol}
			(Check graph with a graphing utility.) slope $m=2$, $y$-intercept $(0,-1)$, $x$-intercept $\left(\frac{1}{2},0  \right)$
		\end{sol}
	\end{ex}
	\begin{ex}
	$y=3-\frac{1}{2}x$
	\begin{sol}
		slope $m=-\frac{1}{2}$, $y$-intercept $(0,3)$, $x$-intercept $\left(6,0  \right)$
	\end{sol}
\end{ex}
\end{multicols}


Find an equation for the line which passes through the given points. What are the intercepts of the graph?
\begin{multicols}{2}
	\begin{ex}
		$P(5,0)$, $Q(0,-8)$
		\begin{sol}
			$y=\frac{8}{5}x-8 $, $y$-intercept $(0,-8)$, $x$-intercept $(5,0)$
		\end{sol}
	\end{ex}
	\begin{ex}
		$P(-1,5)$, $Q(7,5)$
		\begin{sol}
			$y=5$, $y$-intercept $(0,5)$, no $x$-intercept
		\end{sol}
	\end{ex}
\end{multicols}


\begin{ex}
	Jeff can walk comfortably at 3 miles per hour. Find a linear function $d$ that represents the total distance Jeff can walk in $t$ hours, assuming he doesn't take any breaks.
	\begin{sol}
		$d=3t$
	\end{sol}
\end{ex}

\begin{ex}
	A landscape company charges \$45 per cubic yard of mulch plus a delivery charge of \$20. Find a linear function which computes the total cost $C$ (in dollars) to deliver $x$ cubic yards of mulch.
	\begin{sol}
		$C=45x+20$
		\end{sol}
\end{ex}

\begin{ex}
	A salesperson is paid \$200 per week plus 5\% commission on her weekly sales of $x$ dollars. Represent her weekly pay, $W$, as a function of her weekly sales. What must her weekly sales be in order for her to earn \$475.00 for the week?
	\begin{sol}
		$W = 200+0.05x$. She must make \$5500.
	\end{sol}
\end{ex}

\begin{ex}\label{chirps}
	A certain type of cricket chirps at an average rate of 10 chirps/hour when the temperature outside is 15 $^\circ$C but at a rate of 17 chirps/hour when the temperature is 25 $^\circ$C. Represent the average rate of chirps/hour, $r$, as a linear function of the temperature, $C$. What does this model predict for the rate if the temperature is 21 $^\circ$C? How much does the average chirp rate increase as the temperature increases by 1 $^\circ$C?
	\begin{sol}
		$r = 0.7C-0.5$, 14.2 chirps/hour, 0.7 chirps/hour
	\end{sol}
\end{ex}

\begin{ex} \label{FCtemp}\index{Fahrenheit}\index{Celsius}\index{centigrade}\index{temperature}
	Let $F$ represent a temperature in degrees Fahrenheit and $C$, in Celsius. The temperature 0 $^\circ$C corresponds to 32 $^\circ$F while 212 $^\circ$F corresponds to 100 $^\circ$C. Express $F$ as a function of $C$ then express $C$ as a function of $F$.
	\begin{sol}
		$F = \frac{9}{5}C+32$, $C = \frac{5}{9}F-\frac{160}{9}$
	\end{sol}
\end{ex}

\begin{ex}
	With reference to Exercise \ref{FCtemp}, is the temperature measured in $^\circ$C proportional to the temperature measured in $^\circ$F? What quantities in this scenario are proportional? If the temperature of an object increases by 2 $^\circ$C, what is the corresponding change in $^\circ$F?
	\begin{sol}
		No. The change in temperature as measured in one system is proportional to the change in temperature as measured in the other. $\frac{18}{5}$ $^\circ$F.
	\end{sol}
\end{ex}

\begin{ex}\label{FComp}
	Referring to Exercises \ref{chirps} and \ref{FCtemp}, let $F$ be the temperature, measured in $^\circ$F instead of $^\circ$C. Express the chirp rate as a function of $F$. Is the function linear?
	\begin{sol}
		$r = \frac{7}{18}F-\frac{233}{18}$, yes
	\end{sol}

\begin{ex}
	Give a real-world example of a quantity which is a function of some other quantity, but the relationship is non-linear. How can you tell?
\end{ex}
	
\end{ex}

\Closesolutionfile{ans}
\subsection*{Answers\nopunct} \hfill
\begin{multicols}{2}
	\input{ans13}
\end{multicols}


\newpage
\section{Linear inequalities and absolute value} \index{inequalities}

Numbers are assumed to be real unless we specify otherwise.

Let $c$ be a constant and $u$ be some expression. The following are the case when $u$ is defined:

If $c > 0$, $|u| = c \iff u=c$ or $u=-c$.\\
If $c < 0$, $|u| = c$ is false.\\
$|u|=0 \iff u=0$.

If $c > 0$, $|u| < c \iff -c < u < c \iff -c < u$ and $u < c$.\\
If $c > 0$, $|u| > c \iff u > c$ or $u < -c$. \\
If $c < 0$, $|u| < c$ is false.\\
If $c < 0$, $|u| > c$ is true.\\

While these rules seem tricky in abstract form, consider examples:

$|x| = 7 \iff x=7$ or $x=-7$.\\
$|y| = -4$ is false for all $y$.\\
$|x-4|=0 \iff x-4=0$.

$|t| < 2 \iff -2 < t < 2 \iff -2 < t$ and $t < 2$. $t$ must be in the interval $(-2,2)$.\\
$|x| > 3.5 \iff x > 3.5$ or $x < -3.5$. $x$ must be in $(-\infty, -3.5)\cup (3.5,\infty)$.\\
$|u| < -\frac{1}{3}$ is false for all $u$.\\
$|u| > -13$ is true for all $u$.\\

Recall,\index{absolute value} $\ds{|x| = \begin{cases} 
	-x & \text{if } x< 0, \\
	x & \text{if } x \geq 0. 
	\end{cases}  }$

So if $y=|x|$ then $y= \begin{cases} 
-x & \text{if } x< 0, \\
x & \text{if } x \geq 0. 
\end{cases}$

And the graph of $y=|x|$ must look like this:\\
\begin{tikzpicture}
	\drawgridxxyy{-3}{3}{-1}{3}
		\draw[very thick, samples=2, domain=-3:0] plot (\x, {-\x});
		\draw[very thick, samples=2, domain=0:3] plot (\x, {\x}) node [right] {$y=|x|$};
\end{tikzpicture}

$|x| = \sqrt{x^2}$ which is interesting!

$|x|$ also represents the distance from the 0 to $x$ on the number line.

The distance between the numbers $a$ and $b$ on the number line is $|a-b|$ or $|b-a|$.



\Opensolutionfile{ans}[ans14]
\subsection*{Exercises \nopunct} \hfill

Solve the equation or inequality. Give solution set in interval notation if possible.

\begin{multicols}{2}
\begin{ex}
	$3-4x \geq 0$
	\begin{sol}
		$\left(-\infty, \frac{3}{4}\right]$
	\end{sol}
\end{ex}

\begin{ex}
$\ds{ 2t-1 < 3-(4t-3)  }$
	\begin{sol}
		$\left(-\infty, \frac{7}{6}\right)$
	\end{sol}
\end{ex}

\begin{ex}
	$\ds{ -\frac{1}{2} \leq 5x-3 \leq \frac{1}{2}   }$
	\begin{sol}
		$\left[\frac{1}{2}, \frac{7}{10}  \right]$
	\end{sol}
\end{ex}

\begin{ex}
	$\ds{ 2y \leq 3-y < 7  }$
	\begin{sol}
		$\left(-4, 1\right]$
	\end{sol}
\end{ex}

\begin{ex}
	$\ds{ |3t-1|=10  }$
	\begin{sol}
		$t=-3$ or $t=\frac{11}{3}$
		
		 [Solution set $\left\{-3, \frac{11}{3}\right\} $ also fine.]
	\end{sol}
\end{ex}

\begin{ex}
	$\ds{ 4-|y| = 3  }$
	\begin{sol}
		$y=1$ or $y=-1$
	\end{sol}
\end{ex}

\begin{ex}
	$\ds{ 2|5m+1|-3=0  }$
	\begin{sol}
		$m=-\frac{1}{2}$ or $m=\frac{1}{10}$
	\end{sol}
\end{ex}


\begin{ex}
	$\ds{ |7x-1|+2 = 0  }$
	\begin{sol}
		No solution
	\end{sol}
\end{ex}


\begin{ex}
	$\ds{ \frac{|2v+1|-3}{4} = \frac{1}{2} - |2v+1|}$
	\begin{sol}
		$v=-1$ or $v=0$
	\end{sol}
\end{ex}

\begin{ex} \label{distanceIneq}
	$\ds{ |x-2| < 5 }$
	\begin{sol}
		$(-3, 7)$
	\end{sol}
\end{ex}

\begin{ex}
	$\ds{ \frac{|2x+1|}{-4} \geq -\frac{2}{3} }$
	\begin{sol}
		$\left[ -\frac{11}{6}, \frac{5}{6} \right]$
	\end{sol}
\end{ex}

\begin{ex}
	$\ds{ |x+2| = 4-\frac{1}{2}x }$
	
	Hint: Split into two cases -- one for $x+2 \geq 0$ and one for $x+2 < 0$.
	\begin{sol}
		$x=-12$ or $x=\frac{4}{3}$
	\end{sol}
\end{ex}

\begin{ex}
	$\ds{ 3|5-x| = x+1 }$
	
	\begin{sol}
		$x=8$ or $x=\frac{7}{2}$
	\end{sol}
\end{ex}
\begin{ex}
	$\ds{ 3+x = \frac{\ds\left| 1-\frac{x}{2} \right|}{2} }$
	
	\begin{sol}
		$x=-2$
	\end{sol}
\end{ex}

\begin{ex}
	$\ds{ 4-3x = |3x+1|+8 }$
	
	\begin{sol}
		No solution
	\end{sol}
\end{ex}	
\end{multicols}

\begin{ex}
	Which inequality from the above section is a mathematical representation of the statement, ``the quantity $x$ is less than 5 away from the number 2''? Represent the following statements with a similar equation or inquality: ``$x$ is 4 away from -2;'' ``$x$ is 4 away from -2 or closer,'' ``$x$ is more than 4 away from -2.''
	\begin{sol}
		Exercise \ref{distanceIneq}, $|x-(-2)| = 4$, $|x-(-2)| \leq 4$, $|x-(-2)| > 4$. Note $|x+2|$ can also be used in place of $|x-(-2)|$.
	\end{sol}
\end{ex}

Find the $x$ and $y$-intercept(s) of the graph of the function by solving the appropriate equations. Then graph the function by first rewriting it with a piecewise definition in the style of $|x|$. We will learn other ways to draw the graph later.

\begin{multicols}{2}
\begin{ex}
	$y=|x-2|$
	\begin{sol}
		$\ds{y = \begin{cases} 
			2-x & \text{if } x< 2, \\
			x-2 & \text{if } x \geq 2. 
			\end{cases}  }$
		
		$x$-intercept: $(2,0)$. $y$-intercept: $(0,2)$. Check graph on a graphing utility.
	\end{sol}
\end{ex}

\begin{ex}
	$y=5-|2x-3|$
	\begin{sol}
		$\ds{y = \begin{cases} 
			2x+2 & \text{if } x< \frac{3}{2}, \\
			8-2x & \text{if } x \geq \frac{3}{2}. 
			\end{cases}  }$
		
		$x$-intercepts: $(-1,0)$ and $(4,0)$. $y$-intercept: $(0,2)$.
	\end{sol}
\end{ex}
\end{multicols}

\begin{ex}
What temperature values in $^\circ$C are equivalent to the temperature range 50 $^\circ$F to 95 $^\circ$F? Set up and solve an inequality to answer the question. Refer to Exercise \ref{FCtemp} in Section \ref{lineq}.
\begin{sol}
	A good (compound) inequality is:
	
	$50 \leq \frac{9}{5}C + 32 \leq 95$
	
	which reduces to $10 \leq C \leq 35$.
\end{sol}
\end{ex}


\begin{ex}
	Jane is currently 40 inches tall while John is 45 inches tall. Jane is growing at a rate of 3 inches per year while John is growing at a rate of 2.2 inches per year. Assuming they continue to grow at the same rates, when will their difference in height be one inch or less? Let $t$ be the number of years after the current time and solve this problem by setting up and solving an appropriate \textit{single} inequality
	\begin{sol}
		A good inequality is:
		
		$|(40+3t)-(45+2.2t)| \leq 1$.
		
		The solution set is $[5, 7.5]$, so their heights will be no greater than one inch apart between 5 and 7.5 years from now (after which Jane will be more than an inch taller).
	\end{sol}
\end{ex}

\begin{ex}
	If one pound is 16 oz, how many grams are in 2.4 oz? Round to the nearest gram. Refer to Section \ref{conversionsSection}. \begin{sol} 68 g\end{sol}

\end{ex}


\Closesolutionfile{ans}
\subsection*{Answers \nopunct} \hfill
\begin{multicols}{2}
	\input{ans14}
\end{multicols}



\newpage
\section{Powers, order of operations practice}

Assume $b>0$ and  $q \neq 0$. Good facts to know about powers: %, $n \neq 0$, $m \neq 0$.
\index{powers}\index{square root}\index{radicals}
\begin{enumerate}
	\item $b^0=1$
	\item $\ds{b^{-1} = \frac{1}{b}}$
	\item $\ds{b^{-n} = \frac{1}{b^n}}$
	\item $\ds{b^{n+m} = b^nb^m} \ \ \ \leftarrow $ the one fact of powers to rule them all!
	\item $\ds{b^{n-m} = \frac{b^n}{b^m}}$
	\item $\ds{\left(b^n\right)^m = b^{nm}}$
	\item $\ds{b^\frac{1}{q} = \sqrt[q]{b} }$
	\item $\ds{b^\frac{p}{q} = \sqrt[q]{\ds b^p} = \left( \sqrt[q]{b} \right)^p }$
\end{enumerate}

\noindent \qi{Some facts are true if $b\leq 0$, but fractional powers of negative numbers can be pathological, and we like to avoid them when possible.}

\index{order of operations}\index{intimacy (numerical)}
Numerical intimacy:
\begin{itemize}
	\item $5^2\ \ \ \leftarrow$ Very intimate (do first).
	\item $7\cdot3\ \ \ \leftarrow$ Intimate (do after powers).
	\item $6+8\ \ \ \leftarrow$ Meh (do last -- same with subtraction).
\end{itemize}

Addition and subtraction have the same intimacy, ties are broken by the ``work left-to-right'' rule:
\qi{$6-2+1 = 5$ \ \ (not 3!)} 
\qi{This may help: $6-2+1 = 6+(-1\cdot2)+1 = 6+(-2)+1 = 5$.}

Parentheses override these rules and create ``blocks.''

Fraction bars also create ``blocks'' -- the numerator and denominator.

We will not use the symbol ``{\large $\div $}'' but look at it to remind yourself that division is essentially a fraction bar (the two dots are just placeholders for the numerator and denominator).

Gets 'em every time: 
\begin{itemize}
	\item $-3^2 = -\left(3^2\right) = -9$.
	\item $\left(-3\right)^2 = 9$.
\end{itemize}

\Opensolutionfile{ans}[ans15]
\subsection*{Exercises \nopunct} \hfill

Perform the indicated operations and simplify (without a calculator).
\begin{multicols}{3}

\begin{ex}
	$5 - 2 + 3$
	\begin{sol}
	$6$	
	\end{sol}
\end{ex}

\begin{ex}
	$5 - (2+3)$
	\begin{sol}
		$0$
	\end{sol}
\end{ex}

\begin{ex}
	$\dfrac{2}{3} - \dfrac{4}{7}$
	\begin{sol}
	$\dfrac{2}{21}$	
	\end{sol}
\end{ex}

\begin{ex}
	$\dfrac{3}{8} + \dfrac{5}{12}$
	\begin{sol}
	$\dfrac{19}{24}$	
	\end{sol}
\end{ex}

\begin{ex}
	$\dfrac{5-3}{-2-4}$
	\begin{sol}
	$-\dfrac{1}{3}$	
	\end{sol}
\end{ex}

\begin{ex}
	$\dfrac{2(-3)}{3 - (-3)}$
	\begin{sol}
	$-1$	
	\end{sol}
\end{ex}

\begin{ex}
	$\dfrac{2(3)-(4-1)}{2^2 + 1}$
	\begin{sol}
		$\dfrac{3}{5}$
	\end{sol}
\end{ex}

\begin{ex}
	$\dfrac{4 - 5.8}{2 - 2.1}$
	\begin{sol}
		$18$
	\end{sol}
\end{ex}

\begin{ex}
	$\dfrac{1 - 2(-3)}{5(-3) + 7}$
	\begin{sol}
	$-\dfrac{7}{8}$	
	\end{sol}
\end{ex}

\begin{ex}
	$\dfrac{5(3) - 7}{2(3)^2-3(3)-9}$
	\begin{sol}
		 Undefined.
	\end{sol}
\end{ex}

\begin{ex}
	$\dfrac{2((-1)^2-1)}{((-1)^2+1)^2}$
	\begin{sol}
	$0$	
	\end{sol}
\end{ex}

\begin{ex}
	$\dfrac{(-2)^2 - (-2) - 6}{(-2)^2 - 4}$
	\begin{sol}
		Undefined.
	\end{sol}
\end{ex}

\begin{ex}
	$\dfrac{3 - \frac{4}{9}}{-2 - (-3)}$
	\begin{sol}
		$\dfrac{23}{9}$
	\end{sol}
\end{ex}

\begin{ex}
	$\dfrac{\frac{2}{3} - \frac{4}{5}}{4 - \frac{7}{10}}$
	\begin{sol}
		$-\dfrac{4}{99}$
	\end{sol}
\end{ex}

\begin{ex}
	$\dfrac{2\left(\frac{4}{3}\right)}{1 - \left(\frac{4}{3}\right)^2}$
	\begin{sol}
	$-\dfrac{24}{7}$	
	\end{sol}
\end{ex}

\begin{ex}
	$\dfrac{1 - \left(\frac{5}{3}\right)\left(\frac{3}{5}\right)}{1 + \left(\frac{5}{3}\right)\left(\frac{3}{5}\right)}$
	\begin{sol}
	$0$	
	\end{sol}
\end{ex}

\begin{ex}
	$\left(\dfrac{2}{3}\right)^{-5}$
	\begin{sol}
		$\dfrac{243}{32}$
	\end{sol}
\end{ex}

\begin{ex}
	$3^{-1} - 4^{-2}$
	\begin{sol}
		$\dfrac{13}{48}$
	\end{sol}
\end{ex}

\begin{ex}
	$\dfrac{1 + 2^{-3}}{3 - 4^{-1}}$
	\begin{sol}
	$\dfrac{9}{22}$	
	\end{sol}
\end{ex}

\begin{ex}
	$\dfrac{3 \cdot 5^{100}}{12\cdot 5^{98}}$
	\begin{sol}
	$\dfrac{25}{4}$	
	\end{sol}
\end{ex}

\begin{ex}
	$\sqrt{3^2 + 4^2}$
	\begin{sol}
	$5$	
	\end{sol}
\end{ex}

\begin{ex}
	$\sqrt{12} - \sqrt{75}$
	\begin{sol}
		$-3\sqrt{3}$
	\end{sol}
\end{ex}

\begin{ex}
	$(-8)^{2/3} - 9^{-3/2}$
	\begin{sol}
	$\dfrac{107}{27}$	
	\end{sol}
\end{ex}

\begin{ex}
	$\left(-\frac{32}{9}\right)^{-3/5}$
	\begin{sol}
		$-\dfrac{3\sqrt[5]{3}}{8} = -\dfrac{3^{6/5}}{8}$
	\end{sol}
\end{ex}


\begin{ex}
	$\sqrt{(3-4)^2 + (5-2)^2}$
	\begin{sol}
		 $\sqrt{10}$
	\end{sol}
\end{ex}

\begin{ex}
	$\sqrt{(2 - (-1))^2 + \left(\frac{1}{2} - 3\right)^2}$ 
	\begin{sol}
	$\dfrac{\sqrt{61}}{2}$ 	
	\end{sol}
\end{ex}

\begin{ex}
	$\sqrt{(\sqrt{5} - 2\sqrt{5})^2 + (\sqrt{18} - \sqrt{8})^2}$
	\begin{sol}
		$\sqrt{7}$
	\end{sol}
\end{ex}

\begin{ex}
	$\dfrac{-12 + \sqrt{18}}{21}$
	\begin{sol}
	 $\dfrac{-4 + \sqrt{2}}{7}$	
	\end{sol}
\end{ex}

\begin{ex}
	$\dfrac{-2 - \sqrt{(2)^2 - 4(3)(-1)}}{2(3)}$
	\begin{sol}
		$-1$
	\end{sol}
\end{ex}

\begin{ex}
	 $\dfrac{-(-4) + \sqrt{(-4)^2 - 4(1)(-1)}}{2(1)}$
	\begin{sol}
		$2 + \sqrt{5}$
	\end{sol}
\end{ex}
\end{multicols}

\begin{ex}
	$2(-5)(-5+1)^{-1} + (-5)^2(-1)(-5+1)^{-2}$
	\begin{sol}
		$\dfrac{15}{16}$
	\end{sol}
\end{ex}

\begin{ex}
	$3\sqrt{2(4)+1} + 3(4)\left(\frac{1}{2}\right)(2(4)+1)^{-1/2}(2)$
	\begin{sol}
		$13$
	\end{sol}
\end{ex}

\begin{ex}
	$2(-7)\sqrt[3]{1-(-7)} + (-7)^2 \left(\frac{1}{3}\right)(1-(-7))^{-2/3}(-1)$
	\begin{sol}
		$-\dfrac{385}{12}$
	\end{sol}
\end{ex}





\Closesolutionfile{ans}
\subsection*{Answers \nopunct} \hfill
\begin{multicols}{3}
	\input{ans15}
\end{multicols}




\newpage
\section{Power and exponential function graphs, translation}
\label{JimCarSection}
\index{power functions}\index{translation}\index{square root}\index{radicals}
\begin{tikzpicture} [scale=1.52]
\drawgridxxyy{-3}{3}{-3}{3}
\draw[very thick, samples=2, domain=-3:3] plot (\x, {\x}) node [above left] {$x$};
\draw[thick, samples=10, domain=-1.73205:1.73205, smooth] plot (\x, {\x*\x}) node [above right] {\small $x^2$};
\draw[thick, dashed, samples=10, domain=-1.44225:1.44225, smooth] plot (\x, {\x*\x*\x}) node [above ] {\small\ \ \ \  $x^3$};
\draw[thick, dotted, samples=10, domain=-1.31607:1.31607, smooth] plot (\x, {\x*\x*\x*\x}) node [above] {\small $x^4$};

\draw (0.1, -1.3) node [right, fill=white] {\textbf{Power functions}\ \  $y=x^p$};
%\draw (1, -1.7) node [right, fill=white] {$y=x^p$, $p\geq 1$};
\draw (0.3, -1.7) node [right, fill=white] {Positive Integers};
\end{tikzpicture}


\begin{tikzpicture} [scale=1.52]
\drawgridxxyy{-3}{3}{-3}{3}
\draw[very thick, samples=2, domain=-3:3] plot (\x, {\x}) node [above right] {$x$};
\begin{scope}[xscale=-1,rotate=90]
\draw[thick, samples=10, domain=0:1.73205, smooth] plot (\x, {\x*\x}) node [above right] {\small $x^{\frac{1}{2}} = \sqrt{x}$};
\draw[thick, dashed, samples=10, domain=-1.44225:1.44225, smooth] plot (\x, {\x*\x*\x}) node [above right] {\small $x^{\frac{1}{3}} = \sqrt[3]{x}$};
\draw[thick, dotted, samples=5, domain=0:1.31607, smooth] plot (\x, {\x*\x*\x*\x}) node [right] {\small $x^{\frac{1}{4}} = \sqrt[4]{x}$};
\end{scope}
\draw (0.1, -1.3) node [right, fill=white] {\textbf{Power functions}\ \  $y=x^p$};
%\draw (1, -1.7) node [right, fill=white] {$y=x^p$, $0 < p\leq 1$};
\draw (0.3, -1.7) node [right, fill=white] {Fractional Powers};
\end{tikzpicture}

\begin{tikzpicture} [scale=1.52]
\drawgridxxyy{-3}{3}{-3}{3}
%\draw[very thick, samples=2, domain=-3:3] plot (\x, {\x}) node [above left] {$x$};
%\draw[thick, samples=20, domain=-3:-1/3, smooth] plot (\x, {1/\x}) node [below] {\small $x^{-1} = \frac{1}{x}$};

\draw[very thick, samples=20, domain=1/3:3, smooth] plot ({1/\x}, {\x} ) node [above] {\small $\ds \frac{1}{x}$};
\draw[very thick, samples=20, domain=1/3:3, smooth] plot ({-1/\x}, {-\x} ) node [below] {\small $\ds \frac{1}{x}$};

\draw[thick, samples=20, domain=1/3:1.73025, smooth] plot ({1/\x}, {\x*\x} ) node [above] {\small $\ds \frac{1}{x^2}$};
\draw[thick, samples=20, domain=1/3:1.73025, smooth] plot ({-1/\x}, {\x*\x} ) node [above] {\small\ \  $\ds \frac{1}{x^2}$};

\draw[thick, dashed, samples=20, domain=1/3:1.44225, smooth] plot ({1/\x}, {\x*\x*\x} );% node [above right] {\small $\ds \frac{1}{x^3}$};
\draw[thick, dashed, samples=20, domain=1/3:1.44225, smooth] plot ({-1/\x}, {-\x*\x*\x} ) node [below] {\small $\ds \frac{1}{x^3}$};

\draw[thick, dotted, samples=20, domain=1/3:1.31607, smooth] plot ({1/\x}, {\x*\x*\x*\x} );% node [above right] {\small $\ds \frac{1}{x^4}$};
\draw[thick, dotted, samples=20, domain=1/3:1.31607, smooth] plot ({-1/\x}, {\x*\x*\x*\x} ) node [above] {\small $\ds \frac{1}{x^4}$\ \ \ \ };


\draw (0.1, -1.3) node [right, fill=white] {\textbf{Power functions}\ \  $y=x^p$};
%\draw (1, -1.7) node [right, fill=white] {$y=x^p$, $p<0$};
\draw (0.3, -1.7) node [right, fill=white] {Negative Integer Powers};
\draw (1, -2.0) node [right, fill=white] {(Reciprocals)};
\end{tikzpicture}

\index{exponential functions}
\begin{tikzpicture} [scale=0.6]
\drawgridxxyy{-5}{4}{-1}{8}
\draw[very thick, samples=10, domain=-5:3, smooth] plot (\x, {exp(.693147*\x)}) node [above] {\ \ \ \ $y=2^x$};
\draw[very thick, dashed, samples=20, domain=-5:1.29203, smooth] plot (\x, {exp(1.60944*\x)}) node [above] {$y=5^x$};
\draw (-5, 5) node [right, fill=white] {\textbf{Exponential}};
\draw (-5, 4) node [right, fill=white] {\textbf{functions}};
\draw (-5, 3) node [right, fill=white] {$y=b^x$, $b>1$};
\end{tikzpicture}\
\begin{tikzpicture} [scale=0.6]
\drawgridxxyy{-4}{5}{-1}{8}
\begin{scope}[xscale=-1]
\draw[very thick, samples=10, domain=-5:3, smooth] plot (\x, {exp(.693147*\x)}) node [above] {\small $y=\left( \frac{1}{2}\right)^x$\ \ \ \ \ \ \ \ };
\draw[very thick, dashed, samples=20, domain=-5:1.29203, smooth] plot (\x, {exp(1.60944*\x)}) node [above] {\small $y=\left( \frac{1}{5}\right)^x$};
\end{scope}
\draw (0.5, 5) node [right, fill=white] {\textbf{Exponential}};
\draw (0.5, 4) node [right, fill=white] {\textbf{functions}};
\draw (0.3, 3) node [right, fill=white] {\small $y=b^x$, $0<b<1$};
\end{tikzpicture}

\bigskip
Translation transformation: \index{translation transformation}
\fbox{$ u \mapsto (u-a): \ \ \text{add $a$ to $u$-values} $}



\Opensolutionfile{ans}[ans16]
\subsection*{Exercises \nopunct} \hfill


\begin{ex}
	Sketch a graph of the following functions on a single coordinate plane:
	\begin{multicols}{2}
		\ssp
		\item 		$y=x$
		\item 		$y=x^2$
		\item 	$y=x^3$
		\item 	$y=x^4$
		\item	$y=x^9$
		\item 	$y=x^{1.5}$
		\esp
	\end{multicols}
	\begin{sol}
		Check with a graphing utility.
	\end{sol}
\end{ex}

\begin{ex}
Sketch a graph of the following functions on a single coordinate plane:
\begin{multicols}{2}
\ssp
\item 	$y=\sqrt{x}$
\item 	$y=\sqrt[3]{x}$
\item 	$y=x^\frac{1}{5}$
\item 	$y=x^{0.68}$
\esp
\end{multicols}
\begin{sol}
			Check with a graphing utility.
\end{sol}
\end{ex}

\begin{ex}
	Sketch a graph of the following functions on a single coordinate plane:
	\begin{multicols}{2}
		\ssp
		\item 		$y=2^x$
		\item 		$y=3^x$
		\item		$y=2^{-x}$\ \ 	[Hint: rewrite first.]
		\item 		$y={0.68}^x$
		\esp
	\end{multicols}
	\begin{sol}
			Check with a graphing utility.
	\end{sol}
\end{ex}


Sketch the graph of each relation by first sketching the parent relation, then performing the appropriate translation:

\begin{multicols}{3}
	
	\begin{ex}
		$y-3=x^2$
	\end{ex}
	
	\begin{ex}
		$y=(x-3)^2$
	\end{ex}

	
\begin{ex}
	$y= 2^{x-3}$
\end{ex}


	
	\begin{ex}
		$y+4=x^3$
	\end{ex}
	
	\begin{ex}
		$y= \dfrac{1}{x+5}$
	\end{ex}

	
\begin{ex}
	$y+6= \left( \frac{1}{2} \right)^x$
\end{ex}
	
\begin{ex}
	$y-3= \dfrac{1}{x^2}$
\end{ex}

	
\begin{ex}
	$y+9= -\frac{1}{3}(x-6)$
\end{ex}
	
\begin{ex}
	$y+5 = \sqrt{x-4}$
\end{ex}

	
\begin{ex}
	$y= 2^x-8$
\end{ex}

	
\begin{ex}
	$y= \sqrt[3]{x}+1$
\end{ex}

	
\begin{ex}
	$y= (x+1)^2-4$
\end{ex}
	
\end{multicols}

\begin{ex}
	An on-demand publisher charges \$22.50 to print a 600 page book and \$15.50 to print a 400 page book. Find a linear model for the cost of a book, $C$, as a function of the number of pages, $p$. Interpret the slope and $C$-intercept of the function.
	\begin{sol}
		$C = 0.035p+1.5$. The slope 0.035 means it costs 3.5 cents for each additional page. The $C$-intercept of \$1.50 means there is a fixed, or start-up, cost of \$1.50 to make each book.
	\end{sol}
\end{ex}

\begin{ex} \label{JimCar}
	Jim buys a new car for \$30,000. After one year it is only worth \$24,000. If the value of his car is modeled as a linear function of time, what will the value be two years after purchase? Is this a good model for the value of his car in the long-run?
	
	An alternative model for the value of of Jim's car as a function of time is $V = 30,000\cdot 0.80^t$. What does this model predict for the value of his car after one year? Two years? What happens in the long run according to this model?
	\begin{sol}
		The linear model predicts the value to be \$18,000 after two years. This model reduces the value by the same amount, \$6,000 each year. It predicts negative values after 5 years! This does not seem to meet our expectations of the car's depreciating value.
		
		The alternative model predicts the same value, \$24,000, after one year, but predicts a value of \$19,200 after the second year. Each year, the value of the car is reduced by 20\%. Thus over the second year it loses \$4,800 in value. In this model the value of the car continues to decrease, but it is always positive. This seems to fit our intuition about car values more closely. We will study this type of model in Section \ref{expApps}.
	\end{sol}
\end{ex}


\Closesolutionfile{ans}
\subsection*{Answers \nopunct} \hfill
\begin{multicols}{2}
	\input{ans16}
\end{multicols}



	
\newpage
\section{Similar figures, angles, triangles} \label{simfigsection}

Radian measure\index{radians}\index{radian measure}: how many radii are needed to measure the arc?

Conversion: $\pi \text{ radians} = 180 ^\circ$

For $\theta$ in radians:  $\theta = \dfrac{s}{r}$, so $s = r\theta$.

We assume angles are measured in radians if no unit is given.

\bigskip
\begin{tikzpicture}[scale = 1.0]
	\draw (0,0) -- (2,3) -- (2,1) -- (4,1) -- (0,0);
	\draw (0.2,0.1) [above right] node {\small $\alpha$};
	\draw (1.85,2.8) [below] node {\small $\beta$};
	\draw (2,1) [above right] node {\small $\gamma$};
	\draw (3,0.85) [left] node {\small $\theta$};
	\draw (0.8,1.8) node {$a$};
	\draw (2.3,2) node {$b$};
	\draw (3,1.25) node {$c$};
	\draw (2,0.2) node {$d$};
\end{tikzpicture}\hfill
\begin{tikzpicture}[scale = 1.0]
\begin{scope}[xscale=-1]
\draw (0,0) -- (2,3) -- (2,1) -- (4,1) -- (0,0);
\draw (0.2,0.1) [above left] node {\small $\alpha$};
\draw (1.85,2.8) [below] node {\small $\beta$};
\draw (2,1) [above left] node {\small $\gamma$};
\draw (3,0.85) [right] node {\small $\theta$};
\draw (0.8,1.8) node {$a$};
\draw (2.3,2) node {$b$};
\draw (3,1.25) node {$c$};
\draw (2,0.2) node {$d$};
\end{scope}
\draw (1.5,1.5) node {$\longleftarrow $ mirrored};
\end{tikzpicture}


\begin{tikzpicture}[scale = 1.7]
\begin{scope}[rotate=-35]
\draw (0,0) -- (2,3) -- (2,1) -- (4,1) -- (0,0);
	\draw (0.2,0.15) [right] node {\small $\alpha$};
\draw (1.85,2.85) [below] node {\small $\beta$};
	\draw (2,1) [right] node {\small $\gamma$};
\draw (3.1,0.95) [left] node {\small $\theta$};
\draw (0.8,1.8) node {$A$};
\draw (2.3,2) node {$B$};
\draw (3,1.25) node {$C$};
\draw (2,0.2) node {$D$};
\end{scope}
\draw (4.3,0.4) node {$\longleftarrow $ rotated and scaled };
\end{tikzpicture}

Similar figures\index{similar figures}: same shape but possibly translated, scaled (uniformly in all directions), or mirrored.\index{mirror image}\index{reflection}
\begin{itemize}
	\item 
	One way to think about the proportional sizes (constant scale factor/ratio):
	$$ \dfrac{a}{A} = \dfrac{b}{B}= \dfrac{c}{C}= \dfrac{d}{D}$$ or
	$$ \dfrac{A}{a} = \dfrac{B}{b}= \dfrac{C}{c}= \dfrac{D}{d}.$$  
	\item Another way (constant ratios of lengths within a figure):
	$$ \dfrac{a}{c} = \dfrac{A}{C},\ \ \dfrac{b}{d} = \dfrac{B}{D},\ \ \dfrac{c}{d} = \dfrac{C}{D},\ \ \text{etc.}$$
	\item Angles are the same! [Note we implicitly mean they have the same measure; they are not literally the same angles.]
\end{itemize}

\bigskip
Triangles: remember the interior angles always add to a straight angle ($180\dg$ or $\pi$ radians).

\Opensolutionfile{ans}[ans17]
\subsection*{Exercises \nopunct} \hfill

Convert each angle measure from degrees to radians or radians to degrees. Give exact values in terms of $\pi$ if necessary.

\begin{multicols}{3}
	
	\begin{ex}
	$15\dg$
	\begin{sol}
		$\dfrac{\pi}{12}$
	\end{sol}
	\end{ex}	
	
	\begin{ex}
		$240\dg$
		\begin{sol}
			$\dfrac{4\pi}{3}$
		\end{sol}
	\end{ex}

	\begin{ex}
	$\dfrac{\pi}{6}$
	\begin{sol}
		$30\dg$
	\end{sol}
\end{ex}

	\begin{ex}
	$135\dg$
	\begin{sol}
		$\dfrac{3\pi}{4}$
	\end{sol}
\end{ex}

	\begin{ex}
	$\dfrac{\pi}{3}$
	\begin{sol}
		$60\dg$
	\end{sol}
\end{ex}

	\begin{ex}
	$45\dg$
	\begin{sol}
		$\dfrac{\pi}{4}$
	\end{sol}
\end{ex}

	\begin{ex}
	$150\dg$
	\begin{sol}
		$\dfrac{5\pi}{6}$
	\end{sol}
\end{ex}
	\begin{ex}
	$90\dg$
	\begin{sol}
		$\dfrac{\pi}{2}$
	\end{sol}
\end{ex}

	\begin{ex}
	$0.5$
	\begin{sol}
		$\left(\dfrac{90}{\pi}\right)^\dg \approx 28.6\dg$
	\end{sol}
\end{ex}
\end{multicols}

\begin{comment}
\begin{ex}
A yo-yo which is 2.25 inched in diameter spins at a rate of 4500 revolutions per minute. How fast is the edge of the yo-yo spinning in miles per hour? Round your answer to two decimal places.
\begin{sol}
About 30.12 miles per hour.
\end{sol}
\end{ex}
\end{comment}



\begin{ex} \label{7200rpm}
	A computer hard drive contains a circular disk with diameter 2.5 inches and spins at a rate of 7200 RPM (revolutions per minute). Find the linear speed of a point on the edge of the disk in miles per hour. Round your answer to two decimal places.
	
	[Hint: how far does it travel per revolution?]
	\begin{sol}
		About 53.55 miles per hour.
	\end{sol}
\end{ex}

\begin{ex}
A car with 24 in diameter tires is traveling down the road at 60 miles per hour. How fast is the wheel rotating in degrees per second?
	\begin{sol}
		About $5042$ degrees per second
	\end{sol}
\end{ex}

\begin{ex}
	John is trying to approximate the height of a nearby tree. He is 6 feet tall and measures his shadow to be 10.8 feet long. At the same time, the tree's shadow measures 63 feet.
	\ssp
	\item 	How tall is the tree? Draw a diagram.
	\item  Your solution probably uses similar triangles. How do you know the triangles are similar?
	\esp
	\begin{sol}
		\ssp
		\item About 35 feet tall.
		\item  John might as well be standing so that the diagram looks like this:
		
		\begin{tikzpicture} [scale=0.5]
		\draw (0,0) -- (0,5) -- (9,0) -- (0,0);
		\draw (5.5,0) -- (5.5,1.94444);
		\draw [<->] (0,-0.5) -- (9, -0.5);
	  	 \draw (4.5, -0.5) node [below] {\small 63 ft};
		\draw [<->] (5.5,2.2) -- (9, 2.2);
		 \draw (7.25, 2.2) node [above] {\small 10.8 ft};
		\draw [<->] (-0.5,0) -- (-0.5, 5);
		 \draw (-0.5, 2.5) node [left] {\small $h$?};
		\draw [<->] (5,0) -- (5, 1.94444);
		 \draw (5, 1) node [left] {\small 6 ft};
		\draw (0,0) rectangle (0.3,0.3);
		\draw (5.5,0) rectangle (5.8,0.3);
		\draw (7.5, 0.4) node {$\theta$};
		\end{tikzpicture}
		
		It is then clear that the triangles both have a right angle and an angle measuring $\theta$. Thus their third angle also has the same measure and the large and small triangles are similar figures.
		
		Another explanation is the rays from the sun are parallel and therefore make the same angle with the horizontal ground when casting both shadows.
		
		\esp
	\end{sol}
\end{ex}

\begin{ex}
	You have an clinometer, a device which measures the angle of inclination of an object sighted at a distance as in the diagram (similar to a sextant). Devise a method for measuring the height of a building using the clinometer and a single length measurement along the ground. Do not use trigonometric functions, whatever those are\footnote{Look ahead a chapter to find out!}.
	
			\begin{tikzpicture} [scale=0.5]
	\draw (0,0) -- (0,5) -- (9,0) -- (0,0) -- (-3,0) -- (-3,5) -- (0,5);
	\draw (-2,2) rectangle (-1,1);
	\draw (-2,3) rectangle (-1,4);
%	\draw (5.5,0) -- (5.5,1.94444);
%	\draw [<->] (0,-0.5) -- (9, -0.5);
%	\draw (4.5, -0.5) node [below] {\small 63 ft};
%	\draw [<->] (5.5,2.2) -- (9, 2.2);
%	\draw (7.25, 2.2) node [above] {\small 10.8 ft};
	\draw [<->] (-3.5,0) -- (-3.5, 5);
	\draw (-3.5, 2.5) node [left] {\small $h$?};
%	\draw [<->] (5,0) -- (5, 1.94444);
%	\draw (5, 1) node [left] {\small 6 ft};
	\draw (0,0) rectangle (0.3,0.3);
%	\draw (5.5,0) rectangle (5.8,0.3);
	\draw (7.5, 0.4) node {$\theta$};
	\end{tikzpicture}
	
	\begin{sol}
		Move toward or away from the building until the clinometer reads $45\dg$. Then you have created a square as in this diagram:
		
			\begin{tikzpicture} [scale=0.5]
\draw (0,0) -- (-3,0) -- (-3,5) -- (0,5) -- (5,0);
\draw (-2,2) rectangle (-1,1);
\draw (-2,3) rectangle (-1,4);
\draw [thick] (0,0) rectangle (5,5);
%	\draw (5.5,0) -- (5.5,1.94444);
	\draw [<->] (0,-0.5) -- (5, -0.5);
	\draw (2.5, -0.5) node [below] {$\ell$};
%	\draw [<->] (5.5,2.2) -- (9, 2.2);
%	\draw (7.25, 2.2) node [above] {\small 10.8 ft};
\draw [<->] (-3.5,0) -- (-3.5, 5);
\draw [<->] (5.5,0) -- (5.5, 5);
\draw (-3.5, 2.5) node [left] {\small $h$};
\draw (5.5, 2.5) node [right] {\small $h$};
%	\draw [<->] (5,0) -- (5, 1.94444);
%	\draw (5, 1) node [left] {\small 6 ft};
\draw (0,0) rectangle (0.3,0.3);
\draw (4.7,4.7) rectangle (5,5);
%	\draw (5.5,0) rectangle (5.8,0.3);
\draw (4, .4) node {\tiny $45\dg$};
\draw (4.6, 1) node {\tiny $45\dg$};
\end{tikzpicture}

The length along the ground from where you are standing to the base of the building, $\ell$, is then equal to the building's height, $h$.
		\end{sol}
	
\end{ex}

\begin{ex}
	Who got the larger slice of pizza from \textit{The Don's}? (Luckily Jack and Jill have brought their ruler and protractor to lunch.)\label{pizzaTwo}
	
	\begin{tikzpicture} [scale = 0.5]
	\draw (9,0) arc [radius=9, start angle=0, end angle = 34] --(0,0) -- (9,0);
	\draw (1,0) node [above right] {\small $34\dg$};
	\draw [<->] (0,-0.5) -- (9,-0.5) ;
	\draw (4.5, -0.5) node [below] {9 in};
	\draw (0,3) node [right]{\small Jack's Slice};
	
	\draw (18,0) arc [radius=9, start angle=0, end angle = 46] --(10,0) -- (18,0);
	\draw (11,0.2) node [above right] {\small $46\dg$};
	\draw [<->] (10,-0.5) -- (18,-0.5) ;
	\draw (14, -0.5) node [below] {8 in};
	\draw (13,2.5) node [right]{\small Jill's Slice};
	\end{tikzpicture}
	\begin{sol}
		It's close but Jill's slice has a little more area at $\frac{368\pi}{45} \approx 25.7$ in$^2$ versus Jack's $\frac{153\pi}{20} \approx 24.0$ in$^2$.
	\end{sol}
\end{ex}

\begin{ex}
	Whose pizza slice in \#\ref{pizzaTwo} has a longer perimeter?
	\begin{sol}
		Jack's!
	\end{sol}
\end{ex}

\begin{ex}
	James is transferring cake batter from a  9 $\times$ 13 in rectangular pan to a 10 in diameter round pan. Write a function that calculates the height of the batter in the round pan as a function of the height of the batter in the rectangular pan. If the height of the batter in the rectangular pan is 2 in, will it fit in such a round pan with 3 in sides?
	
	\begin{sol}
		$h_{\text{round}} = \dfrac{117}{25\pi}h_{\text{rectangular}}$. Yes, barely. (It will overflow when baking if not when handling.)
		\end{sol}
\end{ex}

\begin{ex} \label{camryangle}
	For a right triangle with a small angle, the height $h$ of the side across from it is not that different from the length of the arc $s$ as pictured.
	
		\begin{tikzpicture} [scale = 1.05]
%	\draw (12,0) arc [radius=9, start angle=0, end angle = 8] -- (0,0) -- (12,0);
%	\draw (12,0) arc [radius=12, start angle=-5, end angle = 8];
	\draw [very thick] (12,0) arc [radius=12, start angle=0, end angle = 8];
	\draw (2.0,0.15) node [right] {\tiny small angle};
	\draw (2.0,1) node [right] {$h \approx s$};
	\draw (0,0) -- (12,0);
	\draw (11.8832,0) -- (11.8832, 1.67008) -- (0,0);
	\draw (11.5832,0) rectangle (11.8832,0.3);
	\draw (12, 0.8) node [left] {$h \longrightarrow\ $};
	\draw (12, 0.8) node [right] {$\ \longleftarrow s$};
%	\draw [<->] (0,-0.5) -- (9,-0.5) ;
%	\draw (4.5, -0.5) node [below] {9 in};
%	\draw (0,3) node [right]{\small Jack's Slice};
	
	\end{tikzpicture}
	
	Professor needs to get his 2000 Toyota Camry to an incline of at least $10\dg$ to get the air out of the coolant system. He has a pair of 6.5 in-tall ramps that he can drive the front wheels onto. The car's wheelbase is 105.2 in. By this estimate, will the ramps be enough?
		
	\begin{sol}
		No, the angle will only be about $3.5\dg$ by this estimate.
	\end{sol}
\end{ex}


\Closesolutionfile{ans}
\subsection*{Answers \nopunct} \hfill
\begin{multicols}{2}
	\input{ans17}
\end{multicols}


\newpage
\section{Pythagoras, distance, circles, midpoints, common right triangles}

\textit{The Pythagorean Theorem}\index{Pythagorean Theorem}: If a triangle has a right angle between $a$ and $b$, then its side lengths satisfy $a^2+b^2=c^2$.

Thus in a right triangle the longest side is across from the right angle. It is also called the \textit{hypotenuse}.\index{hypotenuse}

\textit{Converse of the theorem}: If a triangle's side lengths satisfy $a^2+b^2=c^2$, then the angle between $a$ and $b$ is right. Note this fact \textit{adds} something to the Pythagorean Theorem and is not identical to it.

The \textit{midpoint}\index{midpoint} between $(x_1,y_1)$ and $(x_2,y_2)$ is $\left(\dfrac{x_1+x_2}{2}, \dfrac{y_1+y_2}{2} \right)$. In other words, the midpoint has the average of the $x$-values as its $x$-value and, the average of the $y$-values as its $y$-value.

The distance from $(x_1,y_1)$ to $(x_2,y_2)$ is \fbox{$d=\sqrt{(x_2-x_1)^2+(y_2-y_1)^2}$}.\index{distance formula}

\index{circle equation}The circle centered at $(h,k)$ with radius $r$ has equation \fbox{$(x-h)^2+(y-k)^2 = r^2$}.

Common right triangles:\index{common right triangles}\index{45-45-90 right triangles}\index{30-60-90 right triangles}

\begin{tikzpicture}[scale=1.5]
\pgfmathsetmacro{\ex}{0}
\pgfmathsetmacro{\ey}{0}
\draw (\ex+1.7,\ey) -- (\ex+1.7,\ey+0.3) -- (\ex+2,\ey+0.3);
\draw (\ex,\ey) -- (\ex+2,\ey)  -- (\ex+2,\ey+2) -- (\ex,\ey);
\draw (\ex+0.2,\ey) node [above right] {\small $45^\circ$};
\draw (\ex+2.05,\ey+1.65) node [below left] {\small $45^\circ$};
\draw (\ex+2, \ey+1) node [right] {$1$};
\draw (\ex+1, \ey) node [below] {$1$};
\draw (\ex+1, \ey+1) node [above left] {$\sqrt{2}$};
\draw (\ex+0, \ex-0.5) node[below right] {``45-45-90''};

\pgfmathsetmacro{\ex}{4}
\pgfmathsetmacro{\ey}{0}
\draw (\ex+1.7,\ey) -- (\ex+1.7,\ey+0.3) -- (\ex+2,\ey+0.3);
\draw (\ex,\ey) -- (\ex+2,\ey)  -- (\ex+2,\ey+1.16) -- (\ex,\ey);
\draw (\ex+0.33,\ey-0.04) node [above right] {\small $30^\circ$};
\draw (\ex+2.05,\ey+1.00) node [below left] {\small $60^\circ$};
\draw (\ex+2, \ey+0.58) node [right] {$1$};
\draw (\ex+1, \ey) node [below] {$\sqrt{3}$};
\draw (\ex+1, \ey+0.58) node [above] {$2$};
\draw (\ex+0, \ey-0.5) node[below right] {``30-60-90''};
\end{tikzpicture}


\Opensolutionfile{ans}[ans18]
\subsection*{Exercises \nopunct} \hfill
\begin{ex}
	Find the unknown quantities $a$ and $b$ in the figure.
	
	If the area inside the smaller triangle is about $1.85$ cm$^2$, is the area of the larger triangle given by $1.85\cdot\dfrac{3.6}{2.0}$ cm$^2$? Explain.
	
	
	\begin{tikzpicture}[scale=1.3]
	\draw (0,0) coordinate (a) -- (4,2) coordinate (b) -- (2,2) coordinate (c) -- (0,0);
	\draw (3,2.4) node {$2.0$ cm};
	\draw (0.5,1.5) node {$2.8$ cm};
	\draw (2.8,0.8) node {$4.5$ cm};
	\draw pic[draw=black, -, angle eccentricity=1, angle radius=0.5cm]
	{angle=c--b--a};
	\draw pic[draw=black, -, angle eccentricity=1, angle radius=0.4cm]
{angle=c--b--a};
	\draw pic[draw=black, -, angle eccentricity=1, angle radius=0.25cm]
{angle=a--c--b};
	%pic["$\alpha$", draw=orange, <->, angle eccentricity=1.2, angle radius=1cm]	{angle=a--b--c};
	
\begin{scope}[shift={(6,-2)},rotate=55, xscale=1.6, yscale=1.6]
	\draw (0,0) coordinate (a) -- (4,2) coordinate (b) -- (2,2) coordinate (c) -- (0,0);
	\draw (3,2.4) node {$3.6$ cm};
	\draw (1.0,1.3) node {$a$};
	\draw (2.5,1.0) node {$b$};
\draw pic[draw=black, -, angle eccentricity=1, angle radius=0.5cm]
{angle=c--b--a};
\draw pic[draw=black, -, angle eccentricity=1, angle radius=0.4cm]
{angle=c--b--a};
\draw pic[draw=black, -, angle eccentricity=1, angle radius=0.25cm]
{angle=a--c--b};
\end{scope}
	
	
	\end{tikzpicture}
	\begin{sol}
		$a$ = 5.04 cm. $b$ = 8.1 cm. No, the area is not increased by the same factor. Each cm$^2$ of area in the original figure corresponds to $1.8\times 1.8$ cm$^2$ of area in the larger figure since lengths are increased by the same factor in all directions. Thus the new area is approximately $1.85\times1.8\times 1.8 \approx 5.99$ cm$^2$.
	\end{sol}
\end{ex}

\begin{ex}
	The summit of a mountain is 20,400 ft away from your current location in a straight line. You also measure the angle of inclination of the summit to be $30\dg$. How tall is the mountain? Include a diagram.
	\begin{sol}
		10,200 ft
	\end{sol}
\end{ex}

\begin{ex}
	From the ground you measure the angle of inclination to the top of a building to be $45\dg$. You know the height of the building at that point to be 50 m. How far are you from the top of the building?
	\begin{sol}
		$50\sqrt{2} \approx 70.7$ m
		\end{sol}
\end{ex}

\begin{ex}
	John and Sarah are attempting to gauge how high their kite is flying. John let out the entire length of the string, 500 ft. Sarah judges the kite is directly overhead when she is 350 ft away from John along the flat ground. Assuming the string forms a straight line, how high above the ground is the kite?
	\begin{sol}
		$\sqrt{127500} \approx 357$ ft.
	\end{sol}
\end{ex}

Find all labeled lengths or angles. Use the units of the problem.

\begin{multicols}{2}
	\begin{ex}
		
		
		\begin{tikzpicture} [scale=2.0]
		\draw (0,0) -- (1,0) -- (1, 1.73205) -- (0,0);
		\draw (0.1,0.15) node [above, right] {$60\dg$};
		\draw (0.9,1.35) node {$\theta$};
		\draw (0.85,0) rectangle (1, 0.15);
		\draw (0.4, 1.0) node [left] {\small $48$ mm};
		\draw (1.3, 0.8) node [left] {$a$};
		\draw (0.5, -0.1) node [below] {$b$};
		\end{tikzpicture}
		\begin{sol}
			$\theta$ = $30\dg$, $a = 24\sqrt{3} \approx 41.6$ mm, $b = 24$ mm. 
		\end{sol}
	\end{ex}

\begin{ex}
	
	
		\begin{tikzpicture} [scale=2.0]
	\begin{scope}[rotate=-100]
	\draw (0,0) -- (1,0) -- (1, 1.73205) -- (0,0);
	\draw (0.25,0.15) node {$\phi$};
	\draw (0.89,1.2) node {\small $\pi/6$};
	\draw (0.85,0) rectangle (1, 0.15);
	\draw (0.4, 1.0) node { $c$};
	\draw (1.2, 0.8) node {\small $106$ miles};
	\draw (0.5, -0.2) node {$d$};
		\end{scope}
	\end{tikzpicture}
	\begin{sol}
		$\phi$ = $\pi/3$, $d = \frac{106}{\sqrt{3}} \approx 61.2$ miles, $c = \frac{212}{\sqrt{3}} \approx 122.4$ miles.
	\end{sol}
\end{ex}


\begin{ex}
	
	
	\begin{tikzpicture} [scale=2.0]
	\begin{scope}[rotate=-80]
	\draw (0,0) -- (1,0) -- (1, 2.1) -- (0,0);
%	\draw (0.25,0.15) node {$\phi$};
%	\draw (0.89,1.2) node {\small $\pi/6$};
	\draw (0.85,0) rectangle (1, 0.15);
	\draw (0.3, 1.0) node { $k$};
	\draw (1.2, 0.8) node {\small $55$ in};
	\draw (0.5, -0.3) node {$25$ in};
	\end{scope}
	\end{tikzpicture}
	\begin{sol}
		$k = \sqrt{3650} \approx 60.4$ in
	\end{sol}
\end{ex}

\begin{ex}
	
	
	\begin{tikzpicture} [scale=2.0]
	\begin{scope}[rotate=45]
	\draw (0,0) -- (2,0) -- (2, 2) -- (0,0);
	\draw (0.35,0.15) node {$\alpha$};
	\draw (1.8,1.6) node {$\dfrac{\pi}{4}$};
	\draw (1.85,0) rectangle (2, 0.15);
	\draw (0.7, 1.2) node {\small $85$ cm};
%	\draw (1.2, 0.8) node {\small $85$ cm};
	\draw (1, -0.2) node {$\ell$};
	\draw (2.2, 1) node {$m$};
	\end{scope}
`	\end{tikzpicture}
	\begin{sol}
		$\alpha$ = $\pi/4$, $\ell = m = \frac{85}{\sqrt{2}}\approx 60.1$ cm.
	\end{sol}
\end{ex}
\end{multicols}

Find the distance between the points and their midpoint.

\begin{multicols}{2}
\begin{ex}
	$(4,9)$ and $(0,8)$
	\begin{sol}
		$\sqrt{17}$, $\left(2,\frac{17}{2} \right)$
	\end{sol}
\end{ex}

\begin{ex}
	$(-2,5)$ and $(-7,20)$
	\begin{sol}
		$\sqrt{250}$, $\left(-\frac{9}{2},\frac{25}{2} \right)$
	\end{sol}
\end{ex}
\end{multicols}



Find an equation for each circle then graph it.

\begin{multicols}{2}

	\begin{ex}
	Center $(4,-2)$, radius 3.
	\begin{sol}
		$(x-4)^2+(y+2)^2 = 9$.
	\end{sol}
\end{ex}
	\begin{ex}
	Center $(-10,-5)$, radius 2.
	\begin{sol}
		$(x+10)^2+(y+5)^2 = 4$. Check graph with a graphing utility.
	\end{sol}
\end{ex}
	\begin{ex}
	Center $(7,4)$, diameter 8.
	\begin{sol}
		$(x-7)^2+(y-4)^2 = 16$.
	\end{sol}
\end{ex}
	\begin{ex}
	Center $(3,5)$, passes through $(-1,-2)$.
	\begin{sol}
		$(x-3)^2+(y-5)^2 = 65$.
	\end{sol}
\end{ex}

	\begin{ex}
	Endpoints of a diameter: $\left(\frac{1}{2}, 4\right)$, $\left(\frac{3}{2}, -1\right)$.
	\begin{sol}
		$(x-1)^2+\left(y-\frac{3}{2}\right)^2 = \frac{13}{2}$.
	\end{sol}
\end{ex}
\end{multicols}

\begin{ex}
	Elizabeth and Dave start at the same point but then Elizabeth travels north at 50 mi/h while Dave travels southeast at 30 mi/h. After 2 hours, how far apart are they as the crow flies? If, instead, they double their speeds, will this distance be doubled?
	\begin{sol}
		$\ds 20\sqrt{15\sqrt{2}+34} \approx 148.6$ mi, yes (It will be a side length of a similar figure.)
	\end{sol}
\end{ex}


\Closesolutionfile{ans}
\subsection*{Answers \nopunct} \hfill
\begin{multicols}{2}
	\input{ans18}
\end{multicols}



\chapter{Second Cycle}

\section{Quadratic equations -- review plus something new?}
\index{zero product property}\index{factoring}\index{quadratic expressions}\index{quadratic equations}
The zero product property: \ \ $a\cdot b = 0$ if and only if $a=0$ or $b=0$ (or both).
\begin{itemize}
\item Can be extended to any number of factors.
%\item Used often when solving equations!
\item It's often the reason we factor when trying to solve equations.
\item The zero is critical. The property fails for other numbers, although this is the basis for innumerable student errors.
\end{itemize}

 Use the $AC$ method when attempting to factor a quadratic expression with integral coefficients over the integers or rational numbers.
	
	\qi{Never ``slide and divide.''}

Solving a quadratic by completing the square (CTS)\index{completing the square}\index{CTS|see{completing the square}}:
\begin{align*}
	2x^2-12x+8&=0&\\
	2x^2-12x&=-8 &\leftarrow \text{ Move constant to other side.}\\
	x^2-6x&=-4 &\leftarrow \text{ Divide by $a$ if necessary.}\\
	x^2-6x+9&=-4+9 &\leftarrow \text{ Add ``magic number'' to both sides.}\\
	(x-3)^2&=5 &\leftarrow \text{ Factor on left; simplify on right.}\\
	 & &\text{Now you know if there are solution(s).}\\
	x-3&=\pm\sqrt{5} &\leftarrow \text{ Extract roots if possible.}\\
	x&=3\pm\sqrt{5} &\leftarrow \text{ Add or subtract to isolate $x$.}\\
\end{align*}
In the example, the ``magic number''\index{magic number} is $\left(\dfrac{-6}{2}\right)^2$. In the next step, the square is now complete and factors into $\left( x+ \left(\dfrac{-6}{2}\right)\right)^2$.

Remember the possibilities when solving a quadratic equation are:
\begin{itemize}
	\item no real solutions
	\item a unique solution
	\item two solutions
\end{itemize}

\Opensolutionfile{ans}[ans21]
\subsection*{Exercises -- Professor apologizes for the length of this assignment!} \hfill

\begin{comment}
Factor completely using integers. Check your answer by multiplication.

\begin{multicols}{2}


\begin{ex}
	$2x - 10x^2$
	\begin{sol}
	$2x(1 - 5x)$ 
	\end{sol}
\end{ex}
\begin{ex}
	$12t^5 - 8t^3$
	\begin{sol}
		$4t^3(3t^2-2)$
	\end{sol}
\end{ex}
\begin{ex}
$16xy^2 - 12x^2y$
	\begin{sol}
\item $4xy(4y-3x)$
	\end{sol}
\end{ex}
\begin{ex}
	$5(m+3)^2- 4(m+3)^3$
	\begin{sol}
		$-(m+3)^2(4m+7)$
	\end{sol}
\end{ex}
\begin{ex}
	$(2x-1)(x+3) - 4(2x-1)$
	\begin{sol}
	$(2x-1)(x-1)$	
	\end{sol}
\end{ex}
\begin{ex}
	$t^2(t-5) + t - 5$
	\begin{sol}
	$(t-5)(t^2+1)$	
	\end{sol}
\end{ex}
\begin{ex}
	$w^2 - 121$
	\begin{sol}
		$(w-11)(w+11)$
	\end{sol}
\end{ex}
\begin{ex}
	$49 - 4t^2$
	\begin{sol}
		$(7-2t)(7+2t)$
	\end{sol}
\end{ex}
\begin{ex}
	$81t^4 - 16$
	\begin{sol}
	$(3t-2)(3t+2)(9t^2+4)$	
	\end{sol}
\end{ex}
\begin{ex}
	$9z^2 - 64y^4$
	\begin{sol}
		$(3z-8y^2)(3z+8y^2)$
	\end{sol}
\end{ex}
\begin{ex}
	$(y+3)^2 - 4y^2$
	\begin{sol}
	$-3(y - 3)(y+1)$	
	\end{sol}
\end{ex}
\begin{ex}
	$(x+h)^3 - (x+h)$
	\begin{sol}
	$(x+h)(x+h-1)(x+h+1)$	
	\end{sol}
\end{ex}
\begin{ex}
	$y^2 - 24y + 144$
	\begin{sol}
		$(y-12)^2$
	\end{sol}
\end{ex}
\begin{ex}
	$25t^2 + 10t + 1$
	\begin{sol}
	$(5t+1)^2$	
	\end{sol}
\end{ex}
\begin{ex}
	$12x^3 - 36x^2 + 27x$
	\begin{sol}
	$3x(2x-3)^2$	
	\end{sol}
\end{ex}
\begin{ex}
	$m^4 + 10m^2 + 25$
	\begin{sol}
		$(m^2+5)^2$
	\end{sol}
\end{ex}
\begin{ex}
	$27 - 8x^3$
	\begin{sol}
	$(3-2x)(9 + 6x + 4x^2)$	
	\end{sol}
\end{ex}
\begin{ex}
	$t^6 +t^3$
	\begin{sol}
	$t^3(t+1)(t^2 - t + 1)$	
	\end{sol}
\end{ex}
\begin{ex}
	$x^2 - 5x - 14$
	\begin{sol}
		$(2x-5)(3x-4)$
	\end{sol}
\end{ex}
\begin{ex}
	$y^2 - 12y + 27$
	\begin{sol}
		$(y-9)(y-3)$
	\end{sol}
\end{ex}
\begin{ex}
	$3t^2 + 16t + 5$
	\begin{sol}
		$(3t+1)(t+5)$
	\end{sol}
\end{ex}
\begin{ex}
	$6x^2 - 23x + 20$
	\begin{sol}
		$(2x-5)(3x-4)$
	\end{sol}
\end{ex}
\begin{ex}
	$35+2m - m^2$
	\begin{sol}
	$(7-m)(5+m)$	
	\end{sol}
\end{ex}
\begin{ex}
	$7w - 2w^2 - 3$
	\begin{sol}
		$(-2w+1)(w-3)$
	\end{sol}
\end{ex}
\end{multicols}
\end{comment}

Find all real number solutions. Some problems may be solved by factoring; others may resist. Do \underline{not} use the ``quadratic formula.''
\begin{multicols}{2}

\begin{ex}
	$(7x+3)(x-5) = 0$
	\begin{sol}
		$x = -\dfrac{3}{7}$ or $x = 5$ 
	\end{sol}
\end{ex}
\begin{ex}
	$(2t-1)^2 (t+4) = 0$
	\begin{sol}
		$t = \dfrac{1}{2}$ or $t = -4$
	\end{sol}
\end{ex}
\begin{ex}
	$(y^2 + 4)(3y^2 +y - 10) = 0$
	\begin{sol}
		$y = \dfrac{5}{3}$ or $y = -2$
	\end{sol}
\end{ex}
\begin{ex}
	$4t = t^2$
	\begin{sol}
		$t = 0$ or $t = 4$\
	\end{sol}
\end{ex}
\begin{ex}
	 $y+3 = 2y^2$
	\begin{sol}
		$y = -1$ or $y = \dfrac{3}{2}$
	\end{sol}
\end{ex}
\begin{ex}
	$26x = 8x^2 + 21$
	\begin{sol}
		 $x = \dfrac{3}{2}$ or $x = \dfrac{7}{4}$ 
	\end{sol}
\end{ex}

\begin{ex}
	 $2w^2 + 5w + 2 = - 3(2w+1)$
	\begin{sol}
	$w=-5$ or $w = -\dfrac{1}{2}$	
	\end{sol}
\end{ex}
\begin{ex}
	$x^2(x-3) = 16(x-3)$
	\begin{sol}
		$x=3$ or $x = \pm 4$
	\end{sol}
\end{ex}
\begin{ex}
	 $3\left(x - \dfrac{1}{2}\right)^2 = \dfrac{5}{12}$
	\begin{sol}
		$x = \dfrac{3 \pm \sqrt{5}}{6}$
	\end{sol}
\end{ex}
\begin{ex}
	$(2t+1)^3 = (2t+1)$
	\begin{sol}
		$t = -1$, $t= -\dfrac{1}{2}$
	\end{sol}
\end{ex}
\begin{ex}
	$4 - (5t+3)^2 = 3$
	\begin{sol}
		$t = -\dfrac{4}{5}, -\dfrac{2}{5}$
	\end{sol}
\end{ex}
\begin{ex}
	$a^4 + 4 = 6 - a^2$
	\begin{sol}
		$a = \pm 1$
	\end{sol}
\end{ex}
\begin{ex}
	$3(y^2-3)^2-2 = 10$
	\begin{sol}
		$y = \pm 1$, $\pm \sqrt{5}$
	\end{sol}
\end{ex}

\begin{ex}
	$16x^4 = 9x^2$
	\begin{sol}
		$x = 0$ or $x = \pm \dfrac{3}{4}$ 
	\end{sol}
\end{ex}

\begin{ex}
	$w(6w+11) = 10$
	\begin{sol}
		$w = -\dfrac{5}{2}$ or $w = \dfrac{2}{3}$
	\end{sol}
\end{ex}
\begin{ex}
	$\dfrac{8t^2}{3} = 2t+3$
	\begin{sol}
		$t = -\dfrac{3}{4}$ or $t = \dfrac{3}{2}$
	\end{sol}
\end{ex}
\begin{ex}
	$x^2 + x - 1 = 0$
	\begin{sol}
	$x = \dfrac{-1 \pm \sqrt{5}}{2}$	
	\end{sol}
\end{ex}
\begin{ex}
	$3w^2 = 2-w$
	\begin{sol}
		$w = -1, \dfrac{2}{3}$
	\end{sol}
\end{ex}
\begin{ex}
	$y(y+4) = 1$
	\begin{sol}
		$y = -2 \pm \sqrt{5}$
	\end{sol}
\end{ex}
\begin{ex}
	$\dfrac{z}{2} = 4z^2-1$
	\begin{sol}
		$z = \dfrac{1 \pm \sqrt{65}}{16}$
	\end{sol}
\end{ex}
\begin{ex}
	$0.1v^2 + 0.2v = 0.3$
	\begin{sol}
		$v = -3, 1$
	\end{sol}
\end{ex}
\begin{ex}
	$x^2 = x - 1$
	\begin{sol}
		No real solution.
	\end{sol}
\end{ex}
\begin{ex}
	$3-t = 2(t+1)^2$
	\begin{sol}
		$t = \dfrac{-5 \pm \sqrt{33}}{4}$
	\end{sol}
\end{ex}
\begin{ex}
	$(x-3)^2 = x^2+9$
	\begin{sol}
		$x = 0$ 
	\end{sol}
\end{ex}
\begin{ex}
	$(3y-1)(2y+1) = 5y$
	\begin{sol}
	$y = \dfrac{2 \pm \sqrt{10}}{6}$	
	\end{sol}
\end{ex}
\begin{ex}
	$w^4 + 3w^2 - 1 = 0$
	\begin{sol}
		 $w = \pm \sqrt{\dfrac{\sqrt{13} - 3}{2}}$
	\end{sol}
\end{ex}
\begin{ex}
	$2x^4 +x^2 = 3$
	\begin{sol}
		$x = \pm 1$
	\end{sol}
\end{ex}
\begin{ex}
	$(2-y)^4 = 3(2-y)^2 + 1$
	\begin{sol}
		$y = \dfrac{4 \pm \sqrt{6 + 2 \sqrt{13}}}{2}$
	\end{sol}
\end{ex}
\begin{ex}
	$3x^4 + 6x^2 = 15x^3$
	\begin{sol}
		$x = 0, \dfrac{5 \pm \sqrt{17}}{2}$
	\end{sol}
\end{ex}
\begin{ex}
	$6p + 2 = p^2 + 3p^3$
	\begin{sol}
		$p = -\dfrac{1}{3}, \pm \sqrt{2}$ 
	\end{sol}
\end{ex}
\begin{ex}
	$10v = 7v^3 - v^5$
	\begin{sol}
		$v = 0, \pm \sqrt{2}, \pm \sqrt{5}$
	\end{sol}
\end{ex}
\begin{ex}
	$y^2 - \sqrt{8} y = \sqrt{18} y - 1$
	\begin{sol}
		$y = \dfrac{5\sqrt{2} \pm \sqrt{46}}{2}$
	\end{sol}
\end{ex}
\begin{ex}
	$x^2 \sqrt{3} = x \sqrt{6} + \sqrt{12}$
	\begin{sol}
		$x = \dfrac{\sqrt{2} \pm \sqrt{10}}{2}$
	\end{sol}
\end{ex}
\begin{ex}
	$\dfrac{v^2}{3} = \dfrac{v \sqrt{3}}{2} + 1$
	\begin{sol}
		$v = -\dfrac{\sqrt{3}}{2}, 2\sqrt{3}$
	\end{sol}
\end{ex}
\begin{ex}
	$|x^2 - 3x| = 2$
	\begin{sol}
		$x = 1, 2, \dfrac{3 \pm \sqrt{17}}{2}$
	\end{sol}
\end{ex}
\begin{ex}
	$|x^2 -x + 3| = |4-x^2|$
	\begin{sol}
		$x = -\dfrac{1}{2}, 1, 7$
	\end{sol}
\end{ex}
\end{multicols}

\begin{ex}
	A circle has center $(-1,2)$ and radius 2. Find an equation for the circle, then use it to find its $x$- and $y$-intercept(s).
	\begin{sol}
		$(x+1)^2 + (y-2)^2 = 4$. $x$-intercept: $(-1,0)$, $y$-intercepts: $\left(0, 2-\sqrt{3}\right)$, $\left(0, 2+\sqrt{3}\right)$.
	\end{sol}
\end{ex}
\begin{ex}
	Find the  $x$- and $y$-intercepts of the following relation, then graph it on a graphing utility to check your answer. Visually compare it with the graph of $y=\frac{1}{x}$.
	
	 $$y^2-(x-1)^2 = 4 $$
	\begin{sol}
		No $x$-intercepts. $y$-intercepts: $\left(0, -\sqrt{5}\right)$, $\left(0, \sqrt{5}\right)$. The graphs appear related by rotation and translation.
	\end{sol}
\end{ex}

\begin{ex}
	Find all points on the line $y=2x+1$ which are 4 units from the point $(-1,3)$.
	\begin{sol}
		$(-1,-1)$ and $\left( \frac{11}{5}, \frac{27}{5} \right)$
	\end{sol}
\end{ex}


\Closesolutionfile{ans}
\subsection*{Answers \nopunct} \hfill
\begin{multicols}{2}
	\input{ans21}
\end{multicols}

\newpage
\section{Quadratic function graphs, scaling/stretching, optimization} \label{quadraticgraphs}
\index{quadratic functions}\index{quadratic function graphs}\index{completing the square} \index{mirror transformation}\index{circles}\index{optimization}
\textit{Expanded form}\index{expanded form of a quadratic} of a quadratic function: $$y = ax^2+bx+c$$

\textit{Vertex form}\index{vertex form of a quadratic} of a quadratic function with vertex $(h,k)$ and \index{fat-factor}``fat-factor'' $a$: $$y = a(x-h)^2+k$$

We can see by completing the square that the $x$ value of the vertex is $h=-\dfrac{b}{2a}$. This can quickly locate the vertex when starting with the expanded form.

\bigskip
Recall the translation transformation: \index{translation transformation}


\fbox{$ u \mapsto (u-a): \ \ \text{add $a$ to $u$-values} $}

\bigskip
Scaling/Stretch\index{stretch transformation}\index{scaling transformation} transformation:

\fbox{$ u \mapsto \frac{u}{a}: \ \ \text{multiply $u$-values by $a$} $}


Special case $u \mapsto -u$ (or $\frac{u}{-1}$): mirror over the \textit{other} axis; e.g. changing all $y$-values to their opposites mirrors a graph over the $x$-axis as the signed distance of a point to the $x$-axis is its $y$-value.

If $y$ is a function of $x$ then:
\begin{itemize}
	\item The greatest $y$-value (highest) is called the \textit{maximum value}\index{maximum value} of the function.
	\item The least $y$-value (lowest) is called the \textit{minimum value}\index{minimum value} of the function.
	\item If the $y$ value is always greater as you move to the right within an interval, the function is said to be \textit{increasing} \index{increasing} over the interval.
	 If the $y$-value is always lesser as you move to the right within an interval of $x$-values, the function is said to be \textit{decreasing} \index{decreasing} over the interval.
\end{itemize}

If $y$ is a quadratic function of $x$ then it has either a maximum or minimum value (extreme value\index{extreme value}) at the vertex.


\Opensolutionfile{ans}[ans22]
\subsection*{Exercises \nopunct} \hfill

Graph the quadratic function.  Find the $x$- and $y$-intercepts of each graph, if any exist.  If it is given in expanded form, convert it into vertex form; if it is given in vertex form, convert it into expanded form.  List the intervals on which the function is increasing or decreasing.  Identify the vertex and the axis of symmetry. State the maximum or minimum value of the function.

\begin{multicols}{3}

\begin{ex}
	$y=x^2+2$ 
	\begin{sol}
	Check graphs on a graphing utility \textit{after} working out all parts of the given exercise!
	
	 $y=x^2+2$ is both forms! No $x$-intercepts, $y$-intercept $(0,2)$, decreasing on $(-\infty 0]$, increasing on $[0,\infty)$, axis of symmetry $x=0$, minimum value 2.  
	\end{sol}
\end{ex}
\begin{ex}
	$y=-(x-2)^2$ 
	\begin{sol}
		$y=-x^2-4x-4$. $x$-intercept $(-2,0)$, $y$-intercept $(0,-4)$, increasing on $(-\infty -2]$, decreasing on $[-2,\infty)$, axis of symmetry $x=-2$, maximum value 0.  
	\end{sol}
\end{ex}
\begin{ex}
	$y=x^2+2x-8$ 
	\begin{sol}
		$y=(x+1)^2-9$. $x$-intercepts $(2,0)$ and $(-4,0)$, $y$-intercept $(0,-8)$, decreasing on $(-\infty -1]$, increasing on $[-1,\infty)$, axis of symmetry $x=-1$, minimum value $-9$.  
	\end{sol}
\end{ex}
\begin{ex}
	$y=-2(x+1)^2+4$ 
	\begin{sol}
		$y=-2x^2-4x+2$. $x$-intercepts $(-1-\sqrt{2},0)$ and $(-1+\sqrt{2},0)$, $y$-intercept $(0,2)$, increasing on $(-\infty -1]$, decreasing on $[-1,\infty)$, axis of symmetry $x=-1$, maximum value 4.  
	\end{sol}
\end{ex}
\begin{ex}
	$y=2x^2-4x-1$ 
	\begin{sol}
		$y=2(x-1)^2-3$. $x$-intercepts $\left( \frac{2-\sqrt{6}}{2}, 0 \right)$ and $\left( \frac{2+\sqrt{6}}{2}, 0 \right)$, $y$-intercept $(0,-1)$, deccreasing on $(-\infty, 1]$, increasing on $[1,\infty)$, axis of symmetry $x=1$, minimum value $-3$.  
	\end{sol}
\end{ex}
\begin{ex}
	$y=-3x^2+4x-7$ 
	\begin{sol}
		$y=-3\left( x-\frac{2}{3} \right)^2-\frac{17}{3}$. No $x$-intercepts, $y$-intercept $(0,-7)$, increasing on $\left(-\infty ,\frac{2}{3}\right]$, decreasing on $\left[\frac{2}{3},\infty\right)$, axis of symmetry $x=\frac{2}{3}$, maximum value $-\frac{17}{3}$.  
	\end{sol}
\end{ex}
\begin{ex}
	$y=x^2+x+1$ 
	\begin{sol}
		$y=\left( x+\frac{1}{2} \right)^2+\frac{3}{4}$. No $x$-intercepts, $y$-intercept $(0,1)$, decreasing on $\left(-\infty ,-\frac{1}{2}\right]$, increasing on $\left[-\frac{1}{2},\infty\right)$, axis of symmetry $x=-\frac{1}{2}$, minimum value $\frac{3}{4}$.  
	\end{sol}
\end{ex}

\end{multicols}

Convert the form of each equation to find the center and radius of the circle.

\begin{multicols}{2}
	\begin{ex}
		$x^2-4x+y^2+10y=-25$
		\begin{sol}
			$(x-2)^2+(y+5)^2=4$, center $(2,-5)$, radius $r=2$
		\end{sol}
	\end{ex}
	\begin{ex}
	$-2x^2-36x-2y^2-112=0$
	\begin{sol}
		$(x+9)^2+y^2=25$, center $(-9,0)$, radius $r=5$
	\end{sol}
\end{ex}
	\begin{ex}
	$x^2+y^2+8x-10y-1=0$
	\begin{sol}
		$(x+4)^2+(y-5)^2 = 42$, center $(-4,5)$, radius $r=\sqrt{42}$
	\end{sol}
\end{ex}

\end{multicols}

\begin{ex}
	Find all $x$- and $y$-intercepts of the graph of $ x^2+9y^2-4x-36y+31=0$. Could the graph be a circle? Explain.
	\begin{sol}
		no $x$-intercepts, $y$-intercepts $\left(  0, \frac{6+\sqrt{5}}{3} \right)$ and $\left(  0, \frac{6-\sqrt{5}}{3} \right)$. The graph is not a circle. The equation cannot be made into the appropriate form. The shape is called an ellipse and one can be made from a circle stretched differently in the $x$ and $y$ directions.
	\end{sol}
\end{ex}

\begin{ex}
	Jason participates in the Highland Games. In one event, the hammer throw, the height $h$ in feet of the hammer above the ground $t$ seconds after Jason lets it go is modeled by $h = -16t^2 +  22.08t + 6$.  What is the hammer's maximum height?  What is the hammer's total time in the air? Round your answers to two decimal places.
	\begin{sol}
		About $13.62$ feet, $1.61$ s
	\end{sol}
\end{ex}
\begin{ex}
	 Skippy wishes to plant a rectangular vegetable garden along one side of his house.  In his garage, he found 60 feet of fencing.  Since one side of the garden will border the house, Skippy doesn't need fencing along that side.   What is the maximum area of the garden? What dimensions give it?
	 
	 \begin{center}
	 	\begin{tikzpicture}[scale=0.4]
	 	\draw [ultra thick] (4,14) -- (24,14);
	 	\draw [pattern= dots] (10,8) rectangle (18,14);
	 	\draw (14,14) node[above] {Skippy's House};
	 	\draw (14,10.5) node[above, fill=white] {Garden};
	 	\draw [ultra thick] (6,14) -- (22,14);
	 	\end{tikzpicture}
	 \end{center}
	\begin{sol}
		The maximum area is 450 ft$^2$. To make such a garden, Skippy should make the fence parallel to the house 30 ft long and the other two, 15 ft each.
	\end{sol}
\end{ex}
\begin{ex}
	The two towers of a suspension bridge are 400 feet apart.  The parabolic cable\footnote{The weight of the bridge deck forces the bridge cable into a parabola. A free hanging cable such as a power line forms a catenary instead of a parabola.} attached to the tops of the towers is 10 feet above the point on the bridge deck that is midway between the towers.  If the towers are 100 feet tall, find the height of the cable directly above a point of the bridge deck that is 50 feet to the right of the left-hand tower.
	\begin{sol}
		$60.625$ ft
	\end{sol}
\end{ex}
\begin{ex} %UW 7.9
	Sylvia has an apple orchard. One season her 100 trees yielded 140 apples per tree. She wants to increase her production by adding more trees to the orchard. However, she knows that for every 10 additional trees she plants, she will lose 4 apples per tree (i.e. the yield per tree will decrease by 4 apples). How many trees should she have in the orchard to maximize her production of apples?
	\begin{sol}
		225 trees
	\end{sol}
\end{ex}
\begin{ex} %UW 7.10
	Rosalie is organizing a circus performance to raise money for a charity. She is trying to decide how much to charge for tickets. From past experience, she knows that the number of people who will attend is a linear function of the price per ticket. If she charges 5 dollars, 1200 people will attend. If she charges 7 dollars, 970 people will attend. How much should she charge per ticket to make the most money?
	\begin{sol}
		\$$7.72$
	\end{sol}
\end{ex}
\begin{ex}
	Find the point on the line $y=4-2x$ that is nearest to the origin. Do \underline{not} use the fact that the shortest distance is along a line perpendicular to the original line (except to check your answer if you like).
	
	Hint: distance $d$ will be minimized if and only if its square, $d^2$ is minimized. This is a classic trick for minimizing distance.
	\begin{sol}
		$\left(\frac{8}{5}, \frac{4}{5}\right)$
	\end{sol}
\end{ex}
\begin{ex} %UW 7.11
	A Norman window is a rectangle with a semicircle on top. Suppose that the perimeter of a particular Norman window is to be 24 feet. What should its dimensions be in order to maximize the area of the window and, therefore, allow in as much light as possible?
	\begin{sol}
		The radius of the circular part should be $\dfrac{24}{4+\pi} \approx 3.36$ ft. The short side of the rectangular part should also be equal to this.
	\end{sol}
\end{ex}



\Closesolutionfile{ans}
\subsection*{Answers \nopunct} \hfill
\begin{multicols}{2}
	\input{ans22}
\end{multicols}



\newpage
\section{Functions revisited, graphical meaning of certain statements}


\index{function} Abstract perspective of $y$ as a function of $x$:
\begin{itemize}
	\item A \textit{relation} between $y$ and $x$ is any set of points in the $xy$-plane. The pairs of $x$ and $y$ values are said to be related.\index{relation}
	\item A relation represents $y$ as a \textit{function} of $x$ if every $x$-value is related to only one $y$-value.
	\item i.e. every possible ``input'' corresponds to exactly one ``output.'' 
	\item The set of usable inputs ($x$-values) is called the \textit{domain}\index{domain} of the function.
	\item The set of outputs ($y$-values) is called the \textit{range}\index{range} of the function.
	\item function notation\index{function notation} $f(2) = 5$, read ``$f$ of 2 equals 5,'' means the function $f$ identifies the output 5 with the input 2.
	\qi{Importantly the parentheses in $f(2)=5$ do not have the same meaning as normal parentheses.}
	\item The following have the same meaning: ``$f(1.5)$ exists,'' ``$1.5$ is in the domain of $f$,'' ``$f(1.5)$ is defined.''
	\item The following have the same meaning: ``$f(1)$ does not exist,'' ``$1$ is not in the domain of $f$,'' ``$f(1)$ is undefined.''
\end{itemize}
Consider a relation between $x$ and $y$. If every possible vertical line would intersect the graph at most once (i.e. once or not at all), then the relation is said to pass the \textit{Vertical Line Test}\index{Vertical Line Test}. If there is a line that would intersect the graph at least twice, it is said to fail.

A relation between $x$ and $y$ that passes the Vertical Line Test represents $y$ as a function of $x$. The test is a simple visual check to ensure every $x$ value is related to only one $y$-value.

Useful statements:
\begin{itemize}
	\item $f(x)=0$\ \ \ \ $\longleftarrow$ Solutions are the \textit{zeros} (sometimes called \textit{roots}) of the function. These are the $x$-values of the $x$-intercepts of the graph of $y=f(x)$.
	\item $f(x)>0$\ \ \ \ $\longleftarrow$ Solution set is the set of $x$ values where the graph of $y=f(x)$ is above the $x$-axis (i.e. the set of inputs where outputs are positive).
	\item $f(x)<0$\ \ \ \ $\longleftarrow$ Solution set is the set of $x$ values where the graph of $y=f(x)$ is below the $x$-axis (i.e. the set of inputs where outputs are negative).
	\item $f(x)=g(x)$\ \ \ \ $\longleftarrow$ Solutions are the $x$-values of the intersection points of the graphs of $y=f(x)$ and $y=g(x)$. This is ``equating $y$-values'' in algebra class.
	\item $f(x)>g(x)$\ \ \ \ $\longleftarrow$ Solution set is the set of $x$ values where the graph of $y=f(x)$ is above the graph of $y=g(x)$ (i.e. the set of inputs where $f$'s output is greater than $g$'s.)
\end{itemize}

\Opensolutionfile{ans}[ans23]
\subsection*{Exercises \nopunct} \hfill

	Let $f(x) = x^2-4$ and $g(x) = x+2$.
	
	\begin{multicols}{2}
	\begin{ex}
		What is $f(1)$?
		\begin{sol}
			$-3$
		\end{sol}
	\end{ex}
	\begin{ex}
	What is $g(0)$?
	\begin{sol}
		$2$ -- this is the way to find the $y$-intercept of the graph of a function by the way. 
	\end{sol}
	\end{ex}
	\begin{ex}
	What is $f(b)$?
	\begin{sol}
		$b^2-4$
	\end{sol}
\end{ex}
	\begin{ex}
	Solve $f(x) = 0$.
	\begin{sol}
		$x=-2$ or $x=2$
	\end{sol}
\end{ex}
	\begin{ex}
	Solve $g(x) = 0$.
	\begin{sol}
		$x=-2$
	\end{sol}
\end{ex}
	\begin{ex}
	Solve $f(x) = g(x)$.
	\begin{sol}
		$x=-2$ or $x=3$
	\end{sol}
\end{ex}
	\begin{ex}
	Graph $y=f(x)$ and $y=g(x)$ on the same plane. Investigate the graphical meaning of each previous exercise.
\end{ex}
	\begin{ex}
	Solve $f(x) \leq g(x)$.
	\begin{sol}
		$[-2,3]$
	\end{sol}
\end{ex}
	\end{multicols}

\begin{tikzpicture} [scale=0.6]
\drawgridxxyy{-10}{11}{-6}{7}
\draw[color=darkgreen, thick, samples=2, domain=-10:11] plot (\x, {\x/8}) node [above, fill=white] {\small $y=h(x)$};

\draw[color=blue, thick, samples=15, domain=-9.4018:11, smooth] plot (\x, {-(\x+6)*(\x-10)/11}) node [below left, fill=white] {\small $y=g(x)$};

\draw[color=red, thick, samples=25, domain=-9.413:8, smooth] plot (\x, {\x*(\x+8)*(\x+3)*(\x-7)/200}) node [right, fill=white] {\small $y=f(x)$};
\end{tikzpicture}

Use the diagram above to graphically solve each problem. Estimate numerical quantities to the nearest tenth.

\begin{multicols}{2}
\begin{ex}
	What is $f(-6)$?
	\begin{sol}
		$-2.3$
	\end{sol}
\end{ex}
\begin{ex}
	What is $g(5)$?
	\begin{sol}
		$5.0$
	\end{sol}
\end{ex}
\begin{ex}
	Solve $f(x)=0$.
	\begin{sol}
		$x=-8.0$ or $x=-3.0$ or $x=0.0$ or $x=7.0$
	\end{sol}
\end{ex}
\begin{ex}
	What are the zeros of $g$?
	\begin{sol}
		$-6.0$, $10.0$
	\end{sol}
\end{ex}
\begin{ex}
	Solve $h(x)=1$.
	\begin{sol}
		$x=8.0$
	\end{sol}
\end{ex}
\begin{ex}
	Solve $f(x)=5$.
	\begin{sol}
		$x=-9.1$ or $x=7.7$
	\end{sol}
\end{ex}
\begin{ex}
	Solve $f(x) > 0$.
	\begin{sol}
		$(-\infty,-8.0)\cup(-3.0,0.0)\cup(7.0,\infty)$
	\end{sol}
\end{ex}
\begin{ex}
	Solve $g(x) \geq 0$.
	\begin{sol}
		$[-6.0,10.0]$
	\end{sol}
\end{ex}
\begin{ex}
	Solve $h(x) < 0$.
	\begin{sol}
		$(-\infty, 0.0)$
	\end{sol}
\end{ex}
\begin{ex}
	Solve $f(x) = g(x)$.
	\begin{sol}
		$x=-7.1$ or $x=7.5$
	\end{sol}
\end{ex}
\begin{ex}
	Solve $h(x) = f(x)$.
	\begin{sol}
		$x=-7.7$ or $x=-3.5$ or $x=0.0$ or $x=7.1$
	\end{sol}
\end{ex}
\begin{ex}
	Solve $g(x) = h(x)$.
	\begin{sol}
		$x=-6.6$ or $x=9.2$
	\end{sol}
\end{ex}
\begin{ex}
	Solve $f(x) \leq g(x)$.
	\begin{sol}
		$[-7.1, 7.5]$
	\end{sol}
\end{ex}
\begin{ex}
	Solve $h(x) < f(x)$.
	\begin{sol}
		$(-\infty, -7.7)\cup (-3.5, 0.0)\cup (7.1, \infty)$
	\end{sol}
\end{ex}
\begin{ex}
	$f(x) < -6$
	\begin{sol}
		no solutions
	\end{sol}
\end{ex}
\begin{ex}
	$f(x) \geq -6$
	\begin{sol}
		$(-\infty, \infty)$
	\end{sol}
\end{ex}
\begin{ex}
	List all extreme values $f$.
	\begin{sol}
		$f$ has no maximum value, but it has local maximum $0.6$ when $x=-1.4$. $f$ has minimum value $-5.2$ when $x=4.6$. $f$ has a local minimum value of $-2.3$ when $x=-6.2$.
	\end{sol}
\end{ex}
\end{multicols}


Graph $y=f(x)$ and determine the domain and range.
\begin{multicols}{2}
	\begin{ex}
		$f(x) = \sqrt{x-3}-5$
		
		\begin{sol}
			As usual, check graph with a graphing utility after completing all parts of the exercise.
			
			domain: $[3,\infty)$, range $[-5, \infty)$
		\end{sol}
	\end{ex}

	\begin{ex}
	$f(x) = (x+7)^5$
	
	\begin{sol}
		domain: $(-\infty,\infty)$, range $(-\infty,\infty)$
	\end{sol}
\end{ex}

	\begin{ex}
	$f(x) = \dfrac{1}{x^2+4x+4}$
	
	\begin{sol}
		Note $f(x) = \dfrac{1}{(x+2)^2}$. domain: $(-\infty,-2)\cup(-2,\infty)$, range $(0,\infty)$
	\end{sol}
\end{ex}

	\begin{ex}
	$f(x) =-3x^2-6x+1$
	
	\begin{sol}
		domain: $(-\infty,\infty)$, range $(-\infty,4]$
	\end{sol}
\end{ex}


\begin{ex}
	$f(x) = 7.3^x$
	
	\begin{sol}
		domain: $(-\infty,\infty)$, range $(0,\infty)$
	\end{sol}
\end{ex}
\end{multicols}


\Closesolutionfile{ans}
\subsection*{Answers \nopunct} \hfill
\begin{multicols}{2}
	\input{ans23}
\end{multicols}


\newpage
\section{Working without the graph (maybe...)}




\Opensolutionfile{ans}[ans24]
\subsection*{Exercises \nopunct} Do analytically without graphing then check on a graphing utility.\hfill

\begin{ex}
	Give an example of a relation between $x$ and $y$ we studied in Chapter 1 that does not represent $y$ as a function of $x$. Explain algebraically and graphically.
	\begin{sol}
		One example is a circle. Look at the circle $x^2+y^2=1$. If we try to isolate $y$ and write it explicitly as a function of $x$ we get $y=\sqrt{1-x^2}$ or $y=-\sqrt{1-x^2}$. That is, for most $x$-values, there are two possible $y$-values. Therefore the relation is not a function. Graphically, we see that most vertical lines intersect the graph of a circle in two places, so there are usually two $y$ values for a given $x$.
	\end{sol}
\end{ex}

Determine the domain of each function.

\begin{multicols}{2}
	\begin{ex}
		$f(x) = \dfrac{16}{(x-2)(x+3)}$
		\begin{sol}
			$(-\infty-3)\cup(-3, 2)\cup (2, \infty) $. That is, all real numbers except $-3$ and $2$.
		\end{sol}
	\end{ex}
	\begin{ex}
	$h(x) = \dfrac{x+1}{x^2-5x-6}$
	\begin{sol}
		All real numbers except $-1$ and $6$.
	\end{sol}
\end{ex}
	\begin{ex}
	$g(x) = \dfrac{3}{x}+\dfrac{4}{(x-1)^2} - \dfrac{1}{x-\sqrt{2}}$
	\begin{sol}
		All real numbers except $0$, $1$ and $\sqrt{2}$.
	\end{sol}
\end{ex}

	\begin{ex}
		$k(x) = \sqrt[3]{x^2-1}$
	\begin{sol}
		All real numbers.
	\end{sol}
\end{ex}
	\begin{ex}
		$\ell(x) = \sqrt{3x+7}$
	\begin{sol}
		$\left[ -\frac{7}{3}, \infty \right) $
	\end{sol}
\end{ex}
	\begin{ex}
	$r(t) = \dfrac{\sqrt{10-2t}}{t^2-7t}$
	\begin{sol}
		$(-\infty, 0)\cup(0, 5]$
	\end{sol}
\end{ex}
	\begin{ex}
	$s(t) = \dfrac{t^2-7t}{\sqrt{10-2t}}$
	\begin{sol}
		$(-\infty, 5)$
	\end{sol}
\end{ex}
\end{multicols}

	\begin{ex} \label{oneToone}
	In the relation $x = (y-4)^3$ is  $y$ a function of $x$?
	\begin{sol}
		Explicitly $x$ is written as a function of $y$, but the equation is equivalent to $y = \sqrt[3]{x}+4$, so $y$ is a function of $x$ as well.
	\end{sol}
\end{ex}
\begin{comment}
\begin{ex}
	In \#\ref{oneToone} we saw a relation where both $y$ was a function of $x$ and $x$ was a function of $y$. Sometimes such functions are called one-to-one\index{one-to-one} as each $y$ corresponds to only one $x$ (generally, functions can have multiple $x$ values for the same $y$). Is the function $y=x^2-9$ one-to-one?
	\begin{sol}
		No as $x$ is not a function of $y$.
	\end{sol}
\end{ex}
\end{comment}

\begin{ex}
	Find the intersections of the graphs of $f(x) = x^3+5x$ and $g(x) = 8x^2-7x$.
	\begin{sol}
		$(0,0)$, $(2, 18)$, $(6, 246)$
	\end{sol}
\end{ex}

\begin{ex}
	The distance between a point $P$ and a line $\ell$ (which doesn't contain $P$) is defined to be the distance from $P$ to the nearest point on $\ell$. Call the nearest point $Q$. It can be shown that the line containing $P$ and $Q$ is perpendicular to $\ell$. Use this  to find the distance from the point $(-1,-2)$ to the line $y=3x+6$.
	\begin{sol}
		The distance is $\frac{5}{\sqrt{10}}$.
	\end{sol}
\end{ex}

\begin{ex}
	The height of a certain soccer ball in feet $t$ seconds after it has been kicked is given by the function $h(t) = -16t^2+102t$. What is the maximum height of the ball? When does it hit the ground? What does the inequality $h(t) > 100$ represent? Solve it.
	\begin{sol}
		The maximum height is $\frac{2601}{16} \approx 162.6$ ft. It hits the ground at $t = 51/8 =6.375$ s. Any time $t$ that satisfies the inequality corresponds to a time that the ball is more than 100 ft high. The solution set is $\left(\frac{51-\sqrt{1001}}{16}, \frac{51+\sqrt{1001}}{16} \right)$, so the ball is over 100 ft high between about $1.2$ s after being kicked and $5.2$ s after being kicked.
	\end{sol}
\end{ex}

\begin{ex}
	Find the intersection of the graphs of the upper semi-circle $y = \sqrt{9-x^2}$ and the line $y = \dfrac{1}{2}x$.
	
	\begin{sol}
		The only intersection is $\left( \frac{6}{\sqrt{5}}, \frac{3}{\sqrt{5}} \right)$. You probably solved this by squaring both sides of the equation. Be careful as this can create extraneous solutions! Why? If one side of the equation is positive and the other negative, then their squares can be equal without both sides being equal. This is such an equation. There is another potential problem: $(\sqrt{9-x^2})^2 = 9-x^2$ but only for $-3 \leq x \leq 3$. A solution outside this interval would also be extraneous. So many potential issues for such a simple-seeming exercise!
	\end{sol}
\end{ex}

	Solve the inequality analytically then draw a graph representing the two functions ($y=LHS$ and $y=RHS$).
	
	\begin{multicols}{2}
		\begin{ex}
			$-\dfrac{1}{3}x+5 \geq 2x-8$
			\begin{sol}
				$x \leq \frac{39}{7}$. i.e. $(-\infty, \frac{39}{7}] $
			\end{sol}
		\end{ex}
	
		\begin{ex}
		$2x^2+3x > x+4$
	\begin{sol}
		$ (-\infty -2)\cup (1,\infty) $
	\end{sol}
\end{ex}

		\begin{ex}
	$x^2+4x-10 \leq -2x^2+4x+6$
	\begin{sol}
		$ \left[ -\frac{4}{\sqrt{3}}, \frac{4}{\sqrt{3}} \right] $
	\end{sol}
\end{ex}

		\begin{ex} \label{goldenInequ}
	$x^2 < x+1$
	\begin{sol}
		$\left(  \frac{1-\sqrt{5}}{2}, \frac{1+\sqrt{5}}{2} \right)$
	\end{sol}
\end{ex}
		
	\end{multicols}

\begin{ex}
	What can you say about a number in the solution set of \#17? That is, what can you say about a number between $\frac{1-\sqrt{5}}{2}$ and $\frac{1+\sqrt{5}}{2}$?
	\begin{sol}
		When you square such a number you get a result less than one greater than the original.
		
		For example, $1.6^2 = 2.56$ but $1.6+1=2.6$, and we see $2.56 < 2.6$. Also, $(-0.61)^2 = 0.3721$ but $-0.61+1 = 0.39$. 
		
		What is true about the number $\frac{1+\sqrt{5}}{2}$? Have you heard of this number?
	\end{sol}
\end{ex}

\begin{ex}
	What is the domain of the function $f(x) = \dfrac{\sqrt{21-x-2x^2}}{x^2-2x-8}$?
	\begin{sol}
		$\left[-\frac{7}{2}, -2)\cup(-2, 3 \right]$
	\end{sol}
\end{ex}


\Closesolutionfile{ans}
\subsection*{Answers \nopunct} \hfill
\begin{multicols}{2}
	\input{ans24}
\end{multicols}




\newpage
\section{Logarithms}

\index{good bases}\index{log}\index{ln}\index{logarithm}Good bases for logarithms and exponential functions: $b>0$ and $b\neq 1$,

i.e. $0<b<1$ or $b>1$.

\begin{tikzpicture} [scale=0.7]
	\drawgridxxyyllb{-4}{4}{-1}{10}{}{}
\draw[<->, very thick, samples=15, domain=-4:3.32193, smooth] plot (\x, {exp(.693147*\x)}) node [above] {\ \ \ \ $y=b^x$};
\draw (0,8) node [left] {$b^x$};
\draw [dashed, thick, <-] (0,8) -- (3,8);
\draw [dashed, thick, <-] (3,7.9) -- (3,0);
\draw (3,0) node [below] { $x$};
\draw [thick] (0.2,1) -- (-0.2,1) node [above left] {$1$};

\draw (3.2,4) node [right] {\fbox{$b^x=y \iff \log_b(y)=x$ \small}};	
	
\begin{scope}[shift={(9.5,0)}]
\drawgridxxyyllb{-4}{4}{-1}{10}{}{}
\draw[<->, very thick, samples=15, domain=-4:3.32193, smooth] plot (\x, {exp(.693147*\x)}) node [above] {\ \ \ \ $y=b^x$};
\draw (0,8) node [left] {$y$};
\draw [dashed, thick, ->] (0,8) -- (2.9,8);
\draw [dashed, thick, ->] (3,8) -- (3,0.05);
\draw (3,0) node [below] { $\log_b (y)$};
\draw [thick] (0.2,1) -- (-0.2,1) node [above left] {$1$};
\end{scope}	
\end{tikzpicture}

For example:

\begin{tikzpicture} [scale=0.7]
\drawgridxxyyllb{-4}{4}{-1}{10}{}{}
\draw[<->, very thick, samples=15, domain=-4:3.32193, smooth] plot (\x, {exp(.693147*\x)}) node [above] {\ \ \ \ $y=2^x$};
\draw (0,8) node [left] {$2^3=8$};
\draw [dashed, thick, <-] (0,8) -- (3,8);
\draw [dashed, thick, <-] (3,7.9) -- (3,0);
\draw (3,0) node [below] { $3$};
\draw [thick] (0.2,1) -- (-0.2,1) node [above left] {$1$};

\draw (3.2,4) node [right] {\fbox{$2^3=8 \iff \log_2(8)=3$ \small}};	

\begin{scope}[shift={(9.5,0)}]
\drawgridxxyyllb{-4}{4}{-1}{10}{}{}
\draw[<->, very thick, samples=15, domain=-4:3.32193, smooth] plot (\x, {exp(.693147*\x)}) node [above] {\ \ \ \ $y=2^x$};
\draw (0,8) node [left] {$8$};
\draw [dashed, thick, ->] (0,8) -- (2.9,8);
\draw [dashed, thick, ->] (3,8) -- (3,0.05);
\draw (3,0) node [below] { $\log_2 (8) = 3$};
\draw [thick] (0.2,1) -- (-0.2,1) node [above left] {$1$};
\end{scope}	
\end{tikzpicture}

Euler's number, $e = 2.71828 \cdots$

The natural logarithm has base $e$ and will be denoted by ``$\ln$''. The common logarithm has base 10 and will be denoted ``$\log$'' (no subscript).\footnote{This usage agrees with most pocket calculators and lower-level math textbooks, but advanced books and computer languages often use the convention that the natural logarithm is denoted by ``$\log$''.}

Visualized conversion between ``basic'' exponential and logarithmic equations:
\begin{center}
\begin{tikzpicture}
\draw (0.7,1) node [right] {$b^x = y$};
\draw [->] (1.1,0.95) -- (1.65,0.2);
\draw [->] (1.60,0.85) -- (1.0,0.25); 
\draw (0,0) node [right] {$\log_b(y) = x$};
\end{tikzpicture}\ \ \ \ \ \ \ \ \ \ \ \ \ 
\begin{tikzpicture}
\draw (0.7,0) node [right] {$b^x = y$};
\draw [->] (1.1,0.80) -- (1.65,0.15);
\draw [->] (1.60,0.85) -- (1.15,0.25); 
\draw (0,1) node [right] {$\log_b(y) = x$};
\end{tikzpicture}
\end{center}



\Opensolutionfile{ans}[ans25]
\subsection*{Exercises \nopunct} \hfill

Rewrite each exponential equation in logarithmic form and each logarithmic equation in exponential form.

\begin{multicols}{3}

\begin{ex}
		$2^{3} = 8$
	\begin{sol}
		$\log_{2}(8) = 3$
	\end{sol}
\end{ex}
\begin{ex}
	$5^{-3} = \frac{1}{125}$  
	\begin{sol}
		$\log_{5}\left(\frac{1}{125}\right) = -3$
	\end{sol}
\end{ex}
\begin{ex}
	 $4^{5/2} = 32$  
	\begin{sol}
		$\log_{4}(32) = \frac{5}{2}$
	\end{sol}
\end{ex}
\begin{ex}
	$\left(\frac{1}{3}\right)^{-2} = 9$
	\begin{sol}
		$\log_{\frac{1}{3}}(9) = -2$
	\end{sol}
\end{ex}
\begin{ex}
	$\left(\frac{4}{25}\right)^{-1/2} = \frac{5}{2}$ 
	\begin{sol}
		$\log_{\frac{4}{25}}\left(\frac{5}{2}\right) = -\frac{1}{2}$
	\end{sol}
\end{ex}
\begin{ex}
	$10^{-3} = 0.001$ 
	\begin{sol}
		$\log(0.001) = -3$
	\end{sol}
\end{ex}
\begin{ex}
	$e^{0}  = 1$
	\begin{sol}
		$\ln(1) = 0$
	\end{sol}
\end{ex}
\begin{ex}
	$\log_{5}(25) = 2$ 
	\begin{sol}
		 $5^{2} = 25$
	\end{sol}
\end{ex}
\begin{ex}
	$\log_{25} (5) = \frac{1}{2}$ 
	\begin{sol}
		 $(25)^{\frac{1}{2}} = 5$
	\end{sol}
\end{ex}
\begin{ex}
	$\log_{3} \left(\frac{1}{81} \right) = -4$  
	\begin{sol}
		$3^{-4} = \frac{1}{81}$
	\end{sol}
\end{ex}
\begin{ex}
	 $\log_{\frac{4}{3}} \left(\frac{3}{4} \right) = -1$ 
	\begin{sol}
		  $\left(\frac{4}{3} \right)^{-1} = \frac{3}{4}$
	\end{sol}
\end{ex}
\begin{ex}
	$\log(100) = 2$  
	\begin{sol}
	 $10^{2} = 100$	
	\end{sol}
\end{ex}
\begin{ex}
	 $\log (0.1) = -1$ 
	\begin{sol}
		 $10^{-1} = 0.1$
	\end{sol}
\end{ex}
\begin{ex}
	$\ln(e) = 1$ 
	\begin{sol}
		 $e^{1} = e$
	\end{sol}
\end{ex}
\begin{ex}
	$\ln\left(\frac{1}{\sqrt{e}}\right) = -\frac{1}{2}$ 
	\begin{sol}
		 $e^{-\frac{1}{2}} = \frac{1}{\sqrt{e}}$
	\end{sol}
\end{ex}
\end{multicols}

Evaluate the expression without a calculator.

\begin{multicols}{3}
\begin{ex}
	$\log_{3} (27)$ 
	\begin{sol}
		 $\log_{3} (27) = 3$
	\end{sol}
\end{ex}
\begin{ex}
	 $\log_{6} (216)$
	\begin{sol}
		$\log_{6} (216) = 3$
	\end{sol}
\end{ex}
\begin{ex}
	 $\log_{2} (32)$
	\begin{sol}
		$\log_{2} (32) = 5$
	\end{sol}
\end{ex}
\begin{ex}
	$\log_{6} \left( \frac{1}{36} \right)$
	\begin{sol}
		$\log_{6} \left( \frac{1}{36} \right) = -2$
	\end{sol}
\end{ex}
\begin{ex}
	$\log_{8} (4)$
	\begin{sol}
		$\log_{8} (4) = \frac{2}{3}$
	\end{sol}
\end{ex}
\begin{ex}
	$\log_{36} (216)$
	\begin{sol}
		$\log_{36} (216) = \frac{3}{2}$
	\end{sol}
\end{ex}
\begin{ex}
	$\log_{\frac{1}{5}} (625)$
	\begin{sol}
		$\log_{\frac{1}{5}} (625) = -4$
	\end{sol}
\end{ex}
\begin{ex}
	$\log_{\frac{1}{6}} (216)$
	\begin{sol}
		$\log_{\frac{1}{6}} (216) = -3$
	\end{sol}
\end{ex}
\begin{ex}
	$\log_{36} (36)$
	\begin{sol}
		$\log_{36} (36)=1$
	\end{sol}
\end{ex}
\begin{ex}
	$\log \left(\frac{1}{1000000}\right)$
	\begin{sol}
		$\log \frac{1}{1000000} = -6$
	\end{sol}
\end{ex}
\begin{ex}
	$\log(0.01)$
	\begin{sol}
		$\log(0.01) = -2$
	\end{sol}
\end{ex}
\begin{ex}
	$\ln\left(e^3\right)$
	\begin{sol}
	$\ln\left(e^3\right) = 3$	
	\end{sol}
\end{ex}
\begin{ex}
	$\log_{4} (8)$
	\begin{sol}
		$\log_{4} (8) = \frac{3}{2}$
	\end{sol}
\end{ex}
\begin{ex}
	$\log_{6} (1)$
	\begin{sol}
		$\log_{6} (1) = 0$
	\end{sol}
\end{ex}
\begin{ex}
	$\log_{13} \left(\sqrt{13}\right)$
	\begin{sol}
		$\log_{13} \left(\sqrt{13}\right) = \frac{1}{2}$
	\end{sol}
\end{ex}
\begin{ex}
	$\log_{36} \left(\sqrt[4]{36}\right)$
	\begin{sol}
		$\log_{36} \left(\sqrt[4]{36}\right) = \frac{1}{4}$
	\end{sol}
\end{ex}
\begin{ex}
	$7^{\log_{7} (3)}$
	\begin{sol}
	$7^{\log_{7} (3)} = 3$	
	\end{sol}
\end{ex}
\begin{ex}
	 $36^{\log_{36}(216)}$
	\begin{sol}
		$36^{\log_{36}(216)} = 216$
	\end{sol}
\end{ex}
\begin{ex}
	$\log_{36} \left(36^{216}\right)$
	\begin{sol}
		$\log_{36} \left(36^{216}\right) = 216$
	\end{sol}
\end{ex}
\begin{ex}
	$\ln \left(e^{5} \right)$
	\begin{sol}
		$\ln(e^{5}) = 5$
	\end{sol}
\end{ex}
\begin{ex}
	$\log \left(\sqrt[9]{10^{11}}\right)$
	\begin{sol}
		$\log \left(\sqrt[9]{10^{11}}\right) = \frac{11}{9}$
	\end{sol}
\end{ex}
\begin{ex}
	$\log\left( \sqrt[3]{10^5} \right)$
	\begin{sol}
		$\log\left( \sqrt[3]{10^5} \right) = \frac{5}{3}$
	\end{sol}
\end{ex}
\begin{ex}
	$\ln \left( \frac{1}{\sqrt{e}}\right)$
	\begin{sol}
		 $\ln \left( \frac{1}{\sqrt{e}}\right) = -\frac{1}{2} $
	\end{sol}
\end{ex}
\begin{ex}
	$\log_{5} \left(3^{\log_{3} (5)}\right)$
	\begin{sol}
		$\log_{5} \left(3^{\log_{3} 5}\right) = 1$
	\end{sol}
\end{ex}
\begin{ex}
	$\log\left(e^{\ln(100)}\right)$ 
	\begin{sol}
		$\log\left(e^{\ln(100)}\right) = 2$
	\end{sol}
\end{ex}
\begin{ex}
	$\log_{2}\left(3^{-\log_{3}(2)}\right)$
	\begin{sol}
		 $\log_{2}\left(3^{-\log_{3}(2)}\right) = -1$
	\end{sol}
\end{ex}
\begin{ex}
	$\ln\left(42^{6\log(1)}\right)$
	\begin{sol}
		$\ln\left(42^{6\log(1)}\right) = 0$
	\end{sol}
\end{ex}
\end{multicols}

\begin{ex}
	\index{earthquake ! Richter Scale} Earthquakes are complicated events and it is not our intent to provide a complete discussion of the science involved in them.  Instead, we refer the interested reader to a solid course in Geology\footnote{Rock-solid, perhaps?}. The Richter scale measures the magnitude of an earthquake by comparing the amplitude of the seismic waves of the given earthquake to those of a ``magnitude 0 event'', which was chosen to be a seismograph reading of $0.001$ millimeters recorded on a seismometer 100 kilometers from the earthquake's epicenter.  Specifically, the magnitude of an earthquake is given by \[M(x) = \log \left(\dfrac{x}{0.001}\right)\] where $x$ is the seismograph reading in millimeters of the earthquake recorded 100 kilometers from the epicenter.  
	\ssp
	\item Show that $M(0.001) = 0$.
	\item Compute $M(80,000)$.
	\item Show that an earthquake which registered 6.7 on the Richter scale had a seismograph reading ten times larger than one which measured 5.7.
	\item Find two news stories about recent earthquakes which give their magnitudes on the Richter scale.  How many times larger was the seismograph reading of the earthquake with larger magnitude?
	\esp
	\begin{sol}
		\ssp
		\item $M(0.001) = \log \left(\frac{0.001}{0.001} \right) = \log(1) = 0$.
		\item $M(80,000) = \log \left(\frac{80,000}{0.001} \right) = \log(80,000,000) \approx 7.9$.
		\item Solutions vary but should be written carefully. No calculator is needed. Try calculating the two readings then dividing.
		\esp
	\end{sol}
\end{ex}

\begin{ex}
	Remember Jim and his new car from Section \ref{JimCarSection} \#\ref{JimCar}? The ``better'' model for his car's value in dollars was $V = 30,000\cdot 0.80^t$ where $t$ is the number of years after purchase. When will his car be worth \$20,000? Answer in years and days (use a calculator to evaluate the logarithm).
	\begin{sol}
		About $1.81706$ years after purchase. This is about 1 year and a little over 298 days.
	\end{sol}
\end{ex}

\begin{ex}
	Estimate $\log(3,467,822)$ without a calculator.
	\begin{sol}
		Somewhere between 6 and 7.
	\end{sol}
\end{ex}


\begin{ex}
	Estimate $\log_2(25)$ without a calculator.
	\begin{sol}
		Somewhere between $4$ and $5$.
	\end{sol}
\end{ex}

\begin{ex}
	Estimate $\log(0.004509)$ without a calculator.
	\begin{sol}
		Somewhere between $-2$ and $-3$.
	\end{sol}
\end{ex}



\Closesolutionfile{ans}
\subsection*{Answers \nopunct} \hfill
\begin{multicols}{2}
	\input{ans25}
\end{multicols}

\newpage
\section{Log graphs, asymptotes, end-behavior, limit language}
\index{exponential functions}\index{logarithms}\index{logarithm graphs}\index{limit}

\begin{tikzpicture} [scale=0.6]
\drawgridxxyyb{-5}{10}{-5}{10}
\draw[<->, dashed,very thick, samples=15, domain=-5:3.32193, smooth] plot (\x, {exp(.693147*\x)}) node [above] {\ \ \ \ $y=b^x$};
\draw [thick, <->, dotted] (-4.5,-4.5) -- (9,9) node [above right] {$y=x$};
\draw [thick] (0.2,1) -- (-0.2,1) node [above left] {$1$};

\draw[<->,very thick, samples=15, domain=-5:3.32193, smooth] plot ({exp(.693147*\x)}, \x) node [above] {\ \ \ \ $y=\log_b(x)$};
\draw [thick] (1,0.2) -- (1,-0.2) node [below right] {$1$};
\draw (9, 2.5) node [below] {\fbox{$b > 1$}};
\end{tikzpicture}


\begin{tikzpicture} [scale=0.6]
\drawgridxxyyb{-4}{10}{-4}{10}
\draw[<->, dashed,very thick, samples=15, domain=-5:3.32193, smooth] plot (-\x, {exp(.693147*\x)}) node [above] {\ \ \ \ $y=b^x$};
\draw [thick, <->, dotted] (-3.5,-3.5) -- (8,8) node [above right] {$y=x$};
\draw [thick] (0.2,1) -- (-0.2,1) node [below left] {$1$};

\draw[<->,very thick, samples=15, domain=-5:3.32193, smooth] plot ({exp(.693147*\x)}, -\x) node [above] {\ \ \ \ $y=\log_b(x)$};
\draw [thick] (1,0.2) -- (1,-0.2) node [below left] {$1$};
\draw (9, -0.8) node [below] {\fbox{$0<b<1$}};
\end{tikzpicture}

End-behavior\index{end-behavior}\index{asymptotes} examples (think of the graphs):
\begin{itemize}
	\item As $x \to \infty$, $2^x \to \infty$.
	\item As $x \to -\infty$, $2^x \to 0$. (Graph has horizontal asymptote $y=0$\index{asymptote}\index{horizontal asymptote}).
	\item If it's understood that $y=2^x$, you can also say: as $x \to \infty$, $y \to \infty$, and as $x \to -\infty$, $y \to 0$.
	\item If it's understood that $f(x)=2^x$, you can also say: as $x \to \infty$, $f(x) \to \infty$, and as $x \to -\infty$, $f(x) \to 0$.
\end{itemize}

\bigskip
\begin{itemize}
	\item As $x \to \infty$, $0.45^x \to 0$. (Graph has horizontal asymptote $y=0$).
\item As $x \to -\infty$, $0.45^x \to \infty$.
\end{itemize}

\bigskip
\begin{itemize}
	\item As $x \to \infty$, $x^2 \to \infty$. 
	\item As $x \to -\infty$, $x^2 \to \infty$.
\end{itemize}


\bigskip
\begin{itemize}
	\item As $x \to \infty$, $x^3 \to \infty$. 
	\item As $x \to -\infty$, $x^3 \to -\infty$.
\end{itemize}


\bigskip
\begin{itemize}
	\item As $x \to \infty$, $-3x^2+45x+2300 \to -\infty$. 
	\item As $x \to -\infty$, $-3x^2+45x+2300 \to -\infty$.
\end{itemize}

\bigskip
\begin{itemize}
	\item As $x \to \infty$, $\log_3 (x) \to \infty$. 
	\item Nothing is on the left!
\end{itemize}

\bigskip
\begin{itemize}
	\item As $x \to \infty$, $\log_{0.5} (x) \to -\infty$. 
	\item Nothing is on the left!
\end{itemize}

\bigskip
\begin{itemize}
	\item As $x \to \infty$, $x^{-1} \to 0$.  (Graph has horizontal asymptote $y=0$).
	\item As $x \to -\infty$, $x^{-1} \to 0$. (Graph has horizontal asymptote $y=0$). 
\end{itemize}

\bigskip
\begin{itemize}
	\item As $x \to \infty$, $\dfrac{1}{(x-7)^2}+3 \to 3$.  (Graph has horizontal asymptote $y=3$).
	\item As $x \to -\infty$, $\dfrac{1}{(x-7)^2}+3 \to 3$. (Graph has horizontal asymptote $y=3$). 
\end{itemize}



All of the following statements imply the graph has a vertical asymptote\index{vertical asymptote}:
\begin{itemize}
	\item As $x \to 0^+$, $x^{-1} \to \infty$.  (Graph has vertical asymptote $x=0$).
	\item As $x \to 0^-$, $x^{-1} \to \infty$.  (Graph has vertical asymptote $x=0$).
\end{itemize}

\bigskip
\begin{itemize}
	\item As $x \to 0^+$, $\log_2 (x) \to -\infty$.  (Graph has vertical asymptote $x=0$).
	\item Nothing is to the left at $x=0$.
\end{itemize}

\bigskip
\begin{itemize}
	\item As $x \to 7^+$, $\dfrac{1}{(x-7)^2}+3 \to \infty$.  (Graph has vertical asymptote $x=7$).
	\item As $x \to 7^-$, $\dfrac{1}{(x-7)^2}+3 \to \infty$.  (Graph has vertical asymptote $x=7$).
\end{itemize}


\Opensolutionfile{ans}[ans26]
\subsection*{Exercises \nopunct} \hfill

Solve the equation or inequality. Simplify if possible.
\begin{multicols}{3}
	\begin{ex}
		$5^x=19$
		\begin{sol}
		$x=\log_5 (19)$
		\end{sol}
	\end{ex}
	\begin{ex}
	$6\cdot 7^x = 2058$
	\begin{sol}
		$x=\log_7(343) = 3$
	\end{sol}
\end{ex}
	\begin{ex}
	$2^{4x}=8$
	\begin{sol}
		$x=\frac{3}{4}$
	\end{sol}
\end{ex}
%	\begin{ex}
%	$3^{x-1}=27$
%	\begin{sol}
%		$x=4$
%	\end{sol}
%\end{ex}
	\begin{ex}
	$5^{2x-1}=125$
	\begin{sol}
		$x=2$
	\end{sol}
\end{ex}
	\begin{ex}
	$8^x=\dfrac{1}{128}$
	\begin{sol}
		$x=-\frac{7}{3}$
	\end{sol}
\end{ex}
	\begin{ex}
	$2^{x^3-x}=1$
	\begin{sol}
		$x=-1, 0, 1$
		
		[This means $x=-1$ or $x=0$ or $x=1$.]
	\end{sol}
\end{ex}
	\begin{ex}
	$3^{2x}=5$
	\begin{sol}
		$x=\frac{1}{2}\log_3(5)$
	\end{sol}
\end{ex}
	\begin{ex}
	$5^{-x}=2$
	\begin{sol}
		$x=-\log_5 (2)$
	\end{sol}
\end{ex}
	\begin{ex}
	$5^{x}=-2$
	\begin{sol}
		No solution.
	\end{sol}
\end{ex}
	\begin{ex}
	$3^{x-1}=29$
	\begin{sol}
		$x=1+\log_3(29)$
	\end{sol}
\end{ex}
	\begin{ex}
	$1.005^{12x}=3$
	\begin{sol}
		$x=\frac{\log_{1.005}(3)}{12}$
	\end{sol}
\end{ex}
	\begin{ex}
	$e^{-5730k}=\dfrac{1}{2}$
	\begin{sol}
		$k=-\frac{\ln(1/2)}{5730}$
	\end{sol}
\end{ex}
	\begin{ex}
	$2000e^{0.1t}=4000$
	\begin{sol}
		$t=10\ln(2)$
	\end{sol}
\end{ex}
	\begin{ex}
	$\dfrac{100e^x}{e^x+2} = 50$
	\begin{sol}
		$x = \ln(2)$
	\end{sol}
\end{ex}
	\begin{ex}
	$\dfrac{5000}{1+2e^{-3t}} = 2500$
	\begin{sol}
		$x = \frac{1}{3}\ln(2)$
	\end{sol}
\end{ex}
\begin{ex}
	$e^x>53$
	\begin{sol}
		$(\ln(53),\infty)$
	\end{sol}
\end{ex}
\begin{ex}
	$1000(1.005)^{12t}\geq3000$
	\begin{sol}
		$\left[\frac{\log_{1.005}(3)}{12},\infty\right)$
	\end{sol}
\end{ex}
\begin{ex}
	$25\left(\dfrac{4}{5}\right)^x\geq 10$
	\begin{sol}
		$\left(-\infty, \log_{4/5}(2/5)\right]$
	\end{sol}
\end{ex}
\begin{ex}
	$\log(x)=2.4$
	\begin{sol}
		$x=10^{2.4}$
	\end{sol}
\end{ex}
\begin{ex}
	$\log_2(x)=5$
	\begin{sol}
		$x=2^5=32$
	\end{sol}
\end{ex}
\begin{ex}
	$\log_3(7-2x)=2$
	\begin{sol}
		$x=-1$
	\end{sol}
\end{ex}
\begin{ex}
	$\log_{\frac{1}{2}}(2x-1)=-3$
	\begin{sol}
		$x=\frac{9}{2}$
	\end{sol}
\end{ex}
%\begin{ex}
%	$\log(x^2-99) = 0$
%	\begin{sol}
%		$x=\pm 10$
%		
%		[This means $x=10$ or $x=-10$.]
%	\end{sol}
%\end{ex}
\begin{ex}
	$\log(x^2-3x) = 1$
	\begin{sol}
		$x=-2,5$
	\end{sol}
\end{ex}
\begin{ex}
	$3\ln(x)-2 = 1-\ln(x)$
	\begin{sol}
		$x=e^{3/4}$
	\end{sol}
\end{ex}
\begin{ex}
	$6-\log_5(2x) = 0$
	\begin{sol}
		$x=\frac{15625}{2}$
	\end{sol}
\end{ex}
\begin{ex}
	$2.3< -\log(x) < 5.4$
	\begin{sol}
		$\left(  10^{-5.4}, 10^{-2.3} \right)$
	\end{sol}
\end{ex}
\end{multicols}

Consider the graph of each function. What is the domain and range of the function? Analyze the end-behavior and all vertical and horizontal asymptotes. Justify each with a limit statement.

\begin{multicols}{2}
	\begin{ex}
		$y= x^5$
		\begin{sol}
			Domain $(-\infty, \infty)$, range $[0, \infty)$.
			End-behavior: As $x\to\infty$, $y\to\infty$. As $x\to-\infty$, $y\to-\infty$. No horizontal asymptotes.
			Vertical Asymptotes: No vertical asymptotes.
		\end{sol}
	\end{ex}
	\begin{ex}
	$y= -x^4$
	\begin{sol}
		Domain $(-\infty, \infty)$, range $(-\infty, 0]$
		End-behavior: As $x\to\infty$, $y\to-\infty$. As $x\to-\infty$, $y\to-\infty$. No horizontal asymptotes.
		Vertical Asymptotes: No vertical asymptotes.
	\end{sol}
\end{ex}
	\begin{ex}
	$y= 2x^2-50x-100$
	\begin{sol}
		Domain $(-\infty, \infty)$, range $[-825/2, \infty)$.
		End-behavior: As $x\to\infty$, $y\to\infty$. As $x\to\infty$, $y\to-\infty$. No horizontal asymptotes.
		Vertical Asymptotes: No vertical asymptotes.
	\end{sol}
\end{ex}
	\begin{ex}
	$y= -\dfrac{1}{x}$
	\begin{sol}
		Domain: $(-\infty, 0)\cup(0, \infty)$, range $(-\infty, 0)\cup(0, \infty)$.
		End-behavior: As $x\to\infty$, $y\to0$. As $x\to-\infty$, $y\to0$. Both imply horizontal asymptote $y=0$.
		Vertical Asymptotes: vertical asymptote $x=0$ since as $x\to 0^+$, $y\to-\infty$. Also as $x\to 0^-$, $y\to\infty$.
	\end{sol}
\end{ex}
	\begin{ex}
	$y= \dfrac{1}{(x+5)^3}-2$
	\begin{sol}
		Domain: $(-\infty, -5)\cup(-5, \infty)$, range $(-\infty, -2)\cup(-2, \infty)$.
		End-behavior: As $x\to\infty$, $y\to-2$. As $x\to-\infty$, $y\to-2$. Both imply horizontal asymptote $y=-2$.
		Vertical Asymptotes: vertical asymptote $x=-5$ since as $x\to -5^+$, $y\to\infty$. Also as $x\to -5^-$, $y\to\infty$.
	\end{sol}
\end{ex}
	\begin{ex}
	$y= e^{-x}$
	\begin{sol}
				Domain $(-\infty, \infty)$, range $(0, \infty)$.
		End-behavior: As $x\to\infty$, $y\to0$. This implies horizontal asymptote $y=0$. As $x\to-\infty$, $y\to\infty$.
		Vertical Asymptotes: No vertical asymptote.
	\end{sol}
\end{ex}
	\begin{ex}
	$y= \left( \dfrac{8}{9} \right)^{x}+\dfrac{4}{5}$
	\begin{sol}
		Domain $(-\infty, \infty)$, range $(4/5, \infty)$.
		End-behavior: As $x\to\infty$, $y\to\infty$. As $x\to-\infty$, $y\to\frac{4}{5}$. This implies horizontal asymptote $y=\frac{4}{5}$.
		Vertical Asymptotes: No vertical asymptote.
	\end{sol}
\end{ex}
	\begin{ex}
	$y= 6\log_{0.2}(x)$
	\begin{sol}
		Domain $(0, \infty)$, range $(-\infty, \infty)$.
		End-behavior: As $x\to\infty$, $y\to-\infty$. The graph does not continue left. No horizontal asymptotes.
		Vertical Asymptotes: As $x\to0+$, $y\to\infty$, so there is vertical asymptote $x=0$.
	\end{sol}
\end{ex}
	\begin{ex}
	$y= \ln(x-6)$
	\begin{sol}
			Domain $(6, \infty)$, range $(-\infty, \infty)$.
		End-behavior: As $x\to\infty$, $y\to\infty$. The graph does not continue left. No horizontal asymptotes.
		Vertical Asymptotes: As $x\to6^+$, $y\to-\infty$, so there is vertical asymptote $x=6$.
	\end{sol}
\end{ex}

\end{multicols}

\begin{ex}
	Does the graph of $f(x)=\dfrac{|x|}{x}$ have asymptotes? What is the domain of $f$?
	
	\begin{sol}
		Yes, it has horizontal asymptote $y=1$ (on the right) as $f(x)=1$ for all $x>0$. So as $x\to\infty$, $f(x) \to 1$. Also, though, as $x\to-\infty$, $f(x) \to -1$, so there is also horizontal asymptote $y=-1$. There is no vertical asymptote as the range of the function is $\{1,-1\}$. The domain is all real numbers except 0.
	\end{sol}
\end{ex}

\begin{ex}
 The diameter $D$ of a tumor, in millimeters, $t$ days after it is detected is given by:  \[D(t) = 15e^{0.0277t} \]
 \ssp
 	\item  What was the diameter of the tumor when it was originally detected?
 \item  How long until the diameter of the tumor doubles?
 \esp
	\begin{sol}
 \ssp
\item $D(0) = 15$, so the tumor was 15 millimeters in diameter when it was first detected.
\item  $t = \frac{\ln(2)}{0.0277} \approx 25$ days.  
\esp
	\end{sol}
\end{ex}

\begin{ex}
 The population of Sasquatch in Bigfoot county is modeled by \[P(t) = \dfrac{120}{1 + 3.167e^{-0.05t}}\] where $P(t)$ is the population of Sasquatch $t$ years after $2010$.
 \ssp
\item  Find and interpret $P(0)$.

\item  Find the population of Sasquatch in Bigfoot county in 2013.  Round your answer to the nearest Sasquatch.

\item  When will the population of Sasquatch in Bigfoot county reach 60?  Round your answer to the nearest year.

\item  Find and interpret the end behavior of the graph of $y = P(t)$. Use a graphing utility to help if necessary.
 \esp
 
\begin{sol}
	\ssp
	  \item  $P(0) = \frac{120}{4.167} \approx 29$.  There are 29 Sasquatch in Bigfoot County in 2010.
		
		\item  $P(3) = \frac{120}{1+3.167e^{-0.05(3)}} \approx 32$ Sasquatch.
		
		\item  $t = 20 \ln(3.167) \approx 23$ years.
		
		\item  As $t \rightarrow \infty$, $P(t) \rightarrow 120$.  As time goes by, the Sasquatch Population in Bigfoot County will approach 120.  Graphically,  $y = P(x)$ has a horizontal asymptote $y=120$. Note the domain of this function is understood to be restricted to $[0, \infty)$ so we do not consider the left end.
	\esp
	\end{sol}
\end{ex}




\Closesolutionfile{ans}
\subsection*{Answers \nopunct} \hfill
\begin{multicols}{2}
	\input{ans26}
\end{multicols}


\newpage
\section{Right triangle trigonometry}
\index{trigonometry}\index{SOH-CAH-TOA}\index{sine}\index{cosine}\index{tangent}\index{right triangles}
\begin{multicols}{2}
\begin{tikzpicture} [scale=1.2]
\draw (-0.5,3.6) node [right] {\small \textsc{\underline{Right Triangles}}};
\draw [thick] (-0.5,4) -- (-0.5,-8.5) -- (3, -8.5) -- (3,4) -- (-0.5,4);
\draw (0,0) --(2,0) -- (2,3) --(0,0);
\draw (1.7,0) -- (1.7, .3) -- (2,.3);
\draw (0,0) ++ (0:.4) arc (0:56.31:.4);
\draw (0.4,0) node [above right] {$\theta$};
\draw (2,1.5) node [right] {Opp};
\draw (1,0) node [below] {Adj};
\draw (1,1.5) node [above left] {Hyp};
\draw (0,-1) node [right] {$\sin(\theta)=\frac{\text{Opp}}{\text{Hyp}}$};
\draw (0,-2) node [right] {$\cos(\theta)=\frac{\text{Adj}}{\text{Hyp}}$};
\draw (0,-3) node [right] {$\tan(\theta)=\frac{\text{Opp}}{\text{Adj}}$};

\pgfmathsetmacro{\ex}{0}
\pgfmathsetmacro{\ey}{-5.8}
\draw (\ex+1.7,\ey) -- (\ex+1.7,\ey+0.3) -- (\ex+2,\ey+0.3);
\draw (\ex,\ey) -- (\ex+2,\ey)  -- (\ex+2,\ey+2) -- (\ex,\ey);
\draw (\ex+0.2,\ey) node [above right] {\tiny $45^\circ$};
\draw (\ex+2.07,\ey+1.7) node [below left] {\tiny $45^\circ$};
\draw (\ex+2, \ey+1) node [right] {$1$};
\draw (\ex+1, \ey) node [below] {$1$};
\draw (\ex+1, \ey+1) node [above left] {$\sqrt{2}$};

\pgfmathsetmacro{\ex}{0}
\pgfmathsetmacro{\ey}{-7.8}
\draw (\ex+1.7,\ey) -- (\ex+1.7,\ey+0.3) -- (\ex+2,\ey+0.3);
\draw (\ex,\ey) -- (\ex+2,\ey)  -- (\ex+2,\ey+1.16) -- (\ex,\ey);
\draw (\ex+0.3,\ey-0.05) node [above right] {\tiny $30^\circ$};
\draw (\ex+2.08,\ey+1.05) node [below left] {\tiny $60^\circ$};
\draw (\ex+2, \ey+0.58) node [right] {$1$};
\draw (\ex+1, \ey) node [below] {$\sqrt{3}$};
\draw (\ex+1, \ey+0.58) node [above] {$2$};
\end{tikzpicture}
\columnbreak

Remember all the (non-right) angles in right triangles are acute\index{acute angle}, i.e. $0\dg < \theta < 90\dg$ or $0 < \theta < \frac{\pi}{2}$.

The two acute angles of a right triangle are complementary (Why?).

\textbf{co}sine means\index{complement}\index{cofunction} \textbf{complement}-sine, i.e.
$$\cos(\theta) = \sin\left(90\dg-\theta\right)$$
$$\sin(\theta) = \cos\left(90\dg-\theta\right)$$

or in radians:
$$\cos(\theta) = \sin\left(\frac{\pi}{2}-\theta\right)$$
$$\sin(\theta) = \cos\left(\frac{\pi}{2}-\theta\right)$$

Popular sine/cosines/tangent examples:
\begin{itemize}
	\item $\sin\left (30^\dg \right) = \dfrac{1}{2} = 0.5$
	\item $\cos\left (30^\dg \right) = \dfrac{\sqrt{3}}{2} \approx 0.866$
	\item $\tan\left (30^\dg \right) = \dfrac{1}{\sqrt{3}} = \dfrac{\sqrt{3}}{3} \approx 0.577$
	\item $\sin\left (45^\dg \right) = \dfrac{1}{\sqrt{2}} = \dfrac{\sqrt{2}}{2} \approx 0.707$
\end{itemize}

Popular mnemonic: SOH-CAH-TOA
\end{multicols}



\Opensolutionfile{ans}[ans27]
\subsection*{Exercises -- all refer to right-triangle trigonometric functions only (i.e. all angles acute).} \hfill

Find all missing side-lengths \textit{in terms of the given measurements}. Then use a calculator to approximate them to three significant figures. Use sine, cosine, and tangent only. Do not use the Pythagorean Theorem.

\begin{multicols}{2}

\begin{ex}
\begin{tikzpicture}[line cap=round,line join=round, x=1.0cm,y=1.0cm, scale=2]
\draw [  fill=black,fill opacity=0.1] (0.8154129355465971,0.6793311236835613) -- (0.7090685865510284,0.6793311236835613) -- (0.7090685865510284,0.5729867746879926) -- (0.8154129355465971,0.5729867746879926) -- cycle; 
\draw [shift={(-1.0043513905607917,0.5729867746879928)},  fill=black,fill opacity=0.1] (0,0) -- (0.:0.1503936206312709) arc (0.:45.:0.1503936206312709) -- cycle;
\draw (0.8154129355465971,0.5729867746879926)-- (-1.0043513905607917,0.5729867746879928);
\draw (0.8154129355465971,2.3927511007953814)-- (0.8154129355465971,0.5729867746879926);
\draw (0.8154129355465971,2.3927511007953814)-- (-1.0043513905607917,0.5729867746879928);
\begin{scriptsize}
\draw[  ] (0.7802338812318703,0.6522867146245028) ;
\draw[  ] (-0.10708848049262795,1.7) node {$2.6$};
\draw (-0.25,0.7) node {$45\dg (\pi/4 \text{ radians})$};
\end{scriptsize}
\end{tikzpicture}	
	\begin{sol}
	\begin{tikzpicture}[line cap=round,line join=round, x=1.0cm,y=1.0cm, scale=2]
	\draw[  fill=black,fill opacity=0.1] (0.8154129355465971,0.6793311236835613) -- (0.7090685865510284,0.6793311236835613) -- (0.7090685865510284,0.5729867746879926) -- (0.8154129355465971,0.5729867746879926) -- cycle; 
	\draw [shift={(-1.0043513905607917,0.5729867746879928)},  fill=black,fill opacity=0.1] (0,0) -- (0.:0.1503936206312709) arc (0.:45.:0.1503936206312709) -- cycle;
	\draw (0.8154129355465971,0.5729867746879926)-- (-1.0043513905607917,0.5729867746879928);
	\draw (0.8154129355465971,2.3927511007953814)-- (0.8154129355465971,0.5729867746879926);
	\draw (0.8154129355465971,2.3927511007953814)-- (-1.0043513905607917,0.5729867746879928);
	\begin{scriptsize}
	\draw[  ] (0.7802338812318703,0.6522867146245028) ;
	\draw[  ] (-0.10708848049262795,1.7) node {$2.6$};
	\draw[  ] (-0.25,0.7) node {$45\dg (\pi/4 \text{ radians})$};
	\draw (1.3, 1.5) node {$\frac{2.6}{\sqrt{2}} \approx 1.84$};
	\draw (0.2, 0.3) node {$\frac{2.6}{\sqrt{2}} \approx 1.84$};
	\end{scriptsize}
	\end{tikzpicture}		
\end{sol}
\end{ex}

\begin{ex}
	\begin{tikzpicture}[line cap=round,line join=round, x=1.0cm,y=1.0cm,scale=3]
	\draw[  fill=black,fill opacity=0.1] (1.8633863446044385,0.9391029317397016) -- (1.7570424722467648,0.9394213265671199) -- (1.7567240774193467,0.8330774542094461) -- (1.8630679497770204,0.832759059382028) -- cycle; 
	\draw [shift={(0.9118774984239697,0.8356069349848815)},  fill=black,fill opacity=0.1] (0,0) -- (-0.17154373746154988:0.1503936206312709) arc (-0.17154373746154988:50.394276769152945:0.1503936206312709) -- cycle;
	\draw (0.9118774984239697,0.8356069349848815)-- (1.8630679497770204,0.832759059382028);
	\draw (1.8630679497770204,0.832759059382028)-- (1.8665307965164981,1.9893498703675982);
	\draw (1.8665307965164981,1.9893498703675982)-- (0.9118774984239697,0.8356069349848815);
	\begin{scriptsize}
	\draw[  ] (1.3717821223815354,1.5045172315350384) node {$1.5$};
	\draw[  ] (1.2,.9) node {$51\dg$};
	\end{scriptsize}
	
	\end{tikzpicture}
	\begin{sol}
		\begin{tikzpicture}[line cap=round,line join=round, x=1.0cm,y=1.0cm,scale=3]
		\draw[  fill=black,fill opacity=0.1] (1.8633863446044385,0.9391029317397016) -- (1.7570424722467648,0.9394213265671199) -- (1.7567240774193467,0.8330774542094461) -- (1.8630679497770204,0.832759059382028) -- cycle; 
		\draw [shift={(0.9118774984239697,0.8356069349848815)},  fill=black,fill opacity=0.1] (0,0) -- (-0.17154373746154988:0.1503936206312709) arc (-0.17154373746154988:50.394276769152945:0.1503936206312709) -- cycle;
		\draw (0.9118774984239697,0.8356069349848815)-- (1.8630679497770204,0.832759059382028);
		\draw (1.8630679497770204,0.832759059382028)-- (1.8665307965164981,1.9893498703675982);
		\draw (1.8665307965164981,1.9893498703675982)-- (0.9118774984239697,0.8356069349848815);
		\begin{scriptsize}
		\draw[  ] (1.3717821223815354,1.5045172315350384) node {$1.5$};
		\draw[  ] (1.2,.9) node {$51\dg$};
			\draw (2.3, 1.3) node {$1.5\sin(51\dg) \approx 1.17$};
			\draw (1.5, 0.7) node {$1.5\cos(51\dg) \approx 0.944$};
		\end{scriptsize}
		\end{tikzpicture}
	\end{sol}
\end{ex}


\begin{ex}
	\begin{tikzpicture}[line cap=round,line join=round, x=1.0cm,y=1.0cm, scale=3]
	\draw[  fill=black,fill opacity=0.1] (1.839087401992462,0.9363643386454121) -- (1.7354821227290778,0.9123837908608535) -- (1.7594626705136363,0.8087785115974694) -- (1.8630679497770204,0.832759059382028) -- cycle; 
	\draw [shift={(1.4052952485178145,2.8105149046281124)},  fill=black,fill opacity=0.1] (0,0) -- (253.00954012855865:0.15039362063127087) arc (253.00954012855865:283.03222478094483:0.15039362063127087) -- cycle;
	\draw (0.7201657923021931,0.5682221555170682)-- (1.8630679497770204,0.832759059382028);
	\draw (1.8630679497770204,0.832759059382028)-- (1.4052952485178145,2.8105149046281124);
	\draw (1.4052952485178145,2.8105149046281124)-- (0.7201657923021931,0.5682221555170682);
	\begin{scriptsize}
	\draw[  ] (0.9,1.6) node {$9.2$};
	\draw[  ] (1.4,2.5) node {$\ds\frac{\pi}{6}$};
	\draw[  ] (1.65,2.5) node {$(30\dg)$};
	\end{scriptsize}
	\end{tikzpicture}	
	\begin{sol}
		\begin{tikzpicture}[line cap=round,line join=round, x=1.0cm,y=1.0cm, scale=3]
		\draw[  fill=black,fill opacity=0.1] (1.839087401992462,0.9363643386454121) -- (1.7354821227290778,0.9123837908608535) -- (1.7594626705136363,0.8087785115974694) -- (1.8630679497770204,0.832759059382028) -- cycle; 
		\draw [shift={(1.4052952485178145,2.8105149046281124)},  fill=black,fill opacity=0.1] (0,0) -- (253.00954012855865:0.15039362063127087) arc (253.00954012855865:283.03222478094483:0.15039362063127087) -- cycle;
		\draw (0.7201657923021931,0.5682221555170682)-- (1.8630679497770204,0.832759059382028);
		\draw (1.8630679497770204,0.832759059382028)-- (1.4052952485178145,2.8105149046281124);
		\draw (1.4052952485178145,2.8105149046281124)-- (0.7201657923021931,0.5682221555170682);
		\begin{scriptsize}
		\draw[  ] (0.9,1.6) node {$9.2$};
		\draw[  ] (1.4,2.5) node {$\ds\frac{\pi}{6}$};
		\draw[  ] (1.65,2.5) node {$(30\dg)$};
		
		\draw (2.0, 1.7) node {$\frac{9.2\sqrt{3}}{2} \approx 7.97$};
		\draw (1.6, 0.5) node {$9.2/2 = 4.6$};
		\end{scriptsize}
		\end{tikzpicture}	
	\end{sol}
\end{ex}


\begin{ex}
	\begin{tikzpicture}[line cap=round,line join=round, x=1.0cm,y=1.0cm, scale=2.5]
	\draw[  fill=black,fill opacity=0.1] (1.8991399398967843,0.9327987119808399) -- (1.7991002872979724,0.9688707021006038) -- (1.7630282971782085,0.8688310495017919) -- (1.8630679497770204,0.832759059382028) -- cycle; 
	\draw [shift={(2.5297272234554757,2.681627445050278)},  fill=black,fill opacity=0.1] (0,0) -- (226.09457598618536:0.15039362063127087) arc (226.09457598618536:250.17186622869224:0.15039362063127087) -- cycle;
	\draw (1.0369084495542888,1.130653109943109)-- (1.8630679497770204,0.832759059382028);
	\draw (1.8630679497770204,0.832759059382028)-- (2.5297272234554757,2.681627445050278);
	\draw (2.5297272234554757,2.681627445050278)-- (1.0369084495542888,1.130653109943109);
	\begin{scriptsize}
	\draw[  ] (1.7,2.0058293003059418) node {$2.2$};
	\draw[  ] (2.5398392426177394,2.8) node {$24\dg$};
	\end{scriptsize}
	\end{tikzpicture}
	
	\begin{sol}
		\begin{tikzpicture}[line cap=round,line join=round, x=1.0cm,y=1.0cm, scale=2.5]
		\draw[  fill=black,fill opacity=0.1] (1.8991399398967843,0.9327987119808399) -- (1.7991002872979724,0.9688707021006038) -- (1.7630282971782085,0.8688310495017919) -- (1.8630679497770204,0.832759059382028) -- cycle; 
		\draw [shift={(2.5297272234554757,2.681627445050278)},  fill=black,fill opacity=0.1] (0,0) -- (226.09457598618536:0.15039362063127087) arc (226.09457598618536:250.17186622869224:0.15039362063127087) -- cycle;
		\draw (1.0369084495542888,1.130653109943109)-- (1.8630679497770204,0.832759059382028);
		\draw (1.8630679497770204,0.832759059382028)-- (2.5297272234554757,2.681627445050278);
		\draw (2.5297272234554757,2.681627445050278)-- (1.0369084495542888,1.130653109943109);
		\begin{scriptsize}
		\draw[  ] (1.7,2.0058293003059418) node {$2.2$};
		\draw[  ] (2.5398392426177394,2.8) node {$24\dg$};
		\draw (1.3, 0.7) node {$2.2\sin(24\dg) \approx 0.895$};
		\draw (2.7, 1.3) node {$2.2\cos(24\dg) \approx 2.01$};
		\end{scriptsize}
		\end{tikzpicture}
		
	\end{sol}
\end{ex}


\begin{ex}
	\begin{tikzpicture}[line cap=round,line join=round, x=1.0cm,y=1.0cm, scale=2]
	\draw[  fill=black,fill opacity=0.1] (1.9873631855973626,0.8660513673386772) -- (1.9540708776407134,0.9903466031590193) -- (1.8297756418203712,0.9570542952023702) -- (1.8630679497770204,0.832759059382028) -- cycle; 
	\draw [shift={(3.5401160186425438,1.2819540779865723)},  fill=black,fill opacity=0.1] (0,0) -- (137.7153445839375:0.1819762809638379) arc (137.7153445839375:194.99462373997093:0.1819762809638379) -- cycle;
	\draw (1.1639310901414672,3.442953010256462)-- (1.8630679497770204,0.832759059382028);
	\draw (1.8630679497770204,0.832759059382028)-- (3.5401160186425438,1.2819540779865723);
	\draw (3.5401160186425438,1.2819540779865723)-- (1.1639310901414672,3.442953010256462);
	\begin{scriptsize}
	\draw[  ] (1.35,2.1643943076581804) node {$35$};
	\draw[  ] (3.2,1.3273034152245253) node {$57\dg$};
	\end{scriptsize}
	\end{tikzpicture} 
	\begin{sol}
			\begin{tikzpicture}[line cap=round,line join=round, x=1.0cm,y=1.0cm, scale=2]
		\draw[  fill=black,fill opacity=0.1] (1.9873631855973626,0.8660513673386772) -- (1.9540708776407134,0.9903466031590193) -- (1.8297756418203712,0.9570542952023702) -- (1.8630679497770204,0.832759059382028) -- cycle; 
		\draw [shift={(3.5401160186425438,1.2819540779865723)},  fill=black,fill opacity=0.1] (0,0) -- (137.7153445839375:0.1819762809638379) arc (137.7153445839375:194.99462373997093:0.1819762809638379) -- cycle;
		\draw (1.1639310901414672,3.442953010256462)-- (1.8630679497770204,0.832759059382028);
		\draw (1.8630679497770204,0.832759059382028)-- (3.5401160186425438,1.2819540779865723);
		\draw (3.5401160186425438,1.2819540779865723)-- (1.1639310901414672,3.442953010256462);
		\begin{scriptsize}
		\draw[  ] (1.35,2.1643943076581804) node {$35$};
		\draw[  ] (3.2,1.3273034152245253) node {$57\dg$};
		
		\draw (2.8, 0.7) node {$\frac{35}{\tan(57\dg)} \approx 22.7$};
		\draw (3.1, 2.3) node {$\frac{35}{\sin(57\dg)} \approx 41.7$};
		\end{scriptsize}
		\end{tikzpicture} 
	\end{sol}
\end{ex}



\begin{ex}
	\begin{tikzpicture}[line cap=round,line join=round, x=1.0cm,y=1.0cm, scale=2]
\draw[  fill=black,fill opacity=0.1] (2.377756723469079,1.1791065917858468) -- (2.410247531267721,1.088053156708355) -- (2.5013009663452124,1.1205439645069966) -- (2.468810158546571,1.2115973995844884) -- cycle; 
\draw [shift={(3.2214829278319157,-0.89772045909874)},  fill=black,fill opacity=0.1] (0,0) -- (109.63804237930775:0.13672147330115553) arc (109.63804237930775:135.2622370470228:0.13672147330115553) -- cycle;
\draw (3.2214829278319157,-0.89772045909874)-- (2.468810158546571,1.2115973995844884);
\draw (2.468810158546571,1.2115973995844884)-- (1.4570989676071504,0.8505861557519836);
\draw (1.4570989676071504,0.8505861557519836)-- (3.2214829278319157,-0.89772045909874);
\begin{scriptsize}
\draw[  ] (1.9766128546624109,1.1386792804905388) node {$17$};
\draw[  ] (3.2298930265896697,-1) node {$26\dg$};
\end{scriptsize}
\end{tikzpicture}
\begin{sol}
		\begin{tikzpicture}[line cap=round,line join=round, x=1.0cm,y=1.0cm, scale=2]
	\draw[  fill=black,fill opacity=0.1] (2.377756723469079,1.1791065917858468) -- (2.410247531267721,1.088053156708355) -- (2.5013009663452124,1.1205439645069966) -- (2.468810158546571,1.2115973995844884) -- cycle; 
	\draw [shift={(3.2214829278319157,-0.89772045909874)},  fill=black,fill opacity=0.1] (0,0) -- (109.63804237930775:0.13672147330115553) arc (109.63804237930775:135.2622370470228:0.13672147330115553) -- cycle;
	\draw (3.2214829278319157,-0.89772045909874)-- (2.468810158546571,1.2115973995844884);
	\draw (2.468810158546571,1.2115973995844884)-- (1.4570989676071504,0.8505861557519836);
	\draw (1.4570989676071504,0.8505861557519836)-- (3.2214829278319157,-0.89772045909874);
	\begin{scriptsize}
	\draw[  ] (1.9766128546624109,1.1386792804905388) node {$17$};
	\draw[  ] (3.2298930265896697,-1) node {$26\dg$};
		
	\draw (3.3, 0.7) node {$\frac{17}{\tan(26\dg)} \approx 34.9$};
	\draw (1.5, 0.0) node {$\frac{17}{\sin(57\dg)} \approx 38.8$};
	\end{scriptsize}
	\end{tikzpicture}
\end{sol}
\end{ex}


\begin{ex}
	\begin{tikzpicture}[line cap=round,line join=round, x=1.0cm,y=1.0cm, scale=2]
	\draw[  fill=black,fill opacity=0.1] (2.526641521156807,1.0966486418036485) -- (2.641590278937647,1.1544800044138848) -- (2.583758916327411,1.2694287621947247) -- (2.468810158546571,1.2115973995844884) -- cycle; 
	\draw [shift={(3.7780733724272118,1.8702950413505262)},  fill=black,fill opacity=0.1] (0,0) -- (-153.292775964565:0.18197628096383797) arc (-153.292775964565:-88.29082532240012:0.18197628096383797) -- cycle;
	\draw (3.7780733724272118,1.8702950413505262)-- (2.468810158546571,1.2115973995844884);
	\draw (2.468810158546571,1.2115973995844884)-- (3.8815173770556983,-1.5963762075756252);
	\draw (3.8815173770556983,-1.5963762075756252)-- (3.7780733724272118,1.8702950413505262);
	\begin{scriptsize}
	\draw[  ] (3,1.7) node {$3.5$};
	\draw[  ] (3.6,1.6) node {$65\dg$};
	\end{scriptsize}
	\end{tikzpicture}
	\begin{sol}
		\begin{tikzpicture}[line cap=round,line join=round, x=1.0cm,y=1.0cm, scale=2]
		\draw[  fill=black,fill opacity=0.1] (2.526641521156807,1.0966486418036485) -- (2.641590278937647,1.1544800044138848) -- (2.583758916327411,1.2694287621947247) -- (2.468810158546571,1.2115973995844884) -- cycle; 
		\draw [shift={(3.7780733724272118,1.8702950413505262)},  fill=black,fill opacity=0.1] (0,0) -- (-153.292775964565:0.18197628096383797) arc (-153.292775964565:-88.29082532240012:0.18197628096383797) -- cycle;
		\draw (3.7780733724272118,1.8702950413505262)-- (2.468810158546571,1.2115973995844884);
		\draw (2.468810158546571,1.2115973995844884)-- (3.8815173770556983,-1.5963762075756252);
		\draw (3.8815173770556983,-1.5963762075756252)-- (3.7780733724272118,1.8702950413505262);
		\begin{scriptsize}
		\draw[  ] (3,1.7) node {$3.5$};
		\draw[  ] (3.6,1.6) node {$65\dg$};
		
			\draw (4.4, 0.7) node {$\frac{3.5}{\cos(65\dg)} \approx 8.28$};
		\draw (2.6, -0.5) node {$3.5\tan(65\dg) \approx 7.51$};
		\end{scriptsize}
		\end{tikzpicture}
	\end{sol}
\end{ex}


\begin{ex}
	\begin{tikzpicture}[line cap=round,line join=round, x=1.0cm,y=1.0cm, scale=2.5]
	\draw[  fill=black,fill opacity=0.1] (4.5491175858654795,0.9845725019196684) -- (4.628519751424598,0.8986694636375165) -- (4.71442278970675,0.978071629196635) -- (4.635020624147631,1.0639746674787869) -- cycle; 
	\draw [shift={(3.6896696652326786,0.19016490647022394)},  fill=black,fill opacity=0.1] (0,0) -- (-8.277438555031324:0.16543298269439807) arc (-8.277438555031324:42.747918874248114:0.16543298269439807) -- cycle;
	\draw (5.715062810577266,-0.1044935302572777)-- (4.635020624147631,1.0639746674787869);
	\draw (4.635020624147631,1.0639746674787869)-- (3.6896696652326786,0.19016490647022394);
	\draw (3.6896696652326786,0.19016490647022394)-- (5.715062810577266,-0.1044935302572777);
	\begin{scriptsize}
	\draw[  ] (4,0.65) node {$160$};
	\draw[  ] (4,0.2401184136606855) node {$51\dg$};
	\end{scriptsize}
	\end{tikzpicture} 
	\begin{sol}
		\begin{tikzpicture}[line cap=round,line join=round, x=1.0cm,y=1.0cm, scale=2.5]
		\draw[  fill=black,fill opacity=0.1] (4.5491175858654795,0.9845725019196684) -- (4.628519751424598,0.8986694636375165) -- (4.71442278970675,0.978071629196635) -- (4.635020624147631,1.0639746674787869) -- cycle; 
		\draw [shift={(3.6896696652326786,0.19016490647022394)},  fill=black,fill opacity=0.1] (0,0) -- (-8.277438555031324:0.16543298269439807) arc (-8.277438555031324:42.747918874248114:0.16543298269439807) -- cycle;
		\draw (5.715062810577266,-0.1044935302572777)-- (4.635020624147631,1.0639746674787869);
		\draw (4.635020624147631,1.0639746674787869)-- (3.6896696652326786,0.19016490647022394);
		\draw (3.6896696652326786,0.19016490647022394)-- (5.715062810577266,-0.1044935302572777);
		\begin{scriptsize}
		\draw[  ] (4,0.65) node {$160$};
		\draw[  ] (4,0.2401184136606855) node {$51\dg$};
		
		\draw (4.5, -0.2) node {$\frac{160}{\cos(51\dg)} \approx 254$};
		\draw (5.5, 0.9) node {$160\tan(51\dg) \approx 198$};
		\end{scriptsize}
		\end{tikzpicture} 
	\end{sol}
\end{ex}
\end{multicols}

\begin{ex}
	Silvia is measuring trees again. This time she is using a method common in forestry before laser rangefinders became accessible: She has a length of rope 66 feet long (1 ``chain'') and uses this to place herself 66 feet away from the tree horizontally. Then standing in place she takes a clinometer reading of the angle of inclination to the treetop from eye-level (not ground-level as we assumed previously). She also takes a reading of the angle of declination to the tree base. Draw a diagram then calculate the height of the tree if the angles of inclination and declination are $25\dg$ and $5\dg$, respectively.
	\begin{sol}
		$66\tan(25\dg)+66\tan(5\dg) \approx 36.6$ ft	
		\end{sol}
\end{ex}

\begin{ex} Fill-in the table with exact values using the common right triangles.
	
	
	{\Large
	\begin{tabular}{|c |c | c | c | c |}
		\hline
		$\theta$ & $\theta $& $\sin(\theta)$ & $\cos(\theta)$ & $\tan(\theta)$ \\
		\hline 
	  & & & &	\\
		$\dfrac{\pi}{6}$ & $\ \ 30\dg\ \ $ &$\ \ \ \dfrac{1}{2}\ \ \ $ & $\ \ \ \ \ \ \ \ $&  $\ \ \ \ \ \ \ \ $\\
	  & & & &	\\
		\hline 
			  & & & &	\\
$\dfrac{\pi}{4}$ &  & & & \\
	  & & & &	\\
		\hline 
			  & & & &	\\
$\dfrac{\pi}{3}$ &  & & & \\
	  & & & &	\\
		\hline
	\end{tabular} }
\begin{sol}
	{\small
\begin{tabular}{|c |c | c | c | c |}
	\hline
	$\theta$ & $\theta $& $\sin(\theta)$ & $\cos(\theta)$ & $\tan(\theta)$ \\
	\hline 
	& & & &	\\
	$\dfrac{\pi}{6}$ & $30\dg$ &$\dfrac{1}{2}$ & $\dfrac{\sqrt{3}}{2}$&  $\dfrac{1}{\sqrt{3}}$\\
	& & & &	\\
	\hline 
	& & & &	\\
	$\dfrac{\pi}{4}$ & $45\dg$ &$\dfrac{1}{\sqrt{2}}$ &$\dfrac{1}{\sqrt{2}}$ &  $1$\\
	& & & &	\\
	\hline 
	& & & &	\\
	$\dfrac{\pi}{3}$ & $60\dg$ &$\dfrac{\sqrt{3}}{2}$ &$\dfrac{1}{2}$ &  $\sqrt{3}$\\
	& & & &	\\
	\hline
\end{tabular}	
}
\end{sol}
\end{ex}

\begin{ex} \label{jillnav}
	Jill flies her Piper J-3 Cub at $122$ km/h for three hours along a bearing of $198\dg$. Is she north or south of her original position? How far? Is she east or west of her original position? How far? Round to the nearest kilometer. [Note: bearings are angles measured clockwise from north.]
	\begin{sol}
		She is about 348 km south and 113 km west of her initial position.
	\end{sol}
\end{ex}

\begin{ex}
	After the three hours described in \#\ref{jillnav}, Jill turns $100\dg$ to the right and reduces her speed to 100 km/h. After another eight hours, where is she with respect to her original starting point from \#\ref{jillnav}?
	\begin{sol}
		She is about 27 km north and 819 km west of her initial position.
	\end{sol}
\end{ex}

\begin{ex}
	Visually estimate $\sin(\alpha)$, $\cos(\alpha)$, and $\tan(\alpha)$.
	
	\begin{tikzpicture}[scale=0.6]
		\draw (0,0) -- (4,0) -- (4,9) -- cycle;
		\draw (1.2,0.15) node [above left] {$\alpha$};
	\end{tikzpicture}
	\begin{sol}
		$\sin(\alpha) \approx 0.9$, $\cos(\alpha) \approx 0.4$, $\tan(\alpha) \approx 2.2$
	\end{sol}
	
\end{ex}


\begin{ex}
	Let $\theta$ be an acute angle. What's greater $\sin(\theta)$ or $\tan(\theta)$? Or is it not possible to know with the given information? Explain.
	\begin{sol}
		Both are ratios of side lengths of a right triangle where the length of the side opposite $\theta$ is the numerator. The $\tan(\theta)$ has the smaller denominator though -- the length of the adjacent side instead of the hypotenuse -- so it is larger.
	\end{sol}
\end{ex}

\begin{ex}
	Find the indicated lengths of these non-right triangles accurate to three significant figures. Do not use the Law of Sines or Law of Cosines (these don't exist yet for us -- later we'll see that they're perfect for these triangles.)
	
		\vspace{1cm}
	\begin{tikzpicture}[line cap=round,line join=round,x=1.0cm,y=1.0cm, scale=3]
	\draw [shift={(3.4516233546070385,-0.8682825703917814)},color=black,fill=black,fill opacity=0.1] (0,0) -- (1.240869775130875:0.18197628096383797) arc (1.240869775130875:44.32004597394672:0.18197628096383797) -- cycle;
	\draw [shift={(6.5066191016969235,-0.8021093773140221)},color=black,fill=black,fill opacity=0.1] (0,0) -- (151.24002055130532:0.18197628096383797) arc (151.24002055130532:181.24086977513082:0.18197628096383797) -- cycle;
	\draw (3.4516233546070385,-0.8682825703917814)-- (4.594213821749683,0.24750547486314736);
	\draw (4.594213821749683,0.24750547486314736)-- (6.5066191016969235,-0.8021093773140221);
	\draw (6.5066191016969235,-0.8021093773140221)-- (3.4516233546070385,-0.8682825703917814);
	\begin{scriptsize}
	\draw[color=black] (3.9,-0.25) node {$16$};
	\draw[color=black] (5.7,-0.2) node {$\ell$?};
	\draw[color=black] (3.72,-0.75) node {$43\dg$};
	\draw[color=black] (6.25,-0.753364070437959) node {$31\dg$};
	\end{scriptsize}
	\end{tikzpicture}

	\vspace{1cm}
	
	\begin{tikzpicture}[line cap=round,line join=round,x=1.0cm,y=1.0cm,scale=3]
	\draw [shift={(4.654872582070963,-2.5124681197550554)},color=black,fill=black,fill opacity=0.1] (0,0) -- (7.108729885578895:0.16543298269439824) arc (7.108729885578895:126.19744286066577:0.16543298269439824) -- cycle;
	\draw (3.4516233546070385,-0.8682825703917814)-- (7.281396903982357,-2.1849108140201476);
	\draw (7.281396903982357,-2.1849108140201476)-- (4.654872582070963,-2.5124681197550554);
	\draw (4.654872582070963,-2.5124681197550554)-- (3.4516233546070385,-0.8682825703917814);
	\begin{scriptsize}
	\draw[color=black] (5.7,-1.5) node {$m$?};
	\draw[color=black] (5.9,-2.45) node {$5.2$};
	\draw[color=black] (3.9,-1.5910064061472604) node {$4$};
	\draw[color=black] (4.772329999783987,-2.25) node {$119\dg$};
	\end{scriptsize}
	\end{tikzpicture}
		\vspace{1cm}
		
		\begin{sol}
			$\ell \approx 21.2$, $m \approx 7.95$
		\end{sol}
\end{ex}

\Closesolutionfile{ans}
\subsection*{Answers \nopunct} \hfill
\begin{multicols}{2}
	\input{ans27}
\end{multicols}

\newpage
\section{Reversing trigonometry, basic trig identities}

The Pythagorean Identity\index{Pythagorean identity}: \fbox{$\cos^2(\theta) + \sin^2(\theta) = 1$} for all $\theta$.

\bigskip
Arcsine, Arccosine, Arctangent:\index{arcsin}\index{arccos}\index{arctan}\index{inverse trig}

\begin{align*}
\text{Given } 0 < \theta < 90\dg,\\
\sin(\theta) = R & \iff \arcsin(R) = \theta,\\
\cos(\theta) = R & \iff \arccos(R) = \theta,\\
\tan(\theta) = R & \iff \arctan(R) = \theta.
\end{align*}


\Opensolutionfile{ans}[ans28]
\subsection*{Exercises \nopunct} \hfill
\begin{ex} \label{sincos}
	If $\cos(\theta) = \dfrac{5}{13}$, what is $\sin(\theta)$?
	\begin{sol}
		$\frac{12}{13}$
	\end{sol}
\end{ex}
\begin{ex} \label{sincos}
	If $\tan(\theta) = 0.3$, what is $\cos(\theta)$?
	\begin{sol}
		$\frac{1}{\sqrt{1.09}} \approx 0.958$
	\end{sol}
\end{ex}

\begin{ex} \label{sincos2}
	Let $\theta$ be acute. Write $\cos(\theta)$ in terms of the sine function in two different ways.
	\begin{sol}
		$\cos(\theta) = \sin(90\dg-\theta)$,
		
		 $\cos(\theta) = \sqrt{1-\sin^2(\theta)}$
	\end{sol}
\end{ex}

\begin{ex}
	Repeat \#\ref{sincos2} swapping $\sin$ and $\cos$ in the instructions.
		\begin{sol}
		$\sin(\theta) = \cos(90\dg-\theta)$,
		
		$\sin(\theta) = \sqrt{1-\cos^2(\theta)}$
		
		Note that $\sin$ and $\cos$ are simply swapped in the solution as well.
	\end{sol}
\end{ex}

\begin{ex}
	If you solved \#\ref{sincos} by the Pythagorean Identity, solve it again by drawing a diagram of such a triangle. If you drew a diagram, solve it again by the Pythagorean Identity. [If you did something else, discuss it with your instructor.]
\end{ex}

Find the indicated quantities.
\begin{multicols}{2}
	\begin{ex}
		\begin{tikzpicture} [scale=0.07]
		\draw (0,0) -- (55,0) -- (55,25) -- (0,0);
		\draw (10,2) node {$\theta$};
		\draw (53,0) rectangle (55,2);
		\draw (59,10) node {$25$};
		\draw (27,-3) node {$55$};
		\end{tikzpicture}
		\begin{sol}
			$\theta = \arctan\left(\frac{25}{55}\right) \approx 24.44\dg$
			\end{sol}
	\end{ex}

	\begin{ex}
	\begin{tikzpicture} [scale=0.07]
	\draw (0,0) -- (75,0) -- (75,33) -- (0,0);
	\draw (71,27) node {$\theta$};
	\draw (73,0) rectangle (75,2);
	\draw (30,20) node {$82$};
	\draw (36,-3.5) node {$75$};
	\end{tikzpicture}
	\begin{sol}
		$\theta = \arcsin\left(\frac{75}{82}\right) \approx 66.15\dg$
	\end{sol}
\end{ex}

\begin{ex}
\begin{tikzpicture} [scale=0.01]
\draw (0,0) -- (210,0) -- (0,-498) -- cycle;
\draw (180,-25) node {$\theta$};
\draw (0,0) rectangle (15,-15);
\draw (105,20) node {$210$};
\draw (180,-200) node {$540$};
\end{tikzpicture}
\begin{sol}
	$\theta = \arccos\left(\frac{210}{540}\right) \approx 67.11\dg$
\end{sol}
\end{ex}

\begin{ex}
	\begin{tikzpicture} [scale=0.07]
	\draw (0,0) -- (62,0) -- (62,59) -- (0,0);
	\draw (10,4) node {$\alpha$};
	\draw (58,50) node {$\beta$};
	\draw (27,33) node {$\ell$};
	\draw (60,0) rectangle (62,2);
	\draw (66,27) node {$59$};
	\draw (33,-3) node {$62$};
	\end{tikzpicture}
	\begin{sol}
		$\ell = \sqrt{7325} \approx 85.6$
		
		$\alpha = \arctan\left( \frac{59}{62} \right) \approx 43.6\dg$
		
		$\beta = \arctan\left( \frac{62}{59} \right) \approx 46.4\dg$
	\end{sol}
\end{ex}


\begin{ex}
	\begin{tikzpicture} [scale=0.06]
	\begin{scope}[xscale=-1, rotate=15]
	\draw (0,0) -- (101,0) -- (101,74) -- (0,0);
	\draw (12,4) node {$\beta$};
	\draw (95,64) node {$\alpha$};
	\draw (106,40) node {$h$};
	\draw (98,0) rectangle (101,3);
	\draw (50,50) node {$125$};
	\draw (50,-5) node {$101$};
	\end{scope}
	\end{tikzpicture}
	\begin{sol}
		$h = \sqrt{5424} \approx 73.65$
		
		$\alpha = \arcsin\left( \frac{101}{125} \right) \approx 53.9\dg$
		
		$\beta = \arccos\left( \frac{101}{125} \right) \approx 36.1\dg$
	\end{sol}
\end{ex}
\end{multicols}

\begin{ex}
	Rework \#\ref{camryangle} from Section \ref{simfigsection} for the exact angle. How far is the approximation from the real angle in degrees? What is the error of the approximation if Professor upgrades to 15 inch ramps?
	\begin{sol}
		The angle is $\arcsin\left( \frac{6.5}{105.2} \right)  = 3.5423\cdots\dg$. The approximation gave $3.5401\cdots\dg$, so they are about $0.002\dg$ apart. If Professor upgrades his ramps, the two results differ by about $0.03\dg$. The approximation method is still not bad for this larger angle (about $8.2\dg$), given the application.
	\end{sol}
\end{ex}

\Closesolutionfile{ans}
\subsection*{Answers \nopunct} \hfill
\begin{multicols}{2}
	\input{ans28}
\end{multicols}

\newpage
\chapter{Third Cycle}

\section{Function composition and function arithmetic}

\index{function arithmetic}
These three are defined provided $f(x)$ and $g(x)$ are:
\begin{align*}
(f+g)(x) &:= f(x)+g(x) &\text{[sum of functions]}\\
(f-g)(x) &:= f(x)-g(x) &\text{[difference of functions]}\\
(f \cdot g)(x) &:= f(x)\cdot g(x) &\text{[product of functions]}\\
 & & \\
\left( \frac{f}{g} \right)(x) &:= \frac{f(x)}{g(x)} &\text{[quotient of functions]}\\
 &\text{ provided } g(x)\neq 0. &\\
 & & \\
(f \circ g)(x) &:= f\left( g(x) \right) &\text{[function composition]\index{composition}\index{function composition}}\\
 &\text{ provided } g(x)\text{ is in the domain of } f. &\\
\end{align*}

\index{Calculus}\index{derivative}We build complex functions from simpler ones with these combination methods. In Calculus each of these function-building methods will have a corresponding derivative rule. Derivatives\footnote{Whatever those are. See Section \ref{derivativeSection}.} will then be easy to calculate if we know the derivatives of basic functions and, how complicated functions are built from the simpler ones with these tools. This is the ``calculus of derivatives'' that gave Calculus its name.

\Opensolutionfile{ans}[ans31]
\subsection*{Exercises \nopunct} \hfill
\begin{ex}
	Let $f(x) = 3x+1$ and $g(x) = x^2$ find:
	\begin{multicols}{2}
	\ssp
\item\ \  $(f+g)(2)$
\item\ \  $(f-g)(-1)$
\item\ \  $(f\cdot g)(1)$
\item\ \  $\left( \dfrac{f}{g} \right)(3)$
\item\ \  $(f \circ g)(-2)$
\item\ \  $(g \circ f)(-2)$
\item\ \ $(f \circ f)(1)$
\esp
	\end{multicols}

	\begin{sol}
			\ssp
		\item\ \  $11$
		\item\ \  $-3$
		\item\ \  $4$
		\item\ \  $\frac{10}{9}$
		\item\ \  $13$
		\item\ \  $25$
		\item\ \ $13$
		\esp
	\end{sol}
\end{ex}
\begin{ex}
	Let $f(x) = x^2-x$ and $g(x) = \sqrt{x}$ find:
		\begin{multicols}{2}
	\ssp
	\item\ \  $(f+g)(4)$
	\item\ \  $(g-f)(x)$ and its domain.
	\item\ \  $(f\cdot g)(x)$ and its domain.
	\item\ \  $\left( \dfrac{f}{g} \right)(x)$ and its domain.
	\item\ \  $\left( \dfrac{g}{f} \right)(x)$ and its domain.
	\item\ \  $(f \circ g)(9)$
	\item\ \  $(f \circ g)(x)$ and its domain.
	\item\ \  $(g \circ f)(x)$ and its domain.
	\esp
		\end{multicols}
	\begin{sol}
	\ssp
	\item\ \  $14$
	\item\ \  $\sqrt{x}-x^2+x$, $[0,\infty)$
	\item\ \  $\sqrt{x}(x^2-x)$, $[0,\infty)$
	\item\ \  $\dfrac{x^2-x}{\sqrt{x}}$, $(0,\infty)$
	\item\ \  $\dfrac{\sqrt{x}}{x^2-x}$, $(0,1)\cup(1,\infty)$
	\item\ \  $6$
	\item\ \  $\left( \sqrt{x} \right)^2 - \sqrt{x}$ or $x - \sqrt{x}$. The domain is $[0,\infty)$.
	\item\ \  $\sqrt{x^2-x}$, $(-\infty, 0]\cup [1,\infty)$
\esp
	\end{sol}

\end{ex}

\begin{ex}
	If $f(x) = x^2$ and $g(x) = \sqrt{x-3}$, what is the domain of $(f\circ g)(x)$?
	\begin{sol}
		$[3,\infty)$. For the composition to be defined the inner function, $g(x)$ must be defined even if the expression can be simplified on the domain. For $x$ in $[3,\infty)$, $(f\circ g)(x) = x-3$, but its domain cannot be computed from this expression alone.
	\end{sol}
\end{ex}

Write each function as the sum, difference, product, or quotient of two functions. Neither function can be the zero function.
\begin{multicols}{2}
\begin{ex}
	$h(x) = \sqrt[3]{x+1} +(x-7)^3$
	\begin{sol}
		One possibility is $h(x) = (f+g)(x)$ where $f(x) = \sqrt[3]{x+1}$ and $g(x) = (x-7)^3$.
	\end{sol}
\end{ex}
\begin{ex}
	$k(x) = \dfrac{x^2-0.3x}{x+4}$
	\begin{sol}
		One possibility is $k(x) = \left( \dfrac{f}{g} \right)(x)$ where $f(x) = x^2-0.3x$ and $g(x) = x+4$.
	\end{sol}
\end{ex}
\end{multicols}

Write each function as the composition of two non-identity functions.
\begin{multicols}{2}
	\begin{ex}
		$\ell(x) = (2x+3)^3$
		\begin{sol}
			One possibility is $\ell(x) = (f\circ g)(x)$ where $f(x) = x^3$ and $g(x) = 2x+3$.
		\end{sol}
	\end{ex}
	\begin{ex}
	$H(x) = \sin(7-3x+x^2)$
	\begin{sol}
		One possibility is $H(x) = (f\circ g)(x)$ where $f(x) = \sin(x)$ and $g(x) = 7-3x+x^2$.
	\end{sol}
\end{ex}
	\begin{ex}
	$m(x) = \dfrac{2}{5x+1}$
	\begin{sol}
		One possibility is $m(x) = (f\circ g)(x)$ where $f(x) = \dfrac{2}{x}$ and $g(x) = 5x+1$.
	\end{sol}
	\end{ex}
	\begin{ex}
	$q(x) = \dfrac{|x|+1}{|x|-1}$
	\begin{sol}
		One possibility is $q(x) = (f\circ g)(x)$ where $f(x) = \dfrac{x+1}{x-1}$ and $g(x) = |x|$.
	\end{sol}
\end{ex}
	\begin{ex}
	$w(x) = \dfrac{x^2}{x^4+1}$
	\begin{sol}
		One possibility is $w(x) = (f\circ g)(x)$ where $f(x) = \dfrac{x}{x^2+1}$ and $g(x) = x^2$.
	\end{sol}
\end{ex}
	\begin{ex}
	$n(x) = x^2 - 6x-1$
	\begin{sol}
		One possibility is $n(x) = (f\circ g)(x)$ where $f(x) = x^2-10$ and $g(x) = x-3$.
	\end{sol}
\end{ex}
\end{multicols}

The \textit{difference quotient}\index{difference quotient} of the function $f$ is defined to be $\dfrac{f(x+h) - f(x)}{h}$. Find and simplify the difference quotient of the following functions, assuming $h\neq0$.
\begin{multicols}{2}
\begin{ex} \label{diffquotquest}
	$f(x) = 3x-7$
	\begin{sol}
		$3$
	\end{sol}
\end{ex}
\begin{ex}
	$f(x) = -6x+2$
	\begin{sol}
		$-6$
	\end{sol}
\end{ex}
\begin{ex}
	$f(x) = 4$
	\begin{sol}
		$0$
	\end{sol}
\end{ex}
\begin{ex}
	$f(x) = x^2$
	\begin{sol}
		$2x+h$
	\end{sol}
\end{ex}
\begin{ex}
	$f(x) = 2x^2-3x$
	\begin{sol}
		$4x-3+2h$
	\end{sol}
\end{ex}
\begin{ex}
	$f(x) = x^3+3x-7$
	\begin{sol}
		$3x^2+3 + 3hx+h^2$
	\end{sol}
\end{ex}
\begin{ex}
	$f(x) = \dfrac{1}{x}$
	
	[Hint: combine fractions.]
	\begin{sol}
		$-\dfrac{1}{x(x+h)}$
	\end{sol}
\end{ex}
\begin{ex}
	$f(x) = \sqrt{x}$
	\begin{sol}
		$\frac{1}{\sqrt{x+h}+\sqrt{x}}$
	\end{sol}
\end{ex}
\end{multicols}

\begin{ex}
	Any thoughts about the purpose of the difference quotient?
\end{ex}

\begin{ex}
	The volume $V$ of a cube is a function of its side length $x$ where $x$ is measured in inches. Write out this function. Now, let's assume $x=t+1$ is also a function of time $t$ where $t$ is measured in minutes. Find a formula for $V$ as a function of $t$. Which operation was used to find this function?
	\begin{sol}
		$V(x) =x^3$. $V(t) = (V\circ x)(t) = V(x(t)) = (t+1)^3$. Function composition.
	\end{sol}
\end{ex}

\begin{ex}
	Maria will sell copies of her book, \textit{Abandoned Homesteads of the Virginia Blue Ridge}, for \$10 each. Due to high demand she will be able to sell all her copies.
	\ssp
	\item Write Maria's revenue, $R$, as a function of the number of books sold, $x$.
	\item She will decide beforehand how many to have printed. If it costs her \$20 plus \$$9.80$ per book to have them printed, what is her cost, $C$, as a function of the number of books printed (also $x$)?
	\item Maria's profit, $P$, is her revenue minus her cost. What does this mean in terms of function arithmetic? What is Maria's profit in terms of the number of books printed?
	\item How many books should Maria print if she only wants to break even, i.e. to have zero profit?
	\esp
	\begin{sol}
		\ssp
		\item $R(x) = 10x$
		\item $C(x) = 20+9.8x$
		\item This means $P(x) = (R-C)(x)$ or $P(x) = R(x)-C(x)$. Her profit is $P(x) = 0.2x-20$.
		\item 100 books.
		\esp
	\end{sol}
\end{ex}


\Closesolutionfile{ans}
\subsection*{Answers \nopunct} \hfill
\begin{multicols}{2}
	\input{ans31}
\end{multicols}






\newpage
\section{Transformations from a function perspective}

\label{transformationsthm}\index{transformations of function graphs}\index{transformation of graphs of functions}\index{transformations}\textbf{Transformations.}  Suppose $f$ is a function.  If $A \neq 0$ and $B \neq 0$, then to graph \[g(x) = A f(Bx+H)+K\] 
	
	\begin{enumerate}
		
		\item  Subtract $H$ from each of the $x$-coordinates of the points on the graph of $f$.  This results in a horizontal shift to the left if $H > 0$ or right if $H< 0$.
		
		\item  Divide the $x$-coordinates of the points on the graph obtained in Step 1 by $B$.  This results in a horizontal scaling, but may also include a reflection about the $y$-axis if $B < 0$.
		
		\item  Multiply the $y$-coordinates of the points on the graph obtained in Step 2 by $A$.   This results in a vertical scaling, but may also include a reflection about the $x$-axis if $A < 0$.
		
		\item  Add $K$ to each of the $y$-coordinates of the points on the graph obtained in Step 3.  This results in a vertical shift up if $K > 0$ or down if $K< 0$.
		
	\end{enumerate}


\bigskip
%\textbf{Example.} Graph $f(x) = \sqrt{3-2x} - 1 $ by a series of transformations.


\Opensolutionfile{ans}[ans32]
\subsection*{Exercises \nopunct} \hfill

	Graph each function by starting with the parent function then graphing a series of transformations. Do not combine transformations except reflections and scaling (if desired). For each intermediate graph, describe it in words and give the formula for the intermediate function. Track the coordinates of at least three points and all asymptotes through the transformations. Check answers with a graphing utility.
	\begin{multicols}{2}

	\begin{ex}
	$f(x) = \left(\frac{1}{2}x-5\right)^3$
	\end{ex}
	\begin{ex}
	$g(x) = \sqrt{5-x}$
\end{ex}
	\begin{ex}
	$g(x) = 4\sqrt{x}-8$
\end{ex}
	\begin{ex}
	$h(x) = -\sqrt{x+3}$
\end{ex}
	\begin{ex}
	$k(x) = 2^{-x/3}+2$
\end{ex}
	\begin{ex}
	$\ell(x) = \frac{1}{3}\ln(2x+10)$
\end{ex}
	\begin{ex}
	$m(x) = \dfrac{2}{x+3}-7$
\end{ex}
	\begin{ex}
	$n(x) = \dfrac{1}{4x^2-16x+16}$
\end{ex}
	\begin{ex}
	$a(x) = 2|x+7|+3$
\end{ex}
	\begin{ex}
	$c(x) = 3\sqrt{1-\dfrac{x^2}{4}}$
	
	[Hint: Start with $y=\sqrt{1-x^2}$. What is the shape of the graph?] 
\end{ex}
\begin{ex}
$r(x) = 3\sqrt[3]{2x+5}-7$
\end{ex}
	\end{multicols}

\begin{comment} % Catenary example with figure

\begin{ex}
	Below is a graph of the function $f(x)=\cosh(x)$. This function is called the hyperbolic cosine and one application is the modeling the shape of a suspended cable or chain, hanging under its own weight.
	
	\begin{tikzpicture}
		\drawgridxxyyb{-2.1}{2.1}{-0.3}{4}
		\draw[<->, thick, samples=10, domain=-2:2, smooth] plot (\x, {(exp(\x)+exp(-\x))/2}) node [above] {\small $f(x)=\cosh(x)$};
		\draw[fill] (0,1) circle [radius=0.07] node [below right] {\small $(0,1)$};
		\draw[fill] (.693147,1.25) circle [radius=0.07] node [below right] {\small $\ \ \left(\ln(2),\dfrac{5}{4}\right)$};
		\draw[fill] (-.693147,1.25) circle [radius=0.07] node [below left] {\small $\left(-\ln(2),\dfrac{5}{4}\right)$};
	\end{tikzpicture}
	\begin{tikzpicture}[scale=1.25]
\drawgridxxyyb{-3}{3}{-0.1}{2}
\draw[<->, thick, samples=10, domain=-1:1, smooth] plot (3* \x, {(exp(\x)+exp(-\x))/2}) node [above] {\small $g(x)=$?};
%\draw[fill] (0,1) circle [radius=0.07];
%\draw[fill] (2.0794,1.25) circle [radius=0.056];
%\draw[fill] (-2.0794,1.25) circle [radius=0.056];
%\draw[dashed, <->, very thick] (0.1,0) -- (0.1,1);
\draw[dashed, <->, very thick] (2.0794,0) -- (2.0794,1.25);
\draw[dashed, <->, very thick] (-2.0794,0) -- (-2.0794,1.25);
\draw[dashed, <->, very thick] (-2.0794,-0.3) -- (2.0794,-0.3);
\draw (0,-0.3) node [below] {$w$};
\draw (2.0794,0.6) node [right] {$h$};
\draw (-2.0794,0.6) node [left] {$h$};
%\draw (0.1,0.5) node [right] {$\ell$};
\end{tikzpicture}
\end{ex}
\ssp
\item If the graph of $g(x)$ in the diagram is a transformed version of the graph of $f(x)=\cosh(x)$, what is $g(x)$?
\esp
\end{comment}

\Closesolutionfile{ans}
%\subsection*{Answers \nopunct} \hfill
\begin{multicols}{2}
	\input{ans32}
\end{multicols}


\newpage
\section{Conics, Part I} \label{Parabolas}
The usual conics (conic sections\index{conics}\index{conic sections}):
\begin{itemize}
	\item circle
	\item ellipse
	\item parabola
	\item hyperbola
\end{itemize}

The degenerate conics\index{degenerate conics}:
\begin{itemize}
	\item single point
	\item line
	\item crossing lines
\end{itemize}

General Form: $Ax^2 + By^2+ \underline{Cxy} +Dx +Ey + F = 0$.

\qi{We will focus on the case that $C=0$. If not, the conic has an additional rotation. We know at least one, though: the hyperbola $xy-1=0$.}

We have already studied circles. Recall: \index{circle equation}the circle centered at $(h,k)$ with radius $r$ has equation \fbox{$(x-h)^2+(y-k)^2 = r^2$}.

\bigskip
\index{parabola (geometric)}\index{focus}\index{directrix} \index{latus rectum}\index{focal diameter}\textbf{Parabola}: set of points equidistant from focus and directrix.

\begin{tikzpicture} []
\drawgridxxyyb{-4}{4}{-1.5}{4};
\draw[<->, samples=10, domain=-4:4, smooth] plot (\x, {0.25*\x*\x}) node [left] {$x^2 = 4py$, ($p>0$)\ };
\draw[<->, dashed] (-4.1, -1) -- (4.1, -1);
\draw (0,1) node {$\times$} node[above left] {\small focus};
\draw[<->]   (-0.8, 0.02) -- (-0.8,0.98) node[below left] {\small $p$};
\draw[<->]   (-0.8, -0.98) -- (-0.8,-0.02) node[below left] {\small $p$};
\draw[|-|, dotted ]   (-2, 1) -- (2,1) node[above left] {\small $4p$};
\draw[<->, dashed] (-4.1, -1) -- (4.1, -1) node[above left] {\small directrix};
\draw[] (0,1) node {$\times$};
\draw[fill] (0,0) circle[radius=0.06] node[below right] {vertex};
\end{tikzpicture}


\textit{focal length}, $|p|$: the distance from the vertex to the focus. (Consequently also the distance from the vertex to the directrix.)

\textit{focal diameter}, $|4p|$: the distance across the parabola measured parallel to the directrix through the focus. The corresponding line segment is called the \textit{latus rectum}\index{latus rectum}.

\qi{This is an alternative to the ``fat factor'' method of determining width.}

\index{parabola}The parabola with vertex at the origin opening up or down with \index{focal length}focal length $p$ has equation \fbox{$x^2=4py$}.

The parabola with vertex at the origin opening left or right with focal length $p$ has equation \fbox{$y^2=4px$}.

\qi{For vertex at $(h,k)$ just change $x\mapsto(x-h)$, $y\mapsto(y-k)$.}
\qi{Complete the square if necessary to find out where the vertex (as in Section \ref{quadraticgraphs}.)}

\Opensolutionfile{ans}[ans33]
\subsection*{Exercises \nopunct} \hfill

Sketch the graph of the given parabola.  Find the vertex, focus and directrix.  Include the endpoints of the latus rectum in your sketch.
\begin{multicols}{2}
\begin{ex}
	$(x - 3)^{2} = -16y$
	\begin{sol}
		Check sketches for this section on a graphing utility. [Although your sketch is likely to be correct if you've found all of the following things correctly!]
		
		{\small $(x - 3)^{2} = -16y$}\\
		{\small Vertex $(3, 0)$}\\
		{\small Focus $(3, -4)$}\\
		{\small Directrix $y = 4$}\\
		{\small Endpoints of latus rectum $(-5, -4)$, $(11, -4)$}\\
	\end{sol}
\end{ex}
\begin{ex}
	$(x-1)^2 = 4(y+3)$
	\begin{sol}
		{\small $(x-1)^2 = 4(y+3)$}\\
		{\small Vertex $\left(1, -3\right)$}\\
		{\small Focus $\left(1, -2 \right)$}\\
		{\small Directrix $y = -4$}\\
		{\small Endpoints of latus rectum $\left(3, -2 \right)$, $\left(-1, -2 \right)$}\\
	\end{sol}
\end{ex}
\begin{comment}
\begin{ex}
	$\left(x + \frac{7}{3}\right)^{2} = 2\left(y + \frac{5}{2}\right)$
	\begin{sol}
		{\small $\left(x + \frac{7}{3}\right)^{2} = 2\left(y + \frac{5}{2}\right)$}\\
		{\small Vertex $\left(-\frac{7}{3}, -\frac{5}{2} \right)$}\\
		{\small Focus $\left(-\frac{7}{3}, -2 \right)$}\\
		{\small Directrix $y = -3$}\\
		{\small Endpoints of latus rectum $\left(-\frac{10}{3}, -2 \right)$, $\left(-\frac{4}{3}, -2 \right)$}\\
	\end{sol}
\end{ex}
\end{comment}
\begin{ex}
	$(y + 4)^{2} = 4x$
	\begin{sol}
		{\small $(y + 4)^{2} = 4x$}\\
		{\small Vertex $(0,-4)$} \\
		{\small Focus $(1,-4)$} \\
		{\small Directrix $x = -1$}\\
		{\small Endpoints of latus rectum $(1, -2)$, $(1, -6)$}\\
	\end{sol}
\end{ex}
\begin{ex}
	$(y - 2)^{2} = -12(x + 3)$ 
	\begin{sol}
	{\small $(y - 2)^{2} = -12(x + 3)$} \\
	{\small Vertex $(-3, 2)$} \\
	{\small Focus $(-6, 2)$} \\
	{\small Directrix $x = 0$}\\
	{\small Endpoints of latus rectum $(-6, 8)$, $(-6, -4)$}\\	
	\end{sol}
\end{ex}


\begin{ex}
	$(x+2)^2 = -20(y-5)$
	\begin{sol}
	{\small $(x+2)^2 = -20(y-5)$}\\
	{\small Vertex $\left(-2, 5\right)$}\\
	{\small Focus $\left(-2, 0 \right)$}\\
	{\small Directrix $y = 10$}\\
	{\small Endpoints of latus rectum $\left(-12, 0 \right)$, $\left(8, 0 \right)$}\\	
	\end{sol}
\end{ex}
\begin{ex}
	$(y-4)^2 = 18(x-2)$
	\begin{sol}
	{\small $(y-4)^2 = 18(x-2)$}\\
	{\small Vertex $\left(2, 4\right)$}\\
	{\small Focus $\left( \frac{13}{2}, 4 \right)$}\\
	{\small Directrix $x = -\frac{5}{2}$}\\
	{\small Endpoints of latus rectum $\left(\frac{13}{2}, -5 \right)$, $\left(\frac{13}{2}, 13 \right)$}\\	
	\end{sol}
\end{ex}
\end{multicols}

Change the form of the equation in order to determine the vertex, focus, and directrix.

\begin{multicols}{2}
\begin{ex}
	$y^{2} - 10y - 27x + 133 = 0$
	\begin{sol}
		$(y - 5)^{2} = 27(x - 4)$\\
		Vertex $(4, 5)$\\
		Focus $\left( \frac{43}{4}, 5 \right)$\\
		Directrix $x = -\frac{11}{4}$
	\end{sol}
\end{ex}
\begin{ex}
	$x^2 + 2x - 8y + 49 = 0$
	\begin{sol}
		$(x+1)^2=8(y-6)$ \\
		Vertex $(-1,6)$\\
		Focus $(-1,8)$ \\
		Directrix $y=4$
	\end{sol}
\end{ex}
\begin{ex}
	$2y^2 + 4y +x - 8 = 0$
	\begin{sol}
		$(y+1)^2=-\frac{1}{2}(x-10)$\\
		Vertex $(10,-1)$\\
		Focus $\left(\frac{79}{8}, -1 \right)$\\
		Directrix $x = \frac{81}{8}$
	\end{sol}
\end{ex}
\begin{ex}
	$x^2-10x+12y+1=0$
	\begin{sol}
		$(x-5)^2 = -12(y-2)$\\
		Vertex $(5,2)$\\
		Focus $(5,-1)$ \\
		Directrix $y=5$
	\end{sol}
\end{ex}
\end{multicols}

Find an equation for a parabola which fits the given criteria.

\begin{multicols}{2}
	\begin{ex}
		Vertex $(7, 0)$, focus $(0, 0)$ 
		\begin{sol}
			$y^{2} = -28(x - 7)$
		\end{sol}
	\end{ex}
	\begin{ex}
	Focus $(10, 1)$, directrix $x = 5$
	\begin{sol}
		$(y - 1)^{2} = 10\left(x - \frac{15}{2} \right)$
	\end{sol}
\end{ex}

	\begin{ex}
	The endpoints of latus rectum are $(-2, -7)$ and $(4, -7)$.
	\begin{sol}
		 $(x - 1)^{2} = 6\left(y + \frac{17}{2}\right)$ or\\
		$(x - 1)^{2} = -6\left(y + \frac{11}{2}\right)$
	\end{sol}
\end{ex}
\end{multicols}

\begin{ex}
	The mirror in Carl's flashlight is a paraboloid of revolution.  If the mirror is 5 centimeters in diameter and 2.5 centimeters deep, where should the light bulb be placed so it is at the focus of the mirror?
	\begin{sol}
		The bulb should be placed $0.625$ centimeters above the vertex of the mirror.
	\end{sol}
\end{ex}
\begin{ex}
	A parabolic Wi-Fi antenna is constructed by taking a flat sheet of metal and bending it into a parabolic shape.\footnote{This shape is called a `parabolic cylinder.'}  If the cross section of the antenna is a parabola which is 45 centimeters wide and 25 centimeters deep, where should the receiver be placed to maximize reception?
	\begin{sol}
		The receiver should be placed $5.0625$ centimeters from the vertex of the cross section of the antenna.
	\end{sol}
\end{ex}
\begin{ex} \label{parabolaarch}
	A parabolic arch is constructed which is 6 feet wide at the base and 9 feet tall in the middle. Find the height of the arch exactly 1 foot in from the base of the arch. 
	\begin{sol}
		The arch can be modeled by $x^2=-(y-9)$ or $y=9-x^2$.  One foot in from the base of the arch corresponds to either $x = \pm 2$, so the height is $y=9-(\pm 2)^2=5$ feet.
	\end{sol}
\end{ex}


\Closesolutionfile{ans}
\subsection*{Answers \nopunct} \hfill
\begin{multicols}{2}
	\input{ans33}
\end{multicols}

\newpage
\section{Conics, Part II: ellipse and hyperbola}

For both the ellipse and the hyperbola, let $c$ denote the distance from the center to a focus.

\index{ellipse}\textbf{Ellipse}: set of points who sum of distances to two foci is constant. (Also: a circle stretched different amounts along two perpendicular axes).

\index{major axis}\index{minor axis}\index{focus (ellipse)}\index{vertex (ellipse)}
\begin{tikzpicture} []
\drawgridxxyyb{-5.5}{5.5}{-3.5}{3.5};
    \draw [thick] (0,0) ellipse (5 and 3);
 \draw [dashed] (5,3) -- (-5,3) -- (-5,-3) -- (5,-3) -- (5,3) node [above left] {$\dfrac{x^2}{a^2} + \dfrac{y^2}{b^2} = 1$};
\draw [<->] (-5, -3.3) -- (-0.1, -3.3);
\draw (2, -3.3) node [below] {$a$};
\draw [<->] (5, -3.3) -- (0.1, -3.3);
\draw (-2, -3.3) node [below] {$a$};
\draw [<->] (-5.3, -3) -- (-5.3, -0.1);
\draw (-5.3, 1.5) node [left] {$b$};
\draw [<->] (-5.3, 3) -- (-5.3, 0.1);
\draw (-5.3,-1.5) node [left] {$b$};

\draw (4,0) node {$\times$};

\draw (4,-0.1) node [below] {\small focus};

\draw (-4,0) node {$\times$} node [below] {\small focus};

\draw [very thick, dotted] (0,3) -- (4,0);
%\draw (1.8,2) node[] {$c$};

\draw [very thick, dotted] (0,-0.1) -- (5,-0.1);
%\draw (2,0) node[above] {$c$};

\draw (-2,0) node[above] {\small major axis};
\draw (-2,0) node[below] {\small (longer)};

\draw (0,-1.5) node[above, rotate=90] {\small minor axis};

\draw (0,-1.5) node[below, rotate=90] {\small (shorter)};

\draw[fill] (5,0) circle [radius=.07];
\draw [<-] (5.1, -0.1) -- (5.5, -1) node [below] {\small vertex};

\draw[fill] (-5,0) circle [radius=.07];

\draw [<-] (-4.9, -0.1) -- (-4, -1) node [below right] {\small vertex};

\draw[<->] (-0.1, 1) -- (-4,1);
\draw  (-2,1) node[above] {\small $c = \sqrt{(\text{lg den}) - (\text{sm den})}$};

\end{tikzpicture}

\bigskip
\index{hyperbola}\textbf{Hyperbola}: set of points whose difference of distances to two foci is constant.

\index{transverse axis}\index{conjugate axis}\index{focus (hyperbola)}\index{vertex (hyperbola)} \index{asymptotes (hyperbola)}
\begin{tikzpicture} [scale=0.65]
\drawgridxxyyb{-8}{8}{-7}{7};
%\draw [thick] (0,0) ellipse (5 and 3);
\draw [dashed] (4,3) -- (-4,3) -- (-4,-3) -- (4,-3) -- (4,3) node [above left] {$\dfrac{x^2}{a^2} - \dfrac{y^2}{b^2} = 1$};
\draw [<->] (-4, -3.3) -- (-0.1, -3.3);
\draw (2, -3.3) node [below] {$a$};
\draw [<->] (4, -3.3) -- (0.1, -3.3);
\draw (-2, -3.3) node [below] {$a$};
\draw [<->] (-3.7, -3) -- (-3.7, -0.1);
\draw (-3.7, 1.5) node [right] {$b$};
\draw [<->] (-3.7, 3) -- (-3.7, 0.1);
\draw (-3.7,-1.5) node [right] {$b$};

\draw (5,0) node {$\times$} node [below] {\small focus};

\draw (-5,0) node {$\times$} node [below] {\small focus};

\draw[fill] (-4,0) circle [radius=.1];

\draw[fill] (4,0) circle [radius=.1];

\draw [dotted] (-10,-7.5) -- (0,0);
\draw [dotted, very thick] (0,0) -- (4,3);
\draw [dotted, very thick] (0,0.1) -- (5,0.1);
\draw [dotted] (4,3) -- (10,7.5);
\draw [dotted] (-10,7.5) -- (10,-7.5);

%\begin{scope}[rotate=90]
\draw[<->, domain=-2.5:2.5,smooth, samples=15]  plot ({sqrt(1+\x*\x)*4},{\x*3});
\draw[<->, domain=-2.5:2.5,smooth, samples=15]  plot ({-sqrt(1+\x*\x)*4},{\x*3});
%\end{scope}

\draw (0,-5) node[right] {\small $c= \sqrt{a^2+b^2}$};
\draw (4,0.1) node[above left] {\small vertex};
\draw[<-] (-4.2,0.2) -- (-5, 1) node[above left] {\small vertex};

\draw (0,-1.65) node[above, rotate=90] {\tiny conjugate axis};
\draw (-2, 0) node[above] {\tiny transverse axis};

\end{tikzpicture}


\Opensolutionfile{ans}[ans34]
\subsection*{Exercises \nopunct} \hfill

Graph the ellipse.  Find the center, the lines which contain the major and minor axes, the vertices, the endpoints of the minor axis, and the foci.
\begin{multicols}{2}
	\begin{ex}
	$\dfrac{x^{2}}{169} + \dfrac{y^{2}}{25} = 1$
	\begin{sol}
	Center $(0, 0)$\\
	Major axis along $y = 0$\\
	Minor axis along $x = 0$\\
	Vertices $(13, 0), \, (-13, 0)$\\
	Endpoints of Minor Axis $(0,-5)$, $(0,5)$ \\
	Foci $(12, 0), \, (-12, 0)$\\
	\end{sol}
	\end{ex}

	\begin{ex}
		 $\dfrac{x^2}{9} + \dfrac{y^2}{25} = 1$
		\begin{sol}
			$\dfrac{x^{2}}{9} + \dfrac{y^{2}}{25} = 1$
			
			Center $(0, 0)$\\
			Major axis along $x = 0$\\
			Minor axis along $y = 0$\\
			Vertices $(0,5), \, (0,-5)$\\
			Endpoints of Minor Axis $(-3,0)$, $(3,0)$ \\
			Foci $(0,-4), \, (0,4)$\\
		\end{sol}
	\end{ex}

\begin{ex}
	$\dfrac{(x - 2)^{2}}{4} + \dfrac{(y + 3)^{2}}{9} = 1$
	\begin{sol}
	$\dfrac{(x - 2)^{2}}{4} + \dfrac{(y + 3)^{2}}{9} = 1$
	
	Center $(2, -3)$\\
	Major axis along $x = 2$\\
	Minor axis along $y = -3$\\
	Vertices $(2, 0), \, (2, -6)$\\
	Endpoints of Minor Axis $(0,-3)$, $(4,-3)$\\
	Foci $(2, -3 + \sqrt{5}), \, (2, -3 - \sqrt{5})$\\	
	\end{sol}
\end{ex}

\begin{ex}
	$\dfrac{(x + 5)^{2}}{16} + \dfrac{(y - 4)^{2}}{1} = 1$
	\begin{sol}
		$\dfrac{(x + 5)^{2}}{16} + \dfrac{(y - 4)^{2}}{1} = 1$
		
		Center $(-5, 4)$\\
		Major axis along $y = 4$\\
		Minor axis along $x = -5$\\
		Vertices $(-9, 4), \, (-1, 4)$\\
		Endpoints of Minor Axis $(-5,3)$, $(-5,5)$\\
		Foci $(-5 + \sqrt{15}, 4), \, (-5 - \sqrt{15}, 4)$\\
	\end{sol}
\end{ex}

\begin{ex}
	$\dfrac{(x-1)^2}{9}+\dfrac{(y+3)^2}{4} = 1$
	\begin{sol}
	$\dfrac{(x-1)^2}{9}+\dfrac{(y+3)^2}{4} = 1$
	
	Center $(1, -3)$\\
	Major axis along $y = -3$\\
	Minor axis along $x = 1$\\
	Vertices $(4, -3), \, (-2, -3)$\\
	Endpoints of the Minor Axis $(1,-1)$, $(1,-5)$\\
	Foci $(1+\sqrt{5}, -3), \, (1-\sqrt{5}, -3)$\\	
	\end{sol}
\end{ex}
\end{multicols}

Convert the equation to locate the center of the ellipse. Find the center, the lines which contain the major and minor axes, the vertices, the endpoints of the minor axis, and the foci.
\begin{multicols}{2}
\begin{ex}
	$9x^2+25y^2-54x-50y-119=0$
	\begin{sol}
		$\dfrac{(x-3)^2}{25} + \dfrac{\left(y-1\right)^2}{9} = 1$\\
		Center  $\left(3, 1 \right)$\\
		Major Axis along $y=1$\\
		Minor Axis along $x=3$\\
		Vertices  $\left( 8, 1   \right)$, $(-2, 1)$\\
		Endpoints of Minor Axis $\left(3,4\right)$, $\left(3,-2\right)$\\
		Foci $\left(7,1 \right)$, $\left(-1, 1\right)$
	\end{sol}
\end{ex}

\begin{ex}
	$12x^{2} + 3y^{2} - 30y + 39 = 0$
	\begin{sol}
	$\dfrac{x^{2}}{3} + \dfrac{(y - 5)^{2}}{12} = 1$\\
	Center $(0, 5)$\\
	Major axis along $x = 0$\\
	Minor axis along $y = 5$\\
	Vertices $(0, 5 - 2\sqrt{3}), (0, 5 + 2\sqrt{3})$\\
	Endpoints of Minor Axis $(-\sqrt{3},5)$, $(\sqrt{3},5)$\\
	Foci $(0, 2), (0, 8)$	
	\end{sol}
\end{ex}

\end{multicols}

\begin{ex}
	Find an equation for the ellipse that has center $(3,7)$, vertex $(3,2)$, and focus $(3,3)$.
	\begin{sol}
		$\dfrac{(x-3)^2}{9}+\dfrac{(y-7)^2}{25} = 1$
	\end{sol}
\end{ex}


\begin{ex}
	Find an equation for the ellipse that has foci $(3,0)$ and $(-3,0)$ has minor axis of length 10.
	\begin{sol}
		$\dfrac{x^2}{34}+\dfrac{y^2}{25}=1$
	\end{sol}
\end{ex}


\begin{ex}
	Find an equation for an ellipse were all points have a positive $x$-value and negative $y$-value \textit{except} $(0,-9)$ and $(8,0)$.
	\begin{sol}
				$\dfrac{(x-8)^2}{64}+\dfrac{(7+9)^2}{81} = 1$
	\end{sol}
\end{ex}



\begin{ex}
Halley's Comet follows an elliptical orbit with the Sun at one of the foci. At its closet point (perihelion), Halley's Comet is 	0.586 AU from the Sun. At its farthest point (aphelion), it is 35.082 AU from the Sun. What is the length of the minor axis of Halley's Comet's orbit? Approximate to three significant figures. [Note: 1 AU is the typical distance from the Earth to the Sun. Unlike the Earth's nearly circular orbit, the orbit of Halley's Comet has a high \textit{eccentricity}\index{eccentricity}.]

\begin{tikzpicture}[line cap=round,line join=round,x=1.0cm,y=1.0cm, scale=0.3]
%\clip(-19.891986049333518,-12.704708756628774) rectangle (20.06681136690262,9.097413200317016);
\draw [] (0.,0.) ellipse (18.2cm and 6.499230723708778cm);
\begin{scriptsize}
\draw [fill=white] (-17.,0.) circle (4pt);
\draw[color=black] (-16,0.8) node {Sun};
\draw[color=black] (-4,4) node {Orbit of Halley's Comet};
\end{scriptsize}
\end{tikzpicture}

\begin{sol}
	$9.07$ AU
\end{sol}
\end{ex}


\begin{ex}
\label{ellipsearchex} An elliptical arch is constructed which is 6 feet wide at the base and 9 feet tall in the middle. Find the height of the arch exactly 1 foot in from the base of the arch. Compare your result with your answer to Exercise \ref{parabolaarch} in Section \ref{Parabolas}.	
	\begin{sol}
		$3\sqrt{5} \approx 6.71$ feet.
	\end{sol}
\end{ex}

Graph each hyperbola. Find the center, the lines which contain the transverse and conjugate axes, the vertices, the foci and the equations of the asymptotes.

\begin{multicols}{2}
	
	\begin{ex}
		$\dfrac{x^{2}}{16} - \dfrac{y^{2}}{9} = 1$
		\begin{sol}
		$\dfrac{x^{2}}{16} - \dfrac{y^{2}}{9} = 1$
		
		Center $(0, 0)$\\
		Transverse axis on $y = 0$\\
		Conjugate axis on $x = 0$\\
		Vertices $(4, 0), (-4, 0)$\\
		Foci $(5, 0), (-5, 0)$\\
		Asymptotes $y = \pm \frac{3}{4} x$\\
		
		Check graphs with a graphing utility.	
		\end{sol}
	\end{ex}
	
	\begin{ex}
		 $\dfrac{y^{2}}{9} - \dfrac{x^{2}}{16} = 1$
		\begin{sol}
			$\dfrac{y^{2}}{9} - \dfrac{x^{2}}{16} = 1$
			
			Center $(0, 0)$\\
			Transverse axis on $x = 0$\\
			Conjugate axis on $y = 0$\\
			Vertices $(0, 3), (0, -3)$\\
			Foci $(0, 5), (0, -5)$\\
			Asymptotes $y = \pm \frac{3}{4} x$\\
		\end{sol}
	\end{ex}
	
	\begin{ex}
		 $\dfrac{(x - 2)^{2}}{4} - \dfrac{(y + 3)^{2}}{9} = 1$
		\begin{sol}
		$\dfrac{(x - 2)^{2}}{4} - \dfrac{(y + 3)^{2}}{9} = 1$
		
		Center $(2, -3)$\\
		Transverse axis on $y = -3$\\
		Conjugate axis on $x = 2$\\
		Vertices $(0, -3), (4, -3)$\\
		Foci $(2 + \sqrt{13}, -3), (2 - \sqrt{13}, -3)$\\
		Asymptotes $y = \pm \frac{3}{2}(x - 2) - 3$\\
			
		\end{sol}
	\end{ex}
	
	\begin{ex}
		$\dfrac{(y+2)^2}{16} - \dfrac{(x-5)^2}{20} = 1$
		\begin{sol}
			$\dfrac{(y+2)^2}{16} - \dfrac{(x-5)^2}{20} = 1$
			
			Center $(5, -2)$\\
			Transverse axis on $x=5$\\
			Conjugate axis on $y=-2$\\
			Vertices $(5,2), (5,-6)$\\
			Foci $\left(5,4 \right), \left(5,-8\right)$\\
			Asymptotes $y = \pm \frac{2\sqrt{5}}{5} (x-5)-2$\\
			
		\end{sol}
	\end{ex}
	
	\begin{ex}
		$12x^{2} - 3y^{2} + 30y - 111 = 0$
		\begin{sol}
		$\dfrac{x^{2}}{3} - \dfrac{(y - 5)^{2}}{12} = 1$
		
		Center $(0, 5)$\\
		Transverse axis on $y = 5$\\
		Conjugate axis on $x = 0$\\
		Vertices $(\sqrt{3}, 5), (-\sqrt{3}, 5)$\\
		Foci $(\sqrt{15}, 5), (-\sqrt{15}, 5)$\\
		Asymptotes $y = \pm 2x + 5$
		\end{sol}
	\end{ex}
	
	\begin{ex}
		$18y^{2} - 5x^{2} +  72y + 30x - 63= 0$
		\begin{sol}
		$\dfrac{(y + 2)^{2}}{5} - \dfrac{(x - 3)^{2}}{18} = 1$
		
		Center $(3, -2)$\\
		Transverse axis on $x = 3$\\
		Conjugate axis on $y = -2$\\
		Vertices $(3, -2 + \sqrt{5}), (3, -2 - \sqrt{5})$\\
		Foci $(3, -2 + \sqrt{23}), (3, -2 - \sqrt{23})$\\
		Asymptotes $y = \pm \frac{\sqrt{10}}{6}(x - 3) - 2$	
		\end{sol}
	\end{ex}
	
\end{multicols}

Convert the form of each equation to determine the type of conic section then graph it. Check graphs with a graphing utility.

\begin{multicols}{2}
		
	\begin{ex}
		$x^2-2x-4y-11=0$
		\begin{sol}
			$(x-1)^2 = 4(y+3)$, parabola
		\end{sol}
	\end{ex}
	
	\begin{ex}
		$x^2 + y^2-8x+4y+11=0$
		\begin{sol}
			$(x-4)^2+(y+2)^2 = 9$, circle
		\end{sol}
	\end{ex}
		
\begin{ex}
	$9x^2 + 4y^2-36x+24y + 36=0$
	\begin{sol}
		$\dfrac{(x - 2)^{2}}{4} + \dfrac{(y + 3)^{2}}{9} = 1$\, ellipse
	\end{sol}
\end{ex}

\begin{ex}
	$9x^2-4y^2-36x-24y-36=0$
	\begin{sol}
		$\dfrac{(x - 2)^{2}}{4} - \dfrac{(y + 3)^{2}}{9} = 1$, hyperbola
	\end{sol}
\end{ex}
		
\begin{ex}
	$y^2+8y-4x+16=0$
	\begin{sol}
		$(y + 4)^{2} = 4x$, parabola
	\end{sol}
\end{ex}

\begin{ex}
	$4x^2+y^2-8x+4=0$
	\begin{sol}
		Graph is the single point $(1,0)$.
	\end{sol}
\end{ex}
\end{multicols}


\begin{ex}
	How might you create the equation for a ``crossing lines'' degenerate conic? Create an example then check with a graphing utility.
%	\begin{sol}
%	Hint: try modifying the equation of a hyperbola.
%	\end{sol}
\end{ex}


\Closesolutionfile{ans}
\subsection*{Answers \nopunct} \hfill
\begin{multicols}{2}
	\input{ans34}
\end{multicols}

\newpage
\section{Logarithm and exponent rules}

Remember good bases have $b>0$, $b\neq 1$.

\index{logarithm properties}\index{change of base}\index{exponent properties}\index{product rule (logs)}\index{quotient rule (logs)}\index{power rule (logs)}\index{logarithms}
\bigskip
\begin{tabular}{|c | c | c|}
	\hline
	\textbf{logarithm rule} & \textbf{name} & \textbf{corresponding exponent rule} \\
	\hline
	& & \\
	$\log_b(uv) = \log_b(u) + \log_b(v)$ & $\leftarrow$ product rule & $b^{x+y} = b^xb^y$ \\
	& & \\
	$\log_b\left(\dfrac{u}{v}\right) = \log_b(u) - \log_b(v)$ & $\leftarrow$  quotient rule & $b^{x-y} = \dfrac{b^x}{b^y}$ \\
	& & \\
	$ \log_b\left( u^y \right) =  y\cdot\log_b\left(u\right) $ & $\leftarrow$  power rule & $ \left(b^x\right)^y = b^{xy}$ \\
	& &\\
	$ \log_b\left( b^x \right) = x$\ \  (for all $x$) & inverse properties & $ b^{\log_b(x)} = x$\ \ (for $x>0$) \\		
	& &\\
	$ \log_a(x) = \dfrac{\log_b(x)}{\log_b(a)}$ & change of base & $ a^x = \left[ b^{\log_b(a)} \right]^x = b^{x\cdot \log_b(a)}$ \\	
		& &\\	
	\hline
\end{tabular}

\Opensolutionfile{ans}[ans35]
\subsection*{Exercises \nopunct} \hfill

Expand the given logarithm and simplify.  Assume when necessary that all quantities represent positive real numbers.

\begin{multicols}{2}

\begin{ex}
	$\ln(x^{3}y^{2})$ 
	\begin{sol}
		$3\ln(x) + 2\ln(y)$
	\end{sol}
\end{ex}

\begin{ex}
	$\log_{2}\left(\dfrac{128}{x^{2} + 4}\right)$
	\begin{sol}
		$7 - \log_{2}(x^{2} + 4)$
	\end{sol}
\end{ex}


\begin{ex}
	$\log_{5}\left[ \left(\dfrac{z}{25}\right)^{3}\right]$
	\begin{sol}
		$3\log_{5}(z) - 6$
	\end{sol}
\end{ex}


\begin{ex}
	$\log(1.23 \times 10^{37})$
	\begin{sol}
	 $\log(1.23) + 37$
	\end{sol}
\end{ex}


\begin{ex}
	$\ln\left(\dfrac{\sqrt{z}}{xy}\right)$
	\begin{sol}
		$\frac{1}{2}\ln(z) - \ln(x) - \ln(y)$
	\end{sol}
\end{ex}

\begin{ex}
	$\log_{5} \left(x^2 - 25 \right)$
	\begin{sol}
		$\log_{5}(x-5) + \log_{5}(x+5)$
	\end{sol}
\end{ex}

\begin{ex}
	$\log_{\sqrt{2}} \left(4x^3\right)$
	\begin{sol}
		$3\log_{\sqrt{2}}(x) + 4$
	\end{sol}
\end{ex}


\begin{ex}
	$\log_{\frac{1}{3}}(9x(y^{3} - 8))$
	\begin{sol}
		\small$-2 + \log_{\frac{1}{3}}(x) + \log_{\frac{1}{3}}(y - 2) + \log_{\frac{1}{3}}(y^{2} + 2y + 4)$\normalsize
	\end{sol}
\end{ex}


\begin{ex}
	$\log\left(1000x^3y^5\right)$
	\begin{sol}
		$3 + 3\log(x) + 5 \log(y)$
	\end{sol}
\end{ex}


\begin{ex}
	$\log_{3} \left(\dfrac{x^2}{81y^4}\right)$
	\begin{sol}
	$2\log_{3}(x) - 4 - 4\log_{3}(y)$	
	\end{sol}
\end{ex}

\begin{ex}
	$\ln\left(\sqrt[4]{\dfrac{xy}{ez}}\right)$
	\begin{sol}
		$\frac{1}{4} \ln(x) + \frac{1}{4} \ln(y) - \frac{1}{4} - \frac{1}{4} \ln(z)$
	\end{sol}
\end{ex}

\begin{ex}
	$\log_{6} \left[ \left(\dfrac{216}{x^3y}\right)^4 \right]$
	\begin{sol}
	 $12-12\log_{6}(x) - 4\log_{6}(y)$	
	\end{sol}
\end{ex}


\begin{ex}
	$\log\left(\dfrac{100x\sqrt{y}}{\sqrt[3]{10}}\right)$
	\begin{sol}
		$\frac{5}{3}+\log(x)+\frac{1}{2}\log(y)$
	\end{sol}
\end{ex}


\begin{ex}
	$\log_{\frac{1}{2}}\left(\dfrac{4\sqrt[3]{x^2}}{y\sqrt{z}}\right)$
	\begin{sol}
		$-2+\frac{2}{3}\log_{\frac{1}{2}}(x)-\log_{\frac{1}{2}}(y)-\frac{1}{2}\log_{\frac{1}{2}}(z)$
	\end{sol}
\end{ex}


\begin{ex}
	$\ln \left(\dfrac{\sqrt[3]{x}}{10 \sqrt{yz}}\right)$
	\begin{sol}
		$\frac{1}{3} \ln(x) - \ln(10) - \frac{1}{2}\ln(y)-\frac{1}{2}\ln(z)$
	\end{sol}
\end{ex}

\end{multicols}

Use the properties of logarithms to write the expression as a single logarithm.


\begin{multicols}{2}

\begin{ex}
	$4\ln(x) + 2\ln(y)$ 
	\begin{sol}
		$\ln(x^{4}y^{2})$
	\end{sol}
\end{ex}

\begin{ex}
	$\log_{2}(x) + \log_{2}(y) - \log_{2}(z)$
	\begin{sol}
	$\log_{2}\left(\frac{xy}{z}\right)$	
	\end{sol}
\end{ex}


\begin{ex}
	$\log_{3}(x) - 2 \log_{3}(y)$
	\begin{sol}
		$\log_{3} \left( \frac{x}{y^2} \right)$
	\end{sol}
\end{ex}


\begin{ex}
	$\frac{1}{2}\log_{3}(x) - 2\log_{3}(y) - \log_{3}(z)$
	\begin{sol}
		$\log_{3} \left( \frac{\sqrt{x}}{y^2z} \right)$
	\end{sol}
\end{ex}


\begin{ex}
	$2 \ln(x) -3 \ln(y) - 4\ln(z)$
	\begin{sol}
	$\ln\left( \frac{x^2}{y^3z^4} \right)$	
	\end{sol}
\end{ex}

\begin{ex}
	$\log(x) - \frac{1}{3} \log(z) + \frac{1}{2} \log(y)$
	\begin{sol}
	$\log\left(\frac{x \sqrt{y}}{\sqrt[3]{z}}  \right)$	
	\end{sol}
\end{ex}

\begin{ex}
	$-\frac{1}{3} \ln(x) - \frac{1}{3}\ln(y) + \frac{1}{3} \ln(z)$
	\begin{sol}
		$\ln\left(\sqrt[3]{\frac{z}{xy}}   \right)$
	\end{sol}
\end{ex}


\begin{ex}
	$\log_{5}(x) - 3$
	\begin{sol}
		$\log_{5}\left(\frac{x}{125}\right)$
	\end{sol}
\end{ex}


\begin{ex}
	$3 - \log(x)$
	\begin{sol}
		$\log\left(\frac{1000}{x}\right)$
	\end{sol}
\end{ex}


\begin{ex}
	 $\log_{7}(x) + \log_{7}(x - 3) - 2$
	\begin{sol}
		$\log_{7}\left(\frac{x(x - 3)}{49}\right)$
	\end{sol}
\end{ex}

\begin{ex}
	$\ln(x) + \frac{1}{2}$ 
	\begin{sol}
		$\ln \left(x \sqrt{e} \right)$
	\end{sol}
\end{ex}

\begin{ex}
	$\log_{2}(x) + \log_{4}(x)$ 
	\begin{sol}
		$\log_{2}\left(x^{3/2}\right)$
	\end{sol}
\end{ex}


\begin{ex}
	$\log_{2}(x) + \log_{4}(x-1)$
	\begin{sol}
		$\log_{2}\left(x \sqrt{x-1}\right)$
	\end{sol}
\end{ex}


\begin{ex}
	 $\log_{2}(x) + \log_{\frac{1}{2}}(x - 1)$
	\begin{sol}
		$\log_{2}\left(\frac{x}{x - 1}\right)$ 
	\end{sol}
\end{ex}
\end{multicols}

Use the appropriate change of base formula to convert the given expression to an expression with the indicated base. 

\begin{multicols}{2}
\begin{ex}
	$7^{x - 1}$ to base $e$ 
	\begin{sol}
		$e^{(x - 1)\ln(7)}$
	\end{sol}
\end{ex}

\begin{ex}
	$\log_{3}(x + 2)$ to base 10
	\begin{sol}
		$= \frac{\log(x + 2)}{\log(3)}$
	\end{sol}
\end{ex}

\begin{ex}
	$\left(\dfrac{2}{3}\right)^{x}$ to base $e$
	\begin{sol}
		$e^{x\ln(\frac{2}{3})}$
	\end{sol}
\end{ex}


\begin{ex}
	$\log(x^{2} + 1)$ to base $e$ 
	\begin{sol}
		$ \frac{\ln(x^{2} + 1)}{\ln(10)}$
	\end{sol}
\end{ex}

\end{multicols}


Use the appropriate change of base formula to approximate the logarithm on your calculator using the base-10 or base-$e$ logarithm functions.

\begin{multicols}{2}

\begin{ex}
	$\log_{3}(12)$ 
	\begin{sol}
		 $\log_{3}(12) \approx 2.26186$
	\end{sol}
\end{ex}


\begin{ex}
	 $\log_{5}(80)$
	\begin{sol}
		$\log_{5}(80) \approx 2.72271$
	\end{sol}
\end{ex}

\begin{ex}
	$\log_{6}(72)$
	\begin{sol}
		$\log_{6}(72) \approx 2.38685$
	\end{sol}
\end{ex}

\begin{ex}
	 $\log_{4}\left(\dfrac{1}{10}\right)$
	\begin{sol}
	$\log_{4}\left(\frac{1}{10}\right) \approx -1.66096$	
	\end{sol}
\end{ex}


\begin{ex}
	$\log_{\frac{3}{5}}(1000)$
	\begin{sol}
		 $\log_{\frac{3}{5}}(1000) \approx -13.52273$
	\end{sol}
\end{ex}


\begin{ex}
	 $\log_{\frac{2}{3}}(50)$
	\begin{sol}
		$\log_{\frac{2}{3}}(50) \approx -9.64824$
	\end{sol}
\end{ex}
\end{multicols}

\begin{ex}
	Compare and contrast the graphs of $y = \ln(x^{2})$ and $y = 2\ln(x)$.
\end{ex}

\Closesolutionfile{ans}
\subsection*{Answers \nopunct} \hfill
\begin{multicols}{2}
	\input{ans35}
\end{multicols}


\newpage
\section{The exponential model} \label{expApps}
\index{exponential applications}\index{exponential model}\index{exponential growth}\index{exponential decay}

The exponential model: $ A(t) = A_0 b^t $
\begin{itemize}
	\item	$0 < b < 1$: exponential decay
	\item	$b > 1$: exponential growth
\end{itemize}

\begin{tikzpicture} [scale=0.6]
\drawgridxxyyllb{-5}{4}{-1}{8}{$t$}{$A$}
\draw[very thick, samples=10, domain=-5:3, smooth] plot (\x, {exp(.693147*\x)}) node [above] {\ \ \ \ $A(t) = A_0 b^t$};
%\draw[very thick, dashed, samples=20, domain=-5:1.29203, smooth] plot (\x, {exp(1.60944*\x)}) node [above] {$y=5^x$};
\draw (0.2,1) -- (-0.2,1) node [above left] {$A_0$};
\draw (-5, 5) node [right, fill=white] {\textbf{exponential}};
\draw (-5, 4) node [right, fill=white] {\textbf{growth}};
\draw (-5, 3) node [right, fill=white] {$b>1$};
\end{tikzpicture}\
\begin{tikzpicture} [scale=0.6]
\drawgridxxyyllb{-4}{5.2}{-1}{8}{$t$}{$A$}
\begin{scope}[xscale=-1]
\draw[very thick, samples=10, domain=-5:3, smooth] plot (\x, {exp(.693147*\x)});% node [above] {\small $y=\left( \frac{1}{2}\right)^x$\ \ \ \ \ \ \ \ };
%\draw[very thick, dashed, samples=20, domain=-5:1.29203, smooth] plot (\x, {exp(1.60944*\x)}) node [above] {\small $y=\left( \frac{1}{5}\right)^x$};
\end{scope}
\draw (-0.2,1) -- (0.2,1) node [above right] {$A_0$};
\draw (0.5, 5) node [right, fill=white] {\textbf{exponential}};
\draw (0.5, 4) node [right, fill=white] {\textbf{decay}};
\draw (0.3, 3) node [right, fill=white] {$0<b<1$};
\end{tikzpicture}

\bigskip
Linear model analogy:

\begin{tabular}{|c | c | c|}
	\hline
	  & \textbf{exponential model} & \textbf{linear model} \\
	\hline
	& & \\
	equation & $ A(t) = A_0 b^t $ & $A(t)=A_0 + mt$ \\
	& & \\
	initial amount & $ A_0 $ & $A_0$ \\
	& & \\
	Increase $t$ by 1? & \underline{multiply} by $b$ & \underline{add} $m$\\
		& & \\
	\hline
\end{tabular}


\Opensolutionfile{ans}[ans36]
\subsection*{Exercises \nopunct} Find exact \textit{and} approximate solutions unless otherwise specified.\hfill
\begin{ex}
	Professor's initial investment of \$450 gains 7\% interest each year.
	\ssp
	\item What is it worth after 3.5 years?
	\item When is it worth \$1000?
	\esp
	\begin{sol}
			\ssp
		\item $450\cdot(1.07)^{3.5} \approx \$570$
		\item $\log_{1.07}(20/9) \approx 11.8$ years later
		\esp
	\end{sol}
\end{ex}

\begin{ex}
	Bobby's bacteria culture triples in size every 13 minutes. If there are intially 250 cells...
	\ssp
	\item How many cells are there after $t$ minutes?
	\item How long does it take for the number of cells to double? Does this depend on how many there are?
	\esp
	\begin{sol}
		\ssp
		\item $A(t) = 250\cdot3^{\left( \frac{t}{13} \right)}$
		\item $\log_3(2)\cdot 13 \approx 8.20$ minutes. This does not depend on the amount.
		\esp		
	\end{sol}
\end{ex}

\begin{ex}
	Today Professor won the lottery and bought a new car. Its value decays exponentially. He bought it for \$35,000. After 5 years it will be worth \$20,000.
	\ssp
	\item What will it be worth after 3 years?
	\item What will it be worth after $t$ years?
	\item When will it be worth \$25,000? \$10,000? \$5,000?
	\item How much does it depreciate each year?
	\esp
	\begin{sol}
	\ssp
		\item $35000\cdot \left[ \left( \frac{4}{7} \right)^{\frac{1}{5}} \right]^3$\\
		$= 35000\cdot \left( \frac{4}{7} \right)^{\frac{3}{5}}$\\
		$\approx \$25,017 $
		\item $35000\cdot \left[ \left( \frac{4}{7} \right)^{\frac{1}{5}} \right]^t$\\
		$= 35000\cdot \left( \frac{4}{7} \right)^{\frac{t}{5}}$
		\item $5\cdot\log_{4/7}(5/7) \approx 3.006$ years later, $5\cdot\log_{4/7}(2/7) \approx 11.193$ years later, $5\cdot\log_{4/7}(1/7) \approx 17.386$ years later.  
		\item It depreciates by $\left[ 1-(4/7)^{1/5} \right]$ (about 10.59\%) each year.
	\esp		
	\end{sol}
\end{ex}

\begin{ex} \label{mathonium}
	Professor estimates that mathonium -- a student's ability to do math -- decreases by 25\% each year he or she does not study any math.
	\ssp
	\item What is the half-life of mathonium?
	\item \label{SusieMathonium}If a student is found to have only 17\% of her original mathonium, how long has it been since she has studied math?
	\item How much mathonium is left after 5 years?
	\esp
	\begin{sol}
		\ssp
	\item $\log_{0.75}(0.5) \approx 2.41$ years.
	\item $\log_{0.75}(0.17) \approx 6.16$ years.
		\item $(0.75)^5 \approx 23.7\%$ of the original mathonium
	\esp
	\end{sol}
\end{ex}

\begin{ex}
	Susie is the student from Exercise \#\ref{mathonium}\ref{SusieMathonium}. She thinks that the half-life of mathonium is 2 years. She reasons that since the amount is reduced by 25\% each year, 50\% should be gone after 2 years. Explain Susie's mistake.
\end{ex}


\begin{ex}
	The half-life of the radioactive isotope Carbon-14 is about 5730 years.
	\ssp
	\item Express the amount of Carbon-14 left from an initial $N$ milligrams as a function of time $t$ in years.
	\item What percentage of Carbon-14 is left after 20,000 years?
	\item If an old wooden tool is found in a cave and the amount of Carbon-14 present in it is estimated to be only 42\% of the original amount, approximately how old is the tool?	
	\esp
	\begin{sol}
		\ssp
	\item $A(t) = N\cdot (0.5)^{t/5730}$
	\item $(0.5)^{20000/5730} \approx 8.90\%$
	\item $5730\cdot\log_{0.5}(0.42) \approx 7,170$ years old
	\esp
	\end{sol}
\end{ex}

\begin{ex}
	Apollonius' investment was worth \$50 after 3.2 years and \$75 after 4.5 years. If it appreciates exponentially, was what its initial value?
	\begin{sol}
		$ \dfrac{50}{1.5^{3.2/1.3}} \approx \$18.43 $
	\end{sol}
\end{ex}

\begin{ex} %UW 11.2
	The town of Pinedale, Wyoming is experiencing a population boom. In 1990, the population was 860 and five years later it was 1210.
	\ssp
	\item Find a linear model $\ell(x)$ and an exponential model $p(x)$ for the population of Pinedale in the year $1990+x$.
	\item 		What do these models estimate the population of Pinedale to be in the year 2000?
	\item In what years do the models predict the town reaches 2,000 residents?
	\esp
	\begin{sol}
	\ssp
\item $\ell(x) = 860+70x$, $p(x) = 860 (1210/860)^{x/5}$
\item $\ell(10) = 1560$ people, $p(10) = 860 (1210/860)^{2} \approx 1702$ people.
\item Linear model: 2006, exponential model: 2003 (assuming the population was 860 at the beginning of 1990.)
\esp
	\end{sol}
\end{ex}

\begin{ex} %UW 12.12
	The cities of Abnarca and Bonipto have populations that are growing exponentially. In 1980, Abnarca had a population of 25,000 people. In 1990, its population was 29,000.\\
	Bonipto had a population of 34,000 in 1980. The population of Bonipto doubles every 55 years.
	\ssp
	\item How long does it take the population of Abnarca to double? Round to the nearest year.
	\item When will Abnarca's population equal that of Bonipto?
	\esp
	\begin{sol}
		\ssp
		\item About 47 years.
		\item About 137 years after 1980.
		\esp
	\end{sol}
\end{ex}

\begin{ex} %UW 12.10
	The population of termites and spiders in a certain house are growing exponentially. The house contains 100 termites the day you move in. After 4 days, the house contains 200 termites. Three days after moving in, there are two times as many termites as spiders. Eight days after moving in, there were four times as many termites as spiders.
	
	How long does it take the population of spiders to triple? Round to the nearest hundredth of a day.
	\begin{sol}
		31.7 days
	\end{sol}
\end{ex}

\Closesolutionfile{ans}
\subsection*{Answers \nopunct} \hfill
\begin{multicols}{2}
	\input{ans36}
\end{multicols}


\newpage
\section{Unit circle trigonometry}

\textbf{oriented angles}: direction is recorded by sign: counterclockwise positive, clockwise negative. Necessarily have an initial and terminal side.\index{oriented angles}

\textbf{standard position} (of an angle)\index{standard position (of an angle)}: vertex at origin, initial side in positive $x$ direction.

\textbf{quadrants I--IV}\index{quadrants}: divides the plane in four pieces based on the signs of $x$ and $y$ coordinates (see unit circle diagram). The quadrant of an angle in standard position is the quadrant of the plane containing its terminal side.

\index{trigonometric functions (unit circle definition)}
		\medskip \index{sine (unit circle)}\index{cosine (unit circle)}\index{tangent (unit circle)}
		\textbf{defining the trigonometric functions: }If the angle $\theta$ is in standard position and $(x,y)$ is the intersection of the terminal side and the \textbf{unit circle}:\index{initial side}\index{terminal side}\\
		
		\fbox{
			$
			\begin{array}{ll}
			\cos(\theta)= x & \sec(\theta) = \displaystyle\frac{1}{x} = \frac{1}{\cos(\theta)} \\
			\\
			\sin(\theta) = y &  \csc(\theta) = \displaystyle\frac{1}{y} = \frac{1}{\sin(\theta)} \\
			\\
			\tan(\theta) = \displaystyle\frac{y}{x} = \frac{\sin(\theta)}{\cos(\theta)}\ \ \ \ \ \ \ \  & \cot(\theta) = \displaystyle\frac{x}{y} = \frac{\cos(\theta)}{\sin(\theta)}
			
			\end{array}
			$}
		
		\bigskip For the radius-$r$ circle:
		
		\smallskip
		$\cos(\theta) = \ds \frac{x}{r}$,  $\sin(\theta) = \ds \frac{y}{r}$, $\tan(\theta) = \ds \frac{y}{x}$,
		
		$\sec(\theta) = \ds \frac{r}{x}$,  $\csc(\theta) = \ds \frac{r}{y}$, $\cot(\theta) = \ds \frac{x}{y}$.
		
\index{coterminal angles}\textbf{coterminal angles}: same terminal side when in standard position. Each trig function will give identical output for both angles as $x$ and $y$ are the same.
		
\textbf{Pythagoras, again}\index{Pythagorean identity}: Given the unit circle definition of sine and cosine, it should be clear that for \textit{any} angle $\theta$, $$\cos^2(\theta) + \sin^2(\theta) = 1 \text{\small \ \ \ \ [The Pythagorean identity]}$$
		
\textbf{secant, cosecant, cotangent}:\index{secant}\index{cosecant}\index{cotangent} often considered as reciprocals of the cosine, sine, and tangent (respectively). Note though that when one in each pair is undefined, the other is zero. These new functions also give us all possible ways to for a ratio of two side lengths in a right triangle.

\index{co-functions}\textbf{cofunctions}: the cofunction identites work for all three pairs, i.e. $\text{co\fbox{\tiny sin/sec/tan}}\left(\frac{\pi}{2}-\theta\right) = \fbox{\tiny sin/sec/tan} (\theta)$, and $\text{\fbox{\tiny sin/sec/tan}}\left(\frac{\pi}{2}-\theta\right) = \text{co\fbox{\tiny sin/sec/tan}} (\theta)$.

\textbf{pairing}: notice the six trigonometric functions are paired in two different ways: as co-functions and as reciprocals.


\begin{tikzpicture}[scale=5]
\draw [thick, ->] (-1.07, 0) -- (1.2, 0) node[right] {$x$};
\draw [thick, ->] (0, -1.07) -- (0, 1.2) node[above] {$y$};
\draw [thick] (0,0) circle [radius=1];

\draw (1.1, 0.05) node[above right] {${0,\ 2\pi}$};
\draw[fill] (1,0) circle [radius=0.015];
\draw (1.1, -0.05) node[below right] {\small ${0^\circ,\ 360^\circ}$};

\draw (-0.05, 1.1) node[above left] {$\dfrac{\pi}{2}$};
\draw[fill] (0,1) circle [radius=0.015];
\draw (0.05, 1.15) node[above right] {\small $90^\circ$};

\draw (-1.18, 0) node [below left] {$\pi$};
\draw (-1.18, 0) node [above] {$180^\circ$};
\draw[fill] (-1,0) circle [radius=0.015];

\draw (-0.03, -1.1) node[below left] {$\dfrac{3\pi}{2}$};
\draw[fill] (0,-1) circle [radius=0.015];
\draw (0.03, -1.15) node[below right] {\small $270^\circ$};

\foreach \deg / \rad in {
	30 / \dfrac{\pi}{6},
	45 / \dfrac{\pi}{4},
	60 / \dfrac{\pi}{3},
	120 / \dfrac{2\pi}{3},
	135 / \dfrac{3\pi}{4},
	150 / \dfrac{5\pi}{6},
	210 / \dfrac{7\pi}{6},
	225 / \dfrac{5\pi}{4},
	240 / \dfrac{4\pi}{3},
	300 / \dfrac{5\pi}{3},
	315 / \dfrac{7\pi}{4},
	330 / \dfrac{11\pi}{6} } {
	\begin{scope}[rotate=\deg]
	\draw[gray] (0,0) -- (1.07,0);
	\draw (1.18, 0) node{$\rad$};
	\draw (0.75, 0) node[fill=white] {\small $\deg^\circ$};
	\draw[fill] (1,0) circle [radius=0.015];
	\end{scope}
}
\pgfmathsetmacro{\delta}{0.04};
\pgfmathsetmacro{\stwtw}{0.707107};
\pgfmathsetmacro{\sthtw}{0.866025};

\draw (0.5, 0-\delta) -- (0.5, 0+\delta) node[above] {$\frac{1}{2}$};
\draw (\stwtw, 0-\delta) -- (\stwtw, 0+\delta) node[above] {$\frac{\sqrt{2}}{2}$};
\draw (\sthtw, 0-\delta) -- (\sthtw, 0+\delta) node[above] {$\frac{\sqrt{3}}{2}$};
\draw (1, 0) node[below right] {$1$};

\draw (-0.5, 0-\delta) -- (-0.5, 0+\delta) node[above] {$-\frac{1}{2}$};
\draw (-\stwtw, 0-\delta) -- (-\stwtw, 0+\delta) node[above] {$-\frac{\sqrt{2}}{2}$};
\draw (-\sthtw, 0-\delta) -- (-\sthtw, 0+\delta) node[above] {$-\frac{\sqrt{3}}{2}$};
\draw (-1, 0) node[below left] {$-1$};	

\draw (0+\delta, 0.5) -- (0-\delta, 0.5) node[left] {$\frac{1}{2}$};
\draw (0+\delta, \stwtw) -- (0-\delta, \stwtw) node[left] {$\frac{\sqrt{2}}{2}$};
\draw (0+\delta, \sthtw) -- (0-\delta, \sthtw) node[left] {$\frac{\sqrt{3}}{2}$};
\draw (0, 1) node[above right] {$1$};	

\draw (0+\delta, -0.5) -- (0-\delta, -0.5) node[left] {$-\frac{1}{2}$};
\draw (0+\delta, -\stwtw) -- (0-\delta, -\stwtw) node[left] {$-\frac{\sqrt{2}}{2}$};
\draw (0+\delta, -\sthtw) -- (0-\delta, -\sthtw) node[left] {$-\frac{\sqrt{3}}{2}$};
\draw (0, -1) node[below right] {$-1$};	

\draw (1.2,1.2) node {\large Q I};
\draw (-1.2,1.2) node {\large Q II};
\draw (-1.2,-1.2) node {\large Q III};
\draw (1.2,-1.2) node {\large Q IV};
\draw (-.65, -0.05) node [below] {\small $\pi$ radians = $180^\circ$};
\draw (.85, -0.15) node [left] {\small $x^2+y^2=1$};		
\draw [->] (.85, -0.16) -- (0.97, -0.2);
\draw (-1.0,1.25) node [right] {\small \textsc{\underline{Unit Circle}}};
\end{tikzpicture}\index{unit circle diagram}

\Opensolutionfile{ans}[ans37]
\subsection*{Exercises \nopunct} \hfill
\begin{ex}
	For each angle, convert from degree to radian measure or vice-versa, determine its quadrant, and give two coterminal angles, one of which is positive and the other negative.
	\begin{multicols}{2}
	\ssp
\item $330\dg$
\item $-135\dg$
\item $\dfrac{3\pi}{4}$
\item $-\dfrac{2\pi}{3}$
\item $\dfrac{\pi}{12}$
\item $270\dg$
\esp
	\end{multicols}

	\begin{sol}
	\ssp
		\item $\dfrac{11\pi}{6}$, quadrant IV, $690\dg$, $-30\dg$
		\item $-\dfrac{3\pi}{4}$, quadrant III, $225\dg$, $-495\dg$
		\item $135\dg$, quadrant II, $\dfrac{11\pi}{4}$, $-\dfrac{5\pi}{4}$
		\item $-120\dg$, quadrant III, $\dfrac{4\pi}{3}$, $-\dfrac{8\pi}{3}$
		\item $15\dg$, quadrant I, $\dfrac{13\pi}{12}$, $-\dfrac{11\pi}{12}$
		\item $\dfrac{3\pi}{2}$, not actually in any quadrant! $630\dg$, $-90\dg$
	\esp
	\end{sol}
\end{ex}

\begin{ex}
	Find the exact sine and cosine of each angle.
	\begin{multicols}{2}
		\ssp
		\item $\theta = \dfrac{\pi}{6}$
		\item $\theta = -\dfrac{\pi}{3}$
		\item $\theta = \dfrac{5\pi}{4}$
		\item $\theta = 90\dg$
		\item $\theta = \pi$
		\item $\theta = \dfrac{7\pi}{6}$
		\esp
	\end{multicols}
	
	\begin{sol}
		\ssp
		\item $\sin(\theta) = \dfrac{1}{2}$, $\cos(\theta) = \dfrac{\sqrt{3}}{2}$
		\item $\sin(\theta) = -\dfrac{1}{2}$, $\cos(\theta) = \dfrac{\sqrt{3}}{2}$
		\item $\sin(\theta) = -\dfrac{\sqrt{2}}{2}$, $\cos(\theta) = -\dfrac{\sqrt{2}}{2}$
		\item $\sin(\theta) = 1$, $\cos(\theta) = 0$
		\item $\sin(\theta) = 0$, $\cos(\theta) = -1$
		\item $\sin(\theta) = -\dfrac{1}{2}$, $\cos(\theta) = -\dfrac{\sqrt{3}}{2}$
		\esp
	\end{sol}
\end{ex}

\begin{ex}
	Find the exact value of $\sec\left( \dfrac{\pi}{4}\right)$.
	\begin{sol}
		$\sqrt{2}$ -- Remember it's the reciprocal of the cosine.
	\end{sol}
\end{ex}


\begin{ex}
	Find the exact value of $\csc\left( \dfrac{5\pi}{3}\right)$.
	\begin{sol}
		$-\dfrac{2\sqrt{3}}{3}$
	\end{sol}
\end{ex}


\begin{ex}
	Find the exact value of $\cot\left( -\dfrac{\pi}{6}\right)$.
	\begin{sol}
		$-\sqrt{3}$
	\end{sol}
\end{ex}


\begin{ex}
	Find the exact value of $\tan\left( 90\dg\right)$.
	\begin{sol}
		This is undefined. Notice all the trigonometric functions \underline{except} sine and cosine are undefined for some integral multiples of $90\dg$ ($\pi/2$ rad) because either the $x$ or $y$ value on the unit circle will be zero (and division by zero is undefined).
	\end{sol}
\end{ex}

\begin{ex}
Which pairs of the trigonometric functions have the same domains?
	\begin{sol}
		sine and cosine (defined for all angles), tangent and secant (defined except for angles where the $x$ value on the unit circle is zero), cotangent and cosecant (defined except for angles where the $y$ value on the unit circle is zero). This is yet another way to pair up the trig functions!
	\end{sol}
\end{ex}

\begin{ex}
	What is the range of the sine function? The cosine?
	\begin{sol}
		Both are $[-1,1]$.
		\end{sol}
\end{ex}

\begin{ex}
	Plot the angles in standard position. Use your drawing to estimate the sine, cosine, and tangent of each angle to one or two significant figures. Use a protractor if you have one, otherwise estimate starting from the common angles. Check your results against a calculator. Remember to include the si\underline{gn} in your approximations.
		\begin{multicols}{3}
		\ssp
		\item $20\dg$
		\item $115\dg$
		\item $-80\dg$
		\item $220\dg$
		\esp
	\end{multicols}
\end{ex}

\begin{ex}
	If $\sin(\theta) = -\dfrac{7}{25}$ and $\theta$ is in Q IV, what is $\cos(\theta)$?\ \ \  [``Q'' for quadrant.]
	\begin{sol}
		$\dfrac{24}{25}$
	\end{sol}
\end{ex}


\begin{ex}
	If $\cos(\theta) = \dfrac{4}{9}$ and $\theta$ is in Q I, what is $\sin(\theta)$?
	\begin{sol}
		$\dfrac{\sqrt{65}}{9}$
	\end{sol}
\end{ex}

\begin{ex}
	If $\sin(\theta) = \dfrac{5}{13}$ and $\theta$ is in Q II, what is $\tan(\theta)$?
	\begin{sol}
		$-\dfrac{5}{12}$
	\end{sol}
\end{ex}


\begin{ex}
	If $\tan(\theta) = \dfrac{12}{5}$ and $\pi < \theta < \dfrac{3\pi}{2}$, what is $\sec(\theta)$?
	\begin{sol}
		$-\dfrac{13}{5}$
	\end{sol}
\end{ex}


\begin{ex}
	If $\cot(\theta) = 2$ and $0 < \theta < \dfrac{\pi}{2}$, what is $\cos(\theta)$?
	\begin{sol}
		$\dfrac{2}{\sqrt{5}}$
	\end{sol}
\end{ex}

\begin{ex}
	A pebble stuck in the 24-in diameter tire of his broken-down car starts on the top, but Bob manages to push the car 5 ft. How far off the ground is the pebble afterward? Give an answer in inches, rounded to the nearest tenth.
	\begin{sol}
		15.4 in
	\end{sol}
\end{ex}

\begin{ex}\label{ferriswheel}
	John and Amy are riding a Ferris wheel 64 ft in diameter. They start on the ground and take the ride for one revolution of the wheel which takes 5 minutes. What is their height off the ground $t$ minutes after getting on the ride? Use a trigonometric function in radians. [Hints: First find their angle as a function of time. How many radians do they travel each minute?]
	\begin{sol}
		$h(t) = 32\sin\left( \dfrac{2\pi}{5}t - \dfrac{\pi}{2} \right) + 32$ [Other correct answers possible.]
	\end{sol}
\end{ex}


\Closesolutionfile{ans}
\subsection*{Answers \nopunct} \hfill
\begin{multicols}{2}
	\input{ans37}
\end{multicols}


\newpage
\section{Unit circle trigonometry continued}

\index{trig equations}The problems with ``baby'' trig. equations [$\sin/\cos/\tan(\theta) = \text{const}$]:
\begin{enumerate}
	\item There may be multiple places on the unit circle where you get the desired $x$, $y$, or $y/x$ output. It's often two places, but is sometimes one.
	\item All coterminal angles give the same outputs.
\end{enumerate}

\index{even function}The function $f$ is \textbf{even} if $f(-x) = f(x)$ for all $x$ in the domain. Even functions have symmetry over the $y$-axis.

\index{odd function}The function $f$ is \textbf{odd} if $f(-x) = -f(x)$ for all $x$ in the domain. Odd functions have symmetry ``through the origin'' which can also be thought of as two reflections or a 180$\dg$ rotation.

Cosine and Secant are even functions, the other four trig. functions are odd.

\index{periodic function}A function $f$ is \textbf{periodic} if there is a positive constant $K$ such that $f(x+K)= f(x)$ for all $x$ in the domain. The smallest such number is called the \textbf{period}\index{period}.
\qi{All the trig. functions are periodic since we could choose $K=2\pi$. The period of the tangent and cotangent, however, is $\pi$ while the rest have period $2\pi$.}

\index{trig graphs}
	\begin{tikzpicture}[scale=1.2]
	\drawgridxxyyll{-0.5}{6.78}{-1.25}{1.35}{x}{y = \sin(x)}
	\draw[samples=50, domain=0:2*pi, smooth, style=very thick] plot function{sin(x)};
	\draw[samples=10, domain=-.5:0, smooth, style=very thick, dashed] plot function{sin(x)};
	\draw[samples=10, domain=2*pi:6.75, smooth, style=very thick, dashed] plot function{sin(x)};
	\draw [style=thick] (0.1, 1) -- (-0.1,1) node[left] {$1$};
	\draw [style=thick] (0.1, -1) -- (-0.1,-1) node[left] {$-1$};
	\draw [style=thick] (pi,-0.1) -- (pi, 0.1) node[above] {$\pi$};
	\draw [style=thick] (2*pi,-0.1) -- (2*pi, 0.1) node[above] {\large $2\pi$};
	\draw [style=thick] (pi/2,-0.1) -- (pi/2, 0.1) node[above] {$\frac{\pi}{2}$};
	\draw [style=thick] (3*pi/2,-0.1) -- (3*pi/2, 0.1) node[above] {$\frac{3\pi}{2}$};
	\end{tikzpicture}
	
	\bigskip

	\begin{tikzpicture}[scale=1.2]
	\drawgridxxyyll{-0.5}{6.78}{-1.25}{1.35}{x}{y = \cos(x)}
	\draw[samples=50, domain=0:2*pi, smooth, style=very thick] plot function{cos(x)};
	\draw[samples=10, domain=-.5:0, smooth, style=very thick, dashed] plot function{cos(x)};
	\draw[samples=10, domain=2*pi:6.75, smooth, style=very thick, dashed] plot function{cos(x)};
	\draw [style=thick] (0.0, 1) -- (-0.0,1) node[above left] {$1$};
	\draw [style=thick] (0.1, -1) -- (-0.1,-1) node[left] {$-1$};
	\draw [style=thick] (pi,-0.1) -- (pi, 0.1) node[above] {$\pi$};
	\draw [style=thick] (2*pi,-0.1) -- (2*pi, 0.1) node[above] {\large $2\pi$};
	\draw [style=thick] (pi/2,-0.1) -- (pi/2, 0.1) node[above] {$\frac{\pi}{2}$};
	\draw [style=thick] (3*pi/2,-0.1) -- (3*pi/2, 0.1) node[above] {$\frac{3\pi}{2}$};
	\end{tikzpicture}

\bigskip
	\begin{tikzpicture}[scale=1.2]
	\drawgridxxyyll{-3*pi/4-.2}{3*pi/4+.2}{-3.5}{3.5}{x}{y=\tan(x)}
	\draw[samples=50, domain=-1.29:1.29, smooth, style=very thick, <->] plot function{tan(x)};
	\draw[samples=10, domain=1.85:3*pi/4, smooth, style=very thick, dashed, <-] plot function{tan(x)};
	\draw[samples=10, domain=-3*pi/4:-1.85, smooth, style=very thick, dashed, ->] plot function{tan(x)};
	\draw [style=thick] (0.1, 1) -- (-0.1,1) node[left] {$1$};
	\draw [style=thick] (-0.1, -1) -- (0.1,-1) node[right] {$-1$};
	\draw [style=thick] (pi/2,-0.1) -- (pi/2, 0.1) node[above right] {$\frac{\pi}{2}$};
	\draw [style=thick] (-pi/2,-0.1) -- (-pi/2, 0.1) node[above left] {$-\frac{\pi}{2}$};
	\draw [style=thick] (pi/4,-0.1) -- (pi/4, 0.1) node[above] {$\frac{\pi}{4}$};
	\draw [style=thick] (-pi/4,-0.1) -- (-pi/4, 0.1) node[above] {$-\frac{\pi}{4}$};
	\draw [style=dashed] (-pi/2,-3.5) -- (-pi/2,3.5);
	\draw [style=dashed] (pi/2,-3.5) -- (pi/2,3.5);
	
	\draw [<->] (-pi/2,-3.7) -- (pi/2,-3.7);
	\draw (-pi/2,-3.7) node [below right] {\tiny period $\pi$ unlike sine/cosine};
	\end{tikzpicture}

\Opensolutionfile{ans}[ans38]
\subsection*{Exercises \nopunct} \hfill

Solve each equation.

\begin{multicols}{2}
\begin{ex}
	$\sin(\theta) = \dfrac{1}{2}$
	\begin{sol}
		$\sin(\theta) = \dfrac{1}{2}$ when $\theta = \dfrac{\pi}{6} + 2\pi k$ or $\theta = \dfrac{5\pi}{6} + 2\pi k$ for any integer $k$.
	\end{sol}
\end{ex}

\begin{ex}
	$\cos(\theta) = -\dfrac{\sqrt{3}}{2}$
	\begin{sol}
		$\cos(\theta) = -\dfrac{\sqrt{3}}{2}$ when $\theta = \dfrac{5\pi}{6} + 2\pi k$ or $\theta = \dfrac{7\pi}{6} + 2\pi k$ for any integer $k$.
	\end{sol}
\end{ex}

\begin{ex}
	$\sin(\theta) = 0$
	\begin{sol}
		$\sin(\theta) = 0$ when $\theta = \pi k$ for any integer $k$.
	\end{sol}
\end{ex}

\begin{ex}
	 $\cos(\theta) = -1$
	\begin{sol}
	$\cos(\theta) = -1$ when $\theta = (2k + 1)\pi$ for any integer $k$.	
	\end{sol}
\end{ex}

\begin{ex}
	 $\cos(\theta) = \dfrac{\sqrt{2}}{2}$
	\begin{sol}
		 $\cos(\theta) = \dfrac{\sqrt{2}}{2}$ when $\theta = \dfrac{\pi}{4} + 2\pi k$ or $\theta = \dfrac{7\pi}{4} + 2\pi k$ for any integer $k$.
	\end{sol}
\end{ex}

\begin{ex}
	$\sin(\theta) = \dfrac{\sqrt{3}}{2}$
	\begin{sol}
		$\sin(\theta) = \dfrac{\sqrt{3}}{2}$ when $\theta = \dfrac{\pi}{3} + 2\pi k$ or $\theta = \dfrac{2\pi}{3} + 2\pi k$ for any integer $k$.
	\end{sol}
\end{ex}

\begin{ex}
	$\tan(\theta) = \sqrt{3}$
	\begin{sol}
		$\tan(\theta) = \sqrt{3}$ when $\theta = \dfrac{\pi}{3} + \pi k$ for any integer $k$
	\end{sol}
\end{ex}

\begin{ex}
 $\csc(\theta) = -2$ 
	\begin{sol}
	 $\csc(\theta) = -2$ when $\theta = \dfrac{7\pi}{6} + 2\pi k$ or $\theta = \dfrac{11\pi}{6} + 2\pi k$ for any integer $k$	
	\end{sol}
\end{ex}

\begin{ex}
	 $\cot(\theta) = -1$
	\begin{sol}
		 $\cot(\theta) = -1$ when $\theta = \dfrac{3\pi}{4} + \pi k$ for any integer $k$
	\end{sol}
\end{ex}

\begin{ex}
	$\tan(t) = \dfrac{\sqrt{3}}{3}$
	\begin{sol}
		 $\tan(t) = \dfrac{\sqrt{3}}{3}$ when $t = \dfrac{\pi}{6} + \pi k$ for any integer $k$
	\end{sol}
\end{ex}



\end{multicols}


\begin{ex}
	Draw the graphs of $y=\sin(x)$, $y=\cos(x)$, and $y=\tan(x)$. Draw at least two cycles. Label points or asymptotes every $\pi/4$.
\end{ex}

\begin{ex}
	Use your graphs from the previous exercise to sketch the graphs of the other three trig functions as their reciprocals.  Label points or asymptotes every $\pi/4$.
\end{ex}

Graph one cycle of each function. State the period, amplitude, phase shift, and vertical shift of the given function.

\begin{multicols}{2}
	\begin{ex}
		$y = 3\sin(x)$
			\begin{sol}
			period $2\pi$, amplitude 3, phase shift $0$, vertical shift 0.
		\end{sol}
	\end{ex}

	\begin{ex}
	$y = \sin(3x)$
	\begin{sol}
		period $2\pi/3$, amplitude 1, phase shift 0, vertical shift 0.
	\end{sol}
\end{ex}

\begin{ex}
	$y = -2\cos(x)$
	\begin{sol}
		period $2\pi$, amplitude 2, phase shift 0, vertical shift 0.
	\end{sol}
\end{ex}


\begin{ex}
	$y = \cos\left(x - \dfrac{\pi}{2} \right)$
	\begin{sol}
		period $2\pi$, amplitude 1, phase shift $\dfrac{\pi}{2}$, vertical shift 0.
	\end{sol}
\end{ex}


\begin{ex}
	$y = \cos\left(3x - 2\pi \right)+4$
	\begin{sol}
		period $2\pi/3$, amplitude 1, phase shift $\dfrac{2\pi}{3}$, vertical shift 4.
	\end{sol}
\end{ex}


\begin{ex}
	$y = \sin\left(2x - \pi \right)$
	\begin{sol}
		period $\pi$, amplitude 1, phase shift $\dfrac{\pi}{2}$, vertical shift 0.
	\end{sol}
\end{ex}

\begin{ex}
	$y = \tan\left(4x\right)-5$
	\begin{sol}
		period $\pi/4$, amplitude does not apply as graph is not a sinusoid, phase shift $0$, vertical shift $-5$.
	\end{sol}
\end{ex}


\end{multicols}

Determine analytically if the following functions are even, odd, or neither. Confirm your answer by examining a graph on a graphing utility.

\begin{multicols}{2}
	
	\begin{ex}
		$f(x) = 6x^4-3x^2+7$
		\begin{sol}
			even
		\end{sol}
	\end{ex}

	
\begin{ex}
	$g(x) = \dfrac{1}{x^3-x}$
	\begin{sol}
		odd
	\end{sol}
\end{ex}

	
\begin{ex}
	$q(x) = x^2-4x+6$
	\begin{sol}
		neither
	\end{sol}
\end{ex}

	
\begin{ex}
	$k(x) = x\sin(x)$
	\begin{sol}
		even
	\end{sol}
\end{ex}

	
\begin{ex}
	$\ell(x) = \ln(x^3-x)$
	\begin{sol}
		neither
	\end{sol}
\end{ex}

	
\begin{ex}
	$p(x) = \cos^3(x)$
	\begin{sol}
		even
	\end{sol}
\end{ex}
	
\end{multicols}

\begin{ex}
	Are there any functions that are both even and odd?
\end{ex}

\begin{ex}
	Draw a graph of your function from \#\ref{ferriswheel} in the previous homework.
\end{ex}

\begin{ex}
	Write a formula for $f$ given the graph of $y=f(x)$ below:
	
	\begin{tikzpicture}
		\drawgridxxyy{-3}{5.2}{-2}{4}
	\draw[samples=50, domain=-3:5, smooth, style=very thick, <->] plot function{2*cos(2.0944*x)+1};
	\end{tikzpicture}
	\begin{sol}
	One possibility is $f(x) = 2\cos\left( \dfrac{2\pi}{3}x \right)+1$.
	\end{sol}
\end{ex}


\Closesolutionfile{ans}
\subsection*{Answers \nopunct} \hfill
\begin{multicols}{2}
	\input{ans38}
\end{multicols}



\chapter{Fourth Cycle}



\section{Derivatives}
\label{derivativeSection}

The slope of a secant between two points on a curve: $\ds \frac{\Delta y}{\Delta x}$.

 Leibniz's\index{Leibniz} ``$d$'' notation: $\ds \frac{\Delta y}{\Delta x} \to \frac{dy}{dx}$ as $\Delta x \to 0$.
 \qi{$\dfrac{dy}{dx}$ represents the slope of the tangent line (the line the curve would look like if you zoomed-in.)}

 Assuming $y=f(x)$, $\ds \frac{dy}{dx}$ also depends on $x$, i.e. $\ds \frac{dy}{dx} = f'(x)$ where $f'$ is the \textbf{derivative} of $f$.
	\qi{i.e. the derivative is a function on functions (sometimes called an \textbf{operator}).}
	

\index{derivative}It is common to use function notation and replace $\Delta x$ with $h$ and $\ds \frac{\Delta y}{\Delta x}$ with the difference quotient\index{difference quotient}. The definition then says: for a fixed $x$ in the domain of $f$, if it can be determined that $\dfrac{f(x+h) - f(x)}{h} \to f'(x)$ as $h \to 0$, we say $f'(x)$ is the \textbf{derivative} of $f$ at $x$.
	
	\begin{itemize}
		\item The limit may not exist for all $x$ in the domain of $f$.
		\item {The limit is ``two-sided'' (since there are two ways to approach zero).}
		\item{The derivative will not exist where a function has a sharp corner (no tangent), a vertical tangent (no slope), is undefined, is undefined to one side, or in certain pathological cases.}
	\end{itemize}

	
	\begin{tikzpicture}[scale=1]
	\drawgridxxyyllb{-0.5}{3.8}{-1.15}{3.5}{}{}
	\draw[black, thick, domain=-0.6:3.6, smooth] plot (\x, {-\x*\x + 3*\x+1}) node [above left] {\small $y=f(x)$};
	\draw[fill] (1,3) circle [radius=0.05];
	\draw [<->] (2,4) -- (-0.5, 1.5) node [left] {\small $m_{\text{tan}} = \dfrac{dy}{dx} = f'(x)$};
	\draw [dashed] (1,3) -- (1,0) node [below] {\small $x$};
	\draw [dashed] (1.75,3.1875) -- (1.75,0) node [below] {\small $x+h$};
	\draw [<->] (-0.25, 2.6875) -- (2.75,3.4375) node [right] {\small $m_{\text{sec}} = \dfrac{\Delta y}{\Delta x} = \dfrac{f(x+h)-f(x)}{h}$};

	\draw[fill] (1.75,3.1875) circle [radius=0.05];
	
%	\draw  (3,5) node [right] {\small $y=x^3-4x$};
%	\draw[gray, thick, domain=-3.5:2.5, smooth] plot (\x, {-\x - 2});

%	\draw[fill] (1,-3) circle [radius=0.05];
%	\draw (1.5,-3) node [right] {$(1,-3)$};

	\end{tikzpicture}

\Opensolutionfile{ans}[ans41]
\subsection*{Exercises \nopunct} \hfill


Compute the derivatives of the following functions using the definition given here (Do not use the Calculus of derivatives, if you have learned any of that). Give your answer using Liebniz's notation or function notation  as appropriate (e.g. if $y=x^2$, then report $\frac{dy}{dx} =2x$, but if $f(x) = x^2$, naturally report $f'(x)=2x$).

\begin{multicols}{2}

\begin{ex}
	$f(x) = 3x^2$
	\begin{sol}
		$f'(x)=6x$
	\end{sol}
\end{ex}

\begin{ex}
	$f(x) = x^2-5$
	\begin{sol}
		$f'(x)=2x$
	\end{sol}
\end{ex}

\begin{ex}
	$f(x) = -3x+2$
	\begin{sol}
		$f'(x)=-3$
	\end{sol}
\end{ex}

\begin{ex}
	$f(x) = 3x^2+11x-4$
	\begin{sol}
		$f'(x)=6x+11$
	\end{sol}
\end{ex}


\begin{ex}
$y = x^3-2x$
	\begin{sol}
$\dfrac{dy}{dx} = 3x^2-2$
	\end{sol}
\end{ex}

\begin{ex}
	$y = 1/x$
	\begin{sol}
			$\dfrac{dy}{dx} = -\dfrac{1}{x^2}$
	\end{sol}
\end{ex}

\begin{ex}
	 $g(x) = \ds \frac{1}{x+3}-2$
	\begin{sol}
		$g'(x) = -\dfrac{1}{(x+3)^2}$
	\end{sol}
\end{ex}

\begin{ex}
	$y = \ds \frac{2}{3x-1}$
	\begin{sol}
		$ \dfrac{dy}{dx} = -\dfrac{6}{(3x-1)^2} $
	\end{sol}
\end{ex}

\begin{ex}
 $\ell = 1/x^2$
	\begin{sol}
$\ell'(x) = -\dfrac{2}{x^3}$
	\end{sol}
\end{ex}

\begin{ex}
	$k(x) = \sqrt{x+3}$
	\begin{sol}
		$k'(x) = \dfrac{1}{2\sqrt{x+3}}$
	\end{sol}
\end{ex}

\end{multicols}

[Note for the next three problems a relevant derivative has already been calculated above.]

\begin{ex}
	Find an equation for the line tangent to the graph of 	$f(x) = 3x^2+11x-4$ at $x=-4$. Sketch the line and tangent then check your work on a graphing utility.
	\begin{sol}
		$y = -13(x+4)$
	\end{sol}
\end{ex}

\begin{ex}
	Find an equation for the line tangent to the graph of $k(x) = \sqrt{x+3}$ at $x=1$. Sketch the line and tangent then check your work on a graphing utility.
	\begin{sol}
		$y-2 = \frac{1}{4}(x-1)$
	\end{sol}
\end{ex}

\begin{ex}
	Sketch the graph of $g(x) = \ds \frac{1}{x+3}-2$. Find and graph the tangent line at $x=-2$.
	\begin{sol}
		Check on a graphing utility.
	\end{sol}
\end{ex}

\begin{ex}
	One powerful application of the derivative is the location of horizontal tangents. The following graph has two -- locate them using a derivative.
	
	\begin{tikzpicture}[scale=0.6]
	\drawgridxxyyb{-4}{4}{-5}{5}
	\draw[black, thick, domain=-2.5:2.5, smooth] plot (\x, {\x*\x*\x - 4*\x});
	\draw  (3,5) node [right] {\small $y=x^3-4x$};
	%\draw[gray, thick, domain=-3:1] plot (\x, {3.0792});
	\draw[gray, thick] (-3,3.0792) -- (1,3.0792);
	\draw[gray, dashed] (-1.1547,0)--(-1.1547,3.0792);
	\draw[gray, dashed] (1.1547,0)--(1.1547,-3.0792);
	\draw[gray, thick] (-1,-3.0792) -- (3,-3.0792);
	%\draw  (-1.1547,0) node [below] {\tiny $-\sqrt{\frac{4}{3}}$};
	%\draw  (1.1547,0) node [above] {\tiny $\sqrt{\frac{4}{3}}$};
	\draw  (-1.1547,0) node [below] {$a$};
	\draw  (1.1547,0) node [above] {$b$};
	%\draw[fill] (1,1) circle [radius=0.1];
	\end{tikzpicture}
	\begin{sol}
		$a=-2/\sqrt{3}$, $b=2/\sqrt{3}$
	\end{sol}
\end{ex}

\begin{ex}
	Find an equation for the tangent line on the graph below:
	
	\begin{tikzpicture}[scale=0.6]
	\drawgridxxyyb{-4}{4}{-5}{5}
	\draw[black, thick, domain=-2.5:2.5, smooth] plot (\x, {\x*\x*\x - 4*\x});
	\draw  (3,5) node [right] {\small $y=x^3-4x$};
	\draw[gray, thick, domain=-3.5:2.5, samples=2] plot (\x, {-\x - 2});
	\draw[fill] (1,-3) circle [radius=0.05];
	\draw (1.5,-3) node [right] {$(1,-3)$};
	\draw (2.5,-4.5) node [right] {???};	
	\end{tikzpicture}
	\begin{sol}
		$y+3=-(x-1)$
	\end{sol}
\end{ex}

\begin{ex}
	A farmer wants to build a 4,000 square foot rectangular pen. The material for 3 sides of the pen cost \$1 per foot while the fourth will cost \$4 per foot. Find the dimensions of the cheapest such pen.

	\begin{sol}
		The expensive side should be 40 feet long, and the adjacent sides, 100 feet.
		\end{sol}
\end{ex}


\begin{ex}
		\begin{multicols}{2}
		A box is being designed with volume 72 in$^3$. The box must have height, $h$, equal to twice its width, $x$. What is the smallest possible surface area of such a box? What dimensions should be used? In what sort of practical scenario might these constraints make sense?
		
		%[Hint: first write the surface area as a function of the width, $x$.]
		\begin{center} 
			\begin{tikzpicture}[scale=0.5]
			\draw (0,0) -- (2,0) -- (2,4) -- (0,4) -- (0,0);
			\draw (2,0) -- (2.5,0.5) -- (2.5,4.5) -- (0.5, 4.5) -- (0,4);
			\draw (2,4) -- (2.5,4.5);
			\draw (0,2) node [left] {$h$};
			\draw (1,0) node [below] {$x$};
			%\draw (2.25,0.25) node[below right] {$d$};
			\end{tikzpicture}
		\end{center}
	\end{multicols}
	\begin{sol}
		The smallest surface area is 108 in$^2$. The box should have $x=3$ in, $h = 6$ in, and should measure 4 in the remaining direction.
		
		This might make sense if we were making a box for a fixed volume (cereal, spam, etc.) on which we wanted to print a logo or photo. Perhaps the logo or photo could be resized but we want it to keep its shape.
		\end{sol}
\end{ex}

\begin{ex}
	What might be the difficulty in calculating derivatives of the trigonometric and logarithmic functions?
	\begin{sol}
		Simplifying the difference quotient the way we have done in this section is difficult especially because of the term $f(x+h)$.
	\end{sol}
\end{ex}


\Closesolutionfile{ans}
\subsection*{Answers \nopunct} \hfill
\begin{multicols}{2}
	\input{ans41}
\end{multicols}

\newpage
\section{One-to-one functions, inverses, the arctangent}

If for all $x$ and $y$, $f(x)=y \iff g(y) = x$ we call $f$ and $g$ \textbf{inverses} of each other (\textbf{inverse functions})\index{inverse functions}.
\qi{This generalizes the idea of ``reversing'' functions.}
\qi{i.e. $g$ takes $y$ back to $x$.}
\qi{To find $g(y)$ from $f(x)$, write out $y=f(x)$ and solve for $x$, if possible. If you want $g(x)$, you must unnaturally switch $x$ and $y$.}
\qi{$y=f(x)$ and $x=g(y)$ necessarily have the same graph, but if you've switched $x$ and $y$, then $y=f(x)$ and $y=g(x)$ have symmetry over the line $y=x$.}
\qi{Domain and range of $f$ and $g$ are swapped.}
\qi{$g$ is sometimes called $f^{-1}$ to avoid using another name and to mark the relationship to $f$. Remember \textit{this does not mean reciprocal}.}
\qi{Another characterization of inverse functions: $f$ and $g$ are inverses if
\begin{enumerate}
	\item $f(g(x)) = x$ for all $x$ in the domain of $g$, and
	\item$g(f(x)) = x$ for all $x$ in the domain of $f$.
\end{enumerate}
}

The following are equivalent:
\begin{enumerate}
	\item An inverse function exists for $f$ (i.e. $f$ is \textbf{invertible}\index{invertible}).
	\item $f$ is \textbf{one-to-one}\index{one-to-one}, i.e. if $f(a)=f(b)$ then $a=b$.
	\item $f$ passes the \textbf{Horizontal Line Test}.\index{Horizontal Line Test}
\end{enumerate}

Sometimes a function fails to be pass the Horizontal Line Test, but we restrict the domain of the function so it does. The resulting invertible function is used to create a sort-of partial inverse function.
\qi{e.g. taking the right or left half of a quadratic function.}
\qi{e.g. the arctangent function. Arctan is the inverse of $y=\tan(x)$ restricted to $-\dfrac{\pi}{2} < x < \dfrac{\pi}{2}$. Thus the arctangent always gives an angle from this interval.}


\Opensolutionfile{ans}[ans42]
\subsection*{Exercises \nopunct} \hfill

Determine if the the given function is one-to-one. If so, find its inverse. Check your answers algebraically and graphically.  Verify that the range of $f$ is the domain of $f^{-1}$ and vice-versa.

\begin{multicols}{2}
\begin{ex}
	$f(x) = 6x - 2$
	\begin{sol}
		$f^{-1}(x) = \dfrac{x + 2}{6}$
	\end{sol}
\end{ex}

\begin{ex}
	$f(x) = \dfrac{x-2}{3} + 4$
	\begin{sol}
		$f^{-1}(x) = 3x-10$
	\end{sol}
\end{ex}

\begin{ex}
	$f(x) = \sqrt{3x-1}+5$
	\begin{sol}
		$f^{-1}(x) = \frac{1}{3}(x-5)^2+\frac{1}{3}$, $x \geq 5$
	\end{sol}
\end{ex}

\begin{ex}
	$f(x) = 2-\sqrt{x - 5}$
	\begin{sol}
		$f^{-1}(x) = (x - 2)^{2} + 5, \; x \leq 2$
	\end{sol}
\end{ex}

\begin{ex}
	$f(x) = \sqrt[5]{3x-1}$
	\begin{sol}
	$f^{-1}(x) = \frac{1}{3} x^{5} + \frac{1}{3}$	
	\end{sol}
\end{ex}


\begin{ex}
	$f(x) = x^2-2x+4$
	\begin{sol}
		$f$ is not one-to-one.
	\end{sol}
\end{ex}

\begin{ex}
	$f(x) = x^2 - 10x$, $x \geq 5$
	\begin{sol}
		$f^{-1}(x) = 5 + \sqrt{x+25}$
	\end{sol}
\end{ex}


\begin{ex}
	$f(x) = \sin(x)$
	\begin{sol}
		$f$ is not one-to-one; it's graph fails the Horizontal Line Test.
	\end{sol}
\end{ex}

\begin{ex}
	$f(x) = x^2-6x+5, \; x \leq 3$
	\begin{sol}
		$f^{-1}(x) = 3 - \sqrt{x+4}$
	\end{sol}
\end{ex}

\begin{ex}
	$f(x) = \ln (3x-1)+6$
	\begin{sol}
		$f^{-1}(x) = \dfrac{e^{x-6}+1}{3}$	
	\end{sol}
\end{ex}


\begin{ex}
	$f(x) = \dfrac{3}{4-x}$
	\begin{sol}
		$f^{-1}(x) = \dfrac{4x-3}{x}$
	\end{sol}
\end{ex}

\begin{ex}
	$f(x) = \dfrac{2x-1}{3x+4}$
	\begin{sol}
		$f^{-1}(x) = \dfrac{4x+1}{2-3x}$
	\end{sol}
\end{ex}

\begin{ex}
	$f(x) = \dfrac{-3x - 2}{x + 3}$ 
	\begin{sol}
	$f^{-1}(x) = \dfrac{-3x - 2}{x + 3}$	
	\end{sol}
\end{ex}

\begin{ex}
	$f(x) = 7\cdot 2^{x-3}-4$
	\begin{sol}
		$f^{-1}(x) = \log_2\left(\dfrac{x+4}{7}\right)+3$	
	\end{sol}
\end{ex}

\end{multicols}

\begin{ex}
	The height of a projectile off the ground, in feet, $t$ seconds after launch is given by $h(t) = -16t^2+224t+10$.
	\ssp
	\item What is an appropriate applied domain for $h$?
	\item Is $h$ invertible?
	\item At what time does the projectile reach its maximum height?
	\item If we restrict $t$ to be between 0 and the time of maximum height, is $h$ invertible?
	\item Find the inverse function suggested in part (d). What does this function calculate?
	\esp
	\begin{sol}
	\ssp
	\item $ \left[0, \dfrac{28+\sqrt{794}}{4}\right] \approx [0, 14.04]$.
	\item No.
	\item $t=7$s after launch.
	\item Yes.
	\item $t = \dfrac{28-\sqrt{794-h}}{4}$. This finds the time after launch given the height, assuming the projectile is not post-peak. 
	\esp		
	\end{sol}
\end{ex}


\begin{ex}
	Find a problem in Section \ref{lineq} where you calculated an inverse function.
	\begin{sol}
		Exercise \ref{FCtemp}.
	\end{sol}
\end{ex}


\begin{ex}
	Find a problem in Section \ref{lineq} where you performed function composition.
\begin{sol}
	Exercise \ref{FComp}.
\end{sol}
\end{ex}

Evaluate each expression. Give exact answers in radians. Note there are \underline{not} multiple correct answers.
\begin{multicols}{2}
\begin{ex}
	$\arctan(0)$
	\begin{sol}
		$0$
	\end{sol}
\end{ex}

\begin{ex}
	$\arctan(1)$
	\begin{sol}
		$\dfrac{\pi}{4}$
	\end{sol}
\end{ex}


\begin{ex} \label{sqrt3tan}
	$\arctan\left(\sqrt{3} \right)$
	\begin{sol}
		$\dfrac{\pi}{3}$
	\end{sol}
\end{ex}


\begin{ex}
	$\arctan\left(-\dfrac{\sqrt{3}}{3} \right)$
	\begin{sol}
		$-\dfrac{\pi}{6}$
	\end{sol}
\end{ex}

\begin{ex}
	$\arctan\left(-1 \right)$
	\begin{sol}
		$-\dfrac{\pi}{4}$
	\end{sol}
\end{ex}

\end{multicols}


\begin{ex}
	Solve the equation exactly. How is this different than \#\ref{sqrt3tan}? $$ \tan(x) = \sqrt{3} $$
	\begin{sol}
		$x = \dfrac{\pi}{3} + 2\pi k$, $k\in\Z$. Here we are asked to solve an equation for which there are an infinite number of solutions (one of which is $\arctan(\sqrt{3})$). In \#\ref{sqrt3tan} we evaluated an expression without any variables. This should yield a unique number if it's defined.
	\end{sol}
\end{ex}

\begin{ex}
	Approximate all solutions to the equation where $0 \leq x < 2\pi$. $$ \tan(x) = -4 $$
	\begin{sol}
		$x \approx 1.816$ or $x \approx 4.957$.
		\end{sol}
\end{ex}

\begin{ex}
	Find the degree measure of the smallest positive angle in standard position with terminal side through the given point. Approximate to the nearest tenth.
	\ssp
	\item $(4,\ 5)$
	\item $(-3,\ 8.5)$
	\item $(-44, -108)$
	\item $(0.115,\ -0.371)$
	\esp
	\begin{sol}
			\ssp
		\item $51.3\dg$
		\item $ 109.4\dg$
		\item $ 247.8\dg$
		\item $ 287.2 \dg$
		\esp
		\end{sol}
\end{ex}

\begin{ex}
	Consider the points $O(0,0)$, $A(3,1)$, and $B(-5,2)$. Find the measure of the angle $AOB$ measured counterclockwise using arctangent
	\begin{sol}
		$\pi + \arctan\left( -\dfrac{2}{5} \right) - \arctan\left( \dfrac{1}{3} \right) \approx 2.439$ rad $\approx 139.8\dg$.
	\end{sol}
\end{ex}


\Closesolutionfile{ans}
\subsection*{Answers \nopunct} \hfill
\begin{multicols}{2}
	\input{ans42}
\end{multicols}


\newpage

\section{Higher-order polynomials and division}

\begin{tikzpicture} [scale=1.6]
\drawgridxxyyb{-1}{4}{-1}{2}

\draw[<->, thick, samples=35, domain=-0.32:1.58, smooth] plot (2*\x, {20*(\x)^3*(\x-1)*(\x-1)*(\x-1.5)}) node [above] {\small $y=20x^3(x-1)^2(x-1.5)=20x^6-70x^5+80x^4-30x^3$};

%x-scale is 2.0!!!!!!!!

\draw[<-] (-0.1,-0.1) -- (-0.5, -0.5) node [below left] {\small zero 0, multiplicity 3};

\draw[<-] (1.9, 0.1) -- (1.5, 0.5) node [above] {\small zero 1, multiplicity 2};

\draw[<-] (3.1, -0.1) -- (3.5, -0.5) node [below right] {\small zero 1.5, multiplicity 1};

\end{tikzpicture}

\bigskip
Higher order polynomials\index{higher order polynomials}\index{polynomials} and their graphs:
\begin{itemize}
	\item \textbf{Leading term}\index{leading term} is easy to compute and reveals end-behavior. [$20x^6$ in the example graph.]
	\qi{The power on the leading term (highest power term) is the degree of the polynomial. [For technical reasons, the zero polynomial is said to have undefined degree.]}
	\item \textbf{Constant term}\index{constant term} is also easy to compute and is the $y$-value of the $y$-intercept. [Implicitly 0 in the example.]
	\item Polynomial functions have domain $\R$ and are also \textbf{smooth}\index{smooth} everywhere. Smooth means a non-vertical tangent exists (i.e. the derivative exists). Thus there are no breaks, sharp corners, or vertical asymptotes in the graph).
	\item When factored, \textbf{zeros}\index{zeros} (i.e. roots or $x$-values of $x$-intercepts) and their \textbf{multiplicities}\index{multiplicity} are revealed.
	\begin{itemize}
		\item The graph ``crosses'' the $x$-axis at \textbf{odd multiplicity} zeros.
		\item The graph ``bounces'' off the $x$-axis at \textbf{even multiplicity} zeros.
		\item This follows from the signs ($\pm$) of the factors near the zero.
		\item The graph is tangential with the $x$-axis at all zeros except those with multiplicity 1.
	\end{itemize}
 \item \textbf{Division of polynomials}\index{polynomial division} is analogous to integer division: If the degree of $d(x)$ is less than or equal $p(x)$ then there exist two polynomials $q(x)$ and $r(x)$ where $r(x)$ is either 0 or has degree less than $d(x)$ such that \fbox{$p(x) = d(x)q(x)+r(x)$}.
 \item Alternatively \fbox{$\dfrac{p(x)}{d(x)} = q(x) + \dfrac{r(x)}{d(x)}$}.
 \item \textbf{The remainder theorem}\index{remainder theorem}: If $p(x)$ is a polynomial of degree 1 or higher, the remainder of division by $(x-a)$ is $p(a)$.
 \item \textbf{The factor theorem}\index{factor theorem}: If $p$ is a nonzero polynomial, $a$ is a zero of $p$ if and only if $(x-a)$ is a factor.
\end{itemize}

\Opensolutionfile{ans}[ans43]
\subsection*{Exercises \nopunct} \hfill

Find the degree, the leading term, the leading coefficient, the constant term and the end behavior of the given polynomial.

\begin{multicols}{2}

\begin{ex}
	$f(x) = 4-x-3x^2$ 
	\begin{sol}
		$f(x) = 4-x-3x^2$ \\
		Degree 2 \\
		Leading term $-3x^{2}$\\
		Leading coefficient $-3$\\
		Constant term $4$\\
		As $x \rightarrow -\infty, \; f(x) \rightarrow -\infty$\\
		As $x \rightarrow \infty, \; f(x) \rightarrow -\infty$
	\end{sol}
\end{ex}

\begin{ex}
	 $g(x) = 3x^5 - 2x^2 + x + 1$
	\begin{sol}
		$g(x) = 3x^5 - 2x^2 + x + 1$ \\
		Degree 5 \\
		Leading term $3x^5$\\
		Leading coefficient $3$\\
		Constant term $1$\\
		As $x \rightarrow -\infty, \; g(x) \rightarrow -\infty$\\
		As $x \rightarrow \infty, \; g(x) \rightarrow \infty$
	\end{sol}
\end{ex}

\begin{ex}
	$q(r) = 1 - 16r^{4}$
	\begin{sol}
		$q(r) = 1 - 16r^{4}$\\
		Degree 4 \\
		Leading term $-16r^{4}$\\
		Leading coefficient $-16$\\
		Constant term $1$\\
		As $r \rightarrow -\infty, \; q(r) \rightarrow -\infty$\\
		As $r \rightarrow \infty, \; q(r) \rightarrow -\infty$
	\end{sol}
\end{ex}

\begin{ex}
	$Z(b) = 42b - b^{3}$
	\begin{sol}
	$Z(b) = 42b - b^{3}$\\
	Degree 3 \\
	Leading term $-b^{3}$\\
	Leading coefficient $-1$\\
	Constant term $0$\\
	As $b \rightarrow -\infty, \; Z(b) \rightarrow \infty$\\
	As $b \rightarrow \infty, \; Z(b) \rightarrow -\infty$	
	\end{sol}
\end{ex}

\begin{ex}
	$f(x) = \sqrt{3}x^{17} + 22.5x^{10} - \pi x^{7} + \frac{1}{3}$
	\begin{sol}
		$f(x) = \sqrt{3}x^{17} + 22.5x^{10} - \pi x^{7} + \frac{1}{3}$\\
		Degree 17 \\
		Leading term $\sqrt{3}x^{17}$\\
		Leading coefficient $\sqrt{3}$\\
		Constant term $\frac{1}{3}$\\
		As $x \rightarrow -\infty, \; f(x) \rightarrow -\infty$\\
		As $x \rightarrow \infty, \; f(x) \rightarrow \infty$
	\end{sol}
\end{ex}

\begin{ex}
	 $s(t) = -4.9t^{2} + v_{\mbox{\tiny $0$}}t + s_{\mbox{\tiny $0$}}$
	\begin{sol}
		$s(t) = -4.9t^{2} + v_{\mbox{\tiny $0$}}t + s_{\mbox{\tiny $0$}}$\\
		Degree 2 \\
		Leading term $-4.9t^{2}$\\
		Leading coefficient $-4.9$\\
		Constant term $s_{\mbox{\tiny $0$}}$\\
		As $t \rightarrow -\infty, \; s(t) \rightarrow -\infty$\\
		As $t \rightarrow \infty, \; s(t) \rightarrow -\infty$
	\end{sol}
\end{ex}

\begin{ex}
	$P(x) = (x - 1)(x - 2)(x - 3)(x - 4)$
	\begin{sol}
		$P(x) = (x - 1)(x - 2)(x - 3)(x - 4)$\\
		Degree 4 \\
		Leading term $x^{4}$\\
		Leading coefficient $1$\\
		Constant term $24$\\
		As $x \rightarrow -\infty, \; P(x) \rightarrow \infty$\\
		As $x \rightarrow \infty, \; P(x) \rightarrow \infty$
	\end{sol}
\end{ex}

\begin{ex}
	$p(t) = -t^2(3 - 5t)(t^{2} + t + 4)$
	\begin{sol}
	$p(t) = -t^2(3 - 5t)(t^{2} + t + 4)$\\
	Degree 5 \\
	Leading term $5t^{5}$\\
	Leading coefficient $5$\\
	Constant term $0$\\
	As $t \rightarrow -\infty, \; p(t) \rightarrow -\infty$\\
	As $t \rightarrow \infty, \; p(t) \rightarrow \infty$	
	\end{sol}
\end{ex}

\begin{ex}
	$f(x) = -2x^3(x+1)(x+2)^2$
	\begin{sol}
	$f(x) = -2x^3(x+1)(x+2)^2$ \\
	Degree 6 \\
	Leading term $-2x^{6}$\\
	Leading coefficient $-2$\\
	Constant term $0$\\
	As $x \rightarrow -\infty, \; f(x) \rightarrow -\infty$\\
	As $x \rightarrow \infty, \; f(x) \rightarrow -\infty$	
	\end{sol}
\end{ex}

\begin{ex}
	$G(t) = 4(t-2)^2\left(t+\frac{1}{2}\right)$ 
	\begin{sol}
	$G(t) = 4(t-2)^2\left(t+\frac{1}{2}\right)$ \\
	Degree 3 \\
	Leading term $4t^3$\\
	Leading coefficient $4$\\
	Constant term $8$\\
	As $t \rightarrow -\infty, \; G(t) \rightarrow -\infty$\\
	As $t \rightarrow \infty, \; G(t) \rightarrow \infty$	
	\end{sol}
\end{ex}
\end{multicols}


Find the real zeros of the given polynomial and their corresponding multiplicities.  Use this information along with a sign chart to provide a rough sketch of the graph of the polynomial.  Compare your answer with the result from a graphing utility.

\begin{multicols}{2}

\begin{ex}
	$a(x) = x(x + 2)^{2}$
	\begin{sol}
		$a(x) = x(x + 2)^{2}$\\
		$x = 0$ multiplicity 1\\
		$x = -2$ multiplicity 2
	\end{sol}
\end{ex}

\begin{ex}
	$g(x) = x(x + 2)^{3}$
	\begin{sol}
		$g(x) = x(x + 2)^{3}$\\
		$x = 0$ multiplicity 1\\
		$x = -2$ multiplicity 3
	\end{sol}
\end{ex}

\begin{ex}
	 $f(x) = -2(x-2)^2(x+1)$
	\begin{sol}
		$f(x) = -2(x-2)^2(x+1)$\\
		$x=2$ multiplicity 2 \\
		$x=-1$ multiplicity 1
	\end{sol}
\end{ex}

\begin{ex}
$g(x) = (2x+1)^2(x-3)$	
	\begin{sol}
		$g(x) = (2x+1)^2(x-3)$\\
		$x=-\frac{1}{2}$ multiplicity 2 \\
		$x=3$ multiplicity 1
	\end{sol}
\end{ex}

\begin{ex}
	$F(x) = x^{3}(x + 2)^{2}$
	\begin{sol}
		$F(x) = x^{3}(x + 2)^{2}$\\
		$x = 0$ multiplicity 3\\
		$x = -2$ multiplicity 2
	\end{sol}
\end{ex}

\begin{ex}
	$P(x) = (x - 1)(x - 2)(x - 3)(x - 4)$
	\begin{sol}
		$P(x) = (x - 1)(x - 2)(x - 3)(x - 4)$\\
		$x = 1$ multiplicity 1\\
		$x = 2$ multiplicity 1\\
		$x = 3$ multiplicity 1\\
		$x = 4$ multiplicity 1
	\end{sol}
\end{ex}

\begin{ex}
	$Q(x) = (x + 5)^{2}(x - 3)^{4}$
	\begin{sol}
	$Q(x) = (x + 5)^{2}(x - 3)^{4}$\\
	$x = -5$ multiplicity 2\\
	$x = 3$ multiplicity 4	
	\end{sol}
\end{ex}

\begin{ex}
	$h(x) = x^2(x-2)^2(x+2)^2$
	\begin{sol}
		$h(x) = x^2(x-2)^2(x+2)^2$\\
		$x = -2$ multiplicity 2\\
		$x = 0$ multiplicity 2\\
		$x = 2$ multiplicity 2
	\end{sol}
\end{ex}

\begin{ex}
	$H(t) = (3-t)(t^2+1)$
	\begin{sol}
		$H(t) = (3-t)\left(t^2+1\right)$\\
		$x =3$ multiplicity 1
	\end{sol}
\end{ex}

\begin{ex}
	$Z(b) = b(42 - b^{2})$ 
	\begin{sol}
	$Z(b) = b(42 - b^{2})$\\
	$b = -\sqrt{42}$ multiplicity 1\\
	$b = 0$ multiplicity 1\\
	$b = \sqrt{42}$ multiplicity 1\\
	Remember real zeros need not be rational or integral!
	\end{sol}
\end{ex}

\end{multicols}

\begin{ex}
	 You will be constructing shipping boxes where at least one side and the side opposite it are square. The length of the box is the distance between these two square sides, and the girth is the perimeter of the square side. A shipping company requires that the sum of the length plus girth of a box must not exceed 108 in. Find the dimensions of the boxes with maximum volume that meet these criteria.\begin{sol}
	 %	If the square side measures $x\times x$ in$^2$ then the volume is $V(x) = -4x^3+108x^2$. We assume $x>0$ and look for a horizontal tangent.
	 The boxes should be 18 in by 18 in by 36 in.
	 \end{sol}
\end{ex}

Perform the indicated division.  Write the polynomial in the form $p(x) = d(x)q(x) + r(x)$.

\begin{multicols}{2}
	
	\begin{ex}
		$\left(4x^2+3x-1 \right) \div (x-3)$ 
		\begin{sol}
			 $4x^2+3x-1 = (x-3)(4x+15) + 44$
		\end{sol}
	\end{ex}
	
\begin{ex}
	$\left(2x^3-x+1 \right) \div \left(x^{2} +x+1 \right)$ 
	\begin{sol}
		$2x^3-x+1 = \left(x^2+x+1\right)(2x-2)+(-x+3)$
	\end{sol}
\end{ex}
	
\begin{ex}
	$\left(5x^{4} - 3x^{3} + 2x^{2} - 1 \right) \div \left(x^{2} + 4 \right)$
	\begin{sol}
		$5x^{4} - 3x^{3} + 2x^{2} - 1 = \left(x^{2} + 4 \right) \left(5x^{2} - 3x - 18 \right) + (12x + 71)$
	\end{sol}
\end{ex}
	
\begin{ex}
	$\left(-x^{5} + 7x^{3} - x \right) \div \left(x^{3} - x^{2} + 1 \right)$
	
	\begin{sol}
	$-x^{5} + 7x^{3} - x =$\\ $\left(x^{3} - x^{2} + 1 \right) \left(-x^{2} - x + 6 \right) + \left(7x^{2} - 6 \right)$	
	\end{sol}
\end{ex}
	
\begin{ex}
	$\left(9x^{3} + 5 \right) \div \left(2x - 3 \right)$
	\begin{sol}
		$9x^{3} + 5 =(2x - 3) \left(\frac{9}{2}x^{2} + \frac{27}{4}x + \frac{81}{8} \right) + \frac{283}{8}$
	\end{sol}
\end{ex}
	
\begin{ex}
	$\left(4x^2 - x - 23 \right) \div \left(x^{2} - 1 \right)$
	\begin{sol}
		$4x^2 - x - 23 = \left(x^{2} - 1 \right)(4) + (-x - 19)$
	\end{sol}
\end{ex}

\begin{ex}
	 $\left(3x^2-2x+1 \right) \div \left(x-1\right)$ 
	\begin{sol}
		$\left(3x^2-2x+1 \right) = \left(x-1\right) (3x+1)+2$
	\end{sol}
\end{ex}

\begin{ex}
	 $\left(x^2-5 \right) \div \left(x-5\right)$
	\begin{sol}
		$\left(x^2-5 \right)= \left(x-5\right)(x+5) + 20$
	\end{sol}
\end{ex}

\begin{ex}
	$\left(3-4x-2x^2 \right) \div \left(x+1\right)$
	\begin{sol}
		 $\left(3-4x-2x^2 \right) = \left(x+1\right)(-2x-2)+5$
	\end{sol}
\end{ex}

\begin{ex}
	$\left(4x^2-5x +3\right) \div \left(x+3\right)$
	\begin{sol}
		$\left(4x^2-5x +3\right) = \left(x+3\right)(4x-17)+54$
	\end{sol}
\end{ex}
	
\begin{ex}
	 $\left(x^3 + 8 \right) \div \left(x+2\right)$
	\begin{sol}
		 $\left(x^3 + 8 \right) = \left(x+2\right) \left(x^2-2x+4\right) + 0$
	\end{sol}
\end{ex}

\begin{ex}
	 $\left(4x^3 +2x-3 \right) \div \left(x -3\right)$
	\begin{sol}
		$\left(4x^3 +2x-3 \right) = $\\
$\left(x -3\right) \left(4x^2+12x+38\right) + 111$
	\end{sol}
\end{ex}

\begin{ex}
	$\left(18x^2-15x-25\right) \div \left(x - \frac{5}{3} \right)$
	\begin{sol}
		$\left(18x^2-15x-25\right) = \left(x - \frac{5}{3} \right)(18x+15)+0$
	\end{sol}
\end{ex}

\begin{ex}
	$\left(4x^2-1 \right) \div \left(x - \frac{1}{2} \right)$
	\begin{sol}
		$\left(4x^2-1 \right) = \left(x - \frac{1}{2} \right)(4x+2)+0$
	\end{sol}
\end{ex}
	
\begin{ex}
	$\left(2x^3 - 3x +1 \right) \div \left(x - \frac{1}{2} \right)$
	\begin{sol}
		$\left(2x^3 - 3x +1 \right) = $\\
		$\left(x - \frac{1}{2} \right) \left(2x^2+x-\frac{5}{2}\right)-\frac{1}{4}$
	\end{sol}
\end{ex}

\end{multicols}


Determine $p(c)$ using the Remainder Theorem for the given polynomial functions and value of $c$.  If $p(c) = 0$, factor $p(x) = (x-c) q(x)$.

\begin{multicols}{2}
	
\begin{ex}
	$p(x) = 2x^2 - x + 1$, $c = 4$ 
	\begin{sol}
		$p(4) = 29$
	\end{sol}
\end{ex}

\begin{ex}
	$p(x) = 4x^2-33x-180$, $c = 12$
	\begin{sol}
		$p(12) =0$, $p(x) = (x-12)(4x+15)$
	\end{sol}
\end{ex}

\begin{ex}
	$p(x) = 2x^3 - x + 6$, $c=-3$
	\begin{sol}
		$p(-3)=-45$
	\end{sol}
\end{ex}

\begin{ex}
	$p(x) = x^3+2x^2+3x+4$, $c =-1$
	\begin{sol}
		$p(-1)=2$
	\end{sol}
\end{ex}
	
\begin{ex}
	 $p(x) =3x^3-6x^2+4x-8$, $c=2$
	\begin{sol}
		$p(2) =0$, $p(x)= (x-2) \left(3x^2+4\right)$
	\end{sol}
\end{ex}

\begin{ex}
	$p(x) = 8x^3+12x^2+6x+1$, $c =-\frac{1}{2}$
	\begin{sol}
		$p\left(-\frac{1}{2}\right) = 0$, $p(x)  = \left(x+\frac{1}{2}\right)\left(8x^2+8x+2\right)$
	\end{sol}
\end{ex}
\end{multicols}



You are given a polynomial and one of its zeros.  Use the techniques in this section to find the rest of the real zeros and factor the polynomial.  


\begin{multicols}{2}

\begin{ex}
	 $x^{3} - 6x^{2} + 11x - 6, \;\; c = 1$ 
	\begin{sol}
		 $x^{3} - 6x^{2} + 11x - 6 = (x - 1)(x - 2)(x - 3)$
	\end{sol}
\end{ex}

\begin{ex}
	$x^{3} - 24x^{2} + 192x - 512, \;\; c = 8$
	\begin{sol}
		$x^{3} - 24x^{2} + 192x - 512 = (x - 8)^{3}$
	\end{sol}
\end{ex}


\begin{ex}
	$3x^{3} + 4x^{2} - x - 2, \;\; c = \frac{2}{3}$
	\begin{sol}
		$3x^{3} + 4x^{2} - x - 2 = 3\left(x - \frac{2}{3}\right)(x + 1)^{2}$
	\end{sol}
\end{ex}

\begin{ex}
	$2x^3-3x^2-11x+6, \;\; c=\frac{1}{2}$
	\begin{sol}
		 $2x^3-3x^2-11x+6 = 2\left(x-\frac{1}{2}\right)(x+2)(x-3)$
	\end{sol}
\end{ex}


\begin{ex}
	$x^3+2x^2-3x-6, \;\; c = -2$
	\begin{sol}
		$x^3+2x^2-3x-6 = (x+2)(x+\sqrt{3})(x-\sqrt{3})$
	\end{sol}
\end{ex}
\end{multicols}


\begin{ex} \label{irrationalFactor}
	Factor $x^2-2x-2$ into a product of linear terms.
	\begin{sol}
		 $x^{2} - 2x - 2 = (x - (1 - \sqrt{3}))(x - (1 + \sqrt{3}))$
	\end{sol}
\end{ex}

\begin{ex}
	The polynomials $x^2-a^2$, $x^3-a^3$,  $x^4-a^4$, and  $x^5-a^5$   obviously have zero $a$. Use division to show how they (start to) factor. What do you think about $x^n-a^n$?
	\begin{sol}
		$(x-a)(x+a)$,\\
		$(x-a)(x^2+ax+a^2)$,\\ $(x-a)(x^3+ax^2 + a^2x +a^3)$,\\ $(x-a)(x^4+ax^3 + a^2x^2 +a^3x + a^4)$
	\end{sol}
\end{ex}

\Closesolutionfile{ans}
\subsection*{Answers \nopunct} \hfill
\begin{multicols}{2}
	\input{ans43}
\end{multicols}

\newpage
\section{The rational zeros theorem, complex numbers, and factoring}

\textbf{The rational zeros theorem}\index{rational zeros theorem}: Let $p(x) = ax^n + \cdots + b$ be a polynomial with \underline{all integral coefficients} and $b\neq0$ [i.e. $a$ is the leading coefficient and $b$, the constant term]. If $r$ is a rational zero of $p$ then $r = \pm \dfrac{b^*}{a^*}$ where $b^*$ is a factor of $b$ and $a^*$ is a factor of $a$.

\qi{Remember to include 1 on the list of factors of $a$ and $b$. These are integral factors as from elementary school.}
\qi{This gives a list of ``good guesses'' for zeros -- any number from all to none of them might be zeros.}

Complex Zeros\index{complex zeros}
\begin{itemize}
	\item What you get when you allow \textbf{imaginary numbers}\index{imaginary numbers}  when solving quadratic equations.
	\item \textbf{Imaginary numbers}: numbers whose square is negative but otherwise seem to satisfy the basic rules of algebra.
	\qi{Denoted by radicals or by the imaginary unit $i=\sqrt{-1}$. Note normally such an expression is undefined.} 
	\qi{Comparisons ($>$, $<$, etc.) don't work since you can't put them on a number line. This implies some things get mucked-up and we will only use imaginary numbers when explicit about it. You will learn the details if you go on in mathematics -- even just for engineering as complex numbers have many uses there!}
	\item A \textbf{complex number}\index{complex numbers} is a number that can be written in the form $a+bi$ where the real part and imaginary part are the real numbers $a$ and $b$. [It's complex in the sense it has two parts.]
	\qi{Complex numbers correspond to solutions of quadratic equations. They are strictly complex (i.e. imaginary part is non-zero) when there are no real solutions.}
	\qi{Quadratic expressions can then always be factored into a product of linear terms, using the complex zeros. Remember some quadratic expressions cannot be factored like this using real numbers alone -- those are called ``irreducible'' quadratics (they are irreducible over the Real Numbers).}
\end{itemize}

\textbf{Conjugate pairs}\index{conjugate pairs}: If a polynomial has real coefficients then complex zeros always occur in conjugate pairs: $a+bi$ and $a-bi$.
\qi{Not surprising as this is how they occur for irreducible quadratic expressions. The product of $(x-z_1)(x-z_2)$ where $z_1$ and $z_2$ are complex conjugates is always a quadratic expression with real coefficients.  }

\textbf{Fundamental theorem of algebra}\index{fundamental theorem of algebra}: every polynomial of degree $n$ has exactly $n$ complex roots, counting multiplicity.

Consequence of these facts: If a polynomial has real coefficients it must factor into a product of linear and/or irreducible quadratic terms. There's no such thing as an irreducible polynomial of degree three or higher. [But this does not imply we can always find them readily. For a good story, look up \'{E}variste Galois. For a second good story, read about his contributions to mathematics.]

\Opensolutionfile{ans}[ans44]
\subsection*{Exercises \nopunct} \hfill

	
Use the Rational Zeros Theorem to make a list of possible rational zeros.

\begin{multicols}{2}


\begin{ex}
	$f(x) = x^{3} - 2x^{2} - 5x + 6$ 
	\begin{sol}
		Possible rational zeros are $\pm 1$, $\pm 2$, $\pm 3$, $\pm 6$
	\end{sol}
\end{ex}


\begin{ex}
	 $f(x) = -2x^{3} + 19x^{2} - 49x + 20$
	\begin{sol}
		Possible rational zeros are  $\pm \frac{1}{2}$, $\pm 1$, $\pm 2$, $\pm \frac{5}{2}$, $\pm 4$, $\pm 5$, $\pm 10$, $\pm 20$ 
	\end{sol}
\end{ex}


\begin{ex}
	$f(x) = x^{3} - 7x^{2} + x - 7$
	\begin{sol}
		  Possible rational zeros are $\pm 1$, $\pm 7$
	\end{sol}
\end{ex}

\begin{ex}
	$f(x) = 3x^{3} + 3x^{2} - 11x - 10$
	\begin{sol}
	 Possible rational zeros are $\pm \frac{1}{3}$, $\pm \frac{2}{3}$, $\pm \frac{5}{3}$, $\pm \frac{10}{3}$, $\pm 1$, $\pm 2$, $\pm 5$, $\pm 10$	
	\end{sol}
\end{ex}
\end{multicols}

Find the real zeros of the polynomial and their multiplicities.

\begin{multicols}{2}

\begin{ex}
	$f(x) = x^{3} - 2x^{2} - 5x + 6$
	\begin{sol}
		$x = -2$, $x = 1$, $x = 3$ (each has mult. 1)
	\end{sol}
\end{ex}


\begin{ex}
	$f(x) = x^{4} + 2x^{3} - 12x^{2} - 40x - 32$
	\begin{sol}
		$x = -2$ (mult. 3), $x = 4$ (mult. 1)
	\end{sol}
\end{ex}


\begin{ex}
	 $f(x) = x^{4} - 9x^{2} - 4x + 12$
	\begin{sol}
		 $x = -2$ (mult. 2), $x = 1$ (mult. 1), $x = 3$ (mult. 1)
	\end{sol}
\end{ex}

\begin{ex}
	 $f(x) = x^{3} - 7x^{2} + x - 7$
	\begin{sol}
		$x = 7$ (mult. 1)
	\end{sol}
\end{ex}


\begin{ex}
	$f(x) = -2x^{3} + 19x^{2} - 49x + 20$
	\begin{sol}
		 $x = \frac{1}{2}$, $x = 4$, $x = 5$ (each has mult. 1)
	\end{sol}
\end{ex}


\begin{ex}
	$f(x) = -17x^{3} + 5x^{2} + 34x - 10$
	\begin{sol}
		$x = \frac{5}{17}$, $x = \pm \sqrt{2}$ (each has mult. 1)
	\end{sol}
\end{ex}


\begin{ex}
	$f(x) = 2x^4+x^3-7x^2-3x+3$
	\begin{sol}
		 $x = -1$, $x = \frac{1}{2}$, $x=\pm \sqrt{3}$ (each mult. 1)
	\end{sol}
\end{ex}


\begin{ex}
	$f(x) = 3x^4-14x^2-5$
	\begin{sol}
		$x = \pm \sqrt{5}$ (each has mult. 1)
	\end{sol}
\end{ex}

\end{multicols}

Find all of the zeros of the polynomial (including complex ones), completely factor it over the real numbers, and completely factor it over the complex numbers.

\begin{multicols}{2}

\begin{ex}
	$f(x) = x^2-4x+13$
	\begin{sol}
		$f(x) = x^2-4x+13 = (x-(2+3i)) (x-(2-3i))$ \\
		Zeros: $x = 2 \pm 3i$ 
	\end{sol}
\end{ex}


\begin{ex}
	$f(x) = x^2 - 2x + 5$
	\begin{sol}
		$f(x) = x^2 - 2x + 5 = (x-(1+2i))(x-(1-2i))$ \\ 
		Zeros:  $x = 1 \pm 2i$
	\end{sol}
\end{ex}


\begin{ex}
	$f(x) = 3x^2 + 2x +10$
	\begin{sol}
		$f(x) = 3x^2 + 2x +10 = 3\left(x-\left(-\frac{1}{3} + \frac{\sqrt{29}}{3} i\right) \right) \left(x-\left(-\frac{1}{3} - \frac{\sqrt{29}}{3} i\right) \right)$
		
		Zeros:  $x = -\frac{1}{3} \pm \frac{\sqrt{29}}{3} i$
	\end{sol}
\end{ex}


\begin{ex}
	$f(x) = x^3-2x^2+9x-18 $
	\begin{sol}
		$f(x) = x^3-2x^2+9x-18 = (x-2) \left(x^2+9\right) = (x-2)(x-3i)(x+3i)$\\
		Zeros:  $x=2, \pm 3i$
	\end{sol}
\end{ex}


\begin{ex}
	$f(x) = x^{3} + 6x^{2} + 6x + 5$
	\begin{sol}
		 $f(x) = x^{3} + 6x^{2} + 6x + 5 = (x + 5)(x^{2} + x + 1) = (x + 5) \left( x - \left( -\frac{1}{2} + \frac{\sqrt{3}}{2}i \right) \right) \left( x - \left(-\frac{1}{2} - \frac{\sqrt{3}}{2}i \right) \right)$ \\
		Zeros: $x = -5, \;  x = -\frac{1}{2} \pm \frac{\sqrt{3}}{2}i $
	\end{sol}
\end{ex}


\begin{ex}
		$f(x) = 3x^{3} - 13x^{2} + 43x - 13 $
	\begin{sol}
		$f(x) = 3x^{3} - 13x^{2} + 43x - 13 = (3x - 1)(x^{2} - 4x + 13) = (3x - 1)(x - (2 + 3i))(x - (2 - 3i))$\\
		Zeros: $x = \frac{1}{3}, \; x = 2 \pm 3i$
	\end{sol}
\end{ex}


\begin{ex}
		$f(x) = x^3 + 3x^2 + 4x + 12 $
	\begin{sol}
		$f(x) = x^3 + 3x^2 + 4x + 12 = (x+3) \left(x^2 + 4 \right) = (x+3)(x+2i)(x-2i)$ \\
		Zeros:  $x = -3, \; \pm 2i$
	\end{sol}
\end{ex}


\begin{ex}
	$f(x) = 4x^{4} - 4x^{3} + 13x^{2} - 12x + 3$
	\begin{sol}
		 $f(x) = 4x^{4} - 4x^{3} + 13x^{2} - 12x + 3 = \left(x - \frac{1}{2}\right)^{2}\left(4x^{2} + 12\right) = 4\left(x - \frac{1}{2}\right)^{2}(x + i\sqrt{3})(x - i\sqrt{3})$\\
		Zeros: $x = \frac{1}{2}, \; x = \pm \sqrt{3}i$
	\end{sol}
\end{ex}


\begin{ex}
	$f(x) = x^4+x^3+7x^2+9x-18$
	\begin{sol}
		 $f(x) = x^4+x^3+7x^2+9x-18 = (x+2)(x-1)\left(x^2+9\right) = (x+2)(x-1)(x+3i)(x-3i)$\\
		Zeros:  $x = -2, \; 1, \; \pm 3i$
	\end{sol}
\end{ex}


\begin{ex}
		 $f(x) = 8x^4+50x^3+43x^2+2x-4$
	\begin{sol}
		 $f(x) = 8x^4+50x^3+43x^2+2x-4 = 8\left(x + \frac{1}{2}\right) \left(x - \frac{1}{4}\right)(x - (-3 + \sqrt{5}))(x - (-3 - \sqrt{5}))$ \\
		Zeros:  $x = -\frac{1}{2}, \; \frac{1}{4}, \; x = -3 \pm \sqrt{5}$
	\end{sol}
\end{ex}


\begin{ex}
		$f(x) = x^4+9x^2+20 $
	\begin{sol}
		$f(x) = x^4+9x^2+20 = \left(x^2+4\right) \left(x^2+5\right) = (x-2i)(x+2i)\left(x - i \sqrt{5}\right)\left(x + i \sqrt{5}\right)$\\
		Zeros:  $x = \pm 2i, \pm i \sqrt{5}$
	\end{sol}
\end{ex}


\begin{ex}
	 $f(x) = x^5 - x^4+7x^3-7x^2+12x-12 $
	\begin{sol}
		  $f(x) = x^5 - x^4+7x^3-7x^2+12x-12 = (x-1) \left(x^2 + 3\right) \left(x^2 + 4 \right) \\
		 = (x-1)(x - i \sqrt{3})(x + i \sqrt{3})(x-2i)(x+2i)$ \\
		Zeros:  $x = 1, \;  \pm  \sqrt{3}i,  \; \pm 2i$
	\end{sol}
\end{ex}

\begin{ex}
	 $f(x) = x^6 - 64 $
	\begin{sol}
		 $f(x) = x^6 - 64 = (x-2)(x+2)\left(x^2+2x+4\right)\left(x^2-2x+4\right) \\
		= (x-2)(x+2)\left( x - \left( -1+i\sqrt{3} \right) \right)\left( x - \left( -1-i\sqrt{3} \right) \right)\left( x - \left( 1+i\sqrt{3} \right) \right)\left( x - \left( 1-i\sqrt{3} \right) \right)$ \\
		Zeros:  $x = \pm 2$, $x = -1 \pm i\sqrt{3}$, $x = 1 \pm i\sqrt{3}$
	\end{sol}
\end{ex}

\end{multicols}

\Closesolutionfile{ans}
\subsection*{Answers \nopunct} \hfill
%\begin{multicols}{2}
	\input{ans44}
%\end{multicols}



\newpage
\section{Complex numbers for their own sake}

Complex conjugate\index{complex conjugate} notation for complex numbers:  $\overline{a+bi} = a-bi$,\ \ \  $\overline{a-bi} = a+bi$.

The four basic arithmetic operations can be performed with any two complex numbers (except division by zero) and the result can be written in $a\pm bi$ form. [See lecture examples.]

\begin{comment}

\begin{itemize}
	\item Addition and subtraction are straightforward. Remember to treat both real and complex parts as a single number at the beginning (parentheses help), then expand, then simply collect real and imaginary parts.
	\item Perform multiplication by expanding as usual (FOIL), replacing $i^2$ with $-1$, then collecting parts again.
	\item The trick for division is ``rationalizing'' using the complex conjugate (just as with radicals since $i=\sqrt{-1}$).
\end{itemize}
\end{comment}

Complex numbers in rectangular form\index{rectangular form} ($a\pm bi$) are visualized as points in the \textbf{complex plane}\index{complex plane} plane using the real part as the horizontal coordinate and imaginary part as the vertical coordinate.


The \textbf{complex exponential}\index{complex exponential} function can be defined for imaginary numbers by \textbf{Euler's Formula}\index{Euler's Formula} as:
\fbox{$e^{i\theta} = \cos\theta + i\sin\theta $}
\qi{Thus the (natural) exponential of an imaginary number  is on the unit circle in the complex plane.}
\qi{Other points in the complex plane can be described by their position on the radius-$r$ circle by $re^{i\theta} = r\cos \theta + i(r\sin \theta)$. This is called their \textbf{polar form}\index{polar form of a complex number}.}
%\qi{As a function on complex numbers, the complex exponential is no longer one-to-one (why?). What could this mean about logarithms if we allow complex numbers?\footnote{For more information, make it through the Calculus sequence then look for a class called ``Complex Variables'' or ``Complex Analysis.''}}

Polar form can help us understand the meaning of multiplication and division of complex numbers using the properties of exponents.

\begin{tikzpicture} [scale=1.2]
	\drawgridxxyyllb{-3.5}{3.5}{-3.5}{3.5}{\small \text{real part}}{\small \text{imaginary part}}
	\draw[dashed] (0,0) circle [radius=3];
	\draw[fill] (-1.72073,2.45746) circle [radius=0.08] node [above left] {$re^{i\theta} = a+bi$};
	\draw (0,0) -- (-1.72073,2.45746);
	\draw (-1, 1.3) node [above right] {$r$};
	\draw [->] (0.5,0) arc [radius=0.5, start angle=0, end angle= 120];
	\draw (0.25,0.45) node [above right] {$\theta$};
	\draw [dashed] (-1.72073,2.45746) -- (-1.72073,0) node [below] {$a$} (-1.72, -0.3) node [below] {$=r\cos\theta$};
		\draw [dashed] (-1.72073,2.45746) -- (0,2.45746) node [right] {$b = r\sin \theta$};
	\draw (0, -0.5) node [right] {$\tan\theta = \dfrac{b}{a}$};
	\draw (0, -1.5) node [right] {$r^2 = a^2+b^2$};
	
\end{tikzpicture}



\Opensolutionfile{ans}[ans45]
\subsection*{Exercises \nopunct} \hfill

Use the given complex numbers $z$ and $w$ to find and simplify the following.  Write your answers in the form $a+bi$. 

\begin{multicols}{3}
	
	\begin{itemize}
		
		\item $z+w$
		\item $zw$
		\item $z^2$
		
	\end{itemize}
	
\end{multicols}

\begin{multicols}{3}
	
	\begin{itemize}
		
		\item $\dfrac{1}{z}$
		\item $\dfrac{z}{w}$
		\item $\dfrac{w}{z}$
		
	\end{itemize}
	
\end{multicols}

\begin{multicols}{3}
	
	\begin{itemize}
		
		\item $\overline{z}$
		\item $z\overline{z}$
		\item $(\overline{z})^2$
		
	\end{itemize}
	
\end{multicols}

\begin{ex}
	$z = 2+3i$ and $w = 4i$
	\begin{sol}
		For $z = 2+3i$ and $w = 4i$
		
		\begin{multicols}{3}
			
			\begin{itemize}
				
				\item $z+w = 2+7i$
				
				\item $zw = -12+8i$
				
				\item $z^2 = -5 + 12i$
				
			\end{itemize}
			
		\end{multicols}
		
		\begin{multicols}{3}
			
			\begin{itemize}
				
				\item $\frac{1}{z} = \frac{2}{13} - \frac{3}{13} \, i$
				
				\item $\frac{z}{w} = \frac{3}{4} - \frac{1}{2} \, i$
				
				\item $\frac{w}{z} = \frac{12}{13} + \frac{8}{13} \,i$
				
			\end{itemize}
			
		\end{multicols}
		
		\begin{multicols}{3}
			
			\begin{itemize}
				
				\item $\overline{z} = 2-3i$
				
				\item $z\overline{z} = 13$
				
				\item $(\overline{z})^2 = -5-12i$
				
			\end{itemize}
			
		\end{multicols}
	\end{sol}
\end{ex}

\begin{ex}
	 $z = i$ and $w = -1+2i$

	\begin{sol}
		For  $z = i$ and $w = -1+2i$
		
		\begin{multicols}{3}
			
			\begin{itemize}
				
				\item $z+w = -1+3i$
				
				\item $zw = -2-i$
				
				\item $z^2 = -1$
				
			\end{itemize}
			
		\end{multicols}
		
		\begin{multicols}{3}
			
			\begin{itemize}
				
				\item $\frac{1}{z} = -i$
				
				\item $\frac{z}{w} = \frac{2}{5} - \frac{1}{5} \, i$
				
				\item $\frac{w}{z} = 2+i$
				
			\end{itemize}
			
		\end{multicols}
		
		\begin{multicols}{3}
			
			\begin{itemize}
				
				\item $\overline{z} = -i$
				
				\item $z\overline{z} = 1$
				
				\item $(\overline{z})^2 = -1$
				
			\end{itemize}
			
		\end{multicols}
		
	\end{sol}
\end{ex}

\begin{ex}
	 $z = 3-5i$ and $w = 2+7i$

	\begin{sol}
		For  $z = 3-5i$ and $w = 2+7i$
		
		\begin{multicols}{3}
			
			\begin{itemize}
				
				\item $z+w = 5+2i$
				
				\item $zw = 41+11i$
				
				\item $z^2 = -16-30i$
				
			\end{itemize}
			
		\end{multicols}
		
		\begin{multicols}{3}
			
			\begin{itemize}
				
				\item $\frac{1}{z} = \frac{3}{34} + \frac{5}{34} \,i$
				
				\item $\frac{z}{w} = -\frac{29}{53} - \frac{31}{53} \, i$
				
				\item $\frac{w}{z} = -\frac{29}{34} + \frac{31}{34} \,i$
				
			\end{itemize}
			
		\end{multicols}
		
		\begin{multicols}{3}
			
			\begin{itemize}
				
				\item $\overline{z} = 3+5i$
				
				\item $z\overline{z} = 34$
				
				\item $(\overline{z})^2 = -16+30i$
				
			\end{itemize}
			
		\end{multicols}
	\end{sol}
\end{ex}


\begin{ex}
	$z = -5+i$ and  $w = 4+2i$

	\begin{sol}
		For  $z = -5+i$ and  $w = 4+2i$
		
		\begin{multicols}{3}
			
			\begin{itemize}
				
				\item $z+w = -1+3i$
				
				\item $zw = -22-6i$
				
				\item $z^2 = 24-10i$
				
			\end{itemize}
			
		\end{multicols}
		
		\begin{multicols}{3}
			
			\begin{itemize}
				
				\item $\frac{1}{z} = -\frac{5}{26} - \frac{1}{26} \,i$
				
				\item $\frac{z}{w} = -\frac{9}{10} + \frac{7}{10} \, i$
				
				\item $\frac{w}{z} = -\frac{9}{13} - \frac{7}{13} \,i$
				
			\end{itemize}
			
		\end{multicols}
		
		\begin{multicols}{3}
			
			\begin{itemize}
				
				\item $\overline{z} = -5-i$
				
				\item $z\overline{z} = 26$
				
				\item $(\overline{z})^2 = 24+10i$
				
			\end{itemize}
			
		\end{multicols}
		
	\end{sol}
\end{ex}


\begin{ex}
	 $z = \sqrt{2} - i\sqrt{2}$ and $w = \sqrt{2} + i\sqrt{2}$

	\begin{sol}
		For  $z = \sqrt{2} - i\sqrt{2}$ and $w = \sqrt{2} + i\sqrt{2}$
		
		\begin{multicols}{3}
			
			\begin{itemize}
				
				\item $z+w = 2\sqrt{2}$
				
				\item $zw = 4$
				
				\item $z^2 = -4i$
				
			\end{itemize}
			
		\end{multicols}
		
		\begin{multicols}{3}
			
			\begin{itemize}
				
				\item $\frac{1}{z} = \frac{\sqrt{2}}{4} + \frac{\sqrt{2}}{4} \,i$
				
				\item $\frac{z}{w} = -i$
				
				\item $\frac{w}{z} = i$
				
			\end{itemize}
			
		\end{multicols}
		
		\begin{multicols}{3}
			
			\begin{itemize}
				
				\item $\overline{z} = \sqrt{2}+i\sqrt{2}$
				
				\item $z\overline{z} = 4$
				
				\item $(\overline{z})^2 = 4i$
				
			\end{itemize}
		\end{multicols}
	\end{sol}
\end{ex}


\begin{ex}
	 $z = \frac{1}{2} + \frac{\sqrt{3}}{2} \, i$ and $w = -\frac{1}{2} + \frac{\sqrt{3}}{2} \,i$

	\begin{sol}
		For   $z = \frac{1}{2} + \frac{\sqrt{3}}{2} \, i$ and $w = -\frac{1}{2} + \frac{\sqrt{3}}{2} \,i$
		
		\begin{multicols}{3}
			
			\begin{itemize}
				
				\item $z+w = i\sqrt{3}$
				
				\item $zw = -1$
				
				\item $z^2 = -\frac{1}{2} + \frac{\sqrt{3}}{2} \,i$
				
			\end{itemize}
			
		\end{multicols}
		
		\begin{multicols}{3}
			
			\begin{itemize}
				
				\item $\frac{1}{z} = \frac{1}{2} - \frac{\sqrt{3}}{2} \, i$
				
				\item $\frac{z}{w} = \frac{1}{2} - \frac{\sqrt{3}}{2} \, i$
				
				\item $\frac{w}{z} = \frac{1}{2} + \frac{\sqrt{3}}{2} \, i$
				
			\end{itemize}
			
		\end{multicols}
		
		\begin{multicols}{3}
			
			\begin{itemize}
				
				\item $\overline{z} = \frac{1}{2} - \frac{\sqrt{3}}{2} \, i$
				
				\item $z\overline{z} = 1$
				
				\item $(\overline{z})^2 = -\frac{1}{2} - \frac{\sqrt{3}}{2} \, i$
				
			\end{itemize}
			
		\end{multicols}
	\end{sol}
\end{ex}

Convert each complex number to rectangular form.

\begin{multicols}{2}
	
	\begin{ex}
		$3e^{i\pi}$
		\begin{sol}
			$-3$
		\end{sol}
	\end{ex}
	
	\begin{ex}
			$\ds 44e^{\frac{\pi}{2}i}$
		\begin{sol}
			$44i$
		\end{sol}
	\end{ex}
		
\begin{ex}
	$\ds 10e^{\frac{\pi}{6}i}$
	\begin{sol}
		$5\sqrt{3}+5i$
	\end{sol}
\end{ex}

		
\begin{ex}
	$\ds -10e^{\frac{\pi}{6}i}$
	\begin{sol}
		$-5\sqrt{3}-5i$
	\end{sol}
\end{ex}

		
\begin{ex}
	$\ds 10e^{\left(\frac{\pi}{6} - \pi \right)i}$
	\begin{sol}
		$-5\sqrt{3}-5i$ Notice this is the same answer as the previous problem -- negative $r$ can be interpreted as a reflection through the origin or $180\dg$ rotation.
	\end{sol}
\end{ex}

		
\begin{ex}
	$\ds e^{\frac{3\pi}{2}i}$
	\begin{sol}
		$-i$
	\end{sol}
\end{ex}
		
\begin{ex}
	$\ds 7e^{\frac{2\pi}{3}i}$
	\begin{sol}
		$-\frac{7}{2}+\frac{7\sqrt{3}}{2}i$
	\end{sol}
\end{ex}

		
\begin{ex}
	$ \frac{1}{2}e^{\frac{7\pi}{6}i}$
	\begin{sol}
		$-\frac{\sqrt{3}}{4}-\frac{1}{4}i$
	\end{sol}
\end{ex}
	
\end{multicols}

Convert each complex number to polar form.

\begin{multicols}{2}

\begin{ex}
	$1-i$
	\begin{sol}
	$\ds \sqrt{2}e^{\frac{7\pi}{4}i}$ (Note $\frac{7\pi}{4}i$ is the exponent.)
	\end{sol}
\end{ex}

\begin{ex}
	$-3-3i$
	\begin{sol}
		$\ds 3\sqrt{2}e^{\frac{5\pi}{4}i}$
	\end{sol}
\end{ex}


\begin{ex}
	$2+\sqrt{12}i$
	\begin{sol}
		$\ds 4e^{\frac{\pi}{3}i}$
	\end{sol}
\end{ex}


\begin{ex}
	$-\sqrt{3}-i$
	\begin{sol}
		$\ds 2e^{\frac{7\pi}{6}i}$
	\end{sol}
\end{ex}

\begin{ex}
	$17+30i$
	\begin{sol}
		$\ds \sqrt{1189} e^{\arctan(30/17)i} \approx 34.48e^{1.055i}$
	\end{sol}
\end{ex}

\begin{ex}
	$3i$
	\begin{sol}
		$\ds 3e^{\frac{\pi}{2}i}$
	\end{sol}
\end{ex}


\begin{ex}
	$-3i$
	\begin{sol}
		$\ds 3e^{\frac{3\pi}{2}i}$
	\end{sol}
\end{ex}


\begin{ex}
	$-5$
	\begin{sol}
		$\ds 5e^{i\pi} = -5e^{0i}$. Remember the polar form is not unique. Real numbers are essentially in both rectangular form ($a = a+0i$) and polar form ($a = ae^{0i}$) at the same time!
	\end{sol}
\end{ex}

\end{multicols}

\begin{ex}
	Let $z=1-i$, $w=3i$ and $v=\frac{1}{2}e^{\frac{7\pi}{6}i}$. Give the result of each operation in \underline{both} forms.
\begin{multicols}{2}
	\ssp
	\item $z\cdot w$
	\item $w\cdot v$
	\item $\dfrac{v}{w}$
	\item $\dfrac{z}{v}$
	\item $z \cdot \overline w$
	\item $w \cdot \overline v$
	\item $v \cdot \overline v$
	\esp
\end{multicols}
\begin{sol}
	\begin{multicols}{2}
		\ssp
		\item $3+3i$, $3\sqrt{2}e^{\frac{\pi}{4}i}$
		\item $\frac{3}{4}-\frac{3\sqrt{3}}{4}i$, $\frac{3}{2}e^{\frac{5\pi}{3}i}$
		\item $-\frac{1}{12}+\frac{\sqrt{3}}{12}i$, $\frac{1}{6}e^{\frac{2\pi}{3}i}$	
		\item $(1-\sqrt{3})+(\sqrt{3}+1)i$, $2\sqrt{2}e^{\frac{7\pi}{12}i}$
		\item $-3-3i$, $3\sqrt{2}e^{\frac{5\pi}{4}i}$
		\item $-\frac{3}{4}-\frac{3\sqrt{3}}{4}i$, $\frac{3}{2}e^{\frac{4\pi}{3}i}$
		\item $\frac{1}{4}$, $\frac{1}{4}e^{0i}$
		\esp
	\end{multicols}
	\end{sol}

\end{ex}


\Closesolutionfile{ans}
\subsection*{Answers \nopunct} \hfill
%\begin{multicols}{2}
	\input{ans45}
%\end{multicols}


\newpage
\section{More equation fun!}

The logarithm with base $b$, $f(x) = \log_b(x)$ and the exponential function with base $b$, $g(x) = b^x$ are inverses!
\qi{Remember we defined $\log_b(y)=x \iff b^x=y$.}

When solving equations with logarithms, beware of domain-created issues.

\Opensolutionfile{ans}[ans46]
\subsection*{Exercises \nopunct} \hfill

Solve each equation. For equations involving $\theta$, restrict your solutions to $0 \leq \theta < 2\pi$.

\begin{multicols}{2}

\begin{ex}
	$2^{(x^{3} - x)} = 1$ 
	\begin{sol}
		$x = -1, \, 0, \, 1$
	\end{sol}
\end{ex}

\begin{ex}
	$\sin(\theta) = \cos(\theta)$
	\begin{sol}
		$\theta = \dfrac{\pi}{4},\ \dfrac{5\pi}{4}$
	\end{sol}
\end{ex}

\begin{ex}
	$\log_{2}\left(x^{3}\right) = \log_{2}(x)$
	\begin{sol}
		$x = 1$
	\end{sol}
\end{ex}

\begin{ex}
	 $9 \cdot 3^{7x} = \left(\frac{1}{9}\right)^{2x}$ 
	\begin{sol}
		$x=-\frac{2}{11}$
	\end{sol}
\end{ex}


\begin{ex}
	$-7\cdot 5^{x} = 2$  
	\begin{sol}
		No solution.
	\end{sol}
\end{ex}


\begin{ex}
	$\ln\left(8-x^2\right)=\ln(2-x)$
	\begin{sol}
		 $x=-2$
	\end{sol}
\end{ex}

\begin{ex}
	 $\dfrac{100e^{x}}{e^{x}+2}=50$ 
	\begin{sol}
		$x =  \ln(2)$
	\end{sol}
\end{ex}


\begin{ex}
	$\log_{5}\left(18-x^2\right) = \log_{5}(6-x)$
	\begin{sol}
		$x=-3,\, 4$
	\end{sol}
\end{ex}

\begin{ex}
	$e^{2x} = 2e^{x}$ 
	\begin{sol}
		$x =  \ln(2)$
	\end{sol}
\end{ex}


\begin{ex}
	$\log_{\frac{1}{2}} (2x-1) = -3$
	\begin{sol}
		$x=\frac{9}{2}$
	\end{sol}
\end{ex}

\begin{ex}
	$7e^{2x} = 28e^{-6x}$ 
	\begin{sol}
		$x = -\frac{1}{8} \ln\left(\frac{1}{4} \right) = \frac{1}{4}\ln(2)$
	\end{sol}
\end{ex}


\begin{ex}
	$\sin(3\theta) = \dfrac{1}{2}$
	\begin{sol}
		$\theta = \dfrac{\pi}{18},\ \dfrac{5\pi}{18},\ \dfrac{13\pi}{18},\ \dfrac{17\pi}{18},\ \dfrac{25\pi}{18},\ \dfrac{29\pi}{18}$
	\end{sol}
\end{ex}


\begin{ex}
	$\tan\left(\theta-\pi\right) = 1$
	\begin{sol}
		$\theta = \dfrac{\pi}{4},\ \dfrac{5\pi}{4}$
	\end{sol}
\end{ex}

\begin{ex}
	$\log(x^2-3x) = 1$
	\begin{sol}
		$x=-2,\, 5$
	\end{sol}
\end{ex}

\begin{ex}
	$3^{(x - 1)} = 2^{x}$ 
	\begin{sol}
		$x = \frac{\ln(3)}{\ln(3) - \ln(2)}$
	\end{sol}
\end{ex}

\begin{ex}
	$3^{(x - 1)} = \left(\frac{1}{2}\right)^{(x + 5)}$ 
	\begin{sol}
		$x = \frac{\ln(3) + 5\ln\left(\frac{1}{2}\right)}{\ln(3) - \ln\left(\frac{1}{2}\right)} = \frac{\ln(3)-5\ln(2)}{\ln(3)+\ln(2)}$
	\end{sol}
\end{ex}


\begin{ex}
	$\tan^2(\theta) = 1-\sec(\theta)$
	\begin{sol}
		$\theta = 0,\  \frac{2\pi}{3},\ \frac{4\pi}{3}$
	\end{sol}
\end{ex}

\begin{ex}
	$3\ln(x)-2=1-\ln(x)$
	\begin{sol}
		$x=e^{3/4}$
	\end{sol}
\end{ex}

\begin{ex}
	$7^{3+7x} = 3^{4-2x}$ 
	\begin{sol}
		$x = \frac{4 \ln(3) - 3 \ln(7)}{7 \ln(7) + 2 \ln(3)}$
	\end{sol}
\end{ex}


\begin{ex}
	$\sec(\theta) = 2\csc(\theta)$
	\begin{sol}
		$\theta = \arctan(2),\ \pi+\arctan(2)$
	\end{sol}
\end{ex}

\begin{ex}
	$e^{2x} - 3e^{x}-10=0$ 
	\begin{sol}
		$x=\ln(5)$
	\end{sol}
\end{ex}


\begin{ex}
	$\log_{3}(x - 4) + \log_{3}(x + 4) = 2$
	\begin{sol}
		 $x = 5$
	\end{sol}
\end{ex}

\begin{ex}
	$e^{2x} = e^{x}+6$ %Ans $x=\ln(2)$
	\begin{sol}
		$x=\ln(3)$
	\end{sol}
\end{ex}

\begin{ex}
	$4^{x} + 2^{x} = 12$ %Ans $x=\frac{\ln(3)}{\ln(2)}$
	\begin{sol}
		$x=\frac{\ln(3)}{\ln(2)}$
	\end{sol}
\end{ex}



\begin{ex}
	 $\ln(x+1) - \ln(x) = 3$ 
	\begin{sol}
		$x = \frac{1}{e^3-1}$
	\end{sol}
\end{ex}

\begin{ex}
	 $e^{x}-3e^{-x}=2$ %Ans $x=\ln(3)$
	\begin{sol}
		 $x=\ln(3)$
	\end{sol}
\end{ex}

\begin{ex}
	 $e^{x}+15e^{-x}=8$ %Ans $x=\ln(2)$, $\ln(5)$
	\begin{sol}
		$x=\ln(3)$, $\ln(5)$
	\end{sol}
\end{ex}

\begin{ex}
	 $\left(\log(x)\right)^2=2\log(x)+15$
	\begin{sol}
		 $x=10^{-3}, \, 10^{5}$
	\end{sol}
\end{ex}

\end{multicols}

\begin{ex}
	The hyperbolic sine function is defined by $\sinh(x) = \dfrac{e^x-e^{-x}}{2}$ (pronounced as ``cinch''). Show that like sine, it's an odd function and that $\sinh(0)=0$. Use a graphing utility to examine its graph. Does it appear to be one-to-one? Is it periodic? Find its inverse.
	\begin{sol}
		It appears to be one-to-one, but not periodic. $\sinh^{-1}(y) = y+\ln\left( \sqrt{y^2+1} \right)$
	\end{sol}
\end{ex}

\Closesolutionfile{ans}
\subsection*{Answers \nopunct} \hfill
\begin{multicols}{2}
	\input{ans46}
\end{multicols}



\newpage
\section{Angle addition for sine and cosine with consequences}


\underline{\textsc{Recall}}:

\qi{co-function identities}
\qi{The other four trig functions can be written in terms of sine and cosine.}
\qi{Cosine and secant are even, the other four trig functions are odd.}


\bigskip
\underline{\textsc{The Pythagorean Identity and Consequences}}

\smallskip
\begin{large}\index{Pythagorean identities}
	$\cos^2(\theta)+ \sin^2(\theta)=1$
\end{large}

\smallskip
\ \ \ \ $1+ \tan^2(\theta)=\sec^2(\theta)$\ \ \ \ \ [Divide by $\cos^2(\theta)$]

\ \ \ \ $\cot^2(\theta)+1 = \csc^2(\theta)$\ \ \ \ \ [Divide by $\sin^2(\theta)$]

\bigskip
\underline{\textsc{Angle Addition and Consequences (when combined with Pythagorean Identity)}}\index{angle addition identities}

\smallskip
\begin{large}
	\ \  $
	\left.
	\begin{array}{lr}
	\cos(A+B) = \cos(A)\cos(B)-\sin(A)\sin(B)\\
	\sin(A+B) = \sin(A)\cos(B)+\cos(A)\sin(B)
	\end{array}
	\right \}
	$ [angle addition]
	
	
	% \ \ \ \ $\cos(2\theta) = \cos^2(\theta)-\sin^2(\theta)$
	
	%  \ \ \ \ $\phantom{\cos(2\theta)} = 2\cos^2(\theta)-1$
	
	%  \ \ \ \ $\phantom{\cos(2\theta)} = 1-2\sin^2(\theta)$
	
	\bigskip
	\ \ \ \ $\sin(2\theta) = 2\sin(\theta)\cos(\theta)$\ \ \ [sine double-angle]
	
	\ \ \ \ 	{\small $\cos(2\theta) = \cos^2(\theta) - \sin^2(\theta)$\ \ \ [cosine double-angle] \index{double angle identities}\index{half-angle identities}\index{power reduction identities}
 \qi{This occasionally useful identity can often be replaced by one of the two below or quickly derived from the angle addition formula if needed. I waffle about whether or not to keep it on the trig sheet.}}
	
	\bigskip
	\ \  $
	\left.
	\begin{array}{lr}
	\cos^2(\theta) = \displaystyle\frac{1+\cos(2\theta)}{2}\\
	
	\\
	\sin^2(\theta) = \displaystyle\frac{1-\cos(2\theta)}{2}
	\end{array}
	\right \}
	${\small [cosine double angle / power reduction / half-angle]}
	
\end{large} 




\Opensolutionfile{ans}[ans47]
\subsection*{Exercises \nopunct} \hfill

Use an angle addition formula to find each exact value. Remember you may only use formulas introduced in our class (see handout if unsure).

 If the form of your result is different from the answer given, try to show it is the same analytically. If this fails, check it by approximation with a calculator [this goes for numerical answers in the entire section].

\begin{multicols}{2}
\begin{ex}
	$\cos\left(75\dg\right)$
	\begin{sol}
		$\dfrac{\sqrt{6}-\sqrt{2}}{4}$
	\end{sol}
\end{ex}

\begin{ex}
	$\sin\left(105\dg\right)$
	\begin{sol}
				$\dfrac{\sqrt{6}+\sqrt{2}}{4}$
	\end{sol}
\end{ex}

\begin{ex}
	$\cos\left(\dfrac{7\pi}{12}\right)$	
	\begin{sol}
		$\dfrac{\sqrt{2}-\sqrt{6}}{4}$
	\end{sol}
\end{ex}

\begin{ex}
	$\sin\left(\dfrac{\pi}{12}\right)$	
	\begin{sol}
		$\dfrac{\sqrt{6}-\sqrt{2}}{4}$		
	\end{sol}
\end{ex}

\begin{ex}
	$\sec\left(-\dfrac{\pi}{12}\right)$	
	\begin{sol}
		$\sqrt{6}-\sqrt{2}$
	\end{sol}
\end{ex}


\begin{ex}
	$\tan\left( \dfrac{25\pi}{12} \right)$	
	\begin{sol}
		$2-\sqrt{3}$
	\end{sol}
\end{ex}

\end{multicols}

\begin{ex}
	
If $\sin(\alpha) = \dfrac{3}{5}$, where $0 < \alpha < \dfrac{\pi}{2}$, and $\cos(\beta) = \dfrac{12}{13}$ where $\dfrac{3\pi}{2} < \beta < 2\pi$, find 

\begin{multicols}{3}
\ssp		
		\item $\sin(\alpha + \beta)$
		\item $\cos(\alpha - \beta)$
		\item $\tan(\alpha - \beta)$
	\esp	
\end{multicols}
	\begin{sol}
	\ssp
	\item $\sin(\alpha + \beta) = \dfrac{16}{65}$
	\item $\cos(\alpha - \beta) = \dfrac{33}{65}$
	\item $\tan(\alpha - \beta) = \dfrac{56}{33}$
	\esp	
	\end{sol}
\end{ex}

Verify each identity:

\begin{ex}
	$\cos(\theta) \sec(\theta) = 1$
		\begin{sol}
		Solutions will vary to identity verifications. Remember you should ultimately rewrite your argument if necessary to start with one side and end with the other. Be sure you are able to justify each step with either a rule from algebra or one of the trigonometric identities from the handout.
	\end{sol}
\end{ex}


\begin{ex}
	$\tan(\theta)\cos(\theta) = \sin(\theta)$
\end{ex}


\begin{ex}
	$\csc(\theta) \cos(\theta) = \cot(\theta)$ 
\end{ex}


\begin{ex}
	$\dfrac{\cos(\theta)}{\sin^{2}(\theta)} = \csc(\theta) \cot(\theta)$
\end{ex}

\begin{ex}
	$\cos(\theta - \pi) = -\cos(\theta)$
\end{ex}

\begin{ex}
	$\tan\left(\theta + \dfrac{\pi}{2} \right) = -\cot(\theta)$
\end{ex}

\begin{ex}
	$\dfrac{1+ \sin(\theta)}{\cos(\theta)} = \sec(\theta) + \tan(\theta)$
\end{ex}

\begin{ex}
	 $\sin(\alpha + \beta) + \sin(\alpha - \beta) = 2\sin(\alpha)\cos(\beta)$ 
\end{ex}


\begin{ex}
	$\dfrac{\sin(\alpha+\beta)}{\sin(\alpha-\beta)} = \dfrac{1+\cot(\alpha) \tan(\beta)}{1 - \cot(\alpha) \tan(\beta)}$ 
\end{ex}


\begin{ex}
	$\dfrac{\cos(\theta)}{1 - \sin^{2}(\theta)} = \sec(\theta)$
\end{ex}

\begin{ex}
	$\dfrac{\sin(t + h) - \sin(t)}{h} = \cos(t) \left(\dfrac{\sin(h)}{h} \right) + \sin(t) \left( \dfrac{\cos(h) - 1}{h} \right)$
	
	Bonus: If $\dfrac{\sin(h)}{h} \to 1$ and $\dfrac{\cos(h) - 1}{h} \to 0$ as $h\to 0$, what must be true?
\end{ex}


\begin{ex}
	$\dfrac{\cos(\theta) + 1}{\cos(\theta) - 1} = \dfrac{1 + \sec(\theta)}{1 - \sec(\theta)}$
\end{ex}


\begin{ex}
	$(\cos(\theta) + \sin(\theta))^2 = 1 + \sin(2\theta)$ 
\end{ex}

\begin{ex}
	 $\dfrac{1}{1-\cos(\theta)} + \dfrac{1}{1+\cos(\theta)} = 2\csc^{2}(\theta)$
\end{ex}


\begin{ex}
	$\dfrac{1}{1-\sin(\theta)} = \sec^{2}(\theta) + \sec(\theta) \tan(\theta)$
\end{ex}


\begin{ex}
	 $(\cos(\theta) - \sin(\theta))^2 = 1 - \sin(2\theta)$
\end{ex}


\begin{ex}
	 $\sec(2\theta) = \dfrac{\cos(\theta)}{\cos(\theta) + \sin(\theta)} + \dfrac{\sin(\theta)}{\cos(\theta)-\sin(\theta)}$ 
\end{ex}


\begin{ex}
	 $-\ln|\sec(\theta) - \tan(\theta)| = \ln|\sec(\theta)+\tan(\theta)|$
\end{ex}

%*For more practice, see 10.3 and 10.4 in Reference Book 1.


\bigskip
Use the power reduction formulas to find each exact value. Compare applicable results to those calculated with angle addition in the first section.

\begin{multicols}{3}

\begin{ex}
	$\cos\left( \dfrac{7\pi}{12} \right)$
	\begin{sol}
		$\cos\left( \dfrac{7\pi}{12} \right) = -\dfrac{\sqrt{2-\sqrt{3}}}{2}$  
	\end{sol}
\end{ex}


\begin{ex}
		$\sin\left( \dfrac{\pi}{12} \right)$
	\begin{sol}
		$\sin\left( \dfrac{\pi}{12} \right) = \dfrac{\sqrt{2-\sqrt{3}}}{2}$ 
	\end{sol}
\end{ex}



\begin{ex}
	 $\cos \left( \dfrac{\pi}{8} \right)$
	\begin{sol}
		 $\cos \left( \dfrac{\pi}{8} \right) = \dfrac{\sqrt{2 + \sqrt{2}}}{2}$
	\end{sol}
\end{ex}

\end{multicols}

\begin{ex}
	If  $\sin(\theta) = -\dfrac{7}{25}$ where $\dfrac{3\pi}{2} < \theta < 2\pi$, find:
	\begin{multicols}{3}
	\ssp
	\item $\sin(2\theta)$
	\item $\sin\left(\dfrac{\theta}{2}\right)$
	\item $\cos(2\theta)$
	\item $\cos\left(\dfrac{\theta}{2}\right)$
	\item $\tan(2\theta)$
	\item $\tan\left(\dfrac{\theta}{2}\right)$
	\esp
	\end{multicols}
	\begin{sol}
		\ssp
		\item $\sin(2\theta) = -\dfrac{336}{625}$
		\item $\sin\left(\frac{\theta}{2}\right) = \dfrac{\sqrt{2}}{10}$
		\item $\cos(2\theta) = \dfrac{527}{625}$
		\item $\cos\left(\frac{\theta}{2}\right) = -\dfrac{7\sqrt{2}}{10}$
		\item $\tan(2\theta) = -\dfrac{336}{527}$
		\item $\tan\left(\frac{\theta}{2}\right) = -\dfrac{1}{7}$
		\esp
	\end{sol}
\end{ex}


\Closesolutionfile{ans}
\subsection*{Answers \nopunct} \hfill
\begin{multicols}{2}
	\input{ans47}
\end{multicols}

\newpage
\section{Inverse trigonometry}

 Remember the handout exists as a reference but also as a guide to the most critical things to memorize. Most precalc books contain more trigonometric facts and identities than these, but these are the most critical for calculus. You'd need to have a real stickler Calc 1 teacher to want you to know more.
 
 \underline{\textsc{Inverse Trig. Functions}}
 
 \medskip
 $\arccos(u)=\theta \iff
 \left\{
 \begin{array}{l}
 \cos(\theta)=u\\
 \text{and } 0 \leq \theta \leq \pi
 \end{array}
 \right.
 $
 
 \medskip
 $\arcsin(u)=\theta \iff
 \left\{
 \begin{array}{l}
 \sin(\theta)=u\\
 \text{and } -\frac{\pi}{2} \leq \theta \leq \frac{\pi}{2}
 \end{array}
 \right.
 $
 
 \medskip
 $\arctan(u)=\theta \iff
 \left\{
 \begin{array}{l}
 \tan(\theta)=u\\
 \text{and } -\frac{\pi}{2} < \theta < \frac{\pi}{2}
 \end{array}
 \right.
 $
 
 \Opensolutionfile{ans}[ans48]
 \subsection*{Exercises \nopunct} \hfill
 
 Evaluate the exact value or state that it is undefined.
 
 \begin{multicols}{2}
 
\begin{ex}
		$\arcsin \left( -1 \right) $
	\begin{sol}
		$\arcsin \left( -1 \right) = -\dfrac{\pi}{2}$ 
	\end{sol}
\end{ex}

\begin{ex}
		$\arcsin \left( -\dfrac{1}{2} \right)$
	\begin{sol}
		$\arcsin \left( -\dfrac{1}{2} \right) = -\dfrac{\pi}{6}$
	\end{sol}
\end{ex}

 
\begin{ex}
	 $\arcsin \left( \dfrac{\sqrt{3}}{2} \right)$
	\begin{sol}
		 $\arcsin \left( \dfrac{\sqrt{3}}{2} \right) = \dfrac{\pi}{3}$
	\end{sol}
\end{ex}

\begin{ex}
		$\arccos \left( -\dfrac{\sqrt{3}}{2} \right)$
	\begin{sol}
		$\arccos \left( -\dfrac{\sqrt{3}}{2} \right) = \dfrac{5\pi}{6}$
	\end{sol}
\end{ex}

 
\begin{ex}
	 $\arccos \left( 0 \right)$
	\begin{sol}
		 $\arccos \left( 0 \right) = \dfrac{\pi}{2}$
	\end{sol}
\end{ex}

\begin{ex}
		 $\arccos \left( \dfrac{1}{2} \right)$
	\begin{sol}
		 $\arccos \left( \dfrac{1}{2} \right) = \dfrac{\pi}{3}$
	\end{sol}
\end{ex}

 
\begin{ex}
			 $\arccos \left(\sqrt{3}\right)$
	\begin{sol}
		Undefined.
	\end{sol}
\end{ex}

\begin{ex}
	$\arctan \left( -\sqrt{3} \right) $
	\begin{sol}
		$\arctan \left( -\sqrt{3} \right) = -\dfrac{\pi}{3}$
	\end{sol}
\end{ex}

 
\begin{ex}
	 $\arctan \left( 0 \right) $
	\begin{sol}
		 $\arctan \left( 0 \right) = 0$ 
	\end{sol}
\end{ex}

\begin{ex}
		$\arctan \left( \dfrac{\sqrt{3}}{3} \right)$
	\begin{sol}
		$\arctan \left( \dfrac{\sqrt{3}}{3} \right) = \dfrac{\pi}{6}$
	\end{sol}
\end{ex}
 
\begin{ex}
		$\arctan \left( 1 \right)$
	\begin{sol}
		$\arctan \left( 1 \right) = \dfrac{\pi}{4}$
	\end{sol}
\end{ex}

 \begin{ex}
	  $\sin\left(\arcsin\left(\dfrac{1}{2}\right)\right)$
	\begin{sol}
		  $\sin\left(\arcsin\left(\dfrac{1}{2}\right)\right) = \dfrac{1}{2}$ 
	\end{sol}
\end{ex}


\begin{ex}
	 $\sin\left(\arcsin\left(-\dfrac{3}{50}\right)\right)$
	\begin{sol}
		 $\sin\left(\arcsin\left(-\dfrac{3}{50}\right)\right) = -\dfrac{3}{50}$
	\end{sol}
\end{ex}

\begin{ex}
	$\sin\left(\arcsin\left(\dfrac{5}{4}\right)\right)$ 
	\begin{sol}
		Undefined.
	\end{sol}
\end{ex}
 \begin{ex}
	 $\cos\left(\arccos\left(\dfrac{5}{13}\right)\right)$
	\begin{sol}
		 $\cos\left(\arccos\left(\dfrac{5}{13}\right)\right) = \dfrac{5}{13}$
	\end{sol}
\end{ex}


\begin{ex}
	$\tan\left(\arctan\left(-1.4\right)\right)$
	\begin{sol}
			$\tan\left(\arctan\left(-1.4\right)\right) = -1.4$
	\end{sol}
\end{ex}

\begin{ex}
	 $\arcsin\left(\sin\left(\dfrac{\pi}{6}\right) \right)$
	\begin{sol}
		 $\arcsin\left(\sin\left(\dfrac{\pi}{6}\right) \right) = \dfrac{\pi}{6}$
	\end{sol}
\end{ex}
 \begin{ex}
	 $\arcsin\left(\sin\left(-\dfrac{\pi}{3}\right) \right) $
	\begin{sol}
		 $\arcsin\left(\sin\left(-\dfrac{\pi}{3}\right) \right) = -\dfrac{\pi}{3}$
	\end{sol}
\end{ex}


\begin{ex}
	$\arcsin\left(\sin\left(\dfrac{3\pi}{4}\right) \right) $
	\begin{sol}
		$\arcsin\left(\sin\left(\dfrac{3\pi}{4}\right) \right) = \dfrac{\pi}{4}$
	\end{sol}
\end{ex}

\begin{ex}
	 $\arccos\left(\cos\left(\dfrac{2\pi}{3}\right) \right) $
	\begin{sol}
		 $\arccos\left(\cos\left(\dfrac{2\pi}{3}\right) \right) = \dfrac{2\pi}{3}$
	\end{sol}
\end{ex}
 \begin{ex}
	 $\arccos\left(\cos\left(\dfrac{3\pi}{2}\right) \right)$
	\begin{sol}
		 $\arccos\left(\cos\left(\dfrac{3\pi}{2}\right) \right) = \dfrac{\pi}{2}$
	\end{sol}
\end{ex}


\begin{ex}
	 $\arctan\left(\tan\left(\pi\right) \right) $
	\begin{sol}
			 $\arctan\left(\tan\left(\pi\right) \right) = 0$
	\end{sol}
\end{ex}

\begin{ex}
	$\arctan\left(\tan\left(\dfrac{\pi}{2}\right) \right)$ 
	\begin{sol}
		Undefined.
	\end{sol}
\end{ex}

\begin{ex}
	 $\arctan\left(\tan\left(\dfrac{\pi}{3}\right) \right)$
	\begin{sol}
		 $\arctan\left(\tan\left(\dfrac{\pi}{3}\right) \right) = \dfrac{\pi}{3}$
	\end{sol}
\end{ex}


\begin{ex}
	$\arctan\left(\tan\left(\dfrac{2\pi}{3}\right) \right)$
	\begin{sol}
		$\arctan\left(\tan\left(\dfrac{2\pi}{3}\right) \right) = -\dfrac{\pi}{3}$
	\end{sol}
\end{ex}

\begin{ex}
	 $\sin\left(\arccos\left(-\dfrac{1}{2}\right)\right)$
	\begin{sol}
		 $\sin\left(\arccos\left(-\dfrac{1}{2}\right)\right) = \dfrac{\sqrt{3}}{2}$
	\end{sol}
\end{ex}


\begin{ex}
	$\sin\left(\arccos\left(\dfrac{3}{5}\right)\right) $
	\begin{sol}
		$\sin\left(\arccos\left(\dfrac{3}{5}\right)\right) = \dfrac{4}{5}$
	\end{sol}
\end{ex}


\begin{ex}
	 $\cos\left(\arctan\left(\sqrt{7} \right)\right)$
	\begin{sol}
		 $\cos\left(\arctan\left(\sqrt{7} \right)\right) = \dfrac{\sqrt{2}}{4}$
	\end{sol}
\end{ex}

\begin{ex}
	  $\tan\left(\arcsin\left(-\dfrac{2\sqrt{5}}{5}\right)\right)$
	\begin{sol}
		  $\tan\left(\arcsin\left(-\dfrac{2\sqrt{5}}{5}\right)\right)=-2$
	\end{sol}
\end{ex}


\begin{ex}
	 $\cot\left(\arcsin\left(\dfrac{12}{13}\right)\right)$
	\begin{sol}
		 $\cot\left(\arcsin\left(\dfrac{12}{13}\right)\right) = \dfrac{5}{12}$
	\end{sol}
\end{ex}


\begin{ex}
	  $\csc\left(\arctan\left(-\dfrac{2}{3}\right)\right)$
	
	\begin{sol}
		  $\csc\left(\arctan\left(-\dfrac{2}{3}\right)\right) = -\dfrac{\sqrt{13}}{2}$
		
	\end{sol}
\end{ex}

 \end{multicols}

Rewrite each quantity as an algebraic expression. (An expression built from polynomials, radicals, function arithmetic, and composition. e.g. $ \dfrac{\sqrt{1-x^2}}{x+2} $.)

\begin{multicols}{2}

\begin{ex}
	$\sin \left( \arccos \left( x \right) \right)$ 
	\begin{sol}
		$\sin \left( \arccos \left( x \right) \right) = \sqrt{1 - x^{2}}$ 
	\end{sol}
\end{ex}


\begin{ex}
	$\cos \left( \arctan \left( x \right) \right)$
	\begin{sol}
		$\cos \left( \arctan \left( x \right) \right) = \dfrac{1}{\sqrt{1 + x^{2}}}$
	\end{sol}
\end{ex}


\begin{ex}
	$\tan \left( \arcsin \left( x \right) \right)$ 
	\begin{sol}
		$\tan \left( \arcsin \left( x \right) \right) = \dfrac{x}{\sqrt{1 - x^{2}}}$ 
	\end{sol}
\end{ex}

\begin{ex}
	$\sec \left( \arctan \left( x \right) \right)$ 
	\begin{sol}
		$\sec \left( \arctan \left( x \right) \right) = \sqrt{1 + x^{2}}$ 
	\end{sol}
\end{ex}

 \begin{ex}
	$\csc \left( \arccos \left( x \right) \right)$ 
	\begin{sol}
		$\csc \left( \arccos \left( x \right) \right) = \dfrac{1}{\sqrt{1 - x^{2}}}$ 
	\end{sol}
\end{ex}
 

 \begin{ex}
 	$\sin \left( 2\arccos \left( x \right) \right)$
 	\begin{sol}
 		$\sin \left( 2\arccos \left( x \right) \right) = 2x\sqrt{1-x^2}$
 	\end{sol}
 \end{ex}

\end{multicols}
 
 
 \begin{ex}
 	Approximate all solutions in the interval $[0,2\pi)$ to four decimal places:\\
 	 $ \sin(x) = \dfrac{7}{11} $
 	\begin{sol}
 		$x \approx 0.6898, 2.4518$
 	\end{sol}
 \end{ex}
 
 \begin{ex}
 Approximate all solutions in the interval $[0,2\pi)$ to four decimal places:\\
 $ \cos(x) = -\dfrac{2}{9} $
 	\begin{sol}
  		$x \approx 1.7949, 4.4883$
 	\end{sol}
 \end{ex}

 
\begin{ex}
	Approximate all solutions in the interval $[0,2\pi)$ to four decimal places:\\
	$ \tan(x) = 117 $
	\begin{sol}
		$x \approx 1.5622, 4.7038$
	\end{sol}
\end{ex}
 
 \begin{ex}
 	Draw the graphs of arcsin, arccos, and arctan by starting with the sine, cosine, and tangent, making the correct restriction, then reflecting over the line $y=x$.
 \end{ex}

 
\begin{ex}
	What is the end-behavior of $y=\arctan x$?
	
\begin{sol}
	$y\to \pi/2$ as $x \to \infty$, and $y\to -\pi/2$ as $x \to -\infty$. This means the graph has (two) horizontal asymptotes, the lines $y=-\pi/2$ and $y=\pi/2$.
\end{sol}
\end{ex}
 
 \Closesolutionfile{ans}
 \subsection*{Answers \nopunct} \hfill
 \begin{multicols}{2}
 	\input{ans48}
 \end{multicols}
 
 
 
 
 
 


\chapter{Fifth Cycle}

\section{Polar coordinates}\index{polar coordinates}

\begin{tikzpicture} [scale=1.2]
\drawgridxxyyb{-3.5}{3.5}{-3.5}{3.5}
\draw[dashed] (0,0) circle [radius=3];
\draw[fill] (-1.72073,2.45746) circle [radius=0.06] node [left] {$r$};
\draw[->] (0,0) -- (-2.29431,3.27661);
%\draw (-1, 1.3) node [above right] {$r$};
\draw [->] (0.5,0) arc [radius=0.5, start angle=0, end angle= 120];
\draw (0.25,0.45) node [above right] {$\theta$};
\draw [dashed] (-1.72073,2.45746) -- (-1.72073,0) node [below] {$x$} (-1.72, -0.3) node [below] {$=r\cos\theta$};
\draw [dashed] (-1.72073,2.45746) -- (0,2.45746) node [right] {$y = r\sin \theta$};
\draw (0, -0.5) node [right] {$\tan\theta = \dfrac{y}{x}$};
\draw (0, -1.5) node [right] {$r^2 = x^2+y^2$};
\end{tikzpicture}

Polar coordinates of a point are not unique.

\Opensolutionfile{ans}[ans51]
\subsection*{Exercises \nopunct} \hfill

Convert from polar coordinates to rectangular coordinates.

\begin{multicols}{3}

\begin{ex}
	$\left( 4, \dfrac{5\pi}{6} \right)$
	\begin{sol}
		$\left( -2\sqrt{3},2 \right)$
	\end{sol}
\end{ex}

\begin{ex}
		$\left( 7, -\dfrac{\pi}{4} \right)$
	\begin{sol}
			$\left( \dfrac{7\sqrt{2}}{2}, -\dfrac{7\sqrt{2}}{2} \right)$
	\end{sol}
\end{ex}

\begin{ex}
	$\left( 10, \dfrac{3\pi}{2} \right)$
	\begin{sol}
		$(0,-10)$
	\end{sol}
\end{ex}


\begin{ex}
	$\left( \dfrac{1}{2}, \dfrac{4\pi}{3}\right)$
	\begin{sol}
		$\left(- \dfrac{1}{4}, -\dfrac{\sqrt{3}}{4} \right)$
	\end{sol}
\end{ex}

\begin{ex}
		$\left( -5, \dfrac{3\pi}{4} \right)$
	\begin{sol}
		$\left( \dfrac{5\sqrt{2}}{2}, -\dfrac{5\sqrt{2}}{2} \right)$
	\end{sol}
\end{ex}

\begin{ex}
	$\left( -2, \pi \right)$
	\begin{sol}
		$(2,0)$
	\end{sol}
\end{ex}


\begin{ex}
	$\left( -2, -\pi \right)$
	\begin{sol}
		$(2,0)$
	\end{sol}
\end{ex}


\begin{ex}
		$\left( 0,  \dfrac{13\pi}{7} \right)$
	\begin{sol}
		$(0,0)$
	\end{sol}
\end{ex}

\end{multicols}

Convert from rectangular coordinates to polar coordinates. Give two possibilities -- one with positive $r$ and one with negative $r$.

\begin{multicols}{3}

\begin{ex}
	$\left( 0 , 5 \right)$
	\begin{sol}
	$\left( 5 , \dfrac{\pi}{2} \right)$, $\left( -5 , \dfrac{3\pi}{2} \right)$  [An answer identical to one or the other but substituting a coterminal angle is also correct. Note the answers $\left( 5 , \dfrac{3\pi}{2} \right)$ and $\left( -5 , \dfrac{\pi}{2} \right)$ do not meet this criteria and are not correct.]
	\end{sol}
\end{ex}

\begin{ex}
	$\left( 3 , \sqrt{3} \right)$
	\begin{sol}
		$\left( 2\sqrt{3} , \dfrac{\pi}{6} \right)$, $\left( -2\sqrt{3} , \dfrac{7\pi}{6} \right)$
	\end{sol}
\end{ex}

\begin{ex}
	$\left( 7 , -7 \right)$
	\begin{sol}
		$\left( 7\sqrt{2} ,  \dfrac{7\pi}{4}\right)$, 		$\left( -7\sqrt{2} ,  \dfrac{3\pi}{4}\right)$
	\end{sol}
\end{ex}

\begin{ex}
	$\left( -3 , 0 \right)$
	\begin{sol}
		$\left( 3 , \pi \right)$, 		$\left( -3 , 0 \right)$
	\end{sol}
\end{ex}

\begin{ex}
	$\left( -4 , -4\sqrt{3} \right)$
	\begin{sol}
		$\left( 8 , \dfrac{4\pi}{3} \right)$, $\left( -8 , \dfrac{\pi}{3} \right)$
	\end{sol}
\end{ex}

\begin{ex}
	$\left( -2 , 13 \right)$
	\begin{sol}
		$\left( \sqrt{173} , \arctan(-13/2)+\pi \right)$, \\
			$\left( -\sqrt{173} , \arctan(-13/2)\right)$
	\end{sol}
\end{ex}
\end{multicols}

Plot the graph of the polar equation by hand. Start by making a Cartesian graph, then convert it to a true polar graph. Pay attention to tangents at the origin. Check your work with a polar plotting tool.

\begin{multicols}{2}

\begin{ex}
	$r = 6\sin(\theta)$\\
	 {\small(This one only: what is the Cartesian equation?)}
	\begin{sol}
		$x^2+(y-3)^2 = 9$
	\end{sol}
\end{ex}

\begin{ex}
	$r=2\sin(2\theta)$
\end{ex}

\begin{ex}
		$r=4\cos(2\theta)$
\end{ex}

\begin{ex}
		$r=5\sin(3\theta)$
\end{ex}

\begin{ex}
	$r=3-3\cos(\theta)$
	\begin{sol}
		Graph is called a cardiod\index{cardiod}.
		\end{sol}
\end{ex}

\begin{ex}
	$r=1-2\cos(\theta)$
	\begin{sol}
		Graph is called a lima\c{c}on \index{lima\c{c}on}(actually a cardiod is a special case of a lima\c{c}on).
		\end{sol}
\end{ex}

\begin{ex}
	$r=3\sin\left( \dfrac{\theta}{2} \right)$
\end{ex}

\end{multicols}

\begin{ex}
	Graph $r=\dfrac{6}{1+2\sin(\theta)}$ by first converting it to a Cartesian equation. Check your work on a plotter that can overlay both types (such as Desmos). What is happening near $\theta = \dfrac{\pi}{6}$?
\end{ex}


\begin{ex}
	Graph $r=\dfrac{2}{1-\cos(\theta)}$ by first converting it to a Cartesian equation. Check your work on a plotter that can overlay both types (such as Desmos). What is happening near $\theta = 0$?
\end{ex}

\begin{ex}
	Surveyors use a device called a total station (total station theodolite) to measure the position of points (property boundaries, etc.) relative to their current position (``station''). The device is mounted on a tripod and rotates left or right until the target position appears through a telescope on the device. Then, the distance to the target is measured using infrared ranging.
	\ssp
	\item Explain how the total station measures positions in polar coordinates.
	\item If point $A$ is sighted at angle $137\dg$ and distance 90 meters while point $B$ is sighted at angle $68\dg$ and distance 75 meters, approximate the distance between $A$ and $B$ to the nearest meter.
	\item If the terrain is not flat, the telescope on the total station can also rotate vertically. The total station is then measuring three dimensional space using a system called spherical coordinates. What are the three spherical coordinates?\index{spherical coordinates} Explain how the name might be fitting.
	\esp
	\begin{sol}
		(b) $94$ m
	\end{sol}
\end{ex}

\begin{ex}
	If the parabola $y=x^2$ is rotated about the origin $45\dg$ clockwise, find a Cartesian equation for the new graph.
	\begin{sol}
		$x^2+y^2-2xy-\sqrt{2}x-\sqrt{2}y=0$
	\end{sol}
\end{ex}


\Closesolutionfile{ans}
\subsection*{Answers \nopunct} \hfill
\begin{multicols}{2}
	\input{ans51}
\end{multicols}



\newpage
\section{Rational functions}

\index{rational functions}Functions of the type $r(x) = \dfrac{p(x)}{q(x)}$ where $p$ and $q$ are polynomials and $q(x)\neq 0$ are called \textbf{rational functions}.

A rational function is \textbf{proper}\index{proper rational function}\index{improper rational function} if the degree of $p$ is less than that of $q$.

End-behavior:
\begin{itemize}
	\item Proper rational functions: $r(x) \to 0$ as $x \to \infty$ and as $x \to -\infty$.
	\item Improper rational functions: divide. The end-behavior is given by the quotient.
	\item Unless we want to determine the limiting polynomial (like a slant asymptote), the end behavior can be determined, then, by the ratio of leading terms (Why?).
\end{itemize}

$x$-values not in the domain: correspond to either holes in the graph or vertical asymptotes.
 \qi{After all common factors are cancelled, does the simplified form also miss this value from its domain? If so, there's a vertical asymptote as outputs will explode near there. If the domain problem has been ``fixed'' by the cancelling, the $x$-value corresponds to a hole.}
 
 Sign-chart\index{sign-chart} as a rough graph tool:
 \begin{itemize}
 	\item Begin with a picture of the domain. Use an open circle for a missing $x$-value.
 	\item Mark all zeros.
 	\item Determize sign for each resulting interval.
 	\item If graphing, the values not in the domain should be investigated (asymptote or hole?). If solving an inequality, they needn't.
 	\item Keep input ($x$-value) information below the line and output information (sign, 0, asymptote v. hole) above.
 \end{itemize}


\Opensolutionfile{ans}[ans52]
\subsection*{Exercises \nopunct} \hfill

Graph each rational function by first making a sign chart. State the domain, intercepts, asymptotes (vertical, horizontal, and slant-asymptotes), end-behavior, and the location of any holes. Check your graph with a graphing utility. [Note: most graphing utilities do not show holes in the graph in the emphasized way we desire.]

\begin{multicols}{2}

\begin{ex}
		$f(x) = \dfrac{4}{x + 2}$
	\begin{sol}
		$f(x) = \dfrac{4}{x + 2}$\\
		Domain: $(-\infty, -2) \cup (-2, \infty)$\\
		No $x$-intercepts\\
		$y$-intercept: $(0, 2)$\\
		Vertical asymptote: $x = -2$\\
		As $x \rightarrow -2^{-}, \; f(x) \rightarrow -\infty$\\
		As $x \rightarrow -2^{+}, \; f(x) \rightarrow \infty$\\
		Horizontal asymptote: $y = 0$\\
		As $x \rightarrow -\infty, \; f(x) \rightarrow 0^{-}$\\
		As $x \rightarrow \infty, \; f(x) \rightarrow 0^{+}$
	\end{sol}
\end{ex}


\begin{ex}
		$f(x) = \dfrac{5x}{6 - 2x}$
	\begin{sol}
		$f(x) = \dfrac{5x}{6 - 2x}$\\
		Domain: $(-\infty, 3) \cup (3, \infty)$\\
		$x$-intercept: $(0, 0)$\\
		$y$-intercept: $(0, 0)$\\
		Vertical asymptote: $x = 3$\\
		As $x \rightarrow 3^{-}, \; f(x) \rightarrow \infty$\\
		As $x \rightarrow 3^{+}, \; f(x) \rightarrow -\infty$\\
		Horizontal asymptote: $y = -\frac{5}{2}$\\
		As $x \rightarrow -\infty, \; f(x) \rightarrow -\frac{5}{2}^{+}$\\
		As $x \rightarrow \infty, \; f(x) \rightarrow -\frac{5}{2}^{-}$\
	\end{sol}
\end{ex}

\begin{ex}
		$f(x) = \dfrac{1}{x^{2}}$
	\begin{sol}
		$f(x) = \dfrac{1}{x^{2}}$\\
		Domain: $(-\infty, 0) \cup (0, \infty)$\\
		No $x$-intercepts\\
		No $y$-intercepts\\
		Vertical asymptote: $x = 0$\\
		As $x \rightarrow 0^{-}, \; f(x) \rightarrow \infty$\\
		As $x \rightarrow 0^{+}, \; f(x) \rightarrow \infty$\\
		Horizontal asymptote: $y = 0$\\
		As $x \rightarrow -\infty, \; f(x) \rightarrow 0^{+}$\\
		As $x \rightarrow \infty, \; f(x) \rightarrow 0^{+}$\
	\end{sol}
\end{ex}

\begin{ex}
	$f(x) = \dfrac{1}{x^{2} + x - 12}$
	\begin{sol}
		$f(x) = \dfrac{1}{x^{2} + x - 12} = \dfrac{1}{(x - 3)(x + 4)}$\\
		Domain: $(-\infty, -4) \cup (-4, 3) \cup (3, \infty)$\\
		No $x$-intercepts\\
		$y$-intercept: $(0, -\frac{1}{12})$\\
		Vertical asymptotes: $x = -4$ and $x = 3$\\
		As $x \rightarrow -4^{-}, \; f(x) \rightarrow \infty$\\
		As $x \rightarrow -4^{+}, \; f(x) \rightarrow -\infty$\\
		As $x \rightarrow 3^{-}, \; f(x) \rightarrow -\infty$\\
		As $x \rightarrow 3^{+}, \; f(x) \rightarrow \infty$\\
		Horizontal asymptote: $y = 0$\\
		As $x \rightarrow -\infty, \; f(x) \rightarrow 0^{+}$\\
		As $x \rightarrow \infty, \; f(x) \rightarrow 0^{+}$
	\end{sol}
\end{ex}

\begin{ex}
		$f(x) = \dfrac{2x - 1}{-2x^{2} - 5x + 3}$
	\begin{sol}
		$f(x) = \dfrac{2x - 1}{-2x^{2} - 5x + 3} = -\dfrac{2x - 1}{(2x - 1)(x + 3)}$\\
		Domain: $(-\infty, -3) \cup (-3, \frac{1}{2}) \cup (\frac{1}{2}, \infty)$\\
		No $x$-intercepts\\
		$y$-intercept: $(0, -\frac{1}{3})$\\
		$f(x) = \dfrac{-1}{x + 3}, \; x \neq \frac{1}{2}$\\
		Hole in the graph at $(\frac{1}{2}, -\frac{2}{7})$\\
		Vertical asymptote: $x = -3$\\
		As $x \rightarrow -3^{-}, \; f(x) \rightarrow \infty$\\
		As $x \rightarrow -3^{+}, \; f(x) \rightarrow -\infty$\\
		Horizontal asymptote: $y = 0$\\
		As $x \rightarrow -\infty, \; f(x) \rightarrow 0^{+}$\\
		As $x \rightarrow \infty, \; f(x) \rightarrow 0^{-}$
	\end{sol}
\end{ex}

\begin{ex}
		$f(x) = \dfrac{x}{x^{2} + x - 12} $
	\begin{sol}
		$f(x) = \dfrac{x}{x^{2} + x - 12} = \dfrac{x}{(x - 3)(x + 4)}$\\
		Domain: $(-\infty, -4) \cup (-4, 3) \cup (3, \infty)$\\
		$x$-intercept: $(0, 0)$\\
		$y$-intercept: $(0, 0)$\\
		Vertical asymptotes: $x = -4$ and $x = 3$\\
		As $x \rightarrow -4^{-}, \; f(x) \rightarrow -\infty$\\
		As $x \rightarrow -4^{+}, \; f(x) \rightarrow \infty$\\
		As $x \rightarrow 3^{-}, \; f(x) \rightarrow -\infty$\\
		As $x \rightarrow 3^{+}, \; f(x) \rightarrow \infty$\\
		Horizontal asymptote: $y = 0$\\
		As $x \rightarrow -\infty, \; f(x) \rightarrow 0^{-}$\\
		As $x \rightarrow \infty, \; f(x) \rightarrow 0^{+}$\
	\end{sol}
\end{ex}

\begin{ex}
		$f(x) = \dfrac{4x}{x^{2} + 4}$
	\begin{sol}
		$f(x) = \dfrac{4x}{x^{2} + 4}$\\
		Domain: $(-\infty,  \infty)$\\
		$x$-intercept:  $(0,0)$\\
		$y$-intercept:  $(0,0)$\\
		No vertical asymptotes \\
		No holes in the graph\\
		Horizontal asymptote: $y = 0$ \\
		As $x \rightarrow -\infty, f(x) \rightarrow 0^{-}$\\
		As $x \rightarrow \infty, f(x) \rightarrow 0^{+}$
	\end{sol}
\end{ex}

\begin{ex}
		$f(x) = \dfrac{4x}{x^{2} -4}$
	\begin{sol}
		$f(x) = \dfrac{4x}{x^{2} -4} = \dfrac{4x}{(x + 2)(x - 2)}$\\
		Domain: $(-\infty, -2) \cup (-2, 2) \cup (2, \infty)$\\
		$x$-intercept:  $(0,0)$\\
		$y$-intercept:  $(0,0)$\\
		Vertical asymptotes: $x = -2, x = 2$\\
		As $x \rightarrow -2^{-}, f(x) \rightarrow -\infty$\\
		As $x \rightarrow -2^{+}, f(x) \rightarrow \infty$\\
		As $x \rightarrow 2^{-}, f(x) \rightarrow -\infty$\\
		As $x \rightarrow 2^{+}, f(x) \rightarrow \infty$\\
		No holes in the graph\\
		Horizontal asymptote: $y = 0$ \\
		As $x \rightarrow -\infty, f(x) \rightarrow 0^{-}$\\
		As $x \rightarrow \infty, f(x) \rightarrow 0^{+}$
	\end{sol}
\end{ex}

\begin{ex}
		$f(x) = \dfrac{x^2-x-12}{x^{2} +x - 6}$
	\begin{sol}
		$f(x) = \dfrac{x^2-x-12}{x^{2} +x - 6} = \dfrac{x-4}{x - 2} \, x \neq -3$\\
		Domain: $(-\infty, -3) \cup (-3, 2) \cup (2, \infty)$\\
		$x$-intercept:  $(4,0)$\\
		$y$-intercept:  $(0,2)$\\
		Vertical asymptote: $x = 2$\\
		As $x \rightarrow 2^{-}, f(x) \rightarrow \infty$\\
		As $x \rightarrow 2^{+}, f(x) \rightarrow -\infty$\\
		Hole at $\left(-3, \frac{7}{5} \right)$ \\
		Horizontal asymptote: $y = 1$ \\
		As $x \rightarrow -\infty, f(x) \rightarrow 1^{+}$\\
		As $x \rightarrow \infty, f(x) \rightarrow 1^{-}$
	\end{sol}
\end{ex}

\begin{ex}
		$f(x) = \dfrac{3x^2-5x-2}{x^{2} -9} $
	\begin{sol}
		$f(x) = \dfrac{3x^2-5x-2}{x^{2} -9} = \dfrac{(3x+1)(x-2)}{(x + 3)(x - 3)}$\\
		Domain: $(-\infty, -3) \cup (-3, 3) \cup (3, \infty)$\\
		$x$-intercepts: $\left(-\frac{1}{3}, 0 \right)$, $(2,0)$\\
		$y$-intercept:  $\left(0, \frac{2}{9} \right)$\\
		Vertical asymptotes: $x = -3, x = 3$\\
		As $x \rightarrow -3^{-}, f(x) \rightarrow \infty$\\
		As $x \rightarrow -3^{+}, f(x) \rightarrow -\infty$\\
		As $x \rightarrow 3^{-}, f(x) \rightarrow -\infty$\\
		As $x \rightarrow 3^{+}, f(x) \rightarrow \infty$\\
		No holes in the graph\\
		Horizontal asymptote: $y = 3$ \\
		As $x \rightarrow -\infty, f(x) \rightarrow 3^{+}$\\
		As $x \rightarrow \infty, f(x) \rightarrow 3^{-}$
	\end{sol}
\end{ex}

\begin{ex}
		$f(x) = \dfrac{x^2-x-6}{x+1}$
	\begin{sol}
		$f(x) = \dfrac{x^2-x-6}{x+1} = \dfrac{(x-3)(x+2)}{x+1}$\\
		Domain: $(-\infty, -1) \cup (-1, \infty)$\\
		$x$-intercepts:  $(-2,0)$, $(3,0)$\\
		$y$-intercept:  $(0,-6)$\\
		Vertical asymptote: $x = -1$\\
		As $x \rightarrow -1^{-}, f(x) \rightarrow \infty$\\
		As $x \rightarrow -1^{+}, f(x) \rightarrow -\infty$\\
		Slant asymptote: $y = x-2$ \\
		As $x \rightarrow -\infty$, $y\to -\infty$\\
As $x \rightarrow -\infty$, $y\to \infty$
	\end{sol}
\end{ex}

\begin{ex}
			$f(x) = \dfrac{x^2-x}{3-x}$
	\begin{sol}
		$f(x) = \dfrac{x^2-x}{3-x} = \dfrac{x(x-1)}{3-x}$\\
		Domain: $(-\infty, 3) \cup (3, \infty)$\\
		$x$-intercepts:  $(0,0)$, $(1,0)$\\
		$y$-intercept:  $(0,0)$\\
		Vertical asymptote: $x = 3$\\
		As $x \rightarrow 3^{-}, f(x) \rightarrow \infty$\\
		As $x \rightarrow 3^{+}, f(x) \rightarrow -\infty$\\
		Slant asymptote: $y = -x-2$ \\
		As $x \rightarrow -\infty$, $y\to \infty$\\
		As $x \rightarrow \infty$, $y\to -\infty$

	\end{sol}
\end{ex}

\begin{ex}
	$f(x) = \dfrac{x^3+2x^2+x}{x^{2} -x-2} $
	\begin{sol}
		$f(x) = \dfrac{x^3+2x^2+x}{x^{2} -x-2} = \dfrac{x(x+1)}{x - 2} \, x \neq -1$\\
		Domain: $(-\infty, -1) \cup (-1, 2) \cup (2, \infty)$\\
		$x$-intercept:  $(0,0)$\\
		$y$-intercept:  $(0,0)$\\
		Vertical asymptote: $x = 2$\\
		As $x \rightarrow 2^{-}, f(x) \rightarrow -\infty$\\
		As $x \rightarrow 2^{+}, f(x) \rightarrow \infty$\\
		Hole at $(-1,0)$ \\
		Slant asymptote: $y = x+3$
		As $x \rightarrow -\infty$, $y\to -\infty$\\
As $x \rightarrow \infty$, $y\to \infty$
	\end{sol}
\end{ex}

\begin{ex}
		$f(x) = \dfrac{-x^{3} + 4x}{x^{2} - 9}$
	\begin{sol}
		$f(x) = \dfrac{-x^{3} + 4x}{x^{2} - 9}$\\
		Domain: $(-\infty, -3) \cup (-3, 3) \cup (3, \infty)$\\
		$x$-intercepts: $(-2, 0), (0, 0), (2, 0)$\\
		$y$-intercept: $(0, 0)$\\
		Vertical asymptotes: $x = -3, x = 3$\\
		As $x \rightarrow -3^{-}, \; f(x) \rightarrow \infty$\\
		As $x \rightarrow -3^{+}, \; f(x) \rightarrow -\infty$\\
		As $x \rightarrow 3^{-}, \; f(x) \rightarrow \infty$\\
		As $x \rightarrow 3^{+}, \; f(x) \rightarrow -\infty$\\
		Slant asymptote: $y = -x$\\
				As $x \rightarrow -\infty$, $y\to \infty$\\
		As $x \rightarrow \infty$, $y\to -\infty$
	\end{sol}
\end{ex}

\begin{ex}
	$f(x) = \dfrac{x^{2} - 2x + 1}{x^{3} + x^{2} - 2x}$
	\begin{sol}
		$f(x) = \dfrac{x^{2} - 2x + 1}{x^{3} + x^{2} - 2x}$\\
		Domain: $(-\infty, -2) \cup (-2, 0) \cup (0, 1) \cup (1, \infty)$\\
		$f(x) = \dfrac{x - 1}{x(x + 2)}, \; x \neq 1$\\
		No $x$-intercepts\\
		No $y$-intercepts\\
		Vertical asymptotes: $x = -2$ and $x = 0$\\
		As $x \rightarrow -2^{-}, \; f(x) \rightarrow -\infty$\\
		As $x \rightarrow -2^{+}, \; f(x) \rightarrow \infty$\\
		As $x \rightarrow 0^{-}, \; f(x) \rightarrow \infty$\\
		As $x \rightarrow 0^{+}, \; f(x) \rightarrow -\infty$\\
		Hole in the graph at $(1, 0)$\\
		Horizontal asymptote: $y = 0$\\
		As $x \rightarrow -\infty, \; f(x) \rightarrow 0^{-}$\\
		As $x \rightarrow \infty, \; f(x) \rightarrow 0^{+}$
	\end{sol}
\end{ex}

\end{multicols}

Solve the equation or inequality. For inequalities use a sign chart.

\begin{multicols}{2}
	
	\begin{ex}
		$\dfrac{3x - 1}{x^{2} + 1} = 1$
		\begin{sol}
			$x = 1$ or $x=2$
		\end{sol}
	\end{ex}
	
	
	\begin{ex}
		$\dfrac{x^{2} - 2x + 1}{x^{3} + x^{2} - 2x} = 1$
		\begin{sol}
			No solution.
		\end{sol}
	\end{ex}
	
	\begin{ex}
		 $\dfrac{x - 3}{x + 2} \leq 0$
		\begin{sol}
			 $(-2, 3]$
		\end{sol}
	\end{ex}
	
	\begin{ex}
		$\dfrac{x}{x^{2} - 1} > 0$
		\begin{sol}
			 $(-1, 0) \cup (1, \infty)$
		\end{sol}
	\end{ex}
	
	\begin{ex}
		$\dfrac{x^3+2x^2+x}{x^2-x-2} \geq 0$
		\begin{sol}
			$(-1,0] \cup (2, \infty)$
		\end{sol}
	\end{ex}
	
	\begin{ex}
		$\dfrac{2x + 17}{x + 1} > x + 5$
		\begin{sol}
			$(-\infty, -6) \cup (-1, 2)$
		\end{sol}
	\end{ex}
	
	\begin{ex}
		$\dfrac{-x^{3} + 4x}{x^{2} - 9} \geq 4x$
		\begin{sol}
			$(-\infty, -3) \cup \left[-2\sqrt{2}, 0\right] \cup \left[2\sqrt{2}, 3\right)$
		\end{sol}
	\end{ex}
		
	\begin{ex}
	$\dfrac{1}{x^{2} + 1} < 0$ 	
		\begin{sol}
			No solution.
		\end{sol}
	\end{ex}
	
\end{multicols}

	\begin{ex}
	 A box with a square base and no top is to be constructed so that it has a volume of 1000 cubic centimeters. What are the dimensions of the box which minimize the surface area? Give exact and approximate answers.
	\begin{sol}
		The base of the box should measure $\sqrt[3]{2000}$ cm by $\sqrt[3]{2000}$ cm (about 12.6 cm by 12.6 cm). The height should be $\frac{1000}{2000^{2/3}} \approx 6.3$ cm.
	\end{sol}
\end{ex}

\Closesolutionfile{ans}
\subsection*{Answers \nopunct} \hfill
\begin{multicols}{2}
	\input{ans52}
\end{multicols}




\newpage
\section{Systems of linear equations and row reduction of matrices}

Solutions of systems of equations in $n$ variables are $n$-tuples.

Systems of equations are called \textbf{consistent} if there is at least one solution and \textbf{inconsistent} otherwise.\index{consistent system of equations}\index{inconsistent system of equations}

This section focuses on systems of linear equations and solving them via row reduction of an \textbf{augmented matrix}\index{augmented matrix}\index{matrix} representation.

\index{matrix}
Row operations\index{row operations} (create an equivalent system of equations):\index{matrix row operations}
\begin{enumerate}
	\item \textbf{Interchange} two rows.
	\item Replace a row with a nonzero \textbf{multiple} of itself.
	\item Replace a row with the \textbf{sum} of itself and multiple of another row.
\end{enumerate}

\bigskip
To be in \index{row echelon form} \textbf{row echelon form} all must be true:
\begin{enumerate}
	\item The first nonzero entry in each row is 1.
	\item A leading 1 of a given row must be to the right of leading 1's above it.
	\item All zero rows must be at the bottom.
\end{enumerate}
If \underline{additionally} the leading 1's are the only nonzero entries in their columns, then the matrix is in \textbf{reduced row echelon form}.\index{reduced row echelon form}

\Opensolutionfile{ans}[ans53]
\subsection*{Exercises \nopunct} \hfill

Solve the system of linear equations by writing it as an augmented matrix and performing row reduction to reduced row echelon form.

\begin{multicols}{2}

\begin{ex}
	$\left\{ \begin{array}{rcr} -5x + y & = & 17  \\ x + y & = & 5  \end{array} \right.$
	\begin{sol}
		 $(-2, 7)$
	\end{sol}
\end{ex}

\begin{ex}
	$\left\{ \begin{array}{rcr} x + y + z & = & 3 \\ 2x - y + z & = & 0 \\ -3x + 5y + 7z & = & 7  \end{array} \right.$
	\begin{sol}
		 $(1, 2, 0)$
	\end{sol}
\end{ex}

\begin{ex}
	 $\left\{ \begin{array}{rcr} 4x - y + z & = & 5 \\ 2y + 6z & = & 30 \\ x + z & = & 5  \end{array} \right.$
	\begin{sol}
	$(-t + 5, -3t + 15, t)$\\
	for all real numbers $t$
	\end{sol}
\end{ex}

\begin{ex}
	 $\left\{ \begin{array}{rcr} x-2y+3z & = & 7 \\ -3x+y+2z & = & -5 \\ 2x+2y+z & = & 3  \end{array} \right.$
	\begin{sol}
		$(2,-1,1)$
	\end{sol}
\end{ex}


\begin{ex}
	$\left\{ \begin{array}{rcr} 3x-2y+z & = & -5 \\ x+3y-z & = & 12 \\ x+y+2z & = & 0  \end{array} \right.$
	\begin{sol}
		 $(1,3,-2)$
	\end{sol}
\end{ex}

\begin{ex}
	 $\left\{ \begin{array}{rcr} 2x-y+z& = & -1 \\ 4x+3y+5z & = & 1 \\  5y+3z & = & 4 \end{array} \right.$
	\begin{sol}
	 	Inconsistent
	\end{sol}
\end{ex}


\begin{ex}
	 $\left\{ \begin{array}{rcr} x-y+z & = & -4 \\ -3x+2y+4z & = & -5 \\ x-5y+2z & = & -18  \end{array} \right.$
	\begin{sol}
		$(1,3,-2)$
	\end{sol}
\end{ex}

\begin{ex}
	 $\left\{ \begin{array}{rcr} 2x-4y+z & = & -7 \\ x-2y+2z & = & -2 \\ -x+4y-2z & = & 3  \end{array} \right.$
	
	\begin{sol}
		 $\left(-3,\frac{1}{2},1\right)$
	\end{sol}
\end{ex}


\begin{ex}
	 $\left\{ \begin{array}{rcr} 2x-y+z & = & 1 \\ 2x+2y-z & = & 1 \\ 3x+6y+4z & = & 9  \end{array} \right.$
	\begin{sol}
		 $\left(\frac{1}{3},\frac{2}{3},1\right)$
	\end{sol}
\end{ex}

\begin{ex}
	 $\left\{ \begin{array}{rcr} x-3y-4z & = & 3 \\ 3x+4y-z & = & 13 \\ 2x-19y-19z & = & 2  \end{array} \right.$
	\begin{sol}
		 $\left(\frac{19}{13} t + \frac{51}{13},-\frac{11}{13} t+\frac{4}{13},t\right)$\\
		for all real numbers $t$
	\end{sol}
\end{ex}


\begin{ex}
	$\left\{ \begin{array}{rcr} x+y+z & = & 4 \\ 2x-4y-z& = & -1 \\ x-y & = & 2 \end{array} \right.$
	\begin{sol}
		Inconsistent
	\end{sol}
\end{ex}

\begin{ex}
	 $\left\{ \begin{array}{rcr} x-y+z & = & 8 \\ 3x+3y-9z & = & -6 \\  7x-2y+5z & = & 39 \end{array} \right.$
	\begin{sol}
		 $\left(4,-3,1\right)$
	\end{sol}
\end{ex}

\begin{ex}
	$\left\{ \begin{array}{rcr} 2x-3y+z & = & -1 \\ 4x-4y+4z & = & -13 \\ 6x-5y+7z & = & -25  \end{array} \right.$
	\begin{sol}
		$\left(-2t - \frac{35}{4},-t - \frac{11}{2},t\right)$\\
		for all real numbers $t$
	\end{sol}
\end{ex}

\begin{ex}
	$\left\{ \begin{array}{rcr} x_{\mbox{\tiny$1$}} - x_{\mbox{\tiny$3$}} & = & -2 \\ 
	2x_{\mbox{\tiny$2$}} - x_{\mbox{\tiny$4$}} & = & 0  \\  
	x_{\mbox{\tiny$1$}} -  2x_{\mbox{\tiny$2$}} + x_{\mbox{\tiny$3$}} & = & 0 \\
	-x_{\mbox{\tiny$3$}} + x_{\mbox{\tiny$4$}} & = & 1  \end{array} \right.$
	\begin{sol}
		$(1, 2, 3, 4)$
	\end{sol}
\end{ex}

\end{multicols}

\begin{ex}
	A local buffet charges $\$7.50$ per person for the basic buffet and $\$9.25$ for the deluxe buffet (which includes crab legs.)  If 27 diners went out to eat and the total bill was $\$227.00$ before taxes, how many chose the basic buffet and how many chose the deluxe buffet?
	\begin{sol}
		$13$ chose the basic buffet and $14$ chose the deluxe buffet.
	\end{sol}
\end{ex}


\begin{ex}
	A $10 \%$ salt solution is to be mixed with pure water to produce 75 gallons of a $3\%$ salt solution.  How much of each are needed?
	\begin{sol}
		$22.5$ gallons of the $10 \%$ solution and $52.5$ gallons of pure water.
	\end{sol}
\end{ex}

\begin{ex}
	Skippy has a total of $\$$10,000 to split between two investments.  One account offers $3\%$ simple interest, and the other account offers $8\%$ simple interest.  For tax reasons, he can only earn $\$500$ in interest the entire year.  How much money should Skippy invest in each account to earn $\$500$ in interest for the year?
	\begin{sol}
		Skippy needs to invest $\$$6000 in the $3\%$ account and $\$$4000 in the $8 \%$ account.
	\end{sol}
\end{ex}

\begin{comment}
\begin{ex}
	It's time for another meal at our local buffet.  This time, 22 diners (5 of whom were children) feasted for $\$162.25$, before taxes.  If the kids buffet is $\$4.50$, the basic buffet is $\$7.50$, and the deluxe buffet (with crab legs) is $\$9.25$, find out how many diners chose the deluxe buffet. 
	\begin{sol}
		This time, 7 diners chose the deluxe buffet.
	\end{sol}
\end{ex}
\end{comment}


\begin{ex}
	Find the quadratic function passing through the points $(-2,1)$, $(1,4)$, and $(3,-2)$.
	\begin{sol}
		$f(x) = -\frac{4}{5} x^2+\frac{1}{5} x + \frac{23}{5}$
	\end{sol}
\end{ex}


\begin{ex}
	At The Crispy Critter's Head Shop and Patchouli Emporium along with their dried up weeds, sunflower seeds and astrological postcards they sell an herbal tea blend.  By weight, Type I herbal tea is 30\% peppermint, 40\% rose hips and 30\% chamomile, Type II has percents 40\%, 20\% and 40\%, respectively, and Type III has percents 35\%, 30\% and 35\%, respectively.  How much of each Type of tea is needed to make 2 pounds of a new blend of tea that is equal parts peppermint, rose hips and chamomile?  
	\begin{sol}
		$\frac{4}{3}- \frac{1}{2}t$ pounds of Type I, $\frac{2}{3} - \frac{1}{2}t$ pounds of Type II and $t$ pounds of Type III where $0 \leq t \leq \frac{4}{3}$.
	\end{sol}
\end{ex}



\Closesolutionfile{ans}
\subsection*{Answers \nopunct} \hfill
\begin{multicols}{2}
	\input{ans53}
\end{multicols}


\newpage
\section{Partial fraction decomposition}
\index{partial fraction decomposition}\index{proper rational function}

Partial Fraction Decomposition (PFD) splits a \underline{proper} rational function into a sum of simpler, proper ones.
\qi{Divide first if not proper. The decomposition will then be a polynomial plus a sum of simple rational terms.}

Three variations on solving for the unknowns:
\begin{enumerate}
	\item ``long method'': get a system of linear equations by equating coefficients then solve it. It should be consistent with a unique solution (this fact can help find errors).
	\item ``shortcut method'': get a coefficient quickly for each zero. Works great for distinct linear factors.
	\item ``hybrid method(s)'': use shortcut method to get as many coefficients as possible, then either use additional points or equating coefficients to get additional equations to solve for the remaining coefficients.
\end{enumerate}

\bigskip
\begin{tabular}{|l|c|}
	\hline
	denominator term... & add to PFD... \\
	\hline
	 & \\
	{\small linear, multiplicity 1:} & \\
	$(mx+b)$ & $ \dfrac{A}{mx+b} $ \\
	 & \\
	\hline
	 & \\
	{\small irreducible quadratic, non-repeated:} & \\
$(ax^2+bx+c)$ & $ \dfrac{Ax+B}{ax^2+bx+c} $ \\			 
& \\
\hline		
& \\
		{\small linear, multiplicity $n$:} & \\
	$(mx+b)^n$ & $ \dfrac{A_1}{(mx+b)} +  \dfrac{A_2}{(mx+b)^2} + \cdots +1 \dfrac{A_n}{(mx+b)^n} $ \\
	 & \\
\hline		
			 & \\

{\small irreducible quadratic, repeated:} & \\
$(ax^2+bx+c)^n$ & $ \dfrac{A_1x+B_1}{(ax^2+bx+c)} + \dfrac{A_2x+B_2}{(ax^2+bx+c)^2} + \cdots + \dfrac{A_nx+B_n}{(ax^2+bx+c)^n}$ \\	
	 & \\
\hline	
\end{tabular}

\Opensolutionfile{ans}[ans54]
\subsection*{Exercises \nopunct} \hfill

Find the partial fraction decomposition of the following rational functions. For the first five, use both the ``long method'' and the shortcut/hybrid method. Solve the resulting equations using any appropriate method but keep work clear.

\begin{multicols}{2}

\begin{ex}
	$\dfrac{2x+3}{x^2+x}$
	\begin{sol}
		$\dfrac{3}{x} - \dfrac{1}{x+1}$
	\end{sol}
\end{ex}

\begin{ex}
	$\dfrac{5x+1}{x^2+x-2}$
	\begin{sol}
		$\dfrac{3}{x+2} + \dfrac{2}{x-1}$
	\end{sol}
\end{ex}


\begin{ex}
	$\dfrac{-2x^2+9x+8}{x^3-x^2-2x}$
	\begin{sol}
		$\dfrac{-1}{x+1} + \dfrac{3}{x-2}-\dfrac{4}{x}$
	\end{sol}
\end{ex}


\begin{ex}
	$\dfrac{2x-1}{x^3+2x^2+x}$
	\begin{sol}
		$-\dfrac{1}{x}+\dfrac{1}{x+1}+ \dfrac{3}{(x+1)^2} $
	\end{sol}
\end{ex}

\begin{ex}
$\dfrac{-2x^2+4x+7}{x^3-2x^2+3x-6}$
	\begin{sol}
	$\dfrac{1}{x-2} - \dfrac{3x+2}{x^2+3}$
	\end{sol}
\end{ex}


\begin{ex}
	$\dfrac{4x^3-6x^2-3x-4}{2x^2-4x}$
	\begin{sol}
		$2x+1+\dfrac{1}{x}-\dfrac{1/2}{x-2}$
	\end{sol}
\end{ex}

\begin{ex}
	$\dfrac{3x^2-4x-5}{x^2+x-2}$
	\begin{sol}
		$3-\dfrac{2}{x-1}-\dfrac{5}{x+2}$
	\end{sol}
\end{ex}

\begin{ex}
	$\dfrac{x^5-x^4+4x^2-2x+2}{x^3-x^2+x}$
	\begin{sol}
		$x^2-1 +\dfrac{2}{x}+\dfrac{x+1}{x^2-x+1}$
	\end{sol}
\end{ex}

\begin{ex}
	$\dfrac{-2x^2-6x+2}{6x^3-9x^2+3x}$
	\begin{sol}
		$\dfrac{1}{x-1/2}+\dfrac{2/3}{x}-\dfrac{2}{x-1}$
	\end{sol}
\end{ex}

\begin{ex}
	$\dfrac{-x^3+7x^2-8x}{x^4-2x^3+2x-1}$
	\begin{sol}
		$\dfrac{1}{x-1}+\dfrac{2}{(x-1)^2} -\dfrac{1}{(x-1)^3}-\dfrac{2}{x+1}$
	\end{sol}
\end{ex}

\end{multicols}

\Closesolutionfile{ans}
\subsection*{Answers \nopunct} \hfill
\begin{multicols}{2}
	\input{ans54}
\end{multicols}



\newpage
%\section{Some hard equations}

\section{Law of Sines and Law of Cosines}

\begin{multicols}{2}

\begin{tikzpicture} [scale=0.75]
\draw (0,0) -- (5,0) -- (7,4) -- (0,0);
\draw (0.7,0) node[above right] {$A$};
\draw (6.7,3.63) node[below left] {$B$};
\draw (5.0,0) node[above left] {$C$};
\draw (2.8,2.2) node {$c$};
\draw (2.5,-0.4) node {$b$};
\draw (6.4,1.8) node {$a$};
\end{tikzpicture}

\textsc{Law of Sines and Law of Cosines\\ (for \underline{all} triangles)}
\index{law of sines}\index{law of cosines}

\medskip
$\ds \frac{\sin(A)}{a} = \frac{\sin(B)}{b} = \frac{\sin(C)}{c}$

\smallskip
$c^2 = a^2 + b^2 - 2ab\cos(C) $

\end{multicols}


\Opensolutionfile{ans}[ans55]
\subsection*{Exercises \nopunct} \hfill

Solve for the remaining side(s) and angle(s) in the triangles. Side-angle pairs are named as in the diagram above. Give approximate answers to two decimal places.

\begin{multicols}{2}
	\begin{ex}
		$A = 13^{\circ}, \; B = 17^{\circ}, \; a = 5$ 
		\begin{sol}
		$\begin{array}{lll}
		A = 13^{\circ} & B = 17^{\circ} & C = 150^{\circ} \\
		a = 5 & b \approx 6.50 & c \approx 11.11 \end{array}$	
		\end{sol}
	\end{ex}
	
	\begin{ex}
		$A = 73.2^{\circ}, \; B = 54.1^{\circ}, \; a = 117$
		\begin{sol}
		$\begin{array}{lll}
		A = 73.2^{\circ} & B = 54.1^{\circ} & C = 52.7^{\circ} \\
		a = 117 & b \approx 99.00 & c \approx 97.22 \end{array}$	
		\end{sol}
	\end{ex}

\begin{ex}
	 $a = 7, \; b = 12, \; C = 59.3^{\circ}$ 
	\begin{sol}
		$\begin{array}{lll}
		A \approx 35.54^{\circ} & B \approx 85.16^{\circ} & C = 59.3^{\circ} \\
		a = 7 & b = 12 & c \approx 10.36 \end{array}$
	\end{sol}
\end{ex}


\begin{ex}
	 $A = 104^{\circ}, \; b = 25, \; c  = 37$
	\begin{sol}
		$\begin{array}{lll}
		A = 104^{\circ} & B \approx 29.40^{\circ} & C \approx 46.60^{\circ} \\
		a \approx 49.41 & b = 25 & c = 37 \end{array}$
	\end{sol}
\end{ex}

\begin{ex}
	$A = 95^{\circ}, \; B = 85^{\circ}, \; a = 33.33$
	\begin{sol}
		\begin{tabular}{l}
			Information does not \\
			produce a triangle \end{tabular}
	\end{sol}
\end{ex}


\begin{ex}
	$a = 153, \; B = 8.2^{\circ}, \; c = 153$
	\begin{sol}
		 $\begin{array}{lll}
		A \approx 85.90^{\circ} & B = 8.2^{\circ} & C \approx 85.90^{\circ} \\
		a = 153 & b \approx 21.88 & c = 153 \end{array}$
	\end{sol}
\end{ex}

\begin{ex}
	$A = 95^{\circ}, \; B = 62^{\circ}, \; a = 33.33$
	\begin{sol}
	 $\begin{array}{lll}
	A = 95^{\circ} & B = 62^{\circ} & C = 23^{\circ} \\
	a = 33.33 & b \approx 29.54 & c \approx 13.07 \end{array}$	
	\end{sol}
\end{ex}



\begin{ex}
	 $a = 3, \; b = 4, \; C = 90^{\circ}$
	\begin{sol}
		$\begin{array}{lll}
		A \approx 36.87^{\circ} & B \approx 53.13^{\circ} & C = 90^{\circ} \\
		a = 3 & b = 4 & c = 5 \end{array}$
	\end{sol}
\end{ex}

\begin{ex}
	$A = 117^{\circ}, \; a = 35, \; b = 42$
	\begin{sol}
		\begin{tabular}{l}
			Information does not \\
			produce a triangle \end{tabular}
	\end{sol}
\end{ex}

\begin{ex}
	 $A = 117^{\circ}, \; a = 45, \; b = 42$
	\begin{sol}
		$\begin{array}{lll}
		A = 117^{\circ} & B \approx 56.3^{\circ} & C \approx 6.7^{\circ} \\
		a = 45 & b = 42 & c \approx 5.89 \end{array}$
	\end{sol}
\end{ex}


\begin{ex}
	$A = 120^{\circ}, \; b = 3, \; c = 4$
	\begin{sol}
		 $\begin{array}{lll}
		A = 120^{\circ} & B \approx 25.28^{\circ} & C \approx 34.72^{\circ} \\
		a = \sqrt{37} & b = 3 & c = 4 \end{array}$
	\end{sol}
\end{ex}
\begin{ex}
	$a = 7, \; b = 10, \; c = 13$ 
	\begin{sol}
		 $\begin{array}{lll}
		A \approx 32.21^{\circ} & B \approx 49.58^{\circ} & C \approx 98.21^{\circ} \\
		a = 7 & b = 10 & c = 13 \end{array}$
	\end{sol}
\end{ex}

\begin{ex}
	$A = 68.7^{\circ}, \; a = 88, \; b = 92$
	\begin{sol}
		 $\begin{array}{lll}
		A = 68.7^{\circ} & B \approx 76.9^{\circ} & C \approx 34.4^{\circ} \\
		a = 88 & b = 92 & c \approx 53.36 \end{array}$
		
		$\begin{array}{lll}
		A = 68.7^{\circ} & B \approx 103.1^{\circ} & C \approx 8.2^{\circ} \\
		a = 88 & b = 92 & c \approx 13.47\end{array}$
	\end{sol}
\end{ex}

\begin{ex}
	 $A = 42^{\circ}, \; a = 17, \; b = 23.5$
	\begin{sol}
	 $\begin{array}{lll}
	A = 42^{\circ} & B \approx 67.66^{\circ} & C \approx 70.34^{\circ} \\
	a = 17 & b = 23.5 & c \approx 23.93 \end{array}$
	
	$\begin{array}{lll}
	A = 42^{\circ} & B \approx 112.34^{\circ} & C \approx 25.66^{\circ} \\
	a = 17 & b = 23.5 & c \approx 11.00 \end{array}$
	\end{sol}
\end{ex}


\begin{ex}
	$a = 1, \; b = 2, \; c = 5$
	\begin{sol}
		Information does not produce a triangle.
		
		These numbers violate the ``triangle inequality'' which states that the sum of the lengths of the two shorter sides of a triangle must be greater than the length of the longest side (Why?).
	\end{sol}
\end{ex}
\end{multicols}

\begin{ex}
	 Skippy and Sally decide to hunt UFOs.  One night, they position themselves 2 miles apart on an abandoned stretch of desert runway.  An hour into their investigation, Skippy spies a UFO hovering over a spot on the runway directly between him and Sally.  He records the angle of inclination from the ground to the craft to be $75^{\circ}$ and radios Sally immediately to find the angle of inclination from her position to the craft is $50^{\circ}$.  How high off the ground is the UFO at this point?  Round your answer to the nearest foot.  (Recall:  1 mile is 5280 feet.)
	\begin{sol}
		9539 ft.
	\end{sol}
\end{ex}

\begin{ex}
	 The angle of depression from an observer in an apartment complex to a gargoyle on the building next door is $55^{\circ}$.  From a point five stories below the original observer, the angle of inclination to the gargoyle is $20^{\circ}$.  Find the distance from each observer to the gargoyle and the distance from the gargoyle to the apartment complex.  Round your answers to the nearest foot.  (Use the rule of thumb that one story of a building is 9 feet.)  
	\begin{sol}
		The gargoyle is about 44 feet from the observer on the upper floor. \\
		The gargoyle is about 27 feet from the observer on the lower floor. \\
		The gargoyle is about 25 feet from the other building.
	\end{sol}
\end{ex}

\begin{ex}
	Alice is using a laser rangefinder to measure the distance between her and objects she is spotting. First, she spots a target 5.1 km away from her.  Alice then turns $62^{\circ}$ to the right and spots a supply truck. Alice measures the supply truck to be 9.4 km away from her. How far is the supply truck from the target? Give an approximate answer to the nearest tenth kilometer.
	\begin{sol}
		8.3 km
	\end{sol}
\end{ex}

\begin{ex}
	 Doctor H and Doctor W are 5.7 miles apart. They both spot a drowning precalculus student. Doctor H measures the angle between Doctor W and the student to be $63^\circ$. Doctor W measures the angle between the student and Doctor H to be $37^\circ$. How far is the drowning student from Doctor H? Give an exact answer, then approximate.
	
	\begin{tikzpicture}[line cap=round,line join=round,x=1.0cm,y=1.0cm, scale=3]
%	\draw [shift={(3.4516233546070385,-0.8682825703917814)},color=black,fill=black,fill opacity=0.1] (0,0) -- (1.240869775130875:0.18197628096383797) arc (1.240869775130875:44.32004597394672:0.18197628096383797) -- cycle;
%	\draw [shift={(6.5066191016969235,-0.8021093773140221)},color=black,fill=black,fill opacity=0.1] (0,0) -- (151.24002055130532:0.18197628096383797) arc (151.24002055130532:181.24086977513082:0.18197628096383797) -- cycle;
	\draw (3.4516233546070385,-0.8682825703917814)-- (4.594213821749683,0.24750547486314736);
	\draw (4.594213821749683,0.24750547486314736)-- (6.5066191016969235,-0.8021093773140221);
	\draw (6.5066191016969235,-0.8021093773140221)-- (3.4516233546070385,-0.8682825703917814);
	\draw[color=black] (3.73,-0.75);% node {$A$};
	\draw[color=black] (3.3,-1) node {Doctor H.};
	\draw[color=black] (5,-0.95) node {5.7 mi};
	\draw[color=black] (6.15,-0.73);% node {$B$};
	\draw[color=black] (6.8,-.9) node {Doctor W.};
	\draw[color=black] (4.55,0.35) node {Student};
	\end{tikzpicture}
	\begin{sol}
		$ 5.7\frac{\sin 37^dg}{\sin 80\dg} \approx 3.5$ miles
	\end{sol}
\end{ex}

\begin{ex}
	A geologist wants to measure the diameter of a crater.   From her camp, it is 4 miles to the northern-most point of the crater and 2 miles to the southern-most point.  If the angle between the two lines of sight is $117^{\circ}$, what is the diameter of the crater?  Round your answer to the nearest hundredth of a mile.
	\begin{sol}
		$5.22$ mi
	\end{sol}
\end{ex}

\begin{ex}
	Jack measures the angle of elevation of the peak of a distant mountain to be $35.0\dg$. He then walks 1000 feet toward the mountain then measures the new angle of elevation to be $40.3\dg$. How high is the peak (over the horizontal plane on which Jack is walking)? Find an exact answer then round to two significant figures. Why might two significant figures be appropriate?
	\begin{sol}
		$1000\frac{\sin139.7\dg \sin 35\dg}{\sin 5.3\dg} \approx 4000$ ft.
		
		If the angles were measured to tenth-of-a-degree accuracy, then the angle $5.3\dg$ in the calculation only has two significant figures, so expecting more accuracy might not be appropriate. Additionally it is unclear how accurate the linear measurement was.
	\end{sol}
\end{ex}

\begin{ex}
	Dr. Herning is vacationing with his family in the Adirondacks. He wishes to measure the distance from the launch point of his canoe to an outcropping of rocks on the other side of the lake. Luckily he has brought his compass and his slide rule. He measures the bearing from the launch to the rocks to be 246.5$^\circ$ (his compass measures azimuth angles). He walks in a straight line to another point on the shore, counting 250 paces. From there he measures the bearing back to the boat launch to be 138.5$^\circ$ and the bearing to the rocks to be 237.5$^\circ$.
	
	\smallskip
	If 13 of Dr. Herning's paces are about 30 feet, how far is the boat launch from the rocks in miles? Find an approximate answer to two significant figures. [There are 5280 feet in a mile.]
	\begin{sol}
		$0.69$ miles
		\end{sol}
\end{ex}


\Closesolutionfile{ans}
\subsection*{Answers \nopunct} \hfill
\begin{multicols}{2}
	\input{ans55}
\end{multicols}


\newpage
\section{Vectors and the dot product}

\textbf{Vectors}\index{vectors} are quantities with a magnitude (``length'') and direction. They are often visualized as arrows.
\qi{Notice the length and direction are what define the vector, not the position. Thus the same vector $\vec{u}$ appears in two places below.}
\qi{The vector $-\vec{u}$ has the same length but opposite direction.}
\qi{Two operations are defined on vectors: addition and scalar multiplication.}

\begin{tikzpicture} [scale = 2]
	\draw [->]  (0,0) -- (2,1) node [right] {$\vec{u}$};
	\draw [->]  (1,-1) -- (3,0) node [right] {$\vec{u}$};
	\draw [<-]  (3,-1) -- (5,0) node [right] {$-\vec{u}$};
	\draw [->, dashed]  (7,0) -- (7,-1) node [right] {$\vec{v}$};

\end{tikzpicture}

\begin{tikzpicture} [scale = 2]
	\draw [->]  (8,-1) -- (9.98,-0.01) node [above left] {$\vec{u}$};
	\draw [->, dashed]  (10,0) -- (10,-0.98) node [above right] {$\vec{v}$};
\draw  [dotted, ->]  (8,-1) -- (9.98,-1);
\draw(9.3,-1) node [above left ] {$\vec{u}+\vec{v}$};
	\draw [->, dashed]  (8,-1) -- (8,-1.99) node [above left] {$\vec{v}$};
\draw [->]  (8,-2) -- (9.98,-1.01) node [below right] {$\vec{u}$};
\end{tikzpicture}

We define $\vec{u}-\vec{v}$ to mean $\vec{u} + \left( -\vec{v} \right)$.

\index{vector additions}\index{scalar multiplication (of a vector)}
\qi{Multiplication by a scalar multiplies the length of the vector by the size of the scalar and reverses orientation if it's negative.}

In the Cartesian plane, vectors can be described by (scalar) \textbf{components}\index{components of a vector}, $\vec{u} = \langle x_0, y_0 \rangle$. The components $x_0$ and $y_0$ correspond to the coordinates of the point the vector would point at if its tail was at the origin. Alternatively, The components representation is $\langle \Delta x, \Delta y \rangle$ where the differences are measured from tail to head. That is, the  $x$-component, $x_0$, describes the horizontal displacement from the tail to the head of the vector, while the $y$-component, $y_0$ describes the vertical displacement.

\begin{tikzpicture}
	\drawgridxxyy{-1}{5.5}{-1}{3}
	\draw [thick,->] (1,3) -- (5,1) node [above right] {$\langle 4, -2\rangle$};
	\draw [thick,->] (0,0) -- (1,2) node [below right] {$\langle 1, 2\rangle$};
\end{tikzpicture}

Components make vector operations easy as well:
\begin{itemize}
	\item addition: \ \ $\langle x_0,y_0\rangle + \langle x_1,y_1\rangle = \langle x_0+x_1,y_0+y_1\rangle$
	\item scalar multiplication: \ \ $c\langle x_0,y_0\rangle = \langle cx_0,cy_0\rangle$
\end{itemize}


In three dimensional Cartesian space, $\vec{u} = \langle x_0, y_0, z_0 \rangle$, but we will mainly stick to two dimensions.


\begin{tikzpicture} [scale = 1.5]
\drawgridxxyyb{-0.25}{1}{-0.25}{1}

\draw [->] (0,0) -- (-0.98,1.96);
\draw[fill=black] (-1,2) circle [radius=0.03] node [above left] {$Q=( x_1,y_1)$};
\draw [->] (0,0) -- (2.97,2.97);
\draw[fill=black] (3,3) circle [radius=0.03] node [right] {$P=( x_0,y_0)$};
\draw (1.5, 1.5) node [below right] {$\vec{p}= \langle x_0,y_0 \rangle$};
\draw (-0.5, 1) node [below left] {$\vec{q}=\langle x_1,y_1 \rangle$};
\draw [->] (3,3) -- (-0.96,2.01);
\draw (0.7,3.4) node [above] {$\overrightarrow{PQ} = \vec{q}-\vec{p} = \langle x_1-x_0,y_1-y_0 \rangle$\ \ \ \ \ \  {\tiny ``final - initial''}};
\draw (0.5,2.5) node [above] {$\overrightarrow{PQ}$};
\draw (0.5,2.3) node [below] {$\vec{q}-\vec{p}$};
\end{tikzpicture}

\qi{The magnitude of a vector in two dimensions is $\| \langle x_0,y_0 \rangle \| = \sqrt{x_0^2+y_0^2}$.}


\qi{The \textbf{dot product}\index{dot product}\index{angle between vectors} of vectors $\vec{u} = \langle x_0,y_0 \rangle$ and $\vec{v} = \langle x_1,y_1 \rangle$ is the number $\vec{u} \cdot \vec{v} = x_0x_1+y_0y_1 $ it satisfies
$$ \vec{u} \cdot \vec{v} = \|\vec{u}\|\, \| \vec{v} \|\, \cos \theta $$
where $\theta$ is the angle between $\vec{u}$ and  $\vec{v}$.\ \ \ \   ($0\leq \theta \leq \pi$ or $0\dg \leq \theta \leq 180\dg$)
\qi{This is incredibly useful, but we will only use it to compute the angle between vectors. Common uses include projections and the computation of the work done by a force.}
\qi{Right angles between non-zero vectors are detected by dot product 0. A pair of vectors with dot product of 0 are often called ``orthogonal.''\index{orthogonal vectors}}
}

\Opensolutionfile{ans}[ans56]
\subsection*{Exercises \nopunct} \hfill


Use the given pair of vectors $\vec{v}$ and $\vec{w}$ to find the following quantities.  State whether the result is a vector or a scalar.  

\hspace{.15in} $\bullet \, \vec{v} + \vec{w} \;\;\;$ \hfill $\bullet \, \vec{w}  - 2\vec{v} \;\;\;$ \hfill $\bullet \, \| \vec{v} + \vec{w} \| \;\;\;$

\begin{multicols}{2}
	
	\begin{ex}
		 $\vec{v} = \left<12, -5\right>$, $\vec{w} = \left<3, 4\right>$ 
		\begin{sol}
				\begin{itemize}
					
					\item  $\vec{v} + \vec{w} = \left<15,-1 \right> $, vector
					\item  $\vec{w}  - 2\vec{v}  = \left<-21,14 \right>$, vector
					
					\item $\| \vec{v} + \vec{w} \| = \sqrt{226}$, scalar
			\end{itemize}
		\end{sol}
	\end{ex}
	
	\begin{ex}
		$\vec{v} = \left<-7, 24 \right>$, $\vec{w} = \left<-5, -12\right>$
		\begin{sol}
				\begin{itemize}
					
					\item  $\vec{v} + \vec{w} = \left<-12,12 \right> $, vector
					\item  $\vec{w}  - 2\vec{v}  = \left<9,-60 \right>$, vector
					
					\item $\| \vec{v} + \vec{w} \| = 12\sqrt{2}$, scalar
					\item  $\| \vec{v} \| + \| \vec{w}\| = 38$, scalar
					
				\end{itemize}

		\end{sol}
	\end{ex}

\begin{ex}
	 $\vec{v} = \left<2, -1 \right>$, $\vec{w} = \left<-2, 4 \right>$
	\begin{sol}
			\begin{itemize}
				
				\item  $\vec{v} + \vec{w} = \left<0,3\right> $, vector
				\item  $\vec{w}  - 2\vec{v}  = \left<-6,6 \right>$, vector
				
				\item $\| \vec{v} + \vec{w} \| = 3$, scalar
			\end{itemize}
			

	\end{sol}
\end{ex}

\begin{ex}
	 $\vec{v} = \left<10, 4 \right>$, $\vec{w} = \left<-2, 5 \right>$
	\begin{sol}
		\begin{itemize}
			
			\item  $\vec{v} + \vec{w} = \left<8,9\right> $, vector
			\item  $\vec{w}  - 2\vec{v}  = \left<-22, -3 \right>$, vector
			\item $\| \vec{v} + \vec{w} \| = \sqrt{145}$, scalar
			
		\end{itemize}
	
	\end{sol}
\end{ex}

\begin{ex}
	 $\vec{v} = \left<-\sqrt{3}, 1\right>$, $\vec{w} = \left<2\sqrt{3}, 2\right>$
	\begin{sol}
			\begin{itemize}
				
				\item  $\vec{v} + \vec{w} = \left<\sqrt{3},3\right> $, vector
				\item  $\vec{w}  - 2\vec{v}  = \left<4\sqrt{3}, 0 \right>$, vector
				\item $\| \vec{v} + \vec{w} \| = 2\sqrt{3}$, scalar
		
			\end{itemize}
		\end{sol}
\end{ex}

\begin{ex}
	$\vec{v} = \left<\frac{3}{5}, \frac{4}{5}\right>$, $\vec{w} = \left<-\frac{4}{5}, \frac{3}{5}\right>$
	\begin{sol}
			\begin{itemize}
				
				\item  $\vec{v} + \vec{w} = \left<-\frac{1}{5},\frac{7}{5}\right> $, vector
				\item  $\vec{w}  - 2\vec{v}  = \left<-2, -1 \right>$, vector
				\item $\| \vec{v} + \vec{w} \| = \sqrt{2}$, scalar
				
			\end{itemize}
	\end{sol}
\end{ex}

\end{multicols}


\begin{ex}
	Given the two points, find the coordinates of the vector $\overrightarrow{AB}$.
	\ssp
	\item \ \ \ $A(3,1)$, $B(-2,7)$.
	\item \ \ \ $A(5,-1,0)$, $B(4,0,-3)$.
	\esp
	\begin{sol}
	\ssp
	\item \ \ \ $\langle -5, 6 \rangle$.
	\item \ \ \ $\langle -1,1,-3 \rangle$.
	\esp
	\end{sol}
\end{ex}

\begin{multicols}{2}
This diagram may be useful if you have never seen quadrantal bearings. In this system, non-cardinal directions are described by an acute angle measured east or west of North or South.

\begin{tikzpicture} [scale=2]
	\draw [thick] (-1,0) -- (1,0) node [right] {\large E};
	\draw [->, thick] (0,-1) -- (0,1) node [above] {\Large N};
	\draw (-1,0) node[left] {\large W}; 
	\draw (0,-1) node[below] {\large S};
	\draw [->] (0,0) -- (0.2588, 0.9659) node [right] {\small N$15\dg$E};
	\draw [->] (0,0) -- (-0.8387, 0.5446) node [left] {\small N$57\dg$W};
	\draw [->] (0,0) -- (0.9397, -0.3420) node [right] {\small S$70\dg$E};
	\draw [->] (0,0) -- (-0.1736, -0.9848) node [left] {\small S$10\dg$W};
	\draw (0.3, -0.8) node [right] {\tiny \fbox{Quadrantal Bearings}}
;
\end{tikzpicture}

\end{multicols}

\begin{ex}
	A small boat leaves the dock at Camp DuNuthin and heads across the Nessie River at 17 miles per hour (that is, with respect to the water) at a bearing of  S$68^{\circ}$W.   The river is flowing due east at 8 miles per hour.  What is the boat's true speed and heading?  Round the speed to the nearest mile per hour and express the heading as a bearing, rounded to the nearest tenth of a degree.  
	\begin{sol}
		The boat's true speed is about 10 miles per hour at a heading of S$50.6^{\circ}$W.
	\end{sol}
\end{ex}

\begin{ex}
	\label{HMSSasquatchVectorBearing} The HMS Sasquatch leaves port with bearing S$20^{\circ}$E maintaining a speed of 42 miles per hour (that is, with respect to the water).  If the ocean current is 5 miles per hour with a bearing of N$60^{\circ}$E, find the HMS Sasquatch's true speed and bearing.  Round the speed to the nearest mile per hour and express the heading as a bearing, rounded to the nearest tenth of a degree. 
	\begin{sol}
		The HMS Sasquatch's true speed is about 41 miles per hour at a heading of S$26.8^{\circ}$E.
	\end{sol}
\end{ex}

\begin{ex}
	If the captain of the HMS Sasquatch in Exercise \ref{HMSSasquatchVectorBearing} wishes to reach Chupacabra Cove, an island 100 miles away at a bearing of  S$20^{\circ}$E from port, in three hours, what speed and heading should she set to take into account the ocean current?   Round the speed to the nearest mile per hour and express the heading as a bearing, rounded to the nearest tenth of a degree.  
	\begin{sol}
		She should maintain a speed of about 35 miles per hour at a heading of S$11.8^{\circ}$E.
	\end{sol}
\end{ex}

\begin{ex}
	In calm air, a plane flying from the Pedimaxus International Airport can reach Cliffs of Insanity Point in two hours by following a bearing of N$8.2^{\circ}$E at 96 miles an hour.  (The distance between the airport and the cliffs is 192 miles.)  If the wind is blowing from the southeast at 25 miles per hour, what speed and bearing should the pilot take so that she makes the trip in two hours along the original heading?  Round the speed to the nearest hundredth of a mile per hour and your angle to the nearest tenth of a degree.
	\begin{sol}
		She should fly at 83.46 miles per hour with a heading of N$22.1^{\circ}$E
	\end{sol}
\end{ex}

\begin{ex}
	The SS Bigfoot leaves Yeti Bay on a course of N$37^{\circ}$W at a speed of 50 miles per hour.  After traveling half an hour, the captain determines he is 30 miles from the bay and his bearing back to the bay is S$40^{\circ}$E.  What is the speed and bearing of the ocean current?  Round the speed to the nearest mile per hour and express the heading as a bearing, rounded to the nearest tenth of a degree.
	\begin{sol}
		The current is moving at about 10 miles per hour bearing N$54.6^{\circ}$W.
	\end{sol}
\end{ex}

\begin{ex}
	A $600$ pound Sasquatch statue is suspended by two cables from a gymnasium ceiling.  If  each cable makes a $60^{\circ}$ angle with the ceiling, find the tension on each cable.  Round your answer to the nearest pound.
	
	[Hints: the cable tension is the magnitude of the force the cable is applying to the statue. Note the statue is not moving, so the sum of all forces on it should be $\vec{0}$.]
	\begin{sol}
		The tension on each of the cables is about $346$ pounds.
	\end{sol}
\end{ex}

\begin{ex}
	A $300$ pound metal star is hanging on two cables which are attached to the ceiling.  The left hand cable makes a $72^{\circ}$ angle with the ceiling while the right hand cable makes a $18^{\circ}$ angle with the ceiling.  What is the tension on each of the cables?  Round your answers to three decimal places.
	\begin{sol}
		The tension on the left hand cable is $285.317$ lbs. and on the right hand cable is $92.705$ lbs.
	\end{sol}
\end{ex}

\begin{ex}
	Professor's brother needs to push his rusty pickup truck up his driveway. He measures the angle of inclination of the driveway to be $5\dg$ and his truck weighs $5,000$ lbs. If the average person can push with a force of 112 lbs, how many friends does Professor's brother need to recruit to push the truck, assuming he steers and doesn't contribute much to pushing?
	
	[Hints: Use an inclined coordinate system. What force is needed to balance out the component of the weight opposite the direction of the pushing?]
	\begin{sol}
		He'll need to recruit at least 4. It's close, though, so 5 would probably be safer.
	\end{sol}
\end{ex}

\bigskip
	Compute the dot product, $\vec{v} \cdot \vec{w}$, then find the angle in degrees between the two vectors. If not a common angle, approximate to two decimal places. Check your work by sketching the two angles with tails at the origin and visually observing the angle.
	
	\begin{multicols}{2}

\begin{ex}
	$\vec{v} = \left\langle -2, -7 \right\rangle$ and $\vec{w} = \left\langle 5, -9 \right\rangle$ 
	\begin{sol}
		 $\vec{v} \cdot \vec{w} = 53$
		
		$\theta =  45^{\circ}$ 
	\end{sol}
\end{ex}

\begin{ex}
	$\vec{v} = \left\langle -6, -5 \right\rangle$ and $\vec{w} = \left\langle 10, -12 \right\rangle$
	\begin{sol}
		 $\vec{v} \cdot \vec{w} = 0$
		
		$\theta =  90^{\circ}$ 
	\end{sol}
\end{ex}

\begin{ex}
	$\vec{v} = \left\langle 1, \sqrt{3} \right\rangle$ and $\vec{w} = \left\langle 1, -\sqrt{3} \right\rangle$
	\begin{sol}
		
		$\vec{v} \cdot \vec{w} = -2$
		
		$\theta =  120^{\circ}$ 
	\end{sol}
\end{ex}

\begin{ex}
	$\vec{v} = \left\langle  3, 4 \right\rangle$ and $\vec{w} = \left\langle -6, -8 \right\rangle$
	\begin{sol}
		
		$\vec{v} \cdot \vec{w} = -50$
		
		$\theta =  180^{\circ}$ 
	\end{sol}
\end{ex}

\begin{ex}
	$\vec{v} = \left\langle -2,1 \right\rangle$ and $\vec{w} = \left\langle 3,6 \right\rangle$
	\begin{sol}
		$\vec{v} \cdot \vec{w} = 0$
		
		$\theta =  90^{\circ}$ 
	\end{sol}
\end{ex}

\begin{ex}
	$\vec{v} = \left\langle 0,-7 \right\rangle$ and $\vec{w} = \left\langle -\sqrt{12},-2 \right\rangle$
	\begin{sol}
		$\vec{v} \cdot \vec{w} = 14$
		
		$\theta =  60^{\circ}$ 
	\end{sol}
\end{ex}

\begin{ex}
	$\vec{v} = \left\langle 1, 17 \right\rangle$ and $\vec{w} = \left\langle -1, 0 \right\rangle$
	\begin{sol}
		 $\vec{v} \cdot \vec{w} = -1$
		
		$\theta \approx  93.37^{\circ}$ 
	\end{sol}
\end{ex}

\begin{ex}
	 $\vec{v} = \left\langle 3, 4 \right\rangle$ and $\vec{w} = \left\langle 5, 12 \right\rangle$
	\begin{sol}
		 $\vec{v} \cdot \vec{w} = 63$
		
		$\theta  \approx  14.25^{\circ}$ 
	\end{sol}
\end{ex}

\begin{ex}
	$\vec{v} = \left\langle -4, -2 \right\rangle$ and $\vec{w} = \left\langle 1, -5 \right\rangle$
	\begin{sol}
		 $\vec{v} \cdot \vec{w} = 6$
		
		$\theta \approx  74.74^{\circ}$ 
	\end{sol}
\end{ex}

\begin{ex}
	$\vec{v} = \left\langle -5, 6 \right\rangle$ and $\vec{w} = \left\langle 4, -7 \right\rangle$
	\begin{sol}
		 $\vec{v} \cdot \vec{w} = -62$
		
		$\theta  \approx  169.94^{\circ}$ 
	\end{sol}
\end{ex}
	\end{multicols}



\Closesolutionfile{ans}
\subsection*{Answers \nopunct} \hfill
\begin{multicols}{2}
	\input{ans56}
\end{multicols}


\backmatter

\printindex

\end{document}


\begin{comment}
%------------ Exercise Prototype
\Opensolutionfile{ans}[ans1]
\subsection*{Exercises \nopunct} \hfill
\begin{ex}
	
	\begin{sol}
		
	\end{sol}
\end{ex}

\begin{ex}
	
	\begin{sol}
		
	\end{sol}
\end{ex}

\Closesolutionfile{ans}
\subsection*{Answers \nopunct} \hfill
\begin{multicols}{2}
	\input{ans1}
\end{multicols}
%------------ /Exercise Prototype
\end{comment}

